\documentclass{beamer}
%\documentclass{article}
\usepackage{animate}
\usepackage{graphics}
\setbeamertemplate{frametitle continuation}[from second][(contd.)]
\setbeamertemplate{itemize items}{--}
\beamertemplatenavigationsymbolsempty

\title{Proximal Policy Optimization with Dynamic Clipping}
\author{Student: Rishikesh Vaishnav\\Mentor: Sicun Gao}

\begin{document}
\maketitle
\begin{frame}[noframenumbering, allowframebreaks]{Introduction}
    Reinforcement Learning
    \begin{itemize}
        \item A general algorithmic technique that seeks to replicate behavioral
            learning.
        \item Basic vocabulary:
        \begin{itemize}
            \item \textbf{Environment}: a general setting with changeable
                parameters in which actions can be performed that affect these
                parameters
            \item \textbf{State}: a specific configuration (i.e. ``snapshot'')
                of an environment
            \item \textbf{Agent}: an entity that learns to accomplish a task in
                a specific evironment
            \item \textbf{Action}: a decision made by the agent that is intended
                to affect subsequent states
            \item \textbf{Episode}: a sequence of states and actions in an
                environment
            \item \textbf{Reward}: a number associated with a state-action pair
        \end{itemize}
        \item Overall goal: train an agent that picks actions such that the sum
            of the rewards over an episode is maximimized.
        \framebreak
        \item Example: cart-pole demo
			\animategraphics[loop,controls,width=\linewidth]{25}
				{cartpole_ex/coalesced/cartpole_ex-}{0}{372}
    \end{itemize}
    \framebreak
    Trust Region Policy Optimization
    \framebreak
    Proximal Policy Optimization
\end{frame}

\begin{frame}[noframenumbering, allowframebreaks]{Potential Shortcomings of PPO}
\end{frame}

\begin{frame}[noframenumbering, allowframebreaks]{Idea}
\end{frame}

\begin{frame}[noframenumbering, allowframebreaks]{Results}
\end{frame}

\begin{frame}[noframenumbering, allowframebreaks]{Future Directions}
\end{frame}

\end{document}
