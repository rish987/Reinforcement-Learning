\documentclass[tikz, border=1pt]{standalone}
\usepackage{pgfplots, pgfplotstable}
\pgfplotsset{compat=newest}
\usepackage{tikz}
\tikzset{declare function={gaussian(\s,\m)=(1/(\s * sqrt(2 * pi))) * exp((-1 
/ 2) * (((x - \m)/\s) ^ 2));}}

\newcommand{\getvalid}[1]
{
    \pgfmathsetmacro{\C}{
        ((\meanold ^ 2) * (\std ^ 2)) 
        - ((\mean ^ 2) * (\stdold ^ 2)) 
        - (2 * (\std ^ 2) * (\stdold ^ 2) * ln((\std / \stdold)
        * (1 + (#1 * \eps))))}

    \pgfmathsetmacro{\sqrtterm}{(\B ^ 2) - (4 * \A * \C)}

    % determine if limits can be calculated here
    \ifdim \A pt = 0.0pt
        \pgfmathsetmacro{\valid}{0}
    \else
        \ifdim \sqrtterm pt > 0.0pt
            \pgfmathsetmacro{\valid}{1}
        \else
            \pgfmathsetmacro{\valid}{0}
        \fi
    \fi

}
\newcommand{\linetooldgaus}[1]
{
    \addplot[dashed] coordinates {(#1, 0) (#1, {gaussian(\stdold, \meanold)})};
}
\newcommand{\filloldgauss}[3]
{
    \addplot [draw=none, fill=#3, domain=#1:#2, opacity=0.5] 
        {gaussian(\stdold, \meanold)} \closedcycle;
}

\pgfmathsetmacro{\eps}{0.2}

\pgfmathsetmacro{\meanold}{0}
\pgfmathsetmacro{\stdold}{1}
\pgfmathsetmacro{\stddecmult}{0.03}
\pgfmathsetmacro{\meandiffmult}{0.05}
\pgfmathsetmacro{\lowx}{-3}
\pgfmathsetmacro{\highx}{3}
\pgfmathsetmacro{\xrange}{\highx - \lowx}
\pgfmathsetmacro{\numsamples}{50}

\newcommand{\probdown}{}
\newcommand{\probup}{}

\newcommand{\newinttable}[2]
{
\pgfplotstablenew[
    create on use/x/.style={
        create col/expr={((\pgfplotstablerow/\numsamples) * \xrange) + \lowx}
    },
    create on use/y/.style={
        create col/expr={
            ((1/(\stdold * sqrt(2 * pi))) * exp((-1 / 2) 
            * (((\thisrow{x} - \meanold)/\stdold) ^ 2))))
            * (((\thisrow{x} > #1) && (\thisrow{x} < #2))?1:0)
        }
    },
    create on use/int/.style={
        create col/expr={\pgfmathaccuma+(\thisrow{y}+\prevrow{y})/2*(\thisrow{x}-\prevrow{x})}
    },
    columns={x,y,int}
]
{\numsamples}
\inttable
}

\begin{document}
\foreach \mult in {0,1,...,20}
{
    % get new mean
    \pgfmathsetmacro{\meandiff}{\mult * \meandiffmult}
    \pgfmathsetmacro{\mean}{\meanold + \meandiff}

    % get new standard deviation
    \pgfmathsetmacro{\stddec}{\mult * \stddecmult}
    \pgfmathsetmacro{\std}{\stdold - \stddec}

    % calculate edges
    \pgfmathsetmacro{\A}{(\std ^ 2) - (\stdold ^ 2)}
    \pgfmathsetmacro{\B}{
        2 * 
        ((\mean * (\stdold ^ 2)) - (\meanold * (\std ^ 2)))
    }
    
    \getvalid{1}

    \ifnum \valid = 1
        \pgfmathsetmacro{\validup}{1}
        \pgfmathsetmacro{\higherup}{((-1 * \B) - sqrt(\sqrtterm)) / (2 * \A)}
        \pgfmathsetmacro{\lowerup}{((-1 * \B) + sqrt(\sqrtterm)) / (2 * \A)}
    \else
        \pgfmathsetmacro{\validup}{0}
    \fi

    \getvalid{-1}
    
    \ifnum \valid = 1
        \pgfmathsetmacro{\validdown}{1}
        \pgfmathsetmacro{\higherdown}{((-1 * \B) - sqrt(\sqrtterm)) / (2 * \A)}
        \pgfmathsetmacro{\lowerdown}{((-1 * \B) + sqrt(\sqrtterm)) / (2 * \A)}
    \else
        \pgfmathsetmacro{\validdown}{0}
    \fi

    \newcommand{\extraticks}{}
    \ifnum \validdown = 1
        \ifnum \validup = 1
            \renewcommand{\extraticks}{\lowerdown, \lowerup, 
                \higherup, \higherdown}
        \fi
    \fi

    \begin{tikzpicture}
        \begin{axis}
            [ 
                axis x line=center,
                axis y line=center,
                every extra x tick/.style={xticklabel 
                    style={color=gray}, tick style=gray},
                extra x ticks = \extraticks,
                extra x tick labels = {},
                xlabel=$a$,
                ymin=0, ymax=1.5,
                legend pos = north west,
                xmin=\lowx, xmax=\highx
            ]
            \addplot [color=black, mark=none, samples=\numsamples] 
                {gaussian(\stdold, \meanold)}; 
            \addplot [color=blue, mark=none, samples=\numsamples]
                {gaussian(\std, \mean)}; 
            \legend{$\pi_{\theta_{old}}(s_t)$, $\pi_{\theta}(s_t)$, 
                $r_t > 1 + \epsilon$, $r_t < 1 - \epsilon$}
            \addlegendimage{only marks, mark=square*,color=black!50}
            \addlegendimage{only marks, mark=square*,color=black!25}
%            \addplot [color=red, mark=none, samples=50] {
%                gaussian(\std, \mean) / gaussian(\stdold, \meanold)}; 
%            \addplot [color=green, mark=none, samples=\numsamples]
%                {rect(1, 2)}; 
            \ifnum \validdown = 1
                \ifnum \validup = 1
                    \linetooldgaus{\lowerdown}
                    \linetooldgaus{\lowerup}
                    \linetooldgaus{\higherup}
                    \linetooldgaus{\higherdown}
                    \filloldgauss{\lowx}{\lowerdown}{black!25}
                    \filloldgauss{\lowerup}{\higherup}{black!50}
                    \filloldgauss{\higherdown}{\highx}{black!25}
                \fi
            \fi
        \end{axis}
        
        \pgfmathsetmacro{\pup}{0}
        \pgfmathsetmacro{\pdown}{0}
        \ifnum \validdown = 1
            \ifnum \validup = 1
                \newinttable{\lowx}{\lowerdown}
                \pgfplotstablegetelem{49}{int}\of\inttable
                \pgfmathsetmacro{\lu}{\pgfplotsretval}

                \newinttable{\lowerup}{\higherup}
                \pgfplotstablegetelem{49}{int}\of\inttable
                \pgfmathsetmacro{\pup}{\pgfplotsretval}

                \newinttable{\higherdown}{\highx}
                \pgfplotstablegetelem{49}{int}\of\inttable
                \pgfmathsetmacro{\lh}{\pgfplotsretval}

                \pgfmathsetmacro{\pdown}{\lh + \lu}
            \fi
        \fi

        \begin{axis}[
          ybar,
          enlargelimits=0.50,
          ymin=0, ymax=1,
          symbolic x coords={$p(r_t < 1 - \epsilon)$,$p(r_t > 1 + \epsilon)$},
          xtick=data,
          nodes near coords, 
          nodes near coords align={vertical},
          at={(0, -240)}
          ]
          \addplot coordinates {($p(r_t < 1 - \epsilon)$,\pdown) ($p(r_t > 1 + \epsilon)$,\pup)};
        \end{axis}
    \end{tikzpicture}
}
\end{document}
