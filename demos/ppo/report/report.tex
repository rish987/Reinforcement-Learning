\documentclass[a4paper]{article}
\setlength\parindent{0pt}

\usepackage{pgfplots}
\usepackage{amsthm, amsmath, amssymb, verbatim, enumerate, mathtools, algorithm}
\usepackage{pgf}
\usepackage{hyperref}
\def\labelitemi{--}
\def\labelitemii{--}
\def\labelitemiii{--}
\def\labelitemiv{--}
\pgfplotsset{compat=newest}

\pagestyle{empty}

\title{Proximal Policy Optimization Demo}
\author{Rishikesh Vaishnav}
\begin{document}
\maketitle
\subsection*{Code}
\begin{itemize}
    \item The code for this project is available at: 
    \url{https://github.com/rish987/Reinforcement-Learning/blob/master/demos/ppo/code/ppo.py}.
\end{itemize}
\subsection*{Implementation Details}
Pseudocode:
\begin{itemize}
    \item Initialize policy parameter $\theta$.
    \item Iterate until convergence:
    \begin{itemize}
        \item Initialize/clear list $S$ of $\{G, \pi_{\theta}(s, a), s, a\}$.
        \item Generate $N_{\tau}$ trajectories $\{\tau\}$, saving 
            $\pi_{\theta}(s, a)$ for each $s, a$ encountered.
        \item For each trajectory $\tau \in \{\tau\}$:
        \begin{itemize}
            \item For each $\{s, a\} \in \tau$:
            \begin{itemize}
                \item Calculate discounted return $G_{\theta}(s, a)$ from this
                    time to end of episode.
                \item Retrieve $\pi_{\theta}(a | s)$ at this $(s, a)$.
                \item Store $(G, \pi_{\theta}(a | s), s, a)$ in $S$.
            \end{itemize}
        \end{itemize}
        \item Use automatic differentiation library to calculate the gradient
            $\nabla_{\theta}L^{CLIP}(\theta)$ of:
        \begin{align*}
            L^{CLIP}(\theta) &= \frac{1}{|S|}\sum_{(G, \pi_{\theta_{old}}(a |
            s), s, a) \in S} min\left(\frac{\pi_{\theta}(a |
            s)}{\pi_{\theta_{old}}(a | s)}G, clip\left(\frac{\pi_{\theta}(a |
            s)}{\pi_{\theta_{old}}(a | s)}, 1 - \epsilon, 1 +
            \epsilon\right)G\right)
        \end{align*}
        \item $\theta = \theta + \alpha\nabla_{\theta}L^{CLIP}(\theta)$.
    \end{itemize}
\end{itemize}
Parameter Settings:
\begin{itemize}
    \item $\epsilon = 0.2$ (as per Schulman et. al.)
    \item $\gamma = 1$ (adjusted to maximize empirical performance)
\end{itemize}
Policy Function Encoding:
\begin{itemize}
    \item Each state-action pair is converted to the feature vector $x(s, a)$.
        Letting $S_{obs}$ and $S_{act}$ be the size of the observation and
        action spaces, respectively, the size of the vector is $S_{obs} \times
        S_{act}$, where all features are $0$ except for the $S_{obs}$ features
        starting at index $S_{obs} \times a$, which are set to the
        environment's parameterization of $s$.
    \begin{itemize}
        \item In this case, $S_{obs} = 4$ and $S_{act} = 2$.
    \end{itemize}
    \item The policy function $\pi(a | s, \theta)$ performs the softmax on a
        parameterized linear mapping of feature vectors:
        \begin{align*}
            \pi(a | s, \theta) &= \frac{e^{\theta^T x(s, a)}}
            {\sum_{b} e^{\theta^T x(s, b)}}\\
        \end{align*}
    \item Gradient calculation was performed on the objective using the 
        autograd library. Various constant learning rates were tested.
\end{itemize}
\subsection*{Results}
\begin{centering}
    \scalebox{0.6}{%% Creator: Matplotlib, PGF backend
%%
%% To include the figure in your LaTeX document, write
%%   \input{<filename>.pgf}
%%
%% Make sure the required packages are loaded in your preamble
%%   \usepackage{pgf}
%%
%% Figures using additional raster images can only be included by \input if
%% they are in the same directory as the main LaTeX file. For loading figures
%% from other directories you can use the `import` package
%%   \usepackage{import}
%% and then include the figures with
%%   \import{<path to file>}{<filename>.pgf}
%%
%% Matplotlib used the following preamble
%%   \usepackage{fontspec}
%%   \setmainfont{DejaVu Serif}
%%   \setsansfont{DejaVu Sans}
%%   \setmonofont{DejaVu Sans Mono}
%%
\begingroup%
\makeatletter%
\begin{pgfpicture}%
\pgfpathrectangle{\pgfpointorigin}{\pgfqpoint{6.400000in}{4.800000in}}%
\pgfusepath{use as bounding box, clip}%
\begin{pgfscope}%
\pgfsetbuttcap%
\pgfsetmiterjoin%
\definecolor{currentfill}{rgb}{1.000000,1.000000,1.000000}%
\pgfsetfillcolor{currentfill}%
\pgfsetlinewidth{0.000000pt}%
\definecolor{currentstroke}{rgb}{1.000000,1.000000,1.000000}%
\pgfsetstrokecolor{currentstroke}%
\pgfsetdash{}{0pt}%
\pgfpathmoveto{\pgfqpoint{0.000000in}{0.000000in}}%
\pgfpathlineto{\pgfqpoint{6.400000in}{0.000000in}}%
\pgfpathlineto{\pgfqpoint{6.400000in}{4.800000in}}%
\pgfpathlineto{\pgfqpoint{0.000000in}{4.800000in}}%
\pgfpathclose%
\pgfusepath{fill}%
\end{pgfscope}%
\begin{pgfscope}%
\pgfsetbuttcap%
\pgfsetmiterjoin%
\definecolor{currentfill}{rgb}{1.000000,1.000000,1.000000}%
\pgfsetfillcolor{currentfill}%
\pgfsetlinewidth{0.000000pt}%
\definecolor{currentstroke}{rgb}{0.000000,0.000000,0.000000}%
\pgfsetstrokecolor{currentstroke}%
\pgfsetstrokeopacity{0.000000}%
\pgfsetdash{}{0pt}%
\pgfpathmoveto{\pgfqpoint{0.800000in}{0.528000in}}%
\pgfpathlineto{\pgfqpoint{5.760000in}{0.528000in}}%
\pgfpathlineto{\pgfqpoint{5.760000in}{4.224000in}}%
\pgfpathlineto{\pgfqpoint{0.800000in}{4.224000in}}%
\pgfpathclose%
\pgfusepath{fill}%
\end{pgfscope}%
\begin{pgfscope}%
\pgfsetbuttcap%
\pgfsetroundjoin%
\definecolor{currentfill}{rgb}{0.000000,0.000000,0.000000}%
\pgfsetfillcolor{currentfill}%
\pgfsetlinewidth{0.803000pt}%
\definecolor{currentstroke}{rgb}{0.000000,0.000000,0.000000}%
\pgfsetstrokecolor{currentstroke}%
\pgfsetdash{}{0pt}%
\pgfsys@defobject{currentmarker}{\pgfqpoint{0.000000in}{-0.048611in}}{\pgfqpoint{0.000000in}{0.000000in}}{%
\pgfpathmoveto{\pgfqpoint{0.000000in}{0.000000in}}%
\pgfpathlineto{\pgfqpoint{0.000000in}{-0.048611in}}%
\pgfusepath{stroke,fill}%
}%
\begin{pgfscope}%
\pgfsys@transformshift{1.016418in}{0.528000in}%
\pgfsys@useobject{currentmarker}{}%
\end{pgfscope}%
\end{pgfscope}%
\begin{pgfscope}%
\pgftext[x=1.016418in,y=0.430778in,,top]{\sffamily\fontsize{10.000000}{12.000000}\selectfont \(\displaystyle 0\)}%
\end{pgfscope}%
\begin{pgfscope}%
\pgfsetbuttcap%
\pgfsetroundjoin%
\definecolor{currentfill}{rgb}{0.000000,0.000000,0.000000}%
\pgfsetfillcolor{currentfill}%
\pgfsetlinewidth{0.803000pt}%
\definecolor{currentstroke}{rgb}{0.000000,0.000000,0.000000}%
\pgfsetstrokecolor{currentstroke}%
\pgfsetdash{}{0pt}%
\pgfsys@defobject{currentmarker}{\pgfqpoint{0.000000in}{-0.048611in}}{\pgfqpoint{0.000000in}{0.000000in}}{%
\pgfpathmoveto{\pgfqpoint{0.000000in}{0.000000in}}%
\pgfpathlineto{\pgfqpoint{0.000000in}{-0.048611in}}%
\pgfusepath{stroke,fill}%
}%
\begin{pgfscope}%
\pgfsys@transformshift{1.920044in}{0.528000in}%
\pgfsys@useobject{currentmarker}{}%
\end{pgfscope}%
\end{pgfscope}%
\begin{pgfscope}%
\pgftext[x=1.920044in,y=0.430778in,,top]{\sffamily\fontsize{10.000000}{12.000000}\selectfont \(\displaystyle 100\)}%
\end{pgfscope}%
\begin{pgfscope}%
\pgfsetbuttcap%
\pgfsetroundjoin%
\definecolor{currentfill}{rgb}{0.000000,0.000000,0.000000}%
\pgfsetfillcolor{currentfill}%
\pgfsetlinewidth{0.803000pt}%
\definecolor{currentstroke}{rgb}{0.000000,0.000000,0.000000}%
\pgfsetstrokecolor{currentstroke}%
\pgfsetdash{}{0pt}%
\pgfsys@defobject{currentmarker}{\pgfqpoint{0.000000in}{-0.048611in}}{\pgfqpoint{0.000000in}{0.000000in}}{%
\pgfpathmoveto{\pgfqpoint{0.000000in}{0.000000in}}%
\pgfpathlineto{\pgfqpoint{0.000000in}{-0.048611in}}%
\pgfusepath{stroke,fill}%
}%
\begin{pgfscope}%
\pgfsys@transformshift{2.823669in}{0.528000in}%
\pgfsys@useobject{currentmarker}{}%
\end{pgfscope}%
\end{pgfscope}%
\begin{pgfscope}%
\pgftext[x=2.823669in,y=0.430778in,,top]{\sffamily\fontsize{10.000000}{12.000000}\selectfont \(\displaystyle 200\)}%
\end{pgfscope}%
\begin{pgfscope}%
\pgfsetbuttcap%
\pgfsetroundjoin%
\definecolor{currentfill}{rgb}{0.000000,0.000000,0.000000}%
\pgfsetfillcolor{currentfill}%
\pgfsetlinewidth{0.803000pt}%
\definecolor{currentstroke}{rgb}{0.000000,0.000000,0.000000}%
\pgfsetstrokecolor{currentstroke}%
\pgfsetdash{}{0pt}%
\pgfsys@defobject{currentmarker}{\pgfqpoint{0.000000in}{-0.048611in}}{\pgfqpoint{0.000000in}{0.000000in}}{%
\pgfpathmoveto{\pgfqpoint{0.000000in}{0.000000in}}%
\pgfpathlineto{\pgfqpoint{0.000000in}{-0.048611in}}%
\pgfusepath{stroke,fill}%
}%
\begin{pgfscope}%
\pgfsys@transformshift{3.727295in}{0.528000in}%
\pgfsys@useobject{currentmarker}{}%
\end{pgfscope}%
\end{pgfscope}%
\begin{pgfscope}%
\pgftext[x=3.727295in,y=0.430778in,,top]{\sffamily\fontsize{10.000000}{12.000000}\selectfont \(\displaystyle 300\)}%
\end{pgfscope}%
\begin{pgfscope}%
\pgfsetbuttcap%
\pgfsetroundjoin%
\definecolor{currentfill}{rgb}{0.000000,0.000000,0.000000}%
\pgfsetfillcolor{currentfill}%
\pgfsetlinewidth{0.803000pt}%
\definecolor{currentstroke}{rgb}{0.000000,0.000000,0.000000}%
\pgfsetstrokecolor{currentstroke}%
\pgfsetdash{}{0pt}%
\pgfsys@defobject{currentmarker}{\pgfqpoint{0.000000in}{-0.048611in}}{\pgfqpoint{0.000000in}{0.000000in}}{%
\pgfpathmoveto{\pgfqpoint{0.000000in}{0.000000in}}%
\pgfpathlineto{\pgfqpoint{0.000000in}{-0.048611in}}%
\pgfusepath{stroke,fill}%
}%
\begin{pgfscope}%
\pgfsys@transformshift{4.630920in}{0.528000in}%
\pgfsys@useobject{currentmarker}{}%
\end{pgfscope}%
\end{pgfscope}%
\begin{pgfscope}%
\pgftext[x=4.630920in,y=0.430778in,,top]{\sffamily\fontsize{10.000000}{12.000000}\selectfont \(\displaystyle 400\)}%
\end{pgfscope}%
\begin{pgfscope}%
\pgfsetbuttcap%
\pgfsetroundjoin%
\definecolor{currentfill}{rgb}{0.000000,0.000000,0.000000}%
\pgfsetfillcolor{currentfill}%
\pgfsetlinewidth{0.803000pt}%
\definecolor{currentstroke}{rgb}{0.000000,0.000000,0.000000}%
\pgfsetstrokecolor{currentstroke}%
\pgfsetdash{}{0pt}%
\pgfsys@defobject{currentmarker}{\pgfqpoint{0.000000in}{-0.048611in}}{\pgfqpoint{0.000000in}{0.000000in}}{%
\pgfpathmoveto{\pgfqpoint{0.000000in}{0.000000in}}%
\pgfpathlineto{\pgfqpoint{0.000000in}{-0.048611in}}%
\pgfusepath{stroke,fill}%
}%
\begin{pgfscope}%
\pgfsys@transformshift{5.534545in}{0.528000in}%
\pgfsys@useobject{currentmarker}{}%
\end{pgfscope}%
\end{pgfscope}%
\begin{pgfscope}%
\pgftext[x=5.534545in,y=0.430778in,,top]{\sffamily\fontsize{10.000000}{12.000000}\selectfont \(\displaystyle 500\)}%
\end{pgfscope}%
\begin{pgfscope}%
\pgftext[x=3.280000in,y=0.240809in,,top]{\sffamily\fontsize{10.000000}{12.000000}\selectfont Episode}%
\end{pgfscope}%
\begin{pgfscope}%
\pgfsetbuttcap%
\pgfsetroundjoin%
\definecolor{currentfill}{rgb}{0.000000,0.000000,0.000000}%
\pgfsetfillcolor{currentfill}%
\pgfsetlinewidth{0.803000pt}%
\definecolor{currentstroke}{rgb}{0.000000,0.000000,0.000000}%
\pgfsetstrokecolor{currentstroke}%
\pgfsetdash{}{0pt}%
\pgfsys@defobject{currentmarker}{\pgfqpoint{-0.048611in}{0.000000in}}{\pgfqpoint{0.000000in}{0.000000in}}{%
\pgfpathmoveto{\pgfqpoint{0.000000in}{0.000000in}}%
\pgfpathlineto{\pgfqpoint{-0.048611in}{0.000000in}}%
\pgfusepath{stroke,fill}%
}%
\begin{pgfscope}%
\pgfsys@transformshift{0.800000in}{0.910968in}%
\pgfsys@useobject{currentmarker}{}%
\end{pgfscope}%
\end{pgfscope}%
\begin{pgfscope}%
\pgftext[x=0.563888in,y=0.858206in,left,base]{\sffamily\fontsize{10.000000}{12.000000}\selectfont \(\displaystyle 25\)}%
\end{pgfscope}%
\begin{pgfscope}%
\pgfsetbuttcap%
\pgfsetroundjoin%
\definecolor{currentfill}{rgb}{0.000000,0.000000,0.000000}%
\pgfsetfillcolor{currentfill}%
\pgfsetlinewidth{0.803000pt}%
\definecolor{currentstroke}{rgb}{0.000000,0.000000,0.000000}%
\pgfsetstrokecolor{currentstroke}%
\pgfsetdash{}{0pt}%
\pgfsys@defobject{currentmarker}{\pgfqpoint{-0.048611in}{0.000000in}}{\pgfqpoint{0.000000in}{0.000000in}}{%
\pgfpathmoveto{\pgfqpoint{0.000000in}{0.000000in}}%
\pgfpathlineto{\pgfqpoint{-0.048611in}{0.000000in}}%
\pgfusepath{stroke,fill}%
}%
\begin{pgfscope}%
\pgfsys@transformshift{0.800000in}{1.362581in}%
\pgfsys@useobject{currentmarker}{}%
\end{pgfscope}%
\end{pgfscope}%
\begin{pgfscope}%
\pgftext[x=0.563888in,y=1.309819in,left,base]{\sffamily\fontsize{10.000000}{12.000000}\selectfont \(\displaystyle 50\)}%
\end{pgfscope}%
\begin{pgfscope}%
\pgfsetbuttcap%
\pgfsetroundjoin%
\definecolor{currentfill}{rgb}{0.000000,0.000000,0.000000}%
\pgfsetfillcolor{currentfill}%
\pgfsetlinewidth{0.803000pt}%
\definecolor{currentstroke}{rgb}{0.000000,0.000000,0.000000}%
\pgfsetstrokecolor{currentstroke}%
\pgfsetdash{}{0pt}%
\pgfsys@defobject{currentmarker}{\pgfqpoint{-0.048611in}{0.000000in}}{\pgfqpoint{0.000000in}{0.000000in}}{%
\pgfpathmoveto{\pgfqpoint{0.000000in}{0.000000in}}%
\pgfpathlineto{\pgfqpoint{-0.048611in}{0.000000in}}%
\pgfusepath{stroke,fill}%
}%
\begin{pgfscope}%
\pgfsys@transformshift{0.800000in}{1.814194in}%
\pgfsys@useobject{currentmarker}{}%
\end{pgfscope}%
\end{pgfscope}%
\begin{pgfscope}%
\pgftext[x=0.563888in,y=1.761432in,left,base]{\sffamily\fontsize{10.000000}{12.000000}\selectfont \(\displaystyle 75\)}%
\end{pgfscope}%
\begin{pgfscope}%
\pgfsetbuttcap%
\pgfsetroundjoin%
\definecolor{currentfill}{rgb}{0.000000,0.000000,0.000000}%
\pgfsetfillcolor{currentfill}%
\pgfsetlinewidth{0.803000pt}%
\definecolor{currentstroke}{rgb}{0.000000,0.000000,0.000000}%
\pgfsetstrokecolor{currentstroke}%
\pgfsetdash{}{0pt}%
\pgfsys@defobject{currentmarker}{\pgfqpoint{-0.048611in}{0.000000in}}{\pgfqpoint{0.000000in}{0.000000in}}{%
\pgfpathmoveto{\pgfqpoint{0.000000in}{0.000000in}}%
\pgfpathlineto{\pgfqpoint{-0.048611in}{0.000000in}}%
\pgfusepath{stroke,fill}%
}%
\begin{pgfscope}%
\pgfsys@transformshift{0.800000in}{2.265806in}%
\pgfsys@useobject{currentmarker}{}%
\end{pgfscope}%
\end{pgfscope}%
\begin{pgfscope}%
\pgftext[x=0.494444in,y=2.213045in,left,base]{\sffamily\fontsize{10.000000}{12.000000}\selectfont \(\displaystyle 100\)}%
\end{pgfscope}%
\begin{pgfscope}%
\pgfsetbuttcap%
\pgfsetroundjoin%
\definecolor{currentfill}{rgb}{0.000000,0.000000,0.000000}%
\pgfsetfillcolor{currentfill}%
\pgfsetlinewidth{0.803000pt}%
\definecolor{currentstroke}{rgb}{0.000000,0.000000,0.000000}%
\pgfsetstrokecolor{currentstroke}%
\pgfsetdash{}{0pt}%
\pgfsys@defobject{currentmarker}{\pgfqpoint{-0.048611in}{0.000000in}}{\pgfqpoint{0.000000in}{0.000000in}}{%
\pgfpathmoveto{\pgfqpoint{0.000000in}{0.000000in}}%
\pgfpathlineto{\pgfqpoint{-0.048611in}{0.000000in}}%
\pgfusepath{stroke,fill}%
}%
\begin{pgfscope}%
\pgfsys@transformshift{0.800000in}{2.717419in}%
\pgfsys@useobject{currentmarker}{}%
\end{pgfscope}%
\end{pgfscope}%
\begin{pgfscope}%
\pgftext[x=0.494444in,y=2.664658in,left,base]{\sffamily\fontsize{10.000000}{12.000000}\selectfont \(\displaystyle 125\)}%
\end{pgfscope}%
\begin{pgfscope}%
\pgfsetbuttcap%
\pgfsetroundjoin%
\definecolor{currentfill}{rgb}{0.000000,0.000000,0.000000}%
\pgfsetfillcolor{currentfill}%
\pgfsetlinewidth{0.803000pt}%
\definecolor{currentstroke}{rgb}{0.000000,0.000000,0.000000}%
\pgfsetstrokecolor{currentstroke}%
\pgfsetdash{}{0pt}%
\pgfsys@defobject{currentmarker}{\pgfqpoint{-0.048611in}{0.000000in}}{\pgfqpoint{0.000000in}{0.000000in}}{%
\pgfpathmoveto{\pgfqpoint{0.000000in}{0.000000in}}%
\pgfpathlineto{\pgfqpoint{-0.048611in}{0.000000in}}%
\pgfusepath{stroke,fill}%
}%
\begin{pgfscope}%
\pgfsys@transformshift{0.800000in}{3.169032in}%
\pgfsys@useobject{currentmarker}{}%
\end{pgfscope}%
\end{pgfscope}%
\begin{pgfscope}%
\pgftext[x=0.494444in,y=3.116271in,left,base]{\sffamily\fontsize{10.000000}{12.000000}\selectfont \(\displaystyle 150\)}%
\end{pgfscope}%
\begin{pgfscope}%
\pgfsetbuttcap%
\pgfsetroundjoin%
\definecolor{currentfill}{rgb}{0.000000,0.000000,0.000000}%
\pgfsetfillcolor{currentfill}%
\pgfsetlinewidth{0.803000pt}%
\definecolor{currentstroke}{rgb}{0.000000,0.000000,0.000000}%
\pgfsetstrokecolor{currentstroke}%
\pgfsetdash{}{0pt}%
\pgfsys@defobject{currentmarker}{\pgfqpoint{-0.048611in}{0.000000in}}{\pgfqpoint{0.000000in}{0.000000in}}{%
\pgfpathmoveto{\pgfqpoint{0.000000in}{0.000000in}}%
\pgfpathlineto{\pgfqpoint{-0.048611in}{0.000000in}}%
\pgfusepath{stroke,fill}%
}%
\begin{pgfscope}%
\pgfsys@transformshift{0.800000in}{3.620645in}%
\pgfsys@useobject{currentmarker}{}%
\end{pgfscope}%
\end{pgfscope}%
\begin{pgfscope}%
\pgftext[x=0.494444in,y=3.567884in,left,base]{\sffamily\fontsize{10.000000}{12.000000}\selectfont \(\displaystyle 175\)}%
\end{pgfscope}%
\begin{pgfscope}%
\pgfsetbuttcap%
\pgfsetroundjoin%
\definecolor{currentfill}{rgb}{0.000000,0.000000,0.000000}%
\pgfsetfillcolor{currentfill}%
\pgfsetlinewidth{0.803000pt}%
\definecolor{currentstroke}{rgb}{0.000000,0.000000,0.000000}%
\pgfsetstrokecolor{currentstroke}%
\pgfsetdash{}{0pt}%
\pgfsys@defobject{currentmarker}{\pgfqpoint{-0.048611in}{0.000000in}}{\pgfqpoint{0.000000in}{0.000000in}}{%
\pgfpathmoveto{\pgfqpoint{0.000000in}{0.000000in}}%
\pgfpathlineto{\pgfqpoint{-0.048611in}{0.000000in}}%
\pgfusepath{stroke,fill}%
}%
\begin{pgfscope}%
\pgfsys@transformshift{0.800000in}{4.072258in}%
\pgfsys@useobject{currentmarker}{}%
\end{pgfscope}%
\end{pgfscope}%
\begin{pgfscope}%
\pgftext[x=0.494444in,y=4.019497in,left,base]{\sffamily\fontsize{10.000000}{12.000000}\selectfont \(\displaystyle 200\)}%
\end{pgfscope}%
\begin{pgfscope}%
\pgftext[x=0.438888in,y=2.376000in,,bottom,rotate=90.000000]{\sffamily\fontsize{10.000000}{12.000000}\selectfont Average Episode Length (10 Runs)}%
\end{pgfscope}%
\begin{pgfscope}%
\pgfpathrectangle{\pgfqpoint{0.800000in}{0.528000in}}{\pgfqpoint{4.960000in}{3.696000in}} %
\pgfusepath{clip}%
\pgfsetrectcap%
\pgfsetroundjoin%
\pgfsetlinewidth{1.505625pt}%
\definecolor{currentstroke}{rgb}{0.000000,0.000000,0.000000}%
\pgfsetstrokecolor{currentstroke}%
\pgfsetdash{}{0pt}%
\pgfpathmoveto{\pgfqpoint{1.025455in}{0.836903in}}%
\pgfpathlineto{\pgfqpoint{1.034491in}{0.840516in}}%
\pgfpathlineto{\pgfqpoint{1.043527in}{1.032000in}}%
\pgfpathlineto{\pgfqpoint{1.052563in}{0.813419in}}%
\pgfpathlineto{\pgfqpoint{1.061600in}{0.865806in}}%
\pgfpathlineto{\pgfqpoint{1.070636in}{0.782710in}}%
\pgfpathlineto{\pgfqpoint{1.079672in}{0.865806in}}%
\pgfpathlineto{\pgfqpoint{1.088708in}{0.869419in}}%
\pgfpathlineto{\pgfqpoint{1.097745in}{0.804387in}}%
\pgfpathlineto{\pgfqpoint{1.106781in}{0.831484in}}%
\pgfpathlineto{\pgfqpoint{1.115817in}{0.826065in}}%
\pgfpathlineto{\pgfqpoint{1.133890in}{0.920000in}}%
\pgfpathlineto{\pgfqpoint{1.142926in}{0.918194in}}%
\pgfpathlineto{\pgfqpoint{1.151962in}{0.957935in}}%
\pgfpathlineto{\pgfqpoint{1.160998in}{0.956129in}}%
\pgfpathlineto{\pgfqpoint{1.170035in}{0.871226in}}%
\pgfpathlineto{\pgfqpoint{1.179071in}{0.835097in}}%
\pgfpathlineto{\pgfqpoint{1.188107in}{0.858581in}}%
\pgfpathlineto{\pgfqpoint{1.197143in}{0.896516in}}%
\pgfpathlineto{\pgfqpoint{1.206180in}{0.851355in}}%
\pgfpathlineto{\pgfqpoint{1.215216in}{0.755613in}}%
\pgfpathlineto{\pgfqpoint{1.224252in}{0.827871in}}%
\pgfpathlineto{\pgfqpoint{1.233288in}{0.889290in}}%
\pgfpathlineto{\pgfqpoint{1.242325in}{0.892903in}}%
\pgfpathlineto{\pgfqpoint{1.251361in}{0.827871in}}%
\pgfpathlineto{\pgfqpoint{1.260397in}{0.797161in}}%
\pgfpathlineto{\pgfqpoint{1.269433in}{0.880258in}}%
\pgfpathlineto{\pgfqpoint{1.287506in}{0.699613in}}%
\pgfpathlineto{\pgfqpoint{1.305578in}{0.977806in}}%
\pgfpathlineto{\pgfqpoint{1.314615in}{0.802581in}}%
\pgfpathlineto{\pgfqpoint{1.323651in}{0.860387in}}%
\pgfpathlineto{\pgfqpoint{1.332687in}{0.809806in}}%
\pgfpathlineto{\pgfqpoint{1.341723in}{0.806194in}}%
\pgfpathlineto{\pgfqpoint{1.350760in}{0.847742in}}%
\pgfpathlineto{\pgfqpoint{1.359796in}{0.768258in}}%
\pgfpathlineto{\pgfqpoint{1.368832in}{0.775484in}}%
\pgfpathlineto{\pgfqpoint{1.377868in}{0.883871in}}%
\pgfpathlineto{\pgfqpoint{1.386905in}{0.970581in}}%
\pgfpathlineto{\pgfqpoint{1.395941in}{0.871226in}}%
\pgfpathlineto{\pgfqpoint{1.404977in}{0.844129in}}%
\pgfpathlineto{\pgfqpoint{1.414013in}{0.952516in}}%
\pgfpathlineto{\pgfqpoint{1.423050in}{0.938065in}}%
\pgfpathlineto{\pgfqpoint{1.432086in}{0.874839in}}%
\pgfpathlineto{\pgfqpoint{1.441122in}{0.824258in}}%
\pgfpathlineto{\pgfqpoint{1.450158in}{0.813419in}}%
\pgfpathlineto{\pgfqpoint{1.459195in}{0.764645in}}%
\pgfpathlineto{\pgfqpoint{1.468231in}{0.831484in}}%
\pgfpathlineto{\pgfqpoint{1.477267in}{0.815226in}}%
\pgfpathlineto{\pgfqpoint{1.486304in}{0.938065in}}%
\pgfpathlineto{\pgfqpoint{1.495340in}{0.820645in}}%
\pgfpathlineto{\pgfqpoint{1.504376in}{0.865806in}}%
\pgfpathlineto{\pgfqpoint{1.513412in}{0.945290in}}%
\pgfpathlineto{\pgfqpoint{1.522449in}{0.891097in}}%
\pgfpathlineto{\pgfqpoint{1.531485in}{0.901935in}}%
\pgfpathlineto{\pgfqpoint{1.540521in}{0.858581in}}%
\pgfpathlineto{\pgfqpoint{1.549557in}{0.889290in}}%
\pgfpathlineto{\pgfqpoint{1.558594in}{0.770065in}}%
\pgfpathlineto{\pgfqpoint{1.567630in}{0.867613in}}%
\pgfpathlineto{\pgfqpoint{1.576666in}{0.948903in}}%
\pgfpathlineto{\pgfqpoint{1.585702in}{0.813419in}}%
\pgfpathlineto{\pgfqpoint{1.594739in}{0.791742in}}%
\pgfpathlineto{\pgfqpoint{1.603775in}{0.867613in}}%
\pgfpathlineto{\pgfqpoint{1.612811in}{0.981419in}}%
\pgfpathlineto{\pgfqpoint{1.621847in}{0.817032in}}%
\pgfpathlineto{\pgfqpoint{1.630884in}{0.864000in}}%
\pgfpathlineto{\pgfqpoint{1.639920in}{0.867613in}}%
\pgfpathlineto{\pgfqpoint{1.648956in}{0.903742in}}%
\pgfpathlineto{\pgfqpoint{1.657992in}{0.789935in}}%
\pgfpathlineto{\pgfqpoint{1.667029in}{0.791742in}}%
\pgfpathlineto{\pgfqpoint{1.676065in}{0.854968in}}%
\pgfpathlineto{\pgfqpoint{1.685101in}{0.831484in}}%
\pgfpathlineto{\pgfqpoint{1.694137in}{0.880258in}}%
\pgfpathlineto{\pgfqpoint{1.703174in}{0.864000in}}%
\pgfpathlineto{\pgfqpoint{1.712210in}{0.822452in}}%
\pgfpathlineto{\pgfqpoint{1.721246in}{0.912774in}}%
\pgfpathlineto{\pgfqpoint{1.730282in}{0.901935in}}%
\pgfpathlineto{\pgfqpoint{1.739319in}{0.856774in}}%
\pgfpathlineto{\pgfqpoint{1.748355in}{0.853161in}}%
\pgfpathlineto{\pgfqpoint{1.757391in}{0.876645in}}%
\pgfpathlineto{\pgfqpoint{1.766427in}{0.842323in}}%
\pgfpathlineto{\pgfqpoint{1.784500in}{0.887484in}}%
\pgfpathlineto{\pgfqpoint{1.793536in}{0.867613in}}%
\pgfpathlineto{\pgfqpoint{1.802572in}{0.952516in}}%
\pgfpathlineto{\pgfqpoint{1.811609in}{0.813419in}}%
\pgfpathlineto{\pgfqpoint{1.820645in}{0.918194in}}%
\pgfpathlineto{\pgfqpoint{1.829681in}{0.824258in}}%
\pgfpathlineto{\pgfqpoint{1.838717in}{0.808000in}}%
\pgfpathlineto{\pgfqpoint{1.847754in}{0.806194in}}%
\pgfpathlineto{\pgfqpoint{1.856790in}{0.844129in}}%
\pgfpathlineto{\pgfqpoint{1.865826in}{0.907355in}}%
\pgfpathlineto{\pgfqpoint{1.874862in}{0.824258in}}%
\pgfpathlineto{\pgfqpoint{1.883899in}{0.869419in}}%
\pgfpathlineto{\pgfqpoint{1.892935in}{0.865806in}}%
\pgfpathlineto{\pgfqpoint{1.901971in}{0.829677in}}%
\pgfpathlineto{\pgfqpoint{1.911007in}{0.847742in}}%
\pgfpathlineto{\pgfqpoint{1.920044in}{0.750194in}}%
\pgfpathlineto{\pgfqpoint{1.929080in}{0.901935in}}%
\pgfpathlineto{\pgfqpoint{1.938116in}{0.724903in}}%
\pgfpathlineto{\pgfqpoint{1.947152in}{0.856774in}}%
\pgfpathlineto{\pgfqpoint{1.956189in}{0.696000in}}%
\pgfpathlineto{\pgfqpoint{1.965225in}{0.885677in}}%
\pgfpathlineto{\pgfqpoint{1.974261in}{0.757419in}}%
\pgfpathlineto{\pgfqpoint{1.983298in}{0.815226in}}%
\pgfpathlineto{\pgfqpoint{1.992334in}{0.885677in}}%
\pgfpathlineto{\pgfqpoint{2.001370in}{0.730323in}}%
\pgfpathlineto{\pgfqpoint{2.010406in}{0.849548in}}%
\pgfpathlineto{\pgfqpoint{2.019443in}{0.757419in}}%
\pgfpathlineto{\pgfqpoint{2.028479in}{0.836903in}}%
\pgfpathlineto{\pgfqpoint{2.037515in}{0.865806in}}%
\pgfpathlineto{\pgfqpoint{2.046551in}{0.961548in}}%
\pgfpathlineto{\pgfqpoint{2.055588in}{0.809806in}}%
\pgfpathlineto{\pgfqpoint{2.064624in}{0.822452in}}%
\pgfpathlineto{\pgfqpoint{2.073660in}{0.894710in}}%
\pgfpathlineto{\pgfqpoint{2.082696in}{0.943484in}}%
\pgfpathlineto{\pgfqpoint{2.091733in}{0.873032in}}%
\pgfpathlineto{\pgfqpoint{2.100769in}{0.847742in}}%
\pgfpathlineto{\pgfqpoint{2.109805in}{0.780903in}}%
\pgfpathlineto{\pgfqpoint{2.118841in}{0.943484in}}%
\pgfpathlineto{\pgfqpoint{2.127878in}{0.844129in}}%
\pgfpathlineto{\pgfqpoint{2.136914in}{0.909161in}}%
\pgfpathlineto{\pgfqpoint{2.145950in}{0.869419in}}%
\pgfpathlineto{\pgfqpoint{2.154986in}{0.791742in}}%
\pgfpathlineto{\pgfqpoint{2.164023in}{0.809806in}}%
\pgfpathlineto{\pgfqpoint{2.173059in}{0.883871in}}%
\pgfpathlineto{\pgfqpoint{2.182095in}{0.896516in}}%
\pgfpathlineto{\pgfqpoint{2.191131in}{0.784516in}}%
\pgfpathlineto{\pgfqpoint{2.200168in}{0.925419in}}%
\pgfpathlineto{\pgfqpoint{2.209204in}{0.827871in}}%
\pgfpathlineto{\pgfqpoint{2.218240in}{0.898323in}}%
\pgfpathlineto{\pgfqpoint{2.227276in}{0.903742in}}%
\pgfpathlineto{\pgfqpoint{2.236313in}{0.930839in}}%
\pgfpathlineto{\pgfqpoint{2.245349in}{0.788129in}}%
\pgfpathlineto{\pgfqpoint{2.254385in}{0.865806in}}%
\pgfpathlineto{\pgfqpoint{2.263421in}{0.813419in}}%
\pgfpathlineto{\pgfqpoint{2.272458in}{0.865806in}}%
\pgfpathlineto{\pgfqpoint{2.281494in}{0.853161in}}%
\pgfpathlineto{\pgfqpoint{2.290530in}{0.811613in}}%
\pgfpathlineto{\pgfqpoint{2.299566in}{0.817032in}}%
\pgfpathlineto{\pgfqpoint{2.308603in}{0.860387in}}%
\pgfpathlineto{\pgfqpoint{2.317639in}{0.934452in}}%
\pgfpathlineto{\pgfqpoint{2.326675in}{0.927226in}}%
\pgfpathlineto{\pgfqpoint{2.335711in}{0.811613in}}%
\pgfpathlineto{\pgfqpoint{2.344748in}{0.759226in}}%
\pgfpathlineto{\pgfqpoint{2.353784in}{0.777290in}}%
\pgfpathlineto{\pgfqpoint{2.362820in}{0.853161in}}%
\pgfpathlineto{\pgfqpoint{2.371856in}{0.976000in}}%
\pgfpathlineto{\pgfqpoint{2.380893in}{0.788129in}}%
\pgfpathlineto{\pgfqpoint{2.389929in}{0.864000in}}%
\pgfpathlineto{\pgfqpoint{2.398965in}{0.876645in}}%
\pgfpathlineto{\pgfqpoint{2.408001in}{0.800774in}}%
\pgfpathlineto{\pgfqpoint{2.417038in}{1.015742in}}%
\pgfpathlineto{\pgfqpoint{2.426074in}{1.077161in}}%
\pgfpathlineto{\pgfqpoint{2.435110in}{0.770065in}}%
\pgfpathlineto{\pgfqpoint{2.444146in}{0.761032in}}%
\pgfpathlineto{\pgfqpoint{2.453183in}{0.970581in}}%
\pgfpathlineto{\pgfqpoint{2.462219in}{0.822452in}}%
\pgfpathlineto{\pgfqpoint{2.471255in}{0.835097in}}%
\pgfpathlineto{\pgfqpoint{2.480291in}{0.777290in}}%
\pgfpathlineto{\pgfqpoint{2.489328in}{0.997677in}}%
\pgfpathlineto{\pgfqpoint{2.498364in}{0.762839in}}%
\pgfpathlineto{\pgfqpoint{2.507400in}{0.883871in}}%
\pgfpathlineto{\pgfqpoint{2.516437in}{0.802581in}}%
\pgfpathlineto{\pgfqpoint{2.525473in}{0.844129in}}%
\pgfpathlineto{\pgfqpoint{2.534509in}{0.813419in}}%
\pgfpathlineto{\pgfqpoint{2.543545in}{0.867613in}}%
\pgfpathlineto{\pgfqpoint{2.552582in}{0.876645in}}%
\pgfpathlineto{\pgfqpoint{2.561618in}{0.873032in}}%
\pgfpathlineto{\pgfqpoint{2.570654in}{0.824258in}}%
\pgfpathlineto{\pgfqpoint{2.579690in}{0.806194in}}%
\pgfpathlineto{\pgfqpoint{2.588727in}{0.829677in}}%
\pgfpathlineto{\pgfqpoint{2.597763in}{0.847742in}}%
\pgfpathlineto{\pgfqpoint{2.606799in}{0.798968in}}%
\pgfpathlineto{\pgfqpoint{2.615835in}{0.802581in}}%
\pgfpathlineto{\pgfqpoint{2.624872in}{0.764645in}}%
\pgfpathlineto{\pgfqpoint{2.642944in}{0.932645in}}%
\pgfpathlineto{\pgfqpoint{2.651980in}{0.831484in}}%
\pgfpathlineto{\pgfqpoint{2.661017in}{0.833290in}}%
\pgfpathlineto{\pgfqpoint{2.670053in}{0.784516in}}%
\pgfpathlineto{\pgfqpoint{2.679089in}{0.916387in}}%
\pgfpathlineto{\pgfqpoint{2.688125in}{0.820645in}}%
\pgfpathlineto{\pgfqpoint{2.697162in}{0.945290in}}%
\pgfpathlineto{\pgfqpoint{2.706198in}{0.992258in}}%
\pgfpathlineto{\pgfqpoint{2.715234in}{0.930839in}}%
\pgfpathlineto{\pgfqpoint{2.724270in}{0.804387in}}%
\pgfpathlineto{\pgfqpoint{2.733307in}{0.976000in}}%
\pgfpathlineto{\pgfqpoint{2.742343in}{0.797161in}}%
\pgfpathlineto{\pgfqpoint{2.751379in}{0.779097in}}%
\pgfpathlineto{\pgfqpoint{2.760415in}{0.831484in}}%
\pgfpathlineto{\pgfqpoint{2.769452in}{0.894710in}}%
\pgfpathlineto{\pgfqpoint{2.778488in}{0.900129in}}%
\pgfpathlineto{\pgfqpoint{2.787524in}{1.033806in}}%
\pgfpathlineto{\pgfqpoint{2.796560in}{0.836903in}}%
\pgfpathlineto{\pgfqpoint{2.805597in}{1.013935in}}%
\pgfpathlineto{\pgfqpoint{2.814633in}{0.867613in}}%
\pgfpathlineto{\pgfqpoint{2.823669in}{0.766452in}}%
\pgfpathlineto{\pgfqpoint{2.832705in}{0.885677in}}%
\pgfpathlineto{\pgfqpoint{2.841742in}{0.824258in}}%
\pgfpathlineto{\pgfqpoint{2.850778in}{0.912774in}}%
\pgfpathlineto{\pgfqpoint{2.859814in}{0.914581in}}%
\pgfpathlineto{\pgfqpoint{2.868850in}{0.903742in}}%
\pgfpathlineto{\pgfqpoint{2.877887in}{0.869419in}}%
\pgfpathlineto{\pgfqpoint{2.886923in}{0.844129in}}%
\pgfpathlineto{\pgfqpoint{2.895959in}{0.826065in}}%
\pgfpathlineto{\pgfqpoint{2.904995in}{0.851355in}}%
\pgfpathlineto{\pgfqpoint{2.914032in}{0.910968in}}%
\pgfpathlineto{\pgfqpoint{2.923068in}{0.934452in}}%
\pgfpathlineto{\pgfqpoint{2.932104in}{0.945290in}}%
\pgfpathlineto{\pgfqpoint{2.941140in}{0.864000in}}%
\pgfpathlineto{\pgfqpoint{2.950177in}{0.820645in}}%
\pgfpathlineto{\pgfqpoint{2.959213in}{0.755613in}}%
\pgfpathlineto{\pgfqpoint{2.968249in}{0.826065in}}%
\pgfpathlineto{\pgfqpoint{2.977285in}{0.797161in}}%
\pgfpathlineto{\pgfqpoint{2.986322in}{0.898323in}}%
\pgfpathlineto{\pgfqpoint{2.995358in}{0.938065in}}%
\pgfpathlineto{\pgfqpoint{3.004394in}{0.858581in}}%
\pgfpathlineto{\pgfqpoint{3.013430in}{0.818839in}}%
\pgfpathlineto{\pgfqpoint{3.022467in}{0.894710in}}%
\pgfpathlineto{\pgfqpoint{3.031503in}{0.876645in}}%
\pgfpathlineto{\pgfqpoint{3.040539in}{0.806194in}}%
\pgfpathlineto{\pgfqpoint{3.049576in}{0.887484in}}%
\pgfpathlineto{\pgfqpoint{3.058612in}{1.165677in}}%
\pgfpathlineto{\pgfqpoint{3.067648in}{0.786323in}}%
\pgfpathlineto{\pgfqpoint{3.076684in}{0.849548in}}%
\pgfpathlineto{\pgfqpoint{3.085721in}{0.813419in}}%
\pgfpathlineto{\pgfqpoint{3.094757in}{0.932645in}}%
\pgfpathlineto{\pgfqpoint{3.103793in}{0.842323in}}%
\pgfpathlineto{\pgfqpoint{3.112829in}{0.829677in}}%
\pgfpathlineto{\pgfqpoint{3.121866in}{0.898323in}}%
\pgfpathlineto{\pgfqpoint{3.130902in}{0.876645in}}%
\pgfpathlineto{\pgfqpoint{3.139938in}{0.963355in}}%
\pgfpathlineto{\pgfqpoint{3.148974in}{0.979613in}}%
\pgfpathlineto{\pgfqpoint{3.158011in}{0.829677in}}%
\pgfpathlineto{\pgfqpoint{3.167047in}{0.847742in}}%
\pgfpathlineto{\pgfqpoint{3.176083in}{0.788129in}}%
\pgfpathlineto{\pgfqpoint{3.185119in}{0.932645in}}%
\pgfpathlineto{\pgfqpoint{3.194156in}{0.798968in}}%
\pgfpathlineto{\pgfqpoint{3.203192in}{0.869419in}}%
\pgfpathlineto{\pgfqpoint{3.212228in}{0.948903in}}%
\pgfpathlineto{\pgfqpoint{3.221264in}{0.880258in}}%
\pgfpathlineto{\pgfqpoint{3.230301in}{0.845935in}}%
\pgfpathlineto{\pgfqpoint{3.239337in}{0.916387in}}%
\pgfpathlineto{\pgfqpoint{3.248373in}{0.943484in}}%
\pgfpathlineto{\pgfqpoint{3.257409in}{1.028387in}}%
\pgfpathlineto{\pgfqpoint{3.266446in}{0.806194in}}%
\pgfpathlineto{\pgfqpoint{3.275482in}{0.936258in}}%
\pgfpathlineto{\pgfqpoint{3.284518in}{0.992258in}}%
\pgfpathlineto{\pgfqpoint{3.293554in}{0.818839in}}%
\pgfpathlineto{\pgfqpoint{3.302591in}{0.768258in}}%
\pgfpathlineto{\pgfqpoint{3.311627in}{0.925419in}}%
\pgfpathlineto{\pgfqpoint{3.320663in}{0.871226in}}%
\pgfpathlineto{\pgfqpoint{3.329699in}{0.864000in}}%
\pgfpathlineto{\pgfqpoint{3.338736in}{0.820645in}}%
\pgfpathlineto{\pgfqpoint{3.347772in}{0.854968in}}%
\pgfpathlineto{\pgfqpoint{3.356808in}{0.786323in}}%
\pgfpathlineto{\pgfqpoint{3.365844in}{0.887484in}}%
\pgfpathlineto{\pgfqpoint{3.374881in}{0.957935in}}%
\pgfpathlineto{\pgfqpoint{3.383917in}{0.840516in}}%
\pgfpathlineto{\pgfqpoint{3.392953in}{0.999484in}}%
\pgfpathlineto{\pgfqpoint{3.401989in}{1.024774in}}%
\pgfpathlineto{\pgfqpoint{3.411026in}{0.826065in}}%
\pgfpathlineto{\pgfqpoint{3.420062in}{0.766452in}}%
\pgfpathlineto{\pgfqpoint{3.429098in}{0.854968in}}%
\pgfpathlineto{\pgfqpoint{3.438134in}{0.860387in}}%
\pgfpathlineto{\pgfqpoint{3.447171in}{0.822452in}}%
\pgfpathlineto{\pgfqpoint{3.456207in}{0.851355in}}%
\pgfpathlineto{\pgfqpoint{3.465243in}{0.869419in}}%
\pgfpathlineto{\pgfqpoint{3.483316in}{1.033806in}}%
\pgfpathlineto{\pgfqpoint{3.492352in}{0.862194in}}%
\pgfpathlineto{\pgfqpoint{3.501388in}{0.938065in}}%
\pgfpathlineto{\pgfqpoint{3.510424in}{0.936258in}}%
\pgfpathlineto{\pgfqpoint{3.519461in}{0.847742in}}%
\pgfpathlineto{\pgfqpoint{3.528497in}{0.974194in}}%
\pgfpathlineto{\pgfqpoint{3.537533in}{0.844129in}}%
\pgfpathlineto{\pgfqpoint{3.546570in}{0.941677in}}%
\pgfpathlineto{\pgfqpoint{3.555606in}{0.808000in}}%
\pgfpathlineto{\pgfqpoint{3.564642in}{1.086194in}}%
\pgfpathlineto{\pgfqpoint{3.573678in}{0.912774in}}%
\pgfpathlineto{\pgfqpoint{3.582715in}{0.966968in}}%
\pgfpathlineto{\pgfqpoint{3.591751in}{1.180129in}}%
\pgfpathlineto{\pgfqpoint{3.600787in}{0.992258in}}%
\pgfpathlineto{\pgfqpoint{3.609823in}{0.770065in}}%
\pgfpathlineto{\pgfqpoint{3.627896in}{0.905548in}}%
\pgfpathlineto{\pgfqpoint{3.636932in}{0.851355in}}%
\pgfpathlineto{\pgfqpoint{3.645968in}{0.874839in}}%
\pgfpathlineto{\pgfqpoint{3.655005in}{0.808000in}}%
\pgfpathlineto{\pgfqpoint{3.664041in}{0.840516in}}%
\pgfpathlineto{\pgfqpoint{3.673077in}{0.853161in}}%
\pgfpathlineto{\pgfqpoint{3.682113in}{0.909161in}}%
\pgfpathlineto{\pgfqpoint{3.691150in}{0.824258in}}%
\pgfpathlineto{\pgfqpoint{3.700186in}{0.880258in}}%
\pgfpathlineto{\pgfqpoint{3.709222in}{0.762839in}}%
\pgfpathlineto{\pgfqpoint{3.718258in}{0.892903in}}%
\pgfpathlineto{\pgfqpoint{3.727295in}{0.900129in}}%
\pgfpathlineto{\pgfqpoint{3.736331in}{0.891097in}}%
\pgfpathlineto{\pgfqpoint{3.745367in}{0.802581in}}%
\pgfpathlineto{\pgfqpoint{3.754403in}{0.811613in}}%
\pgfpathlineto{\pgfqpoint{3.763440in}{0.901935in}}%
\pgfpathlineto{\pgfqpoint{3.772476in}{0.860387in}}%
\pgfpathlineto{\pgfqpoint{3.781512in}{0.865806in}}%
\pgfpathlineto{\pgfqpoint{3.790548in}{0.869419in}}%
\pgfpathlineto{\pgfqpoint{3.799585in}{0.836903in}}%
\pgfpathlineto{\pgfqpoint{3.808621in}{0.793548in}}%
\pgfpathlineto{\pgfqpoint{3.817657in}{0.876645in}}%
\pgfpathlineto{\pgfqpoint{3.826693in}{0.818839in}}%
\pgfpathlineto{\pgfqpoint{3.835730in}{1.116903in}}%
\pgfpathlineto{\pgfqpoint{3.844766in}{0.800774in}}%
\pgfpathlineto{\pgfqpoint{3.853802in}{0.900129in}}%
\pgfpathlineto{\pgfqpoint{3.862838in}{0.791742in}}%
\pgfpathlineto{\pgfqpoint{3.871875in}{0.910968in}}%
\pgfpathlineto{\pgfqpoint{3.880911in}{0.817032in}}%
\pgfpathlineto{\pgfqpoint{3.889947in}{0.898323in}}%
\pgfpathlineto{\pgfqpoint{3.898983in}{0.829677in}}%
\pgfpathlineto{\pgfqpoint{3.908020in}{0.959742in}}%
\pgfpathlineto{\pgfqpoint{3.917056in}{0.894710in}}%
\pgfpathlineto{\pgfqpoint{3.926092in}{0.844129in}}%
\pgfpathlineto{\pgfqpoint{3.935128in}{0.927226in}}%
\pgfpathlineto{\pgfqpoint{3.944165in}{0.844129in}}%
\pgfpathlineto{\pgfqpoint{3.953201in}{0.921806in}}%
\pgfpathlineto{\pgfqpoint{3.962237in}{1.010323in}}%
\pgfpathlineto{\pgfqpoint{3.971273in}{0.925419in}}%
\pgfpathlineto{\pgfqpoint{3.980310in}{0.817032in}}%
\pgfpathlineto{\pgfqpoint{3.989346in}{0.878452in}}%
\pgfpathlineto{\pgfqpoint{3.998382in}{0.883871in}}%
\pgfpathlineto{\pgfqpoint{4.007418in}{0.829677in}}%
\pgfpathlineto{\pgfqpoint{4.016455in}{0.827871in}}%
\pgfpathlineto{\pgfqpoint{4.025491in}{0.923613in}}%
\pgfpathlineto{\pgfqpoint{4.034527in}{1.003097in}}%
\pgfpathlineto{\pgfqpoint{4.043563in}{0.759226in}}%
\pgfpathlineto{\pgfqpoint{4.052600in}{0.874839in}}%
\pgfpathlineto{\pgfqpoint{4.061636in}{0.862194in}}%
\pgfpathlineto{\pgfqpoint{4.070672in}{0.864000in}}%
\pgfpathlineto{\pgfqpoint{4.079709in}{0.871226in}}%
\pgfpathlineto{\pgfqpoint{4.088745in}{0.948903in}}%
\pgfpathlineto{\pgfqpoint{4.097781in}{0.941677in}}%
\pgfpathlineto{\pgfqpoint{4.106817in}{0.829677in}}%
\pgfpathlineto{\pgfqpoint{4.115854in}{0.838710in}}%
\pgfpathlineto{\pgfqpoint{4.124890in}{0.802581in}}%
\pgfpathlineto{\pgfqpoint{4.133926in}{0.916387in}}%
\pgfpathlineto{\pgfqpoint{4.142962in}{0.923613in}}%
\pgfpathlineto{\pgfqpoint{4.151999in}{1.008516in}}%
\pgfpathlineto{\pgfqpoint{4.161035in}{0.976000in}}%
\pgfpathlineto{\pgfqpoint{4.170071in}{0.869419in}}%
\pgfpathlineto{\pgfqpoint{4.179107in}{0.876645in}}%
\pgfpathlineto{\pgfqpoint{4.188144in}{0.838710in}}%
\pgfpathlineto{\pgfqpoint{4.197180in}{0.755613in}}%
\pgfpathlineto{\pgfqpoint{4.206216in}{1.021161in}}%
\pgfpathlineto{\pgfqpoint{4.215252in}{0.948903in}}%
\pgfpathlineto{\pgfqpoint{4.224289in}{0.905548in}}%
\pgfpathlineto{\pgfqpoint{4.233325in}{0.905548in}}%
\pgfpathlineto{\pgfqpoint{4.242361in}{0.808000in}}%
\pgfpathlineto{\pgfqpoint{4.251397in}{0.920000in}}%
\pgfpathlineto{\pgfqpoint{4.260434in}{0.898323in}}%
\pgfpathlineto{\pgfqpoint{4.269470in}{0.974194in}}%
\pgfpathlineto{\pgfqpoint{4.278506in}{1.022968in}}%
\pgfpathlineto{\pgfqpoint{4.287542in}{0.910968in}}%
\pgfpathlineto{\pgfqpoint{4.296579in}{0.914581in}}%
\pgfpathlineto{\pgfqpoint{4.305615in}{0.894710in}}%
\pgfpathlineto{\pgfqpoint{4.314651in}{0.957935in}}%
\pgfpathlineto{\pgfqpoint{4.323687in}{0.802581in}}%
\pgfpathlineto{\pgfqpoint{4.332724in}{0.927226in}}%
\pgfpathlineto{\pgfqpoint{4.341760in}{0.894710in}}%
\pgfpathlineto{\pgfqpoint{4.350796in}{0.954323in}}%
\pgfpathlineto{\pgfqpoint{4.359832in}{0.932645in}}%
\pgfpathlineto{\pgfqpoint{4.368869in}{0.891097in}}%
\pgfpathlineto{\pgfqpoint{4.377905in}{0.808000in}}%
\pgfpathlineto{\pgfqpoint{4.386941in}{0.811613in}}%
\pgfpathlineto{\pgfqpoint{4.395977in}{0.802581in}}%
\pgfpathlineto{\pgfqpoint{4.405014in}{0.914581in}}%
\pgfpathlineto{\pgfqpoint{4.414050in}{0.876645in}}%
\pgfpathlineto{\pgfqpoint{4.423086in}{0.945290in}}%
\pgfpathlineto{\pgfqpoint{4.432122in}{0.957935in}}%
\pgfpathlineto{\pgfqpoint{4.441159in}{0.822452in}}%
\pgfpathlineto{\pgfqpoint{4.459231in}{1.106065in}}%
\pgfpathlineto{\pgfqpoint{4.468267in}{0.907355in}}%
\pgfpathlineto{\pgfqpoint{4.477304in}{0.820645in}}%
\pgfpathlineto{\pgfqpoint{4.486340in}{0.871226in}}%
\pgfpathlineto{\pgfqpoint{4.495376in}{0.847742in}}%
\pgfpathlineto{\pgfqpoint{4.504412in}{0.929032in}}%
\pgfpathlineto{\pgfqpoint{4.513449in}{0.867613in}}%
\pgfpathlineto{\pgfqpoint{4.522485in}{0.733935in}}%
\pgfpathlineto{\pgfqpoint{4.531521in}{0.889290in}}%
\pgfpathlineto{\pgfqpoint{4.540557in}{0.726710in}}%
\pgfpathlineto{\pgfqpoint{4.549594in}{0.775484in}}%
\pgfpathlineto{\pgfqpoint{4.558630in}{0.981419in}}%
\pgfpathlineto{\pgfqpoint{4.567666in}{0.746581in}}%
\pgfpathlineto{\pgfqpoint{4.576702in}{0.780903in}}%
\pgfpathlineto{\pgfqpoint{4.585739in}{0.847742in}}%
\pgfpathlineto{\pgfqpoint{4.594775in}{1.030194in}}%
\pgfpathlineto{\pgfqpoint{4.603811in}{0.806194in}}%
\pgfpathlineto{\pgfqpoint{4.612848in}{0.916387in}}%
\pgfpathlineto{\pgfqpoint{4.621884in}{0.835097in}}%
\pgfpathlineto{\pgfqpoint{4.630920in}{1.033806in}}%
\pgfpathlineto{\pgfqpoint{4.639956in}{0.979613in}}%
\pgfpathlineto{\pgfqpoint{4.648993in}{0.865806in}}%
\pgfpathlineto{\pgfqpoint{4.667065in}{0.997677in}}%
\pgfpathlineto{\pgfqpoint{4.676101in}{0.874839in}}%
\pgfpathlineto{\pgfqpoint{4.685138in}{0.941677in}}%
\pgfpathlineto{\pgfqpoint{4.694174in}{0.829677in}}%
\pgfpathlineto{\pgfqpoint{4.703210in}{0.929032in}}%
\pgfpathlineto{\pgfqpoint{4.712246in}{0.992258in}}%
\pgfpathlineto{\pgfqpoint{4.721283in}{1.008516in}}%
\pgfpathlineto{\pgfqpoint{4.730319in}{0.809806in}}%
\pgfpathlineto{\pgfqpoint{4.739355in}{0.845935in}}%
\pgfpathlineto{\pgfqpoint{4.748391in}{0.820645in}}%
\pgfpathlineto{\pgfqpoint{4.757428in}{0.826065in}}%
\pgfpathlineto{\pgfqpoint{4.766464in}{0.818839in}}%
\pgfpathlineto{\pgfqpoint{4.775500in}{0.945290in}}%
\pgfpathlineto{\pgfqpoint{4.784536in}{0.883871in}}%
\pgfpathlineto{\pgfqpoint{4.793573in}{0.894710in}}%
\pgfpathlineto{\pgfqpoint{4.802609in}{0.867613in}}%
\pgfpathlineto{\pgfqpoint{4.811645in}{0.986839in}}%
\pgfpathlineto{\pgfqpoint{4.820681in}{0.827871in}}%
\pgfpathlineto{\pgfqpoint{4.829718in}{0.820645in}}%
\pgfpathlineto{\pgfqpoint{4.838754in}{1.120516in}}%
\pgfpathlineto{\pgfqpoint{4.847790in}{0.898323in}}%
\pgfpathlineto{\pgfqpoint{4.856826in}{0.829677in}}%
\pgfpathlineto{\pgfqpoint{4.865863in}{0.824258in}}%
\pgfpathlineto{\pgfqpoint{4.874899in}{0.900129in}}%
\pgfpathlineto{\pgfqpoint{4.883935in}{0.797161in}}%
\pgfpathlineto{\pgfqpoint{4.892971in}{0.835097in}}%
\pgfpathlineto{\pgfqpoint{4.902008in}{0.858581in}}%
\pgfpathlineto{\pgfqpoint{4.911044in}{0.836903in}}%
\pgfpathlineto{\pgfqpoint{4.920080in}{0.862194in}}%
\pgfpathlineto{\pgfqpoint{4.929116in}{0.914581in}}%
\pgfpathlineto{\pgfqpoint{4.938153in}{0.977806in}}%
\pgfpathlineto{\pgfqpoint{4.947189in}{0.968774in}}%
\pgfpathlineto{\pgfqpoint{4.956225in}{0.871226in}}%
\pgfpathlineto{\pgfqpoint{4.965261in}{0.970581in}}%
\pgfpathlineto{\pgfqpoint{4.974298in}{0.934452in}}%
\pgfpathlineto{\pgfqpoint{4.983334in}{0.864000in}}%
\pgfpathlineto{\pgfqpoint{4.992370in}{1.015742in}}%
\pgfpathlineto{\pgfqpoint{5.001406in}{0.963355in}}%
\pgfpathlineto{\pgfqpoint{5.010443in}{0.824258in}}%
\pgfpathlineto{\pgfqpoint{5.019479in}{0.815226in}}%
\pgfpathlineto{\pgfqpoint{5.028515in}{0.845935in}}%
\pgfpathlineto{\pgfqpoint{5.037551in}{0.905548in}}%
\pgfpathlineto{\pgfqpoint{5.046588in}{0.918194in}}%
\pgfpathlineto{\pgfqpoint{5.055624in}{0.854968in}}%
\pgfpathlineto{\pgfqpoint{5.064660in}{0.966968in}}%
\pgfpathlineto{\pgfqpoint{5.073696in}{0.802581in}}%
\pgfpathlineto{\pgfqpoint{5.082733in}{0.761032in}}%
\pgfpathlineto{\pgfqpoint{5.091769in}{0.900129in}}%
\pgfpathlineto{\pgfqpoint{5.100805in}{0.853161in}}%
\pgfpathlineto{\pgfqpoint{5.109842in}{0.905548in}}%
\pgfpathlineto{\pgfqpoint{5.118878in}{0.853161in}}%
\pgfpathlineto{\pgfqpoint{5.127914in}{0.909161in}}%
\pgfpathlineto{\pgfqpoint{5.136950in}{0.840516in}}%
\pgfpathlineto{\pgfqpoint{5.145987in}{0.849548in}}%
\pgfpathlineto{\pgfqpoint{5.155023in}{0.826065in}}%
\pgfpathlineto{\pgfqpoint{5.164059in}{1.104258in}}%
\pgfpathlineto{\pgfqpoint{5.173095in}{0.918194in}}%
\pgfpathlineto{\pgfqpoint{5.182132in}{0.788129in}}%
\pgfpathlineto{\pgfqpoint{5.191168in}{0.775484in}}%
\pgfpathlineto{\pgfqpoint{5.200204in}{0.874839in}}%
\pgfpathlineto{\pgfqpoint{5.209240in}{0.961548in}}%
\pgfpathlineto{\pgfqpoint{5.218277in}{0.854968in}}%
\pgfpathlineto{\pgfqpoint{5.227313in}{0.840516in}}%
\pgfpathlineto{\pgfqpoint{5.236349in}{0.905548in}}%
\pgfpathlineto{\pgfqpoint{5.245385in}{0.853161in}}%
\pgfpathlineto{\pgfqpoint{5.254422in}{0.741161in}}%
\pgfpathlineto{\pgfqpoint{5.263458in}{0.878452in}}%
\pgfpathlineto{\pgfqpoint{5.272494in}{1.118710in}}%
\pgfpathlineto{\pgfqpoint{5.281530in}{1.044645in}}%
\pgfpathlineto{\pgfqpoint{5.290567in}{0.822452in}}%
\pgfpathlineto{\pgfqpoint{5.299603in}{0.894710in}}%
\pgfpathlineto{\pgfqpoint{5.308639in}{0.867613in}}%
\pgfpathlineto{\pgfqpoint{5.317675in}{0.891097in}}%
\pgfpathlineto{\pgfqpoint{5.326712in}{1.017548in}}%
\pgfpathlineto{\pgfqpoint{5.335748in}{0.883871in}}%
\pgfpathlineto{\pgfqpoint{5.344784in}{0.844129in}}%
\pgfpathlineto{\pgfqpoint{5.353820in}{0.820645in}}%
\pgfpathlineto{\pgfqpoint{5.362857in}{1.060903in}}%
\pgfpathlineto{\pgfqpoint{5.371893in}{0.921806in}}%
\pgfpathlineto{\pgfqpoint{5.380929in}{0.835097in}}%
\pgfpathlineto{\pgfqpoint{5.389965in}{0.963355in}}%
\pgfpathlineto{\pgfqpoint{5.399002in}{0.939871in}}%
\pgfpathlineto{\pgfqpoint{5.408038in}{0.831484in}}%
\pgfpathlineto{\pgfqpoint{5.417074in}{0.963355in}}%
\pgfpathlineto{\pgfqpoint{5.426110in}{0.943484in}}%
\pgfpathlineto{\pgfqpoint{5.435147in}{0.976000in}}%
\pgfpathlineto{\pgfqpoint{5.444183in}{0.732129in}}%
\pgfpathlineto{\pgfqpoint{5.453219in}{0.965161in}}%
\pgfpathlineto{\pgfqpoint{5.462255in}{0.947097in}}%
\pgfpathlineto{\pgfqpoint{5.471292in}{0.752000in}}%
\pgfpathlineto{\pgfqpoint{5.480328in}{0.871226in}}%
\pgfpathlineto{\pgfqpoint{5.489364in}{0.865806in}}%
\pgfpathlineto{\pgfqpoint{5.498400in}{0.900129in}}%
\pgfpathlineto{\pgfqpoint{5.507437in}{1.012129in}}%
\pgfpathlineto{\pgfqpoint{5.516473in}{1.059097in}}%
\pgfpathlineto{\pgfqpoint{5.525509in}{0.938065in}}%
\pgfpathlineto{\pgfqpoint{5.534545in}{0.945290in}}%
\pgfpathlineto{\pgfqpoint{5.534545in}{0.945290in}}%
\pgfusepath{stroke}%
\end{pgfscope}%
\begin{pgfscope}%
\pgfpathrectangle{\pgfqpoint{0.800000in}{0.528000in}}{\pgfqpoint{4.960000in}{3.696000in}} %
\pgfusepath{clip}%
\pgfsetrectcap%
\pgfsetroundjoin%
\pgfsetlinewidth{1.505625pt}%
\definecolor{currentstroke}{rgb}{0.300000,0.300000,0.300000}%
\pgfsetstrokecolor{currentstroke}%
\pgfsetdash{}{0pt}%
\pgfpathmoveto{\pgfqpoint{1.025455in}{0.874839in}}%
\pgfpathlineto{\pgfqpoint{1.034491in}{0.947097in}}%
\pgfpathlineto{\pgfqpoint{1.043527in}{0.896516in}}%
\pgfpathlineto{\pgfqpoint{1.052563in}{0.829677in}}%
\pgfpathlineto{\pgfqpoint{1.061600in}{0.930839in}}%
\pgfpathlineto{\pgfqpoint{1.070636in}{0.770065in}}%
\pgfpathlineto{\pgfqpoint{1.079672in}{0.860387in}}%
\pgfpathlineto{\pgfqpoint{1.088708in}{0.847742in}}%
\pgfpathlineto{\pgfqpoint{1.097745in}{0.880258in}}%
\pgfpathlineto{\pgfqpoint{1.106781in}{0.818839in}}%
\pgfpathlineto{\pgfqpoint{1.115817in}{0.892903in}}%
\pgfpathlineto{\pgfqpoint{1.124853in}{0.826065in}}%
\pgfpathlineto{\pgfqpoint{1.133890in}{0.836903in}}%
\pgfpathlineto{\pgfqpoint{1.142926in}{0.748387in}}%
\pgfpathlineto{\pgfqpoint{1.151962in}{0.970581in}}%
\pgfpathlineto{\pgfqpoint{1.160998in}{0.936258in}}%
\pgfpathlineto{\pgfqpoint{1.179071in}{0.882065in}}%
\pgfpathlineto{\pgfqpoint{1.188107in}{0.963355in}}%
\pgfpathlineto{\pgfqpoint{1.197143in}{0.811613in}}%
\pgfpathlineto{\pgfqpoint{1.206180in}{0.900129in}}%
\pgfpathlineto{\pgfqpoint{1.215216in}{0.826065in}}%
\pgfpathlineto{\pgfqpoint{1.224252in}{0.878452in}}%
\pgfpathlineto{\pgfqpoint{1.233288in}{0.845935in}}%
\pgfpathlineto{\pgfqpoint{1.242325in}{0.880258in}}%
\pgfpathlineto{\pgfqpoint{1.251361in}{0.920000in}}%
\pgfpathlineto{\pgfqpoint{1.260397in}{0.844129in}}%
\pgfpathlineto{\pgfqpoint{1.269433in}{0.871226in}}%
\pgfpathlineto{\pgfqpoint{1.278470in}{0.873032in}}%
\pgfpathlineto{\pgfqpoint{1.287506in}{0.788129in}}%
\pgfpathlineto{\pgfqpoint{1.296542in}{0.934452in}}%
\pgfpathlineto{\pgfqpoint{1.305578in}{0.929032in}}%
\pgfpathlineto{\pgfqpoint{1.314615in}{0.831484in}}%
\pgfpathlineto{\pgfqpoint{1.323651in}{0.992258in}}%
\pgfpathlineto{\pgfqpoint{1.332687in}{0.918194in}}%
\pgfpathlineto{\pgfqpoint{1.341723in}{0.880258in}}%
\pgfpathlineto{\pgfqpoint{1.350760in}{0.972387in}}%
\pgfpathlineto{\pgfqpoint{1.359796in}{0.885677in}}%
\pgfpathlineto{\pgfqpoint{1.368832in}{0.833290in}}%
\pgfpathlineto{\pgfqpoint{1.377868in}{0.838710in}}%
\pgfpathlineto{\pgfqpoint{1.386905in}{0.860387in}}%
\pgfpathlineto{\pgfqpoint{1.395941in}{0.802581in}}%
\pgfpathlineto{\pgfqpoint{1.404977in}{0.925419in}}%
\pgfpathlineto{\pgfqpoint{1.414013in}{0.883871in}}%
\pgfpathlineto{\pgfqpoint{1.423050in}{0.856774in}}%
\pgfpathlineto{\pgfqpoint{1.432086in}{0.945290in}}%
\pgfpathlineto{\pgfqpoint{1.441122in}{0.932645in}}%
\pgfpathlineto{\pgfqpoint{1.450158in}{0.753806in}}%
\pgfpathlineto{\pgfqpoint{1.459195in}{0.791742in}}%
\pgfpathlineto{\pgfqpoint{1.468231in}{0.780903in}}%
\pgfpathlineto{\pgfqpoint{1.477267in}{0.934452in}}%
\pgfpathlineto{\pgfqpoint{1.486304in}{1.046452in}}%
\pgfpathlineto{\pgfqpoint{1.495340in}{0.845935in}}%
\pgfpathlineto{\pgfqpoint{1.504376in}{0.898323in}}%
\pgfpathlineto{\pgfqpoint{1.513412in}{0.965161in}}%
\pgfpathlineto{\pgfqpoint{1.522449in}{0.860387in}}%
\pgfpathlineto{\pgfqpoint{1.531485in}{0.952516in}}%
\pgfpathlineto{\pgfqpoint{1.540521in}{0.891097in}}%
\pgfpathlineto{\pgfqpoint{1.549557in}{0.874839in}}%
\pgfpathlineto{\pgfqpoint{1.558594in}{0.773677in}}%
\pgfpathlineto{\pgfqpoint{1.567630in}{1.149419in}}%
\pgfpathlineto{\pgfqpoint{1.576666in}{0.795355in}}%
\pgfpathlineto{\pgfqpoint{1.585702in}{0.824258in}}%
\pgfpathlineto{\pgfqpoint{1.594739in}{0.995871in}}%
\pgfpathlineto{\pgfqpoint{1.603775in}{0.903742in}}%
\pgfpathlineto{\pgfqpoint{1.612811in}{1.010323in}}%
\pgfpathlineto{\pgfqpoint{1.621847in}{0.779097in}}%
\pgfpathlineto{\pgfqpoint{1.630884in}{0.956129in}}%
\pgfpathlineto{\pgfqpoint{1.639920in}{0.909161in}}%
\pgfpathlineto{\pgfqpoint{1.648956in}{0.952516in}}%
\pgfpathlineto{\pgfqpoint{1.657992in}{0.894710in}}%
\pgfpathlineto{\pgfqpoint{1.667029in}{0.806194in}}%
\pgfpathlineto{\pgfqpoint{1.676065in}{0.784516in}}%
\pgfpathlineto{\pgfqpoint{1.685101in}{0.889290in}}%
\pgfpathlineto{\pgfqpoint{1.694137in}{0.920000in}}%
\pgfpathlineto{\pgfqpoint{1.703174in}{0.916387in}}%
\pgfpathlineto{\pgfqpoint{1.712210in}{1.003097in}}%
\pgfpathlineto{\pgfqpoint{1.721246in}{0.820645in}}%
\pgfpathlineto{\pgfqpoint{1.730282in}{0.869419in}}%
\pgfpathlineto{\pgfqpoint{1.739319in}{0.883871in}}%
\pgfpathlineto{\pgfqpoint{1.748355in}{0.876645in}}%
\pgfpathlineto{\pgfqpoint{1.757391in}{0.938065in}}%
\pgfpathlineto{\pgfqpoint{1.766427in}{1.024774in}}%
\pgfpathlineto{\pgfqpoint{1.775464in}{0.845935in}}%
\pgfpathlineto{\pgfqpoint{1.784500in}{0.947097in}}%
\pgfpathlineto{\pgfqpoint{1.793536in}{1.122323in}}%
\pgfpathlineto{\pgfqpoint{1.802572in}{0.905548in}}%
\pgfpathlineto{\pgfqpoint{1.811609in}{0.950710in}}%
\pgfpathlineto{\pgfqpoint{1.820645in}{0.780903in}}%
\pgfpathlineto{\pgfqpoint{1.829681in}{0.838710in}}%
\pgfpathlineto{\pgfqpoint{1.838717in}{0.865806in}}%
\pgfpathlineto{\pgfqpoint{1.847754in}{0.921806in}}%
\pgfpathlineto{\pgfqpoint{1.856790in}{0.891097in}}%
\pgfpathlineto{\pgfqpoint{1.865826in}{0.914581in}}%
\pgfpathlineto{\pgfqpoint{1.874862in}{0.858581in}}%
\pgfpathlineto{\pgfqpoint{1.883899in}{0.835097in}}%
\pgfpathlineto{\pgfqpoint{1.892935in}{1.078968in}}%
\pgfpathlineto{\pgfqpoint{1.901971in}{0.900129in}}%
\pgfpathlineto{\pgfqpoint{1.911007in}{0.918194in}}%
\pgfpathlineto{\pgfqpoint{1.920044in}{0.963355in}}%
\pgfpathlineto{\pgfqpoint{1.929080in}{0.914581in}}%
\pgfpathlineto{\pgfqpoint{1.938116in}{0.842323in}}%
\pgfpathlineto{\pgfqpoint{1.947152in}{0.929032in}}%
\pgfpathlineto{\pgfqpoint{1.956189in}{0.976000in}}%
\pgfpathlineto{\pgfqpoint{1.965225in}{0.929032in}}%
\pgfpathlineto{\pgfqpoint{1.974261in}{0.804387in}}%
\pgfpathlineto{\pgfqpoint{1.983298in}{1.010323in}}%
\pgfpathlineto{\pgfqpoint{1.992334in}{0.826065in}}%
\pgfpathlineto{\pgfqpoint{2.001370in}{0.992258in}}%
\pgfpathlineto{\pgfqpoint{2.010406in}{0.869419in}}%
\pgfpathlineto{\pgfqpoint{2.019443in}{0.985032in}}%
\pgfpathlineto{\pgfqpoint{2.028479in}{0.900129in}}%
\pgfpathlineto{\pgfqpoint{2.037515in}{0.934452in}}%
\pgfpathlineto{\pgfqpoint{2.046551in}{0.939871in}}%
\pgfpathlineto{\pgfqpoint{2.055588in}{0.894710in}}%
\pgfpathlineto{\pgfqpoint{2.064624in}{1.059097in}}%
\pgfpathlineto{\pgfqpoint{2.073660in}{0.878452in}}%
\pgfpathlineto{\pgfqpoint{2.082696in}{0.851355in}}%
\pgfpathlineto{\pgfqpoint{2.091733in}{0.815226in}}%
\pgfpathlineto{\pgfqpoint{2.100769in}{0.887484in}}%
\pgfpathlineto{\pgfqpoint{2.118841in}{0.930839in}}%
\pgfpathlineto{\pgfqpoint{2.127878in}{0.985032in}}%
\pgfpathlineto{\pgfqpoint{2.136914in}{0.829677in}}%
\pgfpathlineto{\pgfqpoint{2.154986in}{1.028387in}}%
\pgfpathlineto{\pgfqpoint{2.164023in}{1.006710in}}%
\pgfpathlineto{\pgfqpoint{2.173059in}{0.957935in}}%
\pgfpathlineto{\pgfqpoint{2.182095in}{0.820645in}}%
\pgfpathlineto{\pgfqpoint{2.191131in}{0.965161in}}%
\pgfpathlineto{\pgfqpoint{2.200168in}{0.907355in}}%
\pgfpathlineto{\pgfqpoint{2.209204in}{0.963355in}}%
\pgfpathlineto{\pgfqpoint{2.218240in}{0.865806in}}%
\pgfpathlineto{\pgfqpoint{2.227276in}{0.930839in}}%
\pgfpathlineto{\pgfqpoint{2.236313in}{1.042839in}}%
\pgfpathlineto{\pgfqpoint{2.245349in}{0.896516in}}%
\pgfpathlineto{\pgfqpoint{2.254385in}{0.910968in}}%
\pgfpathlineto{\pgfqpoint{2.263421in}{0.972387in}}%
\pgfpathlineto{\pgfqpoint{2.272458in}{0.822452in}}%
\pgfpathlineto{\pgfqpoint{2.281494in}{1.124129in}}%
\pgfpathlineto{\pgfqpoint{2.290530in}{0.938065in}}%
\pgfpathlineto{\pgfqpoint{2.299566in}{0.829677in}}%
\pgfpathlineto{\pgfqpoint{2.308603in}{0.905548in}}%
\pgfpathlineto{\pgfqpoint{2.317639in}{0.892903in}}%
\pgfpathlineto{\pgfqpoint{2.326675in}{0.833290in}}%
\pgfpathlineto{\pgfqpoint{2.335711in}{0.992258in}}%
\pgfpathlineto{\pgfqpoint{2.344748in}{0.856774in}}%
\pgfpathlineto{\pgfqpoint{2.353784in}{1.073548in}}%
\pgfpathlineto{\pgfqpoint{2.362820in}{1.115097in}}%
\pgfpathlineto{\pgfqpoint{2.371856in}{1.022968in}}%
\pgfpathlineto{\pgfqpoint{2.380893in}{0.988645in}}%
\pgfpathlineto{\pgfqpoint{2.389929in}{0.929032in}}%
\pgfpathlineto{\pgfqpoint{2.398965in}{1.106065in}}%
\pgfpathlineto{\pgfqpoint{2.408001in}{0.918194in}}%
\pgfpathlineto{\pgfqpoint{2.417038in}{0.990452in}}%
\pgfpathlineto{\pgfqpoint{2.426074in}{1.026581in}}%
\pgfpathlineto{\pgfqpoint{2.435110in}{0.894710in}}%
\pgfpathlineto{\pgfqpoint{2.444146in}{0.997677in}}%
\pgfpathlineto{\pgfqpoint{2.453183in}{0.914581in}}%
\pgfpathlineto{\pgfqpoint{2.462219in}{0.992258in}}%
\pgfpathlineto{\pgfqpoint{2.471255in}{0.873032in}}%
\pgfpathlineto{\pgfqpoint{2.480291in}{0.930839in}}%
\pgfpathlineto{\pgfqpoint{2.489328in}{0.818839in}}%
\pgfpathlineto{\pgfqpoint{2.498364in}{1.053677in}}%
\pgfpathlineto{\pgfqpoint{2.507400in}{1.212645in}}%
\pgfpathlineto{\pgfqpoint{2.516437in}{1.017548in}}%
\pgfpathlineto{\pgfqpoint{2.525473in}{0.979613in}}%
\pgfpathlineto{\pgfqpoint{2.534509in}{1.286710in}}%
\pgfpathlineto{\pgfqpoint{2.543545in}{0.999484in}}%
\pgfpathlineto{\pgfqpoint{2.552582in}{0.867613in}}%
\pgfpathlineto{\pgfqpoint{2.561618in}{0.887484in}}%
\pgfpathlineto{\pgfqpoint{2.570654in}{0.990452in}}%
\pgfpathlineto{\pgfqpoint{2.579690in}{0.871226in}}%
\pgfpathlineto{\pgfqpoint{2.588727in}{0.925419in}}%
\pgfpathlineto{\pgfqpoint{2.597763in}{0.965161in}}%
\pgfpathlineto{\pgfqpoint{2.606799in}{0.860387in}}%
\pgfpathlineto{\pgfqpoint{2.615835in}{0.880258in}}%
\pgfpathlineto{\pgfqpoint{2.624872in}{0.871226in}}%
\pgfpathlineto{\pgfqpoint{2.633908in}{0.939871in}}%
\pgfpathlineto{\pgfqpoint{2.642944in}{0.864000in}}%
\pgfpathlineto{\pgfqpoint{2.661017in}{1.001290in}}%
\pgfpathlineto{\pgfqpoint{2.670053in}{1.030194in}}%
\pgfpathlineto{\pgfqpoint{2.679089in}{0.927226in}}%
\pgfpathlineto{\pgfqpoint{2.688125in}{0.932645in}}%
\pgfpathlineto{\pgfqpoint{2.697162in}{1.033806in}}%
\pgfpathlineto{\pgfqpoint{2.706198in}{1.033806in}}%
\pgfpathlineto{\pgfqpoint{2.715234in}{0.921806in}}%
\pgfpathlineto{\pgfqpoint{2.724270in}{0.972387in}}%
\pgfpathlineto{\pgfqpoint{2.733307in}{0.930839in}}%
\pgfpathlineto{\pgfqpoint{2.742343in}{0.999484in}}%
\pgfpathlineto{\pgfqpoint{2.760415in}{1.118710in}}%
\pgfpathlineto{\pgfqpoint{2.769452in}{1.078968in}}%
\pgfpathlineto{\pgfqpoint{2.778488in}{0.990452in}}%
\pgfpathlineto{\pgfqpoint{2.787524in}{0.959742in}}%
\pgfpathlineto{\pgfqpoint{2.796560in}{1.073548in}}%
\pgfpathlineto{\pgfqpoint{2.805597in}{0.887484in}}%
\pgfpathlineto{\pgfqpoint{2.814633in}{1.171097in}}%
\pgfpathlineto{\pgfqpoint{2.823669in}{0.983226in}}%
\pgfpathlineto{\pgfqpoint{2.832705in}{0.999484in}}%
\pgfpathlineto{\pgfqpoint{2.841742in}{0.921806in}}%
\pgfpathlineto{\pgfqpoint{2.850778in}{1.077161in}}%
\pgfpathlineto{\pgfqpoint{2.859814in}{1.095226in}}%
\pgfpathlineto{\pgfqpoint{2.868850in}{0.952516in}}%
\pgfpathlineto{\pgfqpoint{2.877887in}{1.013935in}}%
\pgfpathlineto{\pgfqpoint{2.886923in}{0.988645in}}%
\pgfpathlineto{\pgfqpoint{2.895959in}{1.004903in}}%
\pgfpathlineto{\pgfqpoint{2.904995in}{0.977806in}}%
\pgfpathlineto{\pgfqpoint{2.914032in}{0.824258in}}%
\pgfpathlineto{\pgfqpoint{2.923068in}{1.176516in}}%
\pgfpathlineto{\pgfqpoint{2.932104in}{0.985032in}}%
\pgfpathlineto{\pgfqpoint{2.941140in}{0.903742in}}%
\pgfpathlineto{\pgfqpoint{2.950177in}{0.878452in}}%
\pgfpathlineto{\pgfqpoint{2.959213in}{1.026581in}}%
\pgfpathlineto{\pgfqpoint{2.968249in}{1.022968in}}%
\pgfpathlineto{\pgfqpoint{2.977285in}{0.920000in}}%
\pgfpathlineto{\pgfqpoint{2.986322in}{1.008516in}}%
\pgfpathlineto{\pgfqpoint{2.995358in}{0.936258in}}%
\pgfpathlineto{\pgfqpoint{3.004394in}{0.966968in}}%
\pgfpathlineto{\pgfqpoint{3.013430in}{0.992258in}}%
\pgfpathlineto{\pgfqpoint{3.022467in}{0.986839in}}%
\pgfpathlineto{\pgfqpoint{3.031503in}{0.909161in}}%
\pgfpathlineto{\pgfqpoint{3.040539in}{1.042839in}}%
\pgfpathlineto{\pgfqpoint{3.049576in}{0.862194in}}%
\pgfpathlineto{\pgfqpoint{3.058612in}{1.017548in}}%
\pgfpathlineto{\pgfqpoint{3.067648in}{0.990452in}}%
\pgfpathlineto{\pgfqpoint{3.076684in}{0.983226in}}%
\pgfpathlineto{\pgfqpoint{3.085721in}{1.153032in}}%
\pgfpathlineto{\pgfqpoint{3.094757in}{0.891097in}}%
\pgfpathlineto{\pgfqpoint{3.103793in}{0.842323in}}%
\pgfpathlineto{\pgfqpoint{3.112829in}{0.954323in}}%
\pgfpathlineto{\pgfqpoint{3.121866in}{0.918194in}}%
\pgfpathlineto{\pgfqpoint{3.130902in}{0.853161in}}%
\pgfpathlineto{\pgfqpoint{3.139938in}{0.862194in}}%
\pgfpathlineto{\pgfqpoint{3.148974in}{0.939871in}}%
\pgfpathlineto{\pgfqpoint{3.158011in}{1.028387in}}%
\pgfpathlineto{\pgfqpoint{3.167047in}{0.883871in}}%
\pgfpathlineto{\pgfqpoint{3.176083in}{0.968774in}}%
\pgfpathlineto{\pgfqpoint{3.185119in}{1.171097in}}%
\pgfpathlineto{\pgfqpoint{3.194156in}{1.140387in}}%
\pgfpathlineto{\pgfqpoint{3.203192in}{1.006710in}}%
\pgfpathlineto{\pgfqpoint{3.212228in}{1.068129in}}%
\pgfpathlineto{\pgfqpoint{3.221264in}{0.981419in}}%
\pgfpathlineto{\pgfqpoint{3.230301in}{0.874839in}}%
\pgfpathlineto{\pgfqpoint{3.239337in}{0.999484in}}%
\pgfpathlineto{\pgfqpoint{3.248373in}{0.853161in}}%
\pgfpathlineto{\pgfqpoint{3.257409in}{1.048258in}}%
\pgfpathlineto{\pgfqpoint{3.266446in}{1.024774in}}%
\pgfpathlineto{\pgfqpoint{3.275482in}{1.183742in}}%
\pgfpathlineto{\pgfqpoint{3.293554in}{0.956129in}}%
\pgfpathlineto{\pgfqpoint{3.302591in}{1.026581in}}%
\pgfpathlineto{\pgfqpoint{3.311627in}{0.901935in}}%
\pgfpathlineto{\pgfqpoint{3.329699in}{1.180129in}}%
\pgfpathlineto{\pgfqpoint{3.338736in}{1.003097in}}%
\pgfpathlineto{\pgfqpoint{3.347772in}{1.185548in}}%
\pgfpathlineto{\pgfqpoint{3.356808in}{1.008516in}}%
\pgfpathlineto{\pgfqpoint{3.365844in}{1.185548in}}%
\pgfpathlineto{\pgfqpoint{3.374881in}{1.127742in}}%
\pgfpathlineto{\pgfqpoint{3.383917in}{0.963355in}}%
\pgfpathlineto{\pgfqpoint{3.392953in}{1.127742in}}%
\pgfpathlineto{\pgfqpoint{3.401989in}{0.957935in}}%
\pgfpathlineto{\pgfqpoint{3.411026in}{0.956129in}}%
\pgfpathlineto{\pgfqpoint{3.420062in}{1.346323in}}%
\pgfpathlineto{\pgfqpoint{3.429098in}{0.966968in}}%
\pgfpathlineto{\pgfqpoint{3.438134in}{1.245161in}}%
\pgfpathlineto{\pgfqpoint{3.447171in}{1.120516in}}%
\pgfpathlineto{\pgfqpoint{3.456207in}{1.144000in}}%
\pgfpathlineto{\pgfqpoint{3.465243in}{0.972387in}}%
\pgfpathlineto{\pgfqpoint{3.474279in}{0.936258in}}%
\pgfpathlineto{\pgfqpoint{3.483316in}{1.125935in}}%
\pgfpathlineto{\pgfqpoint{3.492352in}{0.974194in}}%
\pgfpathlineto{\pgfqpoint{3.501388in}{1.003097in}}%
\pgfpathlineto{\pgfqpoint{3.510424in}{0.959742in}}%
\pgfpathlineto{\pgfqpoint{3.519461in}{0.950710in}}%
\pgfpathlineto{\pgfqpoint{3.528497in}{1.060903in}}%
\pgfpathlineto{\pgfqpoint{3.537533in}{1.082581in}}%
\pgfpathlineto{\pgfqpoint{3.546570in}{1.418581in}}%
\pgfpathlineto{\pgfqpoint{3.555606in}{1.089806in}}%
\pgfpathlineto{\pgfqpoint{3.564642in}{1.084387in}}%
\pgfpathlineto{\pgfqpoint{3.573678in}{1.348129in}}%
\pgfpathlineto{\pgfqpoint{3.582715in}{0.856774in}}%
\pgfpathlineto{\pgfqpoint{3.600787in}{1.268645in}}%
\pgfpathlineto{\pgfqpoint{3.609823in}{1.030194in}}%
\pgfpathlineto{\pgfqpoint{3.618860in}{1.131355in}}%
\pgfpathlineto{\pgfqpoint{3.627896in}{0.945290in}}%
\pgfpathlineto{\pgfqpoint{3.636932in}{1.134968in}}%
\pgfpathlineto{\pgfqpoint{3.645968in}{0.990452in}}%
\pgfpathlineto{\pgfqpoint{3.655005in}{1.218065in}}%
\pgfpathlineto{\pgfqpoint{3.664041in}{0.999484in}}%
\pgfpathlineto{\pgfqpoint{3.673077in}{0.997677in}}%
\pgfpathlineto{\pgfqpoint{3.682113in}{1.174710in}}%
\pgfpathlineto{\pgfqpoint{3.691150in}{1.033806in}}%
\pgfpathlineto{\pgfqpoint{3.700186in}{0.921806in}}%
\pgfpathlineto{\pgfqpoint{3.709222in}{1.095226in}}%
\pgfpathlineto{\pgfqpoint{3.718258in}{0.990452in}}%
\pgfpathlineto{\pgfqpoint{3.727295in}{1.122323in}}%
\pgfpathlineto{\pgfqpoint{3.736331in}{1.030194in}}%
\pgfpathlineto{\pgfqpoint{3.745367in}{1.059097in}}%
\pgfpathlineto{\pgfqpoint{3.754403in}{1.160258in}}%
\pgfpathlineto{\pgfqpoint{3.763440in}{1.048258in}}%
\pgfpathlineto{\pgfqpoint{3.772476in}{1.068129in}}%
\pgfpathlineto{\pgfqpoint{3.781512in}{1.283097in}}%
\pgfpathlineto{\pgfqpoint{3.799585in}{1.001290in}}%
\pgfpathlineto{\pgfqpoint{3.808621in}{1.008516in}}%
\pgfpathlineto{\pgfqpoint{3.817657in}{1.151226in}}%
\pgfpathlineto{\pgfqpoint{3.826693in}{1.104258in}}%
\pgfpathlineto{\pgfqpoint{3.835730in}{0.972387in}}%
\pgfpathlineto{\pgfqpoint{3.844766in}{0.985032in}}%
\pgfpathlineto{\pgfqpoint{3.853802in}{0.930839in}}%
\pgfpathlineto{\pgfqpoint{3.862838in}{0.952516in}}%
\pgfpathlineto{\pgfqpoint{3.871875in}{0.932645in}}%
\pgfpathlineto{\pgfqpoint{3.880911in}{0.974194in}}%
\pgfpathlineto{\pgfqpoint{3.889947in}{1.140387in}}%
\pgfpathlineto{\pgfqpoint{3.898983in}{1.060903in}}%
\pgfpathlineto{\pgfqpoint{3.908020in}{0.939871in}}%
\pgfpathlineto{\pgfqpoint{3.917056in}{1.205419in}}%
\pgfpathlineto{\pgfqpoint{3.926092in}{1.205419in}}%
\pgfpathlineto{\pgfqpoint{3.935128in}{0.994065in}}%
\pgfpathlineto{\pgfqpoint{3.944165in}{1.127742in}}%
\pgfpathlineto{\pgfqpoint{3.953201in}{1.035613in}}%
\pgfpathlineto{\pgfqpoint{3.962237in}{0.961548in}}%
\pgfpathlineto{\pgfqpoint{3.971273in}{1.281290in}}%
\pgfpathlineto{\pgfqpoint{3.989346in}{0.977806in}}%
\pgfpathlineto{\pgfqpoint{3.998382in}{1.008516in}}%
\pgfpathlineto{\pgfqpoint{4.007418in}{1.198194in}}%
\pgfpathlineto{\pgfqpoint{4.016455in}{1.089806in}}%
\pgfpathlineto{\pgfqpoint{4.025491in}{1.095226in}}%
\pgfpathlineto{\pgfqpoint{4.034527in}{1.120516in}}%
\pgfpathlineto{\pgfqpoint{4.043563in}{1.194581in}}%
\pgfpathlineto{\pgfqpoint{4.052600in}{1.194581in}}%
\pgfpathlineto{\pgfqpoint{4.061636in}{1.086194in}}%
\pgfpathlineto{\pgfqpoint{4.070672in}{1.008516in}}%
\pgfpathlineto{\pgfqpoint{4.079709in}{1.107871in}}%
\pgfpathlineto{\pgfqpoint{4.088745in}{1.102452in}}%
\pgfpathlineto{\pgfqpoint{4.106817in}{0.918194in}}%
\pgfpathlineto{\pgfqpoint{4.115854in}{1.097032in}}%
\pgfpathlineto{\pgfqpoint{4.124890in}{1.228903in}}%
\pgfpathlineto{\pgfqpoint{4.133926in}{1.274065in}}%
\pgfpathlineto{\pgfqpoint{4.142962in}{1.250581in}}%
\pgfpathlineto{\pgfqpoint{4.151999in}{1.104258in}}%
\pgfpathlineto{\pgfqpoint{4.161035in}{1.093419in}}%
\pgfpathlineto{\pgfqpoint{4.170071in}{0.952516in}}%
\pgfpathlineto{\pgfqpoint{4.179107in}{0.943484in}}%
\pgfpathlineto{\pgfqpoint{4.188144in}{1.292129in}}%
\pgfpathlineto{\pgfqpoint{4.197180in}{1.315613in}}%
\pgfpathlineto{\pgfqpoint{4.206216in}{1.313806in}}%
\pgfpathlineto{\pgfqpoint{4.215252in}{1.225290in}}%
\pgfpathlineto{\pgfqpoint{4.224289in}{1.091613in}}%
\pgfpathlineto{\pgfqpoint{4.233325in}{1.116903in}}%
\pgfpathlineto{\pgfqpoint{4.242361in}{1.106065in}}%
\pgfpathlineto{\pgfqpoint{4.251397in}{1.181935in}}%
\pgfpathlineto{\pgfqpoint{4.260434in}{1.140387in}}%
\pgfpathlineto{\pgfqpoint{4.269470in}{1.069935in}}%
\pgfpathlineto{\pgfqpoint{4.278506in}{1.246968in}}%
\pgfpathlineto{\pgfqpoint{4.287542in}{1.098839in}}%
\pgfpathlineto{\pgfqpoint{4.296579in}{1.241548in}}%
\pgfpathlineto{\pgfqpoint{4.305615in}{0.988645in}}%
\pgfpathlineto{\pgfqpoint{4.314651in}{0.968774in}}%
\pgfpathlineto{\pgfqpoint{4.323687in}{1.190968in}}%
\pgfpathlineto{\pgfqpoint{4.332724in}{1.201806in}}%
\pgfpathlineto{\pgfqpoint{4.341760in}{0.990452in}}%
\pgfpathlineto{\pgfqpoint{4.350796in}{1.246968in}}%
\pgfpathlineto{\pgfqpoint{4.359832in}{1.308387in}}%
\pgfpathlineto{\pgfqpoint{4.368869in}{1.088000in}}%
\pgfpathlineto{\pgfqpoint{4.377905in}{1.118710in}}%
\pgfpathlineto{\pgfqpoint{4.386941in}{1.279484in}}%
\pgfpathlineto{\pgfqpoint{4.395977in}{1.102452in}}%
\pgfpathlineto{\pgfqpoint{4.405014in}{1.051871in}}%
\pgfpathlineto{\pgfqpoint{4.414050in}{1.219871in}}%
\pgfpathlineto{\pgfqpoint{4.423086in}{1.304774in}}%
\pgfpathlineto{\pgfqpoint{4.432122in}{1.151226in}}%
\pgfpathlineto{\pgfqpoint{4.441159in}{1.337290in}}%
\pgfpathlineto{\pgfqpoint{4.450195in}{1.254194in}}%
\pgfpathlineto{\pgfqpoint{4.459231in}{1.284903in}}%
\pgfpathlineto{\pgfqpoint{4.477304in}{1.134968in}}%
\pgfpathlineto{\pgfqpoint{4.486340in}{1.075355in}}%
\pgfpathlineto{\pgfqpoint{4.495376in}{0.994065in}}%
\pgfpathlineto{\pgfqpoint{4.504412in}{1.098839in}}%
\pgfpathlineto{\pgfqpoint{4.513449in}{1.304774in}}%
\pgfpathlineto{\pgfqpoint{4.522485in}{0.932645in}}%
\pgfpathlineto{\pgfqpoint{4.531521in}{1.194581in}}%
\pgfpathlineto{\pgfqpoint{4.540557in}{1.187355in}}%
\pgfpathlineto{\pgfqpoint{4.549594in}{1.046452in}}%
\pgfpathlineto{\pgfqpoint{4.558630in}{1.283097in}}%
\pgfpathlineto{\pgfqpoint{4.567666in}{1.111484in}}%
\pgfpathlineto{\pgfqpoint{4.576702in}{1.089806in}}%
\pgfpathlineto{\pgfqpoint{4.585739in}{1.353548in}}%
\pgfpathlineto{\pgfqpoint{4.594775in}{1.071742in}}%
\pgfpathlineto{\pgfqpoint{4.603811in}{1.185548in}}%
\pgfpathlineto{\pgfqpoint{4.612848in}{1.257806in}}%
\pgfpathlineto{\pgfqpoint{4.621884in}{0.956129in}}%
\pgfpathlineto{\pgfqpoint{4.639956in}{1.234323in}}%
\pgfpathlineto{\pgfqpoint{4.648993in}{1.091613in}}%
\pgfpathlineto{\pgfqpoint{4.658029in}{1.237935in}}%
\pgfpathlineto{\pgfqpoint{4.667065in}{1.042839in}}%
\pgfpathlineto{\pgfqpoint{4.676101in}{1.077161in}}%
\pgfpathlineto{\pgfqpoint{4.685138in}{1.129548in}}%
\pgfpathlineto{\pgfqpoint{4.694174in}{1.147613in}}%
\pgfpathlineto{\pgfqpoint{4.703210in}{1.162065in}}%
\pgfpathlineto{\pgfqpoint{4.712246in}{1.284903in}}%
\pgfpathlineto{\pgfqpoint{4.721283in}{1.057290in}}%
\pgfpathlineto{\pgfqpoint{4.739355in}{1.207226in}}%
\pgfpathlineto{\pgfqpoint{4.748391in}{1.084387in}}%
\pgfpathlineto{\pgfqpoint{4.757428in}{1.187355in}}%
\pgfpathlineto{\pgfqpoint{4.766464in}{1.330065in}}%
\pgfpathlineto{\pgfqpoint{4.775500in}{1.234323in}}%
\pgfpathlineto{\pgfqpoint{4.784536in}{1.080774in}}%
\pgfpathlineto{\pgfqpoint{4.793573in}{1.360774in}}%
\pgfpathlineto{\pgfqpoint{4.802609in}{1.185548in}}%
\pgfpathlineto{\pgfqpoint{4.811645in}{1.035613in}}%
\pgfpathlineto{\pgfqpoint{4.820681in}{1.004903in}}%
\pgfpathlineto{\pgfqpoint{4.829718in}{1.012129in}}%
\pgfpathlineto{\pgfqpoint{4.838754in}{1.335484in}}%
\pgfpathlineto{\pgfqpoint{4.847790in}{1.227097in}}%
\pgfpathlineto{\pgfqpoint{4.856826in}{0.997677in}}%
\pgfpathlineto{\pgfqpoint{4.865863in}{1.131355in}}%
\pgfpathlineto{\pgfqpoint{4.874899in}{0.992258in}}%
\pgfpathlineto{\pgfqpoint{4.883935in}{1.212645in}}%
\pgfpathlineto{\pgfqpoint{4.892971in}{1.073548in}}%
\pgfpathlineto{\pgfqpoint{4.902008in}{1.198194in}}%
\pgfpathlineto{\pgfqpoint{4.911044in}{1.375226in}}%
\pgfpathlineto{\pgfqpoint{4.920080in}{1.138581in}}%
\pgfpathlineto{\pgfqpoint{4.929116in}{1.057290in}}%
\pgfpathlineto{\pgfqpoint{4.938153in}{1.223484in}}%
\pgfpathlineto{\pgfqpoint{4.947189in}{1.133161in}}%
\pgfpathlineto{\pgfqpoint{4.956225in}{0.930839in}}%
\pgfpathlineto{\pgfqpoint{4.965261in}{1.263226in}}%
\pgfpathlineto{\pgfqpoint{4.974298in}{1.172903in}}%
\pgfpathlineto{\pgfqpoint{4.983334in}{1.239742in}}%
\pgfpathlineto{\pgfqpoint{4.992370in}{1.449290in}}%
\pgfpathlineto{\pgfqpoint{5.001406in}{1.312000in}}%
\pgfpathlineto{\pgfqpoint{5.010443in}{1.212645in}}%
\pgfpathlineto{\pgfqpoint{5.019479in}{1.163871in}}%
\pgfpathlineto{\pgfqpoint{5.028515in}{1.147613in}}%
\pgfpathlineto{\pgfqpoint{5.037551in}{1.398710in}}%
\pgfpathlineto{\pgfqpoint{5.046588in}{1.089806in}}%
\pgfpathlineto{\pgfqpoint{5.055624in}{1.145806in}}%
\pgfpathlineto{\pgfqpoint{5.073696in}{0.941677in}}%
\pgfpathlineto{\pgfqpoint{5.082733in}{1.111484in}}%
\pgfpathlineto{\pgfqpoint{5.091769in}{1.207226in}}%
\pgfpathlineto{\pgfqpoint{5.100805in}{1.198194in}}%
\pgfpathlineto{\pgfqpoint{5.109842in}{1.321032in}}%
\pgfpathlineto{\pgfqpoint{5.118878in}{1.039226in}}%
\pgfpathlineto{\pgfqpoint{5.127914in}{1.259613in}}%
\pgfpathlineto{\pgfqpoint{5.136950in}{1.160258in}}%
\pgfpathlineto{\pgfqpoint{5.145987in}{1.268645in}}%
\pgfpathlineto{\pgfqpoint{5.155023in}{1.109677in}}%
\pgfpathlineto{\pgfqpoint{5.164059in}{1.113290in}}%
\pgfpathlineto{\pgfqpoint{5.182132in}{1.360774in}}%
\pgfpathlineto{\pgfqpoint{5.191168in}{1.051871in}}%
\pgfpathlineto{\pgfqpoint{5.200204in}{1.165677in}}%
\pgfpathlineto{\pgfqpoint{5.209240in}{1.366194in}}%
\pgfpathlineto{\pgfqpoint{5.218277in}{1.302968in}}%
\pgfpathlineto{\pgfqpoint{5.227313in}{1.324645in}}%
\pgfpathlineto{\pgfqpoint{5.236349in}{1.257806in}}%
\pgfpathlineto{\pgfqpoint{5.245385in}{1.223484in}}%
\pgfpathlineto{\pgfqpoint{5.254422in}{0.948903in}}%
\pgfpathlineto{\pgfqpoint{5.263458in}{1.218065in}}%
\pgfpathlineto{\pgfqpoint{5.272494in}{1.275871in}}%
\pgfpathlineto{\pgfqpoint{5.281530in}{1.131355in}}%
\pgfpathlineto{\pgfqpoint{5.290567in}{1.107871in}}%
\pgfpathlineto{\pgfqpoint{5.299603in}{1.306581in}}%
\pgfpathlineto{\pgfqpoint{5.308639in}{1.060903in}}%
\pgfpathlineto{\pgfqpoint{5.317675in}{1.066323in}}%
\pgfpathlineto{\pgfqpoint{5.326712in}{1.322839in}}%
\pgfpathlineto{\pgfqpoint{5.335748in}{0.994065in}}%
\pgfpathlineto{\pgfqpoint{5.344784in}{1.209032in}}%
\pgfpathlineto{\pgfqpoint{5.353820in}{1.308387in}}%
\pgfpathlineto{\pgfqpoint{5.362857in}{1.044645in}}%
\pgfpathlineto{\pgfqpoint{5.371893in}{1.239742in}}%
\pgfpathlineto{\pgfqpoint{5.380929in}{1.053677in}}%
\pgfpathlineto{\pgfqpoint{5.389965in}{1.212645in}}%
\pgfpathlineto{\pgfqpoint{5.399002in}{1.017548in}}%
\pgfpathlineto{\pgfqpoint{5.408038in}{1.283097in}}%
\pgfpathlineto{\pgfqpoint{5.417074in}{1.481806in}}%
\pgfpathlineto{\pgfqpoint{5.426110in}{1.225290in}}%
\pgfpathlineto{\pgfqpoint{5.435147in}{1.111484in}}%
\pgfpathlineto{\pgfqpoint{5.444183in}{1.127742in}}%
\pgfpathlineto{\pgfqpoint{5.453219in}{1.075355in}}%
\pgfpathlineto{\pgfqpoint{5.462255in}{1.185548in}}%
\pgfpathlineto{\pgfqpoint{5.471292in}{1.147613in}}%
\pgfpathlineto{\pgfqpoint{5.480328in}{1.218065in}}%
\pgfpathlineto{\pgfqpoint{5.489364in}{1.026581in}}%
\pgfpathlineto{\pgfqpoint{5.498400in}{1.203613in}}%
\pgfpathlineto{\pgfqpoint{5.507437in}{1.113290in}}%
\pgfpathlineto{\pgfqpoint{5.516473in}{1.048258in}}%
\pgfpathlineto{\pgfqpoint{5.525509in}{1.144000in}}%
\pgfpathlineto{\pgfqpoint{5.534545in}{1.171097in}}%
\pgfpathlineto{\pgfqpoint{5.534545in}{1.171097in}}%
\pgfusepath{stroke}%
\end{pgfscope}%
\begin{pgfscope}%
\pgfpathrectangle{\pgfqpoint{0.800000in}{0.528000in}}{\pgfqpoint{4.960000in}{3.696000in}} %
\pgfusepath{clip}%
\pgfsetrectcap%
\pgfsetroundjoin%
\pgfsetlinewidth{1.505625pt}%
\definecolor{currentstroke}{rgb}{0.600000,0.600000,0.600000}%
\pgfsetstrokecolor{currentstroke}%
\pgfsetdash{}{0pt}%
\pgfpathmoveto{\pgfqpoint{1.025455in}{0.797161in}}%
\pgfpathlineto{\pgfqpoint{1.034491in}{0.842323in}}%
\pgfpathlineto{\pgfqpoint{1.043527in}{0.795355in}}%
\pgfpathlineto{\pgfqpoint{1.052563in}{0.827871in}}%
\pgfpathlineto{\pgfqpoint{1.061600in}{1.057290in}}%
\pgfpathlineto{\pgfqpoint{1.070636in}{0.797161in}}%
\pgfpathlineto{\pgfqpoint{1.079672in}{0.988645in}}%
\pgfpathlineto{\pgfqpoint{1.088708in}{0.840516in}}%
\pgfpathlineto{\pgfqpoint{1.097745in}{0.840516in}}%
\pgfpathlineto{\pgfqpoint{1.106781in}{1.026581in}}%
\pgfpathlineto{\pgfqpoint{1.115817in}{0.833290in}}%
\pgfpathlineto{\pgfqpoint{1.124853in}{0.979613in}}%
\pgfpathlineto{\pgfqpoint{1.133890in}{0.853161in}}%
\pgfpathlineto{\pgfqpoint{1.142926in}{0.858581in}}%
\pgfpathlineto{\pgfqpoint{1.151962in}{0.874839in}}%
\pgfpathlineto{\pgfqpoint{1.160998in}{0.968774in}}%
\pgfpathlineto{\pgfqpoint{1.170035in}{1.041032in}}%
\pgfpathlineto{\pgfqpoint{1.179071in}{0.907355in}}%
\pgfpathlineto{\pgfqpoint{1.188107in}{0.920000in}}%
\pgfpathlineto{\pgfqpoint{1.197143in}{0.977806in}}%
\pgfpathlineto{\pgfqpoint{1.206180in}{1.106065in}}%
\pgfpathlineto{\pgfqpoint{1.215216in}{0.970581in}}%
\pgfpathlineto{\pgfqpoint{1.224252in}{1.021161in}}%
\pgfpathlineto{\pgfqpoint{1.233288in}{0.891097in}}%
\pgfpathlineto{\pgfqpoint{1.242325in}{0.847742in}}%
\pgfpathlineto{\pgfqpoint{1.251361in}{1.122323in}}%
\pgfpathlineto{\pgfqpoint{1.260397in}{0.820645in}}%
\pgfpathlineto{\pgfqpoint{1.269433in}{0.995871in}}%
\pgfpathlineto{\pgfqpoint{1.278470in}{1.214452in}}%
\pgfpathlineto{\pgfqpoint{1.287506in}{0.923613in}}%
\pgfpathlineto{\pgfqpoint{1.296542in}{1.051871in}}%
\pgfpathlineto{\pgfqpoint{1.305578in}{1.069935in}}%
\pgfpathlineto{\pgfqpoint{1.314615in}{1.140387in}}%
\pgfpathlineto{\pgfqpoint{1.323651in}{1.091613in}}%
\pgfpathlineto{\pgfqpoint{1.332687in}{0.909161in}}%
\pgfpathlineto{\pgfqpoint{1.341723in}{1.158452in}}%
\pgfpathlineto{\pgfqpoint{1.350760in}{0.910968in}}%
\pgfpathlineto{\pgfqpoint{1.359796in}{1.022968in}}%
\pgfpathlineto{\pgfqpoint{1.368832in}{1.089806in}}%
\pgfpathlineto{\pgfqpoint{1.377868in}{1.277677in}}%
\pgfpathlineto{\pgfqpoint{1.386905in}{1.270452in}}%
\pgfpathlineto{\pgfqpoint{1.395941in}{1.111484in}}%
\pgfpathlineto{\pgfqpoint{1.404977in}{1.232516in}}%
\pgfpathlineto{\pgfqpoint{1.414013in}{1.284903in}}%
\pgfpathlineto{\pgfqpoint{1.423050in}{1.209032in}}%
\pgfpathlineto{\pgfqpoint{1.432086in}{1.290323in}}%
\pgfpathlineto{\pgfqpoint{1.441122in}{1.149419in}}%
\pgfpathlineto{\pgfqpoint{1.450158in}{1.380645in}}%
\pgfpathlineto{\pgfqpoint{1.468231in}{1.366194in}}%
\pgfpathlineto{\pgfqpoint{1.477267in}{1.133161in}}%
\pgfpathlineto{\pgfqpoint{1.486304in}{1.248774in}}%
\pgfpathlineto{\pgfqpoint{1.495340in}{1.169290in}}%
\pgfpathlineto{\pgfqpoint{1.504376in}{1.169290in}}%
\pgfpathlineto{\pgfqpoint{1.513412in}{1.131355in}}%
\pgfpathlineto{\pgfqpoint{1.522449in}{1.337290in}}%
\pgfpathlineto{\pgfqpoint{1.531485in}{1.118710in}}%
\pgfpathlineto{\pgfqpoint{1.540521in}{1.151226in}}%
\pgfpathlineto{\pgfqpoint{1.549557in}{1.140387in}}%
\pgfpathlineto{\pgfqpoint{1.558594in}{1.310194in}}%
\pgfpathlineto{\pgfqpoint{1.567630in}{1.125935in}}%
\pgfpathlineto{\pgfqpoint{1.576666in}{1.230710in}}%
\pgfpathlineto{\pgfqpoint{1.585702in}{1.001290in}}%
\pgfpathlineto{\pgfqpoint{1.594739in}{1.257806in}}%
\pgfpathlineto{\pgfqpoint{1.603775in}{1.638968in}}%
\pgfpathlineto{\pgfqpoint{1.612811in}{1.322839in}}%
\pgfpathlineto{\pgfqpoint{1.621847in}{1.241548in}}%
\pgfpathlineto{\pgfqpoint{1.630884in}{1.232516in}}%
\pgfpathlineto{\pgfqpoint{1.648956in}{1.425806in}}%
\pgfpathlineto{\pgfqpoint{1.657992in}{1.209032in}}%
\pgfpathlineto{\pgfqpoint{1.667029in}{1.440258in}}%
\pgfpathlineto{\pgfqpoint{1.676065in}{1.270452in}}%
\pgfpathlineto{\pgfqpoint{1.685101in}{1.409548in}}%
\pgfpathlineto{\pgfqpoint{1.694137in}{1.452903in}}%
\pgfpathlineto{\pgfqpoint{1.703174in}{1.250581in}}%
\pgfpathlineto{\pgfqpoint{1.712210in}{1.290323in}}%
\pgfpathlineto{\pgfqpoint{1.721246in}{1.107871in}}%
\pgfpathlineto{\pgfqpoint{1.730282in}{1.299355in}}%
\pgfpathlineto{\pgfqpoint{1.739319in}{1.301161in}}%
\pgfpathlineto{\pgfqpoint{1.748355in}{1.268645in}}%
\pgfpathlineto{\pgfqpoint{1.757391in}{1.259613in}}%
\pgfpathlineto{\pgfqpoint{1.766427in}{1.162065in}}%
\pgfpathlineto{\pgfqpoint{1.775464in}{1.330065in}}%
\pgfpathlineto{\pgfqpoint{1.784500in}{1.293935in}}%
\pgfpathlineto{\pgfqpoint{1.793536in}{1.246968in}}%
\pgfpathlineto{\pgfqpoint{1.802572in}{1.319226in}}%
\pgfpathlineto{\pgfqpoint{1.811609in}{1.228903in}}%
\pgfpathlineto{\pgfqpoint{1.820645in}{1.312000in}}%
\pgfpathlineto{\pgfqpoint{1.829681in}{1.339097in}}%
\pgfpathlineto{\pgfqpoint{1.838717in}{1.138581in}}%
\pgfpathlineto{\pgfqpoint{1.847754in}{1.389677in}}%
\pgfpathlineto{\pgfqpoint{1.856790in}{1.358968in}}%
\pgfpathlineto{\pgfqpoint{1.865826in}{1.669677in}}%
\pgfpathlineto{\pgfqpoint{1.874862in}{1.349935in}}%
\pgfpathlineto{\pgfqpoint{1.883899in}{1.200000in}}%
\pgfpathlineto{\pgfqpoint{1.892935in}{1.519742in}}%
\pgfpathlineto{\pgfqpoint{1.901971in}{1.245161in}}%
\pgfpathlineto{\pgfqpoint{1.911007in}{1.196387in}}%
\pgfpathlineto{\pgfqpoint{1.920044in}{1.669677in}}%
\pgfpathlineto{\pgfqpoint{1.929080in}{1.306581in}}%
\pgfpathlineto{\pgfqpoint{1.938116in}{1.375226in}}%
\pgfpathlineto{\pgfqpoint{1.947152in}{1.460129in}}%
\pgfpathlineto{\pgfqpoint{1.956189in}{1.212645in}}%
\pgfpathlineto{\pgfqpoint{1.965225in}{1.593806in}}%
\pgfpathlineto{\pgfqpoint{1.983298in}{1.711226in}}%
\pgfpathlineto{\pgfqpoint{1.992334in}{1.732903in}}%
\pgfpathlineto{\pgfqpoint{2.001370in}{1.463742in}}%
\pgfpathlineto{\pgfqpoint{2.010406in}{1.519742in}}%
\pgfpathlineto{\pgfqpoint{2.019443in}{1.872000in}}%
\pgfpathlineto{\pgfqpoint{2.028479in}{1.324645in}}%
\pgfpathlineto{\pgfqpoint{2.037515in}{1.566710in}}%
\pgfpathlineto{\pgfqpoint{2.046551in}{1.615484in}}%
\pgfpathlineto{\pgfqpoint{2.055588in}{1.545032in}}%
\pgfpathlineto{\pgfqpoint{2.064624in}{1.516129in}}%
\pgfpathlineto{\pgfqpoint{2.073660in}{1.521548in}}%
\pgfpathlineto{\pgfqpoint{2.082696in}{1.622710in}}%
\pgfpathlineto{\pgfqpoint{2.091733in}{1.964129in}}%
\pgfpathlineto{\pgfqpoint{2.100769in}{1.295742in}}%
\pgfpathlineto{\pgfqpoint{2.109805in}{1.615484in}}%
\pgfpathlineto{\pgfqpoint{2.118841in}{2.021935in}}%
\pgfpathlineto{\pgfqpoint{2.127878in}{1.575742in}}%
\pgfpathlineto{\pgfqpoint{2.136914in}{1.579355in}}%
\pgfpathlineto{\pgfqpoint{2.145950in}{1.617290in}}%
\pgfpathlineto{\pgfqpoint{2.154986in}{1.844903in}}%
\pgfpathlineto{\pgfqpoint{2.164023in}{1.480000in}}%
\pgfpathlineto{\pgfqpoint{2.173059in}{1.416774in}}%
\pgfpathlineto{\pgfqpoint{2.182095in}{1.413161in}}%
\pgfpathlineto{\pgfqpoint{2.191131in}{1.629935in}}%
\pgfpathlineto{\pgfqpoint{2.200168in}{1.816000in}}%
\pgfpathlineto{\pgfqpoint{2.209204in}{1.881032in}}%
\pgfpathlineto{\pgfqpoint{2.218240in}{1.508903in}}%
\pgfpathlineto{\pgfqpoint{2.227276in}{1.774452in}}%
\pgfpathlineto{\pgfqpoint{2.236313in}{1.805161in}}%
\pgfpathlineto{\pgfqpoint{2.245349in}{2.076129in}}%
\pgfpathlineto{\pgfqpoint{2.254385in}{1.434839in}}%
\pgfpathlineto{\pgfqpoint{2.263421in}{1.723871in}}%
\pgfpathlineto{\pgfqpoint{2.272458in}{1.864774in}}%
\pgfpathlineto{\pgfqpoint{2.290530in}{1.436645in}}%
\pgfpathlineto{\pgfqpoint{2.299566in}{1.909935in}}%
\pgfpathlineto{\pgfqpoint{2.308603in}{2.316387in}}%
\pgfpathlineto{\pgfqpoint{2.317639in}{1.649806in}}%
\pgfpathlineto{\pgfqpoint{2.326675in}{1.785290in}}%
\pgfpathlineto{\pgfqpoint{2.335711in}{2.045419in}}%
\pgfpathlineto{\pgfqpoint{2.344748in}{1.761806in}}%
\pgfpathlineto{\pgfqpoint{2.353784in}{2.168258in}}%
\pgfpathlineto{\pgfqpoint{2.362820in}{1.819613in}}%
\pgfpathlineto{\pgfqpoint{2.371856in}{1.604645in}}%
\pgfpathlineto{\pgfqpoint{2.380893in}{1.989419in}}%
\pgfpathlineto{\pgfqpoint{2.389929in}{2.058065in}}%
\pgfpathlineto{\pgfqpoint{2.398965in}{1.900903in}}%
\pgfpathlineto{\pgfqpoint{2.408001in}{1.640774in}}%
\pgfpathlineto{\pgfqpoint{2.417038in}{1.344516in}}%
\pgfpathlineto{\pgfqpoint{2.426074in}{1.998452in}}%
\pgfpathlineto{\pgfqpoint{2.435110in}{1.873806in}}%
\pgfpathlineto{\pgfqpoint{2.444146in}{1.799742in}}%
\pgfpathlineto{\pgfqpoint{2.453183in}{2.439226in}}%
\pgfpathlineto{\pgfqpoint{2.462219in}{1.971355in}}%
\pgfpathlineto{\pgfqpoint{2.471255in}{2.755355in}}%
\pgfpathlineto{\pgfqpoint{2.480291in}{1.725677in}}%
\pgfpathlineto{\pgfqpoint{2.489328in}{1.837677in}}%
\pgfpathlineto{\pgfqpoint{2.498364in}{2.437419in}}%
\pgfpathlineto{\pgfqpoint{2.507400in}{1.637161in}}%
\pgfpathlineto{\pgfqpoint{2.516437in}{2.421161in}}%
\pgfpathlineto{\pgfqpoint{2.525473in}{2.005677in}}%
\pgfpathlineto{\pgfqpoint{2.534509in}{2.000258in}}%
\pgfpathlineto{\pgfqpoint{2.543545in}{2.231484in}}%
\pgfpathlineto{\pgfqpoint{2.552582in}{2.016516in}}%
\pgfpathlineto{\pgfqpoint{2.561618in}{1.946065in}}%
\pgfpathlineto{\pgfqpoint{2.570654in}{2.372387in}}%
\pgfpathlineto{\pgfqpoint{2.579690in}{2.345290in}}%
\pgfpathlineto{\pgfqpoint{2.588727in}{1.904516in}}%
\pgfpathlineto{\pgfqpoint{2.597763in}{1.942452in}}%
\pgfpathlineto{\pgfqpoint{2.606799in}{2.171871in}}%
\pgfpathlineto{\pgfqpoint{2.615835in}{2.784258in}}%
\pgfpathlineto{\pgfqpoint{2.624872in}{2.558452in}}%
\pgfpathlineto{\pgfqpoint{2.633908in}{2.778839in}}%
\pgfpathlineto{\pgfqpoint{2.642944in}{2.583742in}}%
\pgfpathlineto{\pgfqpoint{2.651980in}{2.166452in}}%
\pgfpathlineto{\pgfqpoint{2.661017in}{2.144774in}}%
\pgfpathlineto{\pgfqpoint{2.670053in}{2.357935in}}%
\pgfpathlineto{\pgfqpoint{2.679089in}{2.258581in}}%
\pgfpathlineto{\pgfqpoint{2.688125in}{2.453677in}}%
\pgfpathlineto{\pgfqpoint{2.697162in}{2.477161in}}%
\pgfpathlineto{\pgfqpoint{2.706198in}{2.399484in}}%
\pgfpathlineto{\pgfqpoint{2.715234in}{2.260387in}}%
\pgfpathlineto{\pgfqpoint{2.724270in}{2.439226in}}%
\pgfpathlineto{\pgfqpoint{2.733307in}{2.809548in}}%
\pgfpathlineto{\pgfqpoint{2.742343in}{2.121290in}}%
\pgfpathlineto{\pgfqpoint{2.751379in}{2.227871in}}%
\pgfpathlineto{\pgfqpoint{2.760415in}{2.249548in}}%
\pgfpathlineto{\pgfqpoint{2.778488in}{1.918968in}}%
\pgfpathlineto{\pgfqpoint{2.787524in}{2.581935in}}%
\pgfpathlineto{\pgfqpoint{2.796560in}{2.912516in}}%
\pgfpathlineto{\pgfqpoint{2.805597in}{2.365161in}}%
\pgfpathlineto{\pgfqpoint{2.814633in}{2.170065in}}%
\pgfpathlineto{\pgfqpoint{2.823669in}{2.553032in}}%
\pgfpathlineto{\pgfqpoint{2.832705in}{2.677677in}}%
\pgfpathlineto{\pgfqpoint{2.841742in}{2.180903in}}%
\pgfpathlineto{\pgfqpoint{2.850778in}{2.650581in}}%
\pgfpathlineto{\pgfqpoint{2.859814in}{2.968516in}}%
\pgfpathlineto{\pgfqpoint{2.868850in}{2.403097in}}%
\pgfpathlineto{\pgfqpoint{2.877887in}{3.268387in}}%
\pgfpathlineto{\pgfqpoint{2.886923in}{2.462710in}}%
\pgfpathlineto{\pgfqpoint{2.895959in}{2.343484in}}%
\pgfpathlineto{\pgfqpoint{2.904995in}{3.008258in}}%
\pgfpathlineto{\pgfqpoint{2.914032in}{2.361548in}}%
\pgfpathlineto{\pgfqpoint{2.923068in}{2.719226in}}%
\pgfpathlineto{\pgfqpoint{2.932104in}{2.603613in}}%
\pgfpathlineto{\pgfqpoint{2.941140in}{2.648774in}}%
\pgfpathlineto{\pgfqpoint{2.950177in}{2.433806in}}%
\pgfpathlineto{\pgfqpoint{2.959213in}{2.303742in}}%
\pgfpathlineto{\pgfqpoint{2.968249in}{1.924387in}}%
\pgfpathlineto{\pgfqpoint{2.977285in}{2.366968in}}%
\pgfpathlineto{\pgfqpoint{2.986322in}{2.534968in}}%
\pgfpathlineto{\pgfqpoint{2.995358in}{2.477161in}}%
\pgfpathlineto{\pgfqpoint{3.004394in}{2.907097in}}%
\pgfpathlineto{\pgfqpoint{3.013430in}{2.805935in}}%
\pgfpathlineto{\pgfqpoint{3.022467in}{2.880000in}}%
\pgfpathlineto{\pgfqpoint{3.031503in}{2.621677in}}%
\pgfpathlineto{\pgfqpoint{3.040539in}{2.748129in}}%
\pgfpathlineto{\pgfqpoint{3.049576in}{2.695742in}}%
\pgfpathlineto{\pgfqpoint{3.058612in}{2.749935in}}%
\pgfpathlineto{\pgfqpoint{3.067648in}{2.863742in}}%
\pgfpathlineto{\pgfqpoint{3.076684in}{2.892645in}}%
\pgfpathlineto{\pgfqpoint{3.085721in}{2.173677in}}%
\pgfpathlineto{\pgfqpoint{3.094757in}{2.563871in}}%
\pgfpathlineto{\pgfqpoint{3.103793in}{3.161806in}}%
\pgfpathlineto{\pgfqpoint{3.112829in}{3.235871in}}%
\pgfpathlineto{\pgfqpoint{3.121866in}{2.928774in}}%
\pgfpathlineto{\pgfqpoint{3.130902in}{3.365935in}}%
\pgfpathlineto{\pgfqpoint{3.139938in}{3.107613in}}%
\pgfpathlineto{\pgfqpoint{3.148974in}{3.019097in}}%
\pgfpathlineto{\pgfqpoint{3.158011in}{2.679484in}}%
\pgfpathlineto{\pgfqpoint{3.167047in}{3.701935in}}%
\pgfpathlineto{\pgfqpoint{3.176083in}{2.858323in}}%
\pgfpathlineto{\pgfqpoint{3.185119in}{3.042581in}}%
\pgfpathlineto{\pgfqpoint{3.203192in}{3.214194in}}%
\pgfpathlineto{\pgfqpoint{3.212228in}{2.464516in}}%
\pgfpathlineto{\pgfqpoint{3.221264in}{3.262968in}}%
\pgfpathlineto{\pgfqpoint{3.230301in}{2.292903in}}%
\pgfpathlineto{\pgfqpoint{3.239337in}{2.979355in}}%
\pgfpathlineto{\pgfqpoint{3.248373in}{2.876387in}}%
\pgfpathlineto{\pgfqpoint{3.257409in}{3.335226in}}%
\pgfpathlineto{\pgfqpoint{3.266446in}{2.589161in}}%
\pgfpathlineto{\pgfqpoint{3.275482in}{3.387613in}}%
\pgfpathlineto{\pgfqpoint{3.284518in}{2.874581in}}%
\pgfpathlineto{\pgfqpoint{3.293554in}{3.291871in}}%
\pgfpathlineto{\pgfqpoint{3.302591in}{2.984774in}}%
\pgfpathlineto{\pgfqpoint{3.311627in}{3.239484in}}%
\pgfpathlineto{\pgfqpoint{3.320663in}{2.852903in}}%
\pgfpathlineto{\pgfqpoint{3.329699in}{3.057032in}}%
\pgfpathlineto{\pgfqpoint{3.338736in}{3.100387in}}%
\pgfpathlineto{\pgfqpoint{3.347772in}{3.282839in}}%
\pgfpathlineto{\pgfqpoint{3.356808in}{2.585548in}}%
\pgfpathlineto{\pgfqpoint{3.365844in}{2.758968in}}%
\pgfpathlineto{\pgfqpoint{3.374881in}{3.367742in}}%
\pgfpathlineto{\pgfqpoint{3.383917in}{2.993806in}}%
\pgfpathlineto{\pgfqpoint{3.392953in}{2.554839in}}%
\pgfpathlineto{\pgfqpoint{3.401989in}{2.896258in}}%
\pgfpathlineto{\pgfqpoint{3.411026in}{3.566452in}}%
\pgfpathlineto{\pgfqpoint{3.420062in}{3.125677in}}%
\pgfpathlineto{\pgfqpoint{3.429098in}{3.387613in}}%
\pgfpathlineto{\pgfqpoint{3.438134in}{2.357935in}}%
\pgfpathlineto{\pgfqpoint{3.447171in}{2.982968in}}%
\pgfpathlineto{\pgfqpoint{3.456207in}{3.149161in}}%
\pgfpathlineto{\pgfqpoint{3.465243in}{3.266581in}}%
\pgfpathlineto{\pgfqpoint{3.474279in}{3.042581in}}%
\pgfpathlineto{\pgfqpoint{3.483316in}{3.394839in}}%
\pgfpathlineto{\pgfqpoint{3.492352in}{3.205161in}}%
\pgfpathlineto{\pgfqpoint{3.501388in}{3.452645in}}%
\pgfpathlineto{\pgfqpoint{3.510424in}{3.617032in}}%
\pgfpathlineto{\pgfqpoint{3.519461in}{2.692129in}}%
\pgfpathlineto{\pgfqpoint{3.528497in}{3.169032in}}%
\pgfpathlineto{\pgfqpoint{3.537533in}{3.073290in}}%
\pgfpathlineto{\pgfqpoint{3.546570in}{2.896258in}}%
\pgfpathlineto{\pgfqpoint{3.555606in}{2.838452in}}%
\pgfpathlineto{\pgfqpoint{3.564642in}{3.277419in}}%
\pgfpathlineto{\pgfqpoint{3.573678in}{3.564645in}}%
\pgfpathlineto{\pgfqpoint{3.582715in}{2.861935in}}%
\pgfpathlineto{\pgfqpoint{3.591751in}{3.235871in}}%
\pgfpathlineto{\pgfqpoint{3.600787in}{3.309935in}}%
\pgfpathlineto{\pgfqpoint{3.609823in}{2.986581in}}%
\pgfpathlineto{\pgfqpoint{3.618860in}{3.199742in}}%
\pgfpathlineto{\pgfqpoint{3.627896in}{3.329806in}}%
\pgfpathlineto{\pgfqpoint{3.636932in}{3.488774in}}%
\pgfpathlineto{\pgfqpoint{3.645968in}{3.450839in}}%
\pgfpathlineto{\pgfqpoint{3.655005in}{3.676645in}}%
\pgfpathlineto{\pgfqpoint{3.664041in}{3.468903in}}%
\pgfpathlineto{\pgfqpoint{3.673077in}{2.842065in}}%
\pgfpathlineto{\pgfqpoint{3.682113in}{3.318968in}}%
\pgfpathlineto{\pgfqpoint{3.691150in}{2.954065in}}%
\pgfpathlineto{\pgfqpoint{3.700186in}{3.197935in}}%
\pgfpathlineto{\pgfqpoint{3.709222in}{3.234065in}}%
\pgfpathlineto{\pgfqpoint{3.718258in}{3.362323in}}%
\pgfpathlineto{\pgfqpoint{3.727295in}{3.385806in}}%
\pgfpathlineto{\pgfqpoint{3.736331in}{2.880000in}}%
\pgfpathlineto{\pgfqpoint{3.745367in}{3.118452in}}%
\pgfpathlineto{\pgfqpoint{3.754403in}{3.609806in}}%
\pgfpathlineto{\pgfqpoint{3.763440in}{3.131097in}}%
\pgfpathlineto{\pgfqpoint{3.772476in}{3.588129in}}%
\pgfpathlineto{\pgfqpoint{3.781512in}{3.566452in}}%
\pgfpathlineto{\pgfqpoint{3.790548in}{3.550194in}}%
\pgfpathlineto{\pgfqpoint{3.799585in}{3.340645in}}%
\pgfpathlineto{\pgfqpoint{3.808621in}{3.510452in}}%
\pgfpathlineto{\pgfqpoint{3.817657in}{3.232258in}}%
\pgfpathlineto{\pgfqpoint{3.826693in}{3.685677in}}%
\pgfpathlineto{\pgfqpoint{3.835730in}{3.049806in}}%
\pgfpathlineto{\pgfqpoint{3.844766in}{3.517677in}}%
\pgfpathlineto{\pgfqpoint{3.853802in}{3.510452in}}%
\pgfpathlineto{\pgfqpoint{3.862838in}{3.268387in}}%
\pgfpathlineto{\pgfqpoint{3.871875in}{3.306323in}}%
\pgfpathlineto{\pgfqpoint{3.880911in}{3.165419in}}%
\pgfpathlineto{\pgfqpoint{3.889947in}{3.107613in}}%
\pgfpathlineto{\pgfqpoint{3.898983in}{3.430968in}}%
\pgfpathlineto{\pgfqpoint{3.908020in}{3.660387in}}%
\pgfpathlineto{\pgfqpoint{3.917056in}{3.938581in}}%
\pgfpathlineto{\pgfqpoint{3.926092in}{3.425548in}}%
\pgfpathlineto{\pgfqpoint{3.935128in}{3.329806in}}%
\pgfpathlineto{\pgfqpoint{3.944165in}{3.776000in}}%
\pgfpathlineto{\pgfqpoint{3.953201in}{2.793290in}}%
\pgfpathlineto{\pgfqpoint{3.962237in}{2.907097in}}%
\pgfpathlineto{\pgfqpoint{3.971273in}{3.055226in}}%
\pgfpathlineto{\pgfqpoint{3.980310in}{3.040774in}}%
\pgfpathlineto{\pgfqpoint{3.989346in}{3.331613in}}%
\pgfpathlineto{\pgfqpoint{3.998382in}{3.658581in}}%
\pgfpathlineto{\pgfqpoint{4.007418in}{3.199742in}}%
\pgfpathlineto{\pgfqpoint{4.016455in}{2.446452in}}%
\pgfpathlineto{\pgfqpoint{4.025491in}{2.984774in}}%
\pgfpathlineto{\pgfqpoint{4.034527in}{3.145548in}}%
\pgfpathlineto{\pgfqpoint{4.043563in}{3.365935in}}%
\pgfpathlineto{\pgfqpoint{4.052600in}{3.490581in}}%
\pgfpathlineto{\pgfqpoint{4.061636in}{2.901677in}}%
\pgfpathlineto{\pgfqpoint{4.070672in}{3.561032in}}%
\pgfpathlineto{\pgfqpoint{4.079709in}{3.143742in}}%
\pgfpathlineto{\pgfqpoint{4.088745in}{3.701935in}}%
\pgfpathlineto{\pgfqpoint{4.106817in}{3.299097in}}%
\pgfpathlineto{\pgfqpoint{4.115854in}{3.210581in}}%
\pgfpathlineto{\pgfqpoint{4.124890in}{3.456258in}}%
\pgfpathlineto{\pgfqpoint{4.133926in}{3.333419in}}%
\pgfpathlineto{\pgfqpoint{4.142962in}{2.625290in}}%
\pgfpathlineto{\pgfqpoint{4.151999in}{3.356903in}}%
\pgfpathlineto{\pgfqpoint{4.161035in}{3.828387in}}%
\pgfpathlineto{\pgfqpoint{4.170071in}{2.973935in}}%
\pgfpathlineto{\pgfqpoint{4.179107in}{3.743484in}}%
\pgfpathlineto{\pgfqpoint{4.188144in}{3.183484in}}%
\pgfpathlineto{\pgfqpoint{4.197180in}{3.288258in}}%
\pgfpathlineto{\pgfqpoint{4.206216in}{3.631484in}}%
\pgfpathlineto{\pgfqpoint{4.215252in}{3.167226in}}%
\pgfpathlineto{\pgfqpoint{4.224289in}{3.528516in}}%
\pgfpathlineto{\pgfqpoint{4.233325in}{3.461677in}}%
\pgfpathlineto{\pgfqpoint{4.242361in}{3.683871in}}%
\pgfpathlineto{\pgfqpoint{4.251397in}{3.084129in}}%
\pgfpathlineto{\pgfqpoint{4.260434in}{3.241290in}}%
\pgfpathlineto{\pgfqpoint{4.269470in}{3.624258in}}%
\pgfpathlineto{\pgfqpoint{4.278506in}{3.568258in}}%
\pgfpathlineto{\pgfqpoint{4.287542in}{3.465290in}}%
\pgfpathlineto{\pgfqpoint{4.296579in}{3.463484in}}%
\pgfpathlineto{\pgfqpoint{4.305615in}{3.402065in}}%
\pgfpathlineto{\pgfqpoint{4.314651in}{3.552000in}}%
\pgfpathlineto{\pgfqpoint{4.323687in}{3.624258in}}%
\pgfpathlineto{\pgfqpoint{4.332724in}{3.633290in}}%
\pgfpathlineto{\pgfqpoint{4.341760in}{3.570065in}}%
\pgfpathlineto{\pgfqpoint{4.350796in}{3.421935in}}%
\pgfpathlineto{\pgfqpoint{4.359832in}{3.541161in}}%
\pgfpathlineto{\pgfqpoint{4.368869in}{3.633290in}}%
\pgfpathlineto{\pgfqpoint{4.377905in}{3.165419in}}%
\pgfpathlineto{\pgfqpoint{4.386941in}{3.942194in}}%
\pgfpathlineto{\pgfqpoint{4.395977in}{3.759742in}}%
\pgfpathlineto{\pgfqpoint{4.405014in}{3.792258in}}%
\pgfpathlineto{\pgfqpoint{4.414050in}{3.575484in}}%
\pgfpathlineto{\pgfqpoint{4.423086in}{3.140129in}}%
\pgfpathlineto{\pgfqpoint{4.432122in}{3.400258in}}%
\pgfpathlineto{\pgfqpoint{4.441159in}{3.311742in}}%
\pgfpathlineto{\pgfqpoint{4.450195in}{3.685677in}}%
\pgfpathlineto{\pgfqpoint{4.459231in}{3.203355in}}%
\pgfpathlineto{\pgfqpoint{4.468267in}{3.689290in}}%
\pgfpathlineto{\pgfqpoint{4.477304in}{3.629677in}}%
\pgfpathlineto{\pgfqpoint{4.486340in}{3.472516in}}%
\pgfpathlineto{\pgfqpoint{4.495376in}{3.362323in}}%
\pgfpathlineto{\pgfqpoint{4.504412in}{3.423742in}}%
\pgfpathlineto{\pgfqpoint{4.513449in}{3.624258in}}%
\pgfpathlineto{\pgfqpoint{4.522485in}{3.430968in}}%
\pgfpathlineto{\pgfqpoint{4.531521in}{3.635097in}}%
\pgfpathlineto{\pgfqpoint{4.540557in}{3.589935in}}%
\pgfpathlineto{\pgfqpoint{4.549594in}{3.374968in}}%
\pgfpathlineto{\pgfqpoint{4.558630in}{3.864516in}}%
\pgfpathlineto{\pgfqpoint{4.567666in}{3.683871in}}%
\pgfpathlineto{\pgfqpoint{4.576702in}{3.658581in}}%
\pgfpathlineto{\pgfqpoint{4.585739in}{3.127484in}}%
\pgfpathlineto{\pgfqpoint{4.594775in}{3.651355in}}%
\pgfpathlineto{\pgfqpoint{4.603811in}{3.678452in}}%
\pgfpathlineto{\pgfqpoint{4.612848in}{3.156387in}}%
\pgfpathlineto{\pgfqpoint{4.621884in}{3.548387in}}%
\pgfpathlineto{\pgfqpoint{4.630920in}{3.533935in}}%
\pgfpathlineto{\pgfqpoint{4.639956in}{3.329806in}}%
\pgfpathlineto{\pgfqpoint{4.648993in}{3.694710in}}%
\pgfpathlineto{\pgfqpoint{4.658029in}{3.609806in}}%
\pgfpathlineto{\pgfqpoint{4.667065in}{3.830194in}}%
\pgfpathlineto{\pgfqpoint{4.676101in}{3.712774in}}%
\pgfpathlineto{\pgfqpoint{4.685138in}{3.208774in}}%
\pgfpathlineto{\pgfqpoint{4.694174in}{3.671226in}}%
\pgfpathlineto{\pgfqpoint{4.703210in}{3.687484in}}%
\pgfpathlineto{\pgfqpoint{4.712246in}{3.772387in}}%
\pgfpathlineto{\pgfqpoint{4.721283in}{3.369548in}}%
\pgfpathlineto{\pgfqpoint{4.730319in}{3.765161in}}%
\pgfpathlineto{\pgfqpoint{4.739355in}{3.553806in}}%
\pgfpathlineto{\pgfqpoint{4.748391in}{3.842839in}}%
\pgfpathlineto{\pgfqpoint{4.757428in}{3.815742in}}%
\pgfpathlineto{\pgfqpoint{4.766464in}{3.739871in}}%
\pgfpathlineto{\pgfqpoint{4.775500in}{3.463484in}}%
\pgfpathlineto{\pgfqpoint{4.784536in}{3.860903in}}%
\pgfpathlineto{\pgfqpoint{4.793573in}{3.815742in}}%
\pgfpathlineto{\pgfqpoint{4.802609in}{3.916903in}}%
\pgfpathlineto{\pgfqpoint{4.811645in}{3.530323in}}%
\pgfpathlineto{\pgfqpoint{4.820681in}{3.477935in}}%
\pgfpathlineto{\pgfqpoint{4.829718in}{3.221419in}}%
\pgfpathlineto{\pgfqpoint{4.838754in}{3.851871in}}%
\pgfpathlineto{\pgfqpoint{4.847790in}{3.367742in}}%
\pgfpathlineto{\pgfqpoint{4.856826in}{3.712774in}}%
\pgfpathlineto{\pgfqpoint{4.865863in}{3.600774in}}%
\pgfpathlineto{\pgfqpoint{4.874899in}{3.832000in}}%
\pgfpathlineto{\pgfqpoint{4.883935in}{3.423742in}}%
\pgfpathlineto{\pgfqpoint{4.892971in}{3.736258in}}%
\pgfpathlineto{\pgfqpoint{4.902008in}{3.911484in}}%
\pgfpathlineto{\pgfqpoint{4.911044in}{3.555613in}}%
\pgfpathlineto{\pgfqpoint{4.920080in}{3.869935in}}%
\pgfpathlineto{\pgfqpoint{4.929116in}{3.790452in}}%
\pgfpathlineto{\pgfqpoint{4.938153in}{3.721806in}}%
\pgfpathlineto{\pgfqpoint{4.947189in}{3.396645in}}%
\pgfpathlineto{\pgfqpoint{4.956225in}{3.922323in}}%
\pgfpathlineto{\pgfqpoint{4.965261in}{3.561032in}}%
\pgfpathlineto{\pgfqpoint{4.974298in}{3.486968in}}%
\pgfpathlineto{\pgfqpoint{4.983334in}{3.938581in}}%
\pgfpathlineto{\pgfqpoint{4.992370in}{4.001806in}}%
\pgfpathlineto{\pgfqpoint{5.001406in}{3.633290in}}%
\pgfpathlineto{\pgfqpoint{5.010443in}{3.541161in}}%
\pgfpathlineto{\pgfqpoint{5.028515in}{3.922323in}}%
\pgfpathlineto{\pgfqpoint{5.037551in}{3.799484in}}%
\pgfpathlineto{\pgfqpoint{5.046588in}{3.960258in}}%
\pgfpathlineto{\pgfqpoint{5.055624in}{3.774194in}}%
\pgfpathlineto{\pgfqpoint{5.064660in}{3.665806in}}%
\pgfpathlineto{\pgfqpoint{5.073696in}{3.953032in}}%
\pgfpathlineto{\pgfqpoint{5.082733in}{3.165419in}}%
\pgfpathlineto{\pgfqpoint{5.091769in}{3.971097in}}%
\pgfpathlineto{\pgfqpoint{5.100805in}{3.803097in}}%
\pgfpathlineto{\pgfqpoint{5.109842in}{3.732645in}}%
\pgfpathlineto{\pgfqpoint{5.118878in}{3.613419in}}%
\pgfpathlineto{\pgfqpoint{5.127914in}{3.597161in}}%
\pgfpathlineto{\pgfqpoint{5.136950in}{3.812129in}}%
\pgfpathlineto{\pgfqpoint{5.145987in}{3.562839in}}%
\pgfpathlineto{\pgfqpoint{5.155023in}{3.454452in}}%
\pgfpathlineto{\pgfqpoint{5.164059in}{3.362323in}}%
\pgfpathlineto{\pgfqpoint{5.173095in}{3.739871in}}%
\pgfpathlineto{\pgfqpoint{5.182132in}{3.721806in}}%
\pgfpathlineto{\pgfqpoint{5.191168in}{3.633290in}}%
\pgfpathlineto{\pgfqpoint{5.200204in}{3.665806in}}%
\pgfpathlineto{\pgfqpoint{5.209240in}{3.365935in}}%
\pgfpathlineto{\pgfqpoint{5.218277in}{3.743484in}}%
\pgfpathlineto{\pgfqpoint{5.227313in}{3.105806in}}%
\pgfpathlineto{\pgfqpoint{5.236349in}{3.826581in}}%
\pgfpathlineto{\pgfqpoint{5.245385in}{3.606194in}}%
\pgfpathlineto{\pgfqpoint{5.254422in}{4.056000in}}%
\pgfpathlineto{\pgfqpoint{5.263458in}{3.754323in}}%
\pgfpathlineto{\pgfqpoint{5.281530in}{3.580903in}}%
\pgfpathlineto{\pgfqpoint{5.290567in}{3.776000in}}%
\pgfpathlineto{\pgfqpoint{5.299603in}{3.839226in}}%
\pgfpathlineto{\pgfqpoint{5.308639in}{3.570065in}}%
\pgfpathlineto{\pgfqpoint{5.317675in}{3.851871in}}%
\pgfpathlineto{\pgfqpoint{5.326712in}{3.521290in}}%
\pgfpathlineto{\pgfqpoint{5.335748in}{3.837419in}}%
\pgfpathlineto{\pgfqpoint{5.344784in}{3.692903in}}%
\pgfpathlineto{\pgfqpoint{5.362857in}{3.497806in}}%
\pgfpathlineto{\pgfqpoint{5.371893in}{3.579097in}}%
\pgfpathlineto{\pgfqpoint{5.380929in}{3.730839in}}%
\pgfpathlineto{\pgfqpoint{5.389965in}{3.954839in}}%
\pgfpathlineto{\pgfqpoint{5.399002in}{3.810323in}}%
\pgfpathlineto{\pgfqpoint{5.408038in}{3.542968in}}%
\pgfpathlineto{\pgfqpoint{5.417074in}{3.851871in}}%
\pgfpathlineto{\pgfqpoint{5.426110in}{3.810323in}}%
\pgfpathlineto{\pgfqpoint{5.435147in}{3.633290in}}%
\pgfpathlineto{\pgfqpoint{5.444183in}{3.803097in}}%
\pgfpathlineto{\pgfqpoint{5.453219in}{3.631484in}}%
\pgfpathlineto{\pgfqpoint{5.462255in}{3.739871in}}%
\pgfpathlineto{\pgfqpoint{5.471292in}{3.765161in}}%
\pgfpathlineto{\pgfqpoint{5.480328in}{3.456258in}}%
\pgfpathlineto{\pgfqpoint{5.489364in}{3.701935in}}%
\pgfpathlineto{\pgfqpoint{5.498400in}{3.907871in}}%
\pgfpathlineto{\pgfqpoint{5.507437in}{4.007226in}}%
\pgfpathlineto{\pgfqpoint{5.525509in}{3.631484in}}%
\pgfpathlineto{\pgfqpoint{5.534545in}{3.653161in}}%
\pgfpathlineto{\pgfqpoint{5.534545in}{3.653161in}}%
\pgfusepath{stroke}%
\end{pgfscope}%
\begin{pgfscope}%
\pgfpathrectangle{\pgfqpoint{0.800000in}{0.528000in}}{\pgfqpoint{4.960000in}{3.696000in}} %
\pgfusepath{clip}%
\pgfsetbuttcap%
\pgfsetmiterjoin%
\definecolor{currentfill}{rgb}{1.000000,1.000000,1.000000}%
\pgfsetfillcolor{currentfill}%
\pgfsetlinewidth{1.003750pt}%
\definecolor{currentstroke}{rgb}{1.000000,1.000000,1.000000}%
\pgfsetstrokecolor{currentstroke}%
\pgfsetdash{}{0pt}%
\pgfpathmoveto{\pgfqpoint{2.125158in}{0.487927in}}%
\pgfpathlineto{\pgfqpoint{2.199280in}{1.273273in}}%
\pgfpathlineto{\pgfqpoint{1.954842in}{1.296344in}}%
\pgfpathlineto{\pgfqpoint{1.880720in}{0.510998in}}%
\pgfpathclose%
\pgfusepath{stroke,fill}%
\end{pgfscope}%
\begin{pgfscope}%
\pgftext[x=2.046306in,y=0.551172in,left,base,rotate=84.608318]{\sffamily\fontsize{10.000000}{12.000000}\selectfont \(\displaystyle \alpha =\) 0.001}%
\end{pgfscope}%
\begin{pgfscope}%
\pgfpathrectangle{\pgfqpoint{0.800000in}{0.528000in}}{\pgfqpoint{4.960000in}{3.696000in}} %
\pgfusepath{clip}%
\pgfsetbuttcap%
\pgfsetmiterjoin%
\definecolor{currentfill}{rgb}{1.000000,1.000000,1.000000}%
\pgfsetfillcolor{currentfill}%
\pgfsetlinewidth{1.003750pt}%
\definecolor{currentstroke}{rgb}{1.000000,1.000000,1.000000}%
\pgfsetstrokecolor{currentstroke}%
\pgfsetdash{}{0pt}%
\pgfpathmoveto{\pgfqpoint{3.128564in}{1.468373in}}%
\pgfpathlineto{\pgfqpoint{3.186761in}{0.770325in}}%
\pgfpathlineto{\pgfqpoint{3.431436in}{0.790723in}}%
\pgfpathlineto{\pgfqpoint{3.373239in}{1.488772in}}%
\pgfpathclose%
\pgfusepath{stroke,fill}%
\end{pgfscope}%
\begin{pgfscope}%
\definecolor{textcolor}{rgb}{0.300000,0.300000,0.300000}%
\pgfsetstrokecolor{textcolor}%
\pgfsetfillcolor{textcolor}%
\pgftext[x=3.217333in,y=1.420026in,left,base,rotate=274.765740]{\color{textcolor}\sffamily\fontsize{10.000000}{12.000000}\selectfont \(\displaystyle \alpha =\) 0.01}%
\end{pgfscope}%
\begin{pgfscope}%
\pgfpathrectangle{\pgfqpoint{0.800000in}{0.528000in}}{\pgfqpoint{4.960000in}{3.696000in}} %
\pgfusepath{clip}%
\pgfsetbuttcap%
\pgfsetmiterjoin%
\definecolor{currentfill}{rgb}{1.000000,1.000000,1.000000}%
\pgfsetfillcolor{currentfill}%
\pgfsetlinewidth{1.003750pt}%
\definecolor{currentstroke}{rgb}{1.000000,1.000000,1.000000}%
\pgfsetstrokecolor{currentstroke}%
\pgfsetdash{}{0pt}%
\pgfpathmoveto{\pgfqpoint{4.383080in}{3.784109in}}%
\pgfpathlineto{\pgfqpoint{4.411664in}{3.172671in}}%
\pgfpathlineto{\pgfqpoint{4.656920in}{3.184137in}}%
\pgfpathlineto{\pgfqpoint{4.628336in}{3.795574in}}%
\pgfpathclose%
\pgfusepath{stroke,fill}%
\end{pgfscope}%
\begin{pgfscope}%
\definecolor{textcolor}{rgb}{0.600000,0.600000,0.600000}%
\pgfsetstrokecolor{textcolor}%
\pgfsetfillcolor{textcolor}%
\pgftext[x=4.470027in,y=3.732557in,left,base,rotate=272.676609]{\color{textcolor}\sffamily\fontsize{10.000000}{12.000000}\selectfont \(\displaystyle \alpha =\) 0.1}%
\end{pgfscope}%
\begin{pgfscope}%
\pgfsetrectcap%
\pgfsetmiterjoin%
\pgfsetlinewidth{0.803000pt}%
\definecolor{currentstroke}{rgb}{0.000000,0.000000,0.000000}%
\pgfsetstrokecolor{currentstroke}%
\pgfsetdash{}{0pt}%
\pgfpathmoveto{\pgfqpoint{0.800000in}{0.528000in}}%
\pgfpathlineto{\pgfqpoint{0.800000in}{4.224000in}}%
\pgfusepath{stroke}%
\end{pgfscope}%
\begin{pgfscope}%
\pgfsetrectcap%
\pgfsetmiterjoin%
\pgfsetlinewidth{0.803000pt}%
\definecolor{currentstroke}{rgb}{0.000000,0.000000,0.000000}%
\pgfsetstrokecolor{currentstroke}%
\pgfsetdash{}{0pt}%
\pgfpathmoveto{\pgfqpoint{5.760000in}{0.528000in}}%
\pgfpathlineto{\pgfqpoint{5.760000in}{4.224000in}}%
\pgfusepath{stroke}%
\end{pgfscope}%
\begin{pgfscope}%
\pgfsetrectcap%
\pgfsetmiterjoin%
\pgfsetlinewidth{0.803000pt}%
\definecolor{currentstroke}{rgb}{0.000000,0.000000,0.000000}%
\pgfsetstrokecolor{currentstroke}%
\pgfsetdash{}{0pt}%
\pgfpathmoveto{\pgfqpoint{0.800000in}{0.528000in}}%
\pgfpathlineto{\pgfqpoint{5.760000in}{0.528000in}}%
\pgfusepath{stroke}%
\end{pgfscope}%
\begin{pgfscope}%
\pgfsetrectcap%
\pgfsetmiterjoin%
\pgfsetlinewidth{0.803000pt}%
\definecolor{currentstroke}{rgb}{0.000000,0.000000,0.000000}%
\pgfsetstrokecolor{currentstroke}%
\pgfsetdash{}{0pt}%
\pgfpathmoveto{\pgfqpoint{0.800000in}{4.224000in}}%
\pgfpathlineto{\pgfqpoint{5.760000in}{4.224000in}}%
\pgfusepath{stroke}%
\end{pgfscope}%
\begin{pgfscope}%
\pgftext[x=3.280000in,y=4.307333in,,base]{\sffamily\fontsize{12.000000}{14.400000}\selectfont Proximal Policy Optimization Results}%
\end{pgfscope}%
\end{pgfpicture}%
\makeatother%
\endgroup%
} \\
\end{centering}
\begin{itemize}
    \item Clearly, the learning rate had a significant effect on performance.
        Smaller learning rates were prohibitive, but after the learning rate
        passed a certain threshold, the problem became feasable.
    \item It is important to note that the sample complexity is far inferior to
        TRPO. This may reflect a poor choice in hyperparameters more than an
        inherent weakness in the algorithm. However, because the update at each
        iteration was very simple, the code ran significantly faster.
    \item Possible areas of improvement include a dynamic learning rate, more
        effective trajectory per iteration count, and a different epsilon.
\end{itemize}
\end{document}
