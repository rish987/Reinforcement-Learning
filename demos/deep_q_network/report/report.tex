\documentclass[a4paper]{article}
\setlength\parindent{0pt}

\usepackage{pgfplots}
\usepackage{amsthm, amsmath, amssymb, verbatim, enumerate, mathtools, algorithm}
\usepackage{pgf}
\usepackage{hyperref}
\def\labelitemi{--}
\pgfplotsset{compat=newest}

\pagestyle{empty}

\title{Deep Q-Network Demo}
\author{Rishikesh Vaishnav}
\begin{document}
\maketitle
\section*{Monte Carlo Implementation}
\subsection*{Code}
\begin{itemize}
    \item The code for this project is available at: 
    % TODO
\url{https://github.com/rish987/Reinforcement-Learning/blob/master/demos/policy_gradient/code/policy_gradient.py}.
\end{itemize}
\subsection*{Implementation Details}
\begin{itemize}
    \item Unlike the Atari gameplay environment described by Mnih et. al., the
        pole-cart environment is not perceptually aliased. That is, the current
        observation of the state is theoretically all that is needed to
        determine an optimal value. Therefore, the current state can be equated
        with the current observation, without taking into account past
        observations and actions.
    \item Because the observation space of the Atari gameplay environment is
        much larger than the pole-cart environment, it should suffice to use a
        smaller ANN model.
    \item Because the observation space of the pole-cart environment is small
        and not spatially correlated, it is not helpful to use a convolutional
        neural network.
\end{itemize}
\subsection*{Results}
\begin{centering}
\end{centering}
\begin{itemize}
    \item The results can be summarized as follows:
    \begin{itemize}
        \item The largest learning rate initiates learning quickly
            but fails to converge to an optimal policy, likely because it
            overshoots the mark at each parameter update.
        \item The smallest learning rate learns the policy slowly because of
            its smaller updates but does reach a near-optimal policy.
        \item The middle learning rate finds a near-optimal policy relatively
            quickly.
    \end{itemize}
\end{itemize}
\end{document}
