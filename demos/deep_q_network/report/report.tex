\documentclass[a4paper]{article}
\setlength\parindent{0pt}

\usepackage{pgfplots}
\usepackage{amsthm, amsmath, amssymb, verbatim, enumerate, mathtools, algorithm}
\usepackage{pgf}
\usepackage{hyperref}
\def\labelitemi{--}
\pgfplotsset{compat=newest}

\pagestyle{empty}

\title{Deep Q-Network Demo}
\author{Rishikesh Vaishnav}
\begin{document}
\maketitle
\section*{Basic Implementation}
\subsection*{Code}
Manual Implementation
\begin{itemize}
    \item The code for this project is available at: 
    % TODO
    \url{https://github.com/rish987/Reinforcement-Learning/blob/master/demos/deep_q_network/code/deep_q_network.py}.
\end{itemize}
Keras Implementation
\begin{itemize}
    \item The code for this project is available at: 
    % TODO
    \url{https://github.com/rish987/Reinforcement-Learning/blob/master/demos/deep_q_network/code/deep_q_network_keras.py}.
\end{itemize}
\subsection*{Implementation Details}
\begin{itemize}
    \item Unlike the Atari gameplay environment described by Mnih et. al., the
        pole-cart environment is not perceptually aliased. That is, the current
        observation of the state is theoretically all that is needed to
        determine an optimal value. Therefore, the current state can be equated
        with the current observation, without taking into account past
        observations and actions.
    \item Because the observation space of the Atari gameplay environment is
        much larger than the pole-cart environment, it should suffice to use a
        smaller ANN model.
    \item Because the observation space of the pole-cart environment is small
        and not spatially correlated, it is not helpful to use a convolutional
        neural network.
\end{itemize}
Manual Implementation
\begin{itemize}
    \item The model is a simple vanilla neural network with one hidden layer:
    \begin{itemize}
        \item Let $M$ be the number of nodes in the hidden layer.
        \item Let $K$ be the number of output nodes (i.e., number of actions).
        \item Let $\sigma(x)$ be the sigmoid activation function
            $\frac{1}{1 + e^{-x}}$.
        \item The hidden layer is calculated as:
        \begin{align*}
            Z_{m} &= \sigma(\alpha^{T}_{0m} + \alpha^{T}_{m}x(s)), m = 1, \dots, M
        \end{align*}
        \item The output layer is calculated as:
        \begin{align*}
            \hat{q}(s, a_i; \theta) &= \beta_{0i} + \beta^{T}_{i}Z, i = 1, \dots, K
        \end{align*}
    \end{itemize}
    \item Solving for the gradient of the sample error:
    \begin{align*}
        \nabla_{\theta}J_t(\theta) &= -2(y_t - \hat{q}(s_t, a_t, \theta))
        \nabla_{\theta}\hat{q}(s_t, a_t, \theta)\\
        \frac{d}{d\beta_{jk}}\hat{q}(s_t, a_t, \theta) &=
        \begin{cases}
            0 & k \ne a_t\\
            \begin{cases}
                1 & j = 0\\
                \sigma(\alpha^{T}_{0j} + \alpha^{T}_{j}x(s_t)) & j > 0
            \end{cases} &  k = a_t
        \end{cases}, k = 1, \dots, K\\
        \frac{d}{d\alpha_{im}}\hat{q}(s_t, a_t, \theta) &=
        \beta_{mk}
        \sigma'(\alpha^{T}_{0m} + \alpha^{T}_{m}x(s_t))
        \begin{cases}
            1 & i = 0\\
            x(s_t)_{i} & i > 0
        \end{cases}
    \end{align*}
    \item A decaying $\epsilon$ was used, which started at $1.0$ and decayed to
        a minimum value of $0.01$.
    \item A constant decay rate was used for the learning rate $\alpha$.
Keras Implementation
\begin{itemize}
    \item The size of the model seemed to significantly affect the performance
        of the Keras ANN. I found that a model that yielded good (though
        perhaps not optimal) results contained two hidden layers of 24 nodes
        each, with ReLU activation functions.
    \item Other than replacing the model and outsourcing gradient calculation,
        the general algorithmic framework remained the same.
\end{itemize}
\end{itemize}
\subsection*{Results}
Manual Implementation

\begin{centering}
    \scalebox{0.6}{%% Creator: Matplotlib, PGF backend
%%
%% To include the figure in your LaTeX document, write
%%   \input{<filename>.pgf}
%%
%% Make sure the required packages are loaded in your preamble
%%   \usepackage{pgf}
%%
%% Figures using additional raster images can only be included by \input if
%% they are in the same directory as the main LaTeX file. For loading figures
%% from other directories you can use the `import` package
%%   \usepackage{import}
%% and then include the figures with
%%   \import{<path to file>}{<filename>.pgf}
%%
%% Matplotlib used the following preamble
%%   \usepackage{fontspec}
%%   \setmainfont{DejaVu Serif}
%%   \setsansfont{DejaVu Sans}
%%   \setmonofont{DejaVu Sans Mono}
%%
\begingroup%
\makeatletter%
\begin{pgfpicture}%
\pgfpathrectangle{\pgfpointorigin}{\pgfqpoint{6.400000in}{4.800000in}}%
\pgfusepath{use as bounding box, clip}%
\begin{pgfscope}%
\pgfsetbuttcap%
\pgfsetmiterjoin%
\definecolor{currentfill}{rgb}{1.000000,1.000000,1.000000}%
\pgfsetfillcolor{currentfill}%
\pgfsetlinewidth{0.000000pt}%
\definecolor{currentstroke}{rgb}{1.000000,1.000000,1.000000}%
\pgfsetstrokecolor{currentstroke}%
\pgfsetdash{}{0pt}%
\pgfpathmoveto{\pgfqpoint{0.000000in}{0.000000in}}%
\pgfpathlineto{\pgfqpoint{6.400000in}{0.000000in}}%
\pgfpathlineto{\pgfqpoint{6.400000in}{4.800000in}}%
\pgfpathlineto{\pgfqpoint{0.000000in}{4.800000in}}%
\pgfpathclose%
\pgfusepath{fill}%
\end{pgfscope}%
\begin{pgfscope}%
\pgfsetbuttcap%
\pgfsetmiterjoin%
\definecolor{currentfill}{rgb}{1.000000,1.000000,1.000000}%
\pgfsetfillcolor{currentfill}%
\pgfsetlinewidth{0.000000pt}%
\definecolor{currentstroke}{rgb}{0.000000,0.000000,0.000000}%
\pgfsetstrokecolor{currentstroke}%
\pgfsetstrokeopacity{0.000000}%
\pgfsetdash{}{0pt}%
\pgfpathmoveto{\pgfqpoint{0.800000in}{0.528000in}}%
\pgfpathlineto{\pgfqpoint{5.760000in}{0.528000in}}%
\pgfpathlineto{\pgfqpoint{5.760000in}{4.224000in}}%
\pgfpathlineto{\pgfqpoint{0.800000in}{4.224000in}}%
\pgfpathclose%
\pgfusepath{fill}%
\end{pgfscope}%
\begin{pgfscope}%
\pgfsetbuttcap%
\pgfsetroundjoin%
\definecolor{currentfill}{rgb}{0.000000,0.000000,0.000000}%
\pgfsetfillcolor{currentfill}%
\pgfsetlinewidth{0.803000pt}%
\definecolor{currentstroke}{rgb}{0.000000,0.000000,0.000000}%
\pgfsetstrokecolor{currentstroke}%
\pgfsetdash{}{0pt}%
\pgfsys@defobject{currentmarker}{\pgfqpoint{0.000000in}{-0.048611in}}{\pgfqpoint{0.000000in}{0.000000in}}{%
\pgfpathmoveto{\pgfqpoint{0.000000in}{0.000000in}}%
\pgfpathlineto{\pgfqpoint{0.000000in}{-0.048611in}}%
\pgfusepath{stroke,fill}%
}%
\begin{pgfscope}%
\pgfsys@transformshift{1.002796in}{0.528000in}%
\pgfsys@useobject{currentmarker}{}%
\end{pgfscope}%
\end{pgfscope}%
\begin{pgfscope}%
\pgftext[x=1.002796in,y=0.430778in,,top]{\sffamily\fontsize{10.000000}{12.000000}\selectfont \(\displaystyle 0\)}%
\end{pgfscope}%
\begin{pgfscope}%
\pgfsetbuttcap%
\pgfsetroundjoin%
\definecolor{currentfill}{rgb}{0.000000,0.000000,0.000000}%
\pgfsetfillcolor{currentfill}%
\pgfsetlinewidth{0.803000pt}%
\definecolor{currentstroke}{rgb}{0.000000,0.000000,0.000000}%
\pgfsetstrokecolor{currentstroke}%
\pgfsetdash{}{0pt}%
\pgfsys@defobject{currentmarker}{\pgfqpoint{0.000000in}{-0.048611in}}{\pgfqpoint{0.000000in}{0.000000in}}{%
\pgfpathmoveto{\pgfqpoint{0.000000in}{0.000000in}}%
\pgfpathlineto{\pgfqpoint{0.000000in}{-0.048611in}}%
\pgfusepath{stroke,fill}%
}%
\begin{pgfscope}%
\pgfsys@transformshift{1.569265in}{0.528000in}%
\pgfsys@useobject{currentmarker}{}%
\end{pgfscope}%
\end{pgfscope}%
\begin{pgfscope}%
\pgftext[x=1.569265in,y=0.430778in,,top]{\sffamily\fontsize{10.000000}{12.000000}\selectfont \(\displaystyle 25\)}%
\end{pgfscope}%
\begin{pgfscope}%
\pgfsetbuttcap%
\pgfsetroundjoin%
\definecolor{currentfill}{rgb}{0.000000,0.000000,0.000000}%
\pgfsetfillcolor{currentfill}%
\pgfsetlinewidth{0.803000pt}%
\definecolor{currentstroke}{rgb}{0.000000,0.000000,0.000000}%
\pgfsetstrokecolor{currentstroke}%
\pgfsetdash{}{0pt}%
\pgfsys@defobject{currentmarker}{\pgfqpoint{0.000000in}{-0.048611in}}{\pgfqpoint{0.000000in}{0.000000in}}{%
\pgfpathmoveto{\pgfqpoint{0.000000in}{0.000000in}}%
\pgfpathlineto{\pgfqpoint{0.000000in}{-0.048611in}}%
\pgfusepath{stroke,fill}%
}%
\begin{pgfscope}%
\pgfsys@transformshift{2.135733in}{0.528000in}%
\pgfsys@useobject{currentmarker}{}%
\end{pgfscope}%
\end{pgfscope}%
\begin{pgfscope}%
\pgftext[x=2.135733in,y=0.430778in,,top]{\sffamily\fontsize{10.000000}{12.000000}\selectfont \(\displaystyle 50\)}%
\end{pgfscope}%
\begin{pgfscope}%
\pgfsetbuttcap%
\pgfsetroundjoin%
\definecolor{currentfill}{rgb}{0.000000,0.000000,0.000000}%
\pgfsetfillcolor{currentfill}%
\pgfsetlinewidth{0.803000pt}%
\definecolor{currentstroke}{rgb}{0.000000,0.000000,0.000000}%
\pgfsetstrokecolor{currentstroke}%
\pgfsetdash{}{0pt}%
\pgfsys@defobject{currentmarker}{\pgfqpoint{0.000000in}{-0.048611in}}{\pgfqpoint{0.000000in}{0.000000in}}{%
\pgfpathmoveto{\pgfqpoint{0.000000in}{0.000000in}}%
\pgfpathlineto{\pgfqpoint{0.000000in}{-0.048611in}}%
\pgfusepath{stroke,fill}%
}%
\begin{pgfscope}%
\pgfsys@transformshift{2.702202in}{0.528000in}%
\pgfsys@useobject{currentmarker}{}%
\end{pgfscope}%
\end{pgfscope}%
\begin{pgfscope}%
\pgftext[x=2.702202in,y=0.430778in,,top]{\sffamily\fontsize{10.000000}{12.000000}\selectfont \(\displaystyle 75\)}%
\end{pgfscope}%
\begin{pgfscope}%
\pgfsetbuttcap%
\pgfsetroundjoin%
\definecolor{currentfill}{rgb}{0.000000,0.000000,0.000000}%
\pgfsetfillcolor{currentfill}%
\pgfsetlinewidth{0.803000pt}%
\definecolor{currentstroke}{rgb}{0.000000,0.000000,0.000000}%
\pgfsetstrokecolor{currentstroke}%
\pgfsetdash{}{0pt}%
\pgfsys@defobject{currentmarker}{\pgfqpoint{0.000000in}{-0.048611in}}{\pgfqpoint{0.000000in}{0.000000in}}{%
\pgfpathmoveto{\pgfqpoint{0.000000in}{0.000000in}}%
\pgfpathlineto{\pgfqpoint{0.000000in}{-0.048611in}}%
\pgfusepath{stroke,fill}%
}%
\begin{pgfscope}%
\pgfsys@transformshift{3.268671in}{0.528000in}%
\pgfsys@useobject{currentmarker}{}%
\end{pgfscope}%
\end{pgfscope}%
\begin{pgfscope}%
\pgftext[x=3.268671in,y=0.430778in,,top]{\sffamily\fontsize{10.000000}{12.000000}\selectfont \(\displaystyle 100\)}%
\end{pgfscope}%
\begin{pgfscope}%
\pgfsetbuttcap%
\pgfsetroundjoin%
\definecolor{currentfill}{rgb}{0.000000,0.000000,0.000000}%
\pgfsetfillcolor{currentfill}%
\pgfsetlinewidth{0.803000pt}%
\definecolor{currentstroke}{rgb}{0.000000,0.000000,0.000000}%
\pgfsetstrokecolor{currentstroke}%
\pgfsetdash{}{0pt}%
\pgfsys@defobject{currentmarker}{\pgfqpoint{0.000000in}{-0.048611in}}{\pgfqpoint{0.000000in}{0.000000in}}{%
\pgfpathmoveto{\pgfqpoint{0.000000in}{0.000000in}}%
\pgfpathlineto{\pgfqpoint{0.000000in}{-0.048611in}}%
\pgfusepath{stroke,fill}%
}%
\begin{pgfscope}%
\pgfsys@transformshift{3.835139in}{0.528000in}%
\pgfsys@useobject{currentmarker}{}%
\end{pgfscope}%
\end{pgfscope}%
\begin{pgfscope}%
\pgftext[x=3.835139in,y=0.430778in,,top]{\sffamily\fontsize{10.000000}{12.000000}\selectfont \(\displaystyle 125\)}%
\end{pgfscope}%
\begin{pgfscope}%
\pgfsetbuttcap%
\pgfsetroundjoin%
\definecolor{currentfill}{rgb}{0.000000,0.000000,0.000000}%
\pgfsetfillcolor{currentfill}%
\pgfsetlinewidth{0.803000pt}%
\definecolor{currentstroke}{rgb}{0.000000,0.000000,0.000000}%
\pgfsetstrokecolor{currentstroke}%
\pgfsetdash{}{0pt}%
\pgfsys@defobject{currentmarker}{\pgfqpoint{0.000000in}{-0.048611in}}{\pgfqpoint{0.000000in}{0.000000in}}{%
\pgfpathmoveto{\pgfqpoint{0.000000in}{0.000000in}}%
\pgfpathlineto{\pgfqpoint{0.000000in}{-0.048611in}}%
\pgfusepath{stroke,fill}%
}%
\begin{pgfscope}%
\pgfsys@transformshift{4.401608in}{0.528000in}%
\pgfsys@useobject{currentmarker}{}%
\end{pgfscope}%
\end{pgfscope}%
\begin{pgfscope}%
\pgftext[x=4.401608in,y=0.430778in,,top]{\sffamily\fontsize{10.000000}{12.000000}\selectfont \(\displaystyle 150\)}%
\end{pgfscope}%
\begin{pgfscope}%
\pgfsetbuttcap%
\pgfsetroundjoin%
\definecolor{currentfill}{rgb}{0.000000,0.000000,0.000000}%
\pgfsetfillcolor{currentfill}%
\pgfsetlinewidth{0.803000pt}%
\definecolor{currentstroke}{rgb}{0.000000,0.000000,0.000000}%
\pgfsetstrokecolor{currentstroke}%
\pgfsetdash{}{0pt}%
\pgfsys@defobject{currentmarker}{\pgfqpoint{0.000000in}{-0.048611in}}{\pgfqpoint{0.000000in}{0.000000in}}{%
\pgfpathmoveto{\pgfqpoint{0.000000in}{0.000000in}}%
\pgfpathlineto{\pgfqpoint{0.000000in}{-0.048611in}}%
\pgfusepath{stroke,fill}%
}%
\begin{pgfscope}%
\pgfsys@transformshift{4.968077in}{0.528000in}%
\pgfsys@useobject{currentmarker}{}%
\end{pgfscope}%
\end{pgfscope}%
\begin{pgfscope}%
\pgftext[x=4.968077in,y=0.430778in,,top]{\sffamily\fontsize{10.000000}{12.000000}\selectfont \(\displaystyle 175\)}%
\end{pgfscope}%
\begin{pgfscope}%
\pgfsetbuttcap%
\pgfsetroundjoin%
\definecolor{currentfill}{rgb}{0.000000,0.000000,0.000000}%
\pgfsetfillcolor{currentfill}%
\pgfsetlinewidth{0.803000pt}%
\definecolor{currentstroke}{rgb}{0.000000,0.000000,0.000000}%
\pgfsetstrokecolor{currentstroke}%
\pgfsetdash{}{0pt}%
\pgfsys@defobject{currentmarker}{\pgfqpoint{0.000000in}{-0.048611in}}{\pgfqpoint{0.000000in}{0.000000in}}{%
\pgfpathmoveto{\pgfqpoint{0.000000in}{0.000000in}}%
\pgfpathlineto{\pgfqpoint{0.000000in}{-0.048611in}}%
\pgfusepath{stroke,fill}%
}%
\begin{pgfscope}%
\pgfsys@transformshift{5.534545in}{0.528000in}%
\pgfsys@useobject{currentmarker}{}%
\end{pgfscope}%
\end{pgfscope}%
\begin{pgfscope}%
\pgftext[x=5.534545in,y=0.430778in,,top]{\sffamily\fontsize{10.000000}{12.000000}\selectfont \(\displaystyle 200\)}%
\end{pgfscope}%
\begin{pgfscope}%
\pgftext[x=3.280000in,y=0.240809in,,top]{\sffamily\fontsize{10.000000}{12.000000}\selectfont Episode}%
\end{pgfscope}%
\begin{pgfscope}%
\pgfsetbuttcap%
\pgfsetroundjoin%
\definecolor{currentfill}{rgb}{0.000000,0.000000,0.000000}%
\pgfsetfillcolor{currentfill}%
\pgfsetlinewidth{0.803000pt}%
\definecolor{currentstroke}{rgb}{0.000000,0.000000,0.000000}%
\pgfsetstrokecolor{currentstroke}%
\pgfsetdash{}{0pt}%
\pgfsys@defobject{currentmarker}{\pgfqpoint{-0.048611in}{0.000000in}}{\pgfqpoint{0.000000in}{0.000000in}}{%
\pgfpathmoveto{\pgfqpoint{0.000000in}{0.000000in}}%
\pgfpathlineto{\pgfqpoint{-0.048611in}{0.000000in}}%
\pgfusepath{stroke,fill}%
}%
\begin{pgfscope}%
\pgfsys@transformshift{0.800000in}{1.050350in}%
\pgfsys@useobject{currentmarker}{}%
\end{pgfscope}%
\end{pgfscope}%
\begin{pgfscope}%
\pgftext[x=0.563888in,y=0.997589in,left,base]{\sffamily\fontsize{10.000000}{12.000000}\selectfont \(\displaystyle 20\)}%
\end{pgfscope}%
\begin{pgfscope}%
\pgfsetbuttcap%
\pgfsetroundjoin%
\definecolor{currentfill}{rgb}{0.000000,0.000000,0.000000}%
\pgfsetfillcolor{currentfill}%
\pgfsetlinewidth{0.803000pt}%
\definecolor{currentstroke}{rgb}{0.000000,0.000000,0.000000}%
\pgfsetstrokecolor{currentstroke}%
\pgfsetdash{}{0pt}%
\pgfsys@defobject{currentmarker}{\pgfqpoint{-0.048611in}{0.000000in}}{\pgfqpoint{0.000000in}{0.000000in}}{%
\pgfpathmoveto{\pgfqpoint{0.000000in}{0.000000in}}%
\pgfpathlineto{\pgfqpoint{-0.048611in}{0.000000in}}%
\pgfusepath{stroke,fill}%
}%
\begin{pgfscope}%
\pgfsys@transformshift{0.800000in}{1.683119in}%
\pgfsys@useobject{currentmarker}{}%
\end{pgfscope}%
\end{pgfscope}%
\begin{pgfscope}%
\pgftext[x=0.563888in,y=1.630357in,left,base]{\sffamily\fontsize{10.000000}{12.000000}\selectfont \(\displaystyle 40\)}%
\end{pgfscope}%
\begin{pgfscope}%
\pgfsetbuttcap%
\pgfsetroundjoin%
\definecolor{currentfill}{rgb}{0.000000,0.000000,0.000000}%
\pgfsetfillcolor{currentfill}%
\pgfsetlinewidth{0.803000pt}%
\definecolor{currentstroke}{rgb}{0.000000,0.000000,0.000000}%
\pgfsetstrokecolor{currentstroke}%
\pgfsetdash{}{0pt}%
\pgfsys@defobject{currentmarker}{\pgfqpoint{-0.048611in}{0.000000in}}{\pgfqpoint{0.000000in}{0.000000in}}{%
\pgfpathmoveto{\pgfqpoint{0.000000in}{0.000000in}}%
\pgfpathlineto{\pgfqpoint{-0.048611in}{0.000000in}}%
\pgfusepath{stroke,fill}%
}%
\begin{pgfscope}%
\pgfsys@transformshift{0.800000in}{2.315887in}%
\pgfsys@useobject{currentmarker}{}%
\end{pgfscope}%
\end{pgfscope}%
\begin{pgfscope}%
\pgftext[x=0.563888in,y=2.263125in,left,base]{\sffamily\fontsize{10.000000}{12.000000}\selectfont \(\displaystyle 60\)}%
\end{pgfscope}%
\begin{pgfscope}%
\pgfsetbuttcap%
\pgfsetroundjoin%
\definecolor{currentfill}{rgb}{0.000000,0.000000,0.000000}%
\pgfsetfillcolor{currentfill}%
\pgfsetlinewidth{0.803000pt}%
\definecolor{currentstroke}{rgb}{0.000000,0.000000,0.000000}%
\pgfsetstrokecolor{currentstroke}%
\pgfsetdash{}{0pt}%
\pgfsys@defobject{currentmarker}{\pgfqpoint{-0.048611in}{0.000000in}}{\pgfqpoint{0.000000in}{0.000000in}}{%
\pgfpathmoveto{\pgfqpoint{0.000000in}{0.000000in}}%
\pgfpathlineto{\pgfqpoint{-0.048611in}{0.000000in}}%
\pgfusepath{stroke,fill}%
}%
\begin{pgfscope}%
\pgfsys@transformshift{0.800000in}{2.948655in}%
\pgfsys@useobject{currentmarker}{}%
\end{pgfscope}%
\end{pgfscope}%
\begin{pgfscope}%
\pgftext[x=0.563888in,y=2.895894in,left,base]{\sffamily\fontsize{10.000000}{12.000000}\selectfont \(\displaystyle 80\)}%
\end{pgfscope}%
\begin{pgfscope}%
\pgfsetbuttcap%
\pgfsetroundjoin%
\definecolor{currentfill}{rgb}{0.000000,0.000000,0.000000}%
\pgfsetfillcolor{currentfill}%
\pgfsetlinewidth{0.803000pt}%
\definecolor{currentstroke}{rgb}{0.000000,0.000000,0.000000}%
\pgfsetstrokecolor{currentstroke}%
\pgfsetdash{}{0pt}%
\pgfsys@defobject{currentmarker}{\pgfqpoint{-0.048611in}{0.000000in}}{\pgfqpoint{0.000000in}{0.000000in}}{%
\pgfpathmoveto{\pgfqpoint{0.000000in}{0.000000in}}%
\pgfpathlineto{\pgfqpoint{-0.048611in}{0.000000in}}%
\pgfusepath{stroke,fill}%
}%
\begin{pgfscope}%
\pgfsys@transformshift{0.800000in}{3.581424in}%
\pgfsys@useobject{currentmarker}{}%
\end{pgfscope}%
\end{pgfscope}%
\begin{pgfscope}%
\pgftext[x=0.494444in,y=3.528662in,left,base]{\sffamily\fontsize{10.000000}{12.000000}\selectfont \(\displaystyle 100\)}%
\end{pgfscope}%
\begin{pgfscope}%
\pgfsetbuttcap%
\pgfsetroundjoin%
\definecolor{currentfill}{rgb}{0.000000,0.000000,0.000000}%
\pgfsetfillcolor{currentfill}%
\pgfsetlinewidth{0.803000pt}%
\definecolor{currentstroke}{rgb}{0.000000,0.000000,0.000000}%
\pgfsetstrokecolor{currentstroke}%
\pgfsetdash{}{0pt}%
\pgfsys@defobject{currentmarker}{\pgfqpoint{-0.048611in}{0.000000in}}{\pgfqpoint{0.000000in}{0.000000in}}{%
\pgfpathmoveto{\pgfqpoint{0.000000in}{0.000000in}}%
\pgfpathlineto{\pgfqpoint{-0.048611in}{0.000000in}}%
\pgfusepath{stroke,fill}%
}%
\begin{pgfscope}%
\pgfsys@transformshift{0.800000in}{4.214192in}%
\pgfsys@useobject{currentmarker}{}%
\end{pgfscope}%
\end{pgfscope}%
\begin{pgfscope}%
\pgftext[x=0.494444in,y=4.161431in,left,base]{\sffamily\fontsize{10.000000}{12.000000}\selectfont \(\displaystyle 120\)}%
\end{pgfscope}%
\begin{pgfscope}%
\pgftext[x=0.438888in,y=2.376000in,,bottom,rotate=90.000000]{\sffamily\fontsize{10.000000}{12.000000}\selectfont Average Target Policy Episode Length (10 Runs)}%
\end{pgfscope}%
\begin{pgfscope}%
\pgfpathrectangle{\pgfqpoint{0.800000in}{0.528000in}}{\pgfqpoint{4.960000in}{3.696000in}} %
\pgfusepath{clip}%
\pgfsetrectcap%
\pgfsetroundjoin%
\pgfsetlinewidth{1.505625pt}%
\definecolor{currentstroke}{rgb}{0.000000,0.000000,0.000000}%
\pgfsetstrokecolor{currentstroke}%
\pgfsetdash{}{0pt}%
\pgfpathmoveto{\pgfqpoint{1.025455in}{0.724475in}}%
\pgfpathlineto{\pgfqpoint{1.048113in}{0.733966in}}%
\pgfpathlineto{\pgfqpoint{1.070772in}{0.759277in}}%
\pgfpathlineto{\pgfqpoint{1.093431in}{0.759277in}}%
\pgfpathlineto{\pgfqpoint{1.116090in}{0.714983in}}%
\pgfpathlineto{\pgfqpoint{1.138748in}{0.730802in}}%
\pgfpathlineto{\pgfqpoint{1.161407in}{0.711819in}}%
\pgfpathlineto{\pgfqpoint{1.184066in}{0.955435in}}%
\pgfpathlineto{\pgfqpoint{1.206725in}{0.737130in}}%
\pgfpathlineto{\pgfqpoint{1.229383in}{0.721311in}}%
\pgfpathlineto{\pgfqpoint{1.252042in}{0.851028in}}%
\pgfpathlineto{\pgfqpoint{1.274701in}{0.705492in}}%
\pgfpathlineto{\pgfqpoint{1.297360in}{0.752949in}}%
\pgfpathlineto{\pgfqpoint{1.320018in}{0.838373in}}%
\pgfpathlineto{\pgfqpoint{1.342677in}{0.711819in}}%
\pgfpathlineto{\pgfqpoint{1.365336in}{0.721311in}}%
\pgfpathlineto{\pgfqpoint{1.410653in}{0.803571in}}%
\pgfpathlineto{\pgfqpoint{1.433312in}{1.107299in}}%
\pgfpathlineto{\pgfqpoint{1.455971in}{0.822554in}}%
\pgfpathlineto{\pgfqpoint{1.478630in}{0.730802in}}%
\pgfpathlineto{\pgfqpoint{1.501288in}{0.895322in}}%
\pgfpathlineto{\pgfqpoint{1.523947in}{0.800407in}}%
\pgfpathlineto{\pgfqpoint{1.546606in}{0.952271in}}%
\pgfpathlineto{\pgfqpoint{1.569265in}{0.803571in}}%
\pgfpathlineto{\pgfqpoint{1.591923in}{0.838373in}}%
\pgfpathlineto{\pgfqpoint{1.614582in}{0.822554in}}%
\pgfpathlineto{\pgfqpoint{1.637241in}{1.063006in}}%
\pgfpathlineto{\pgfqpoint{1.659899in}{0.797243in}}%
\pgfpathlineto{\pgfqpoint{1.682558in}{0.930124in}}%
\pgfpathlineto{\pgfqpoint{1.705217in}{0.844701in}}%
\pgfpathlineto{\pgfqpoint{1.727876in}{0.901650in}}%
\pgfpathlineto{\pgfqpoint{1.750534in}{1.091480in}}%
\pgfpathlineto{\pgfqpoint{1.773193in}{0.996565in}}%
\pgfpathlineto{\pgfqpoint{1.795852in}{1.091480in}}%
\pgfpathlineto{\pgfqpoint{1.818511in}{1.107299in}}%
\pgfpathlineto{\pgfqpoint{1.841169in}{1.205379in}}%
\pgfpathlineto{\pgfqpoint{1.863828in}{0.936452in}}%
\pgfpathlineto{\pgfqpoint{1.886487in}{0.999729in}}%
\pgfpathlineto{\pgfqpoint{1.909146in}{0.971254in}}%
\pgfpathlineto{\pgfqpoint{1.931804in}{1.116791in}}%
\pgfpathlineto{\pgfqpoint{1.954463in}{0.851028in}}%
\pgfpathlineto{\pgfqpoint{1.977122in}{1.157921in}}%
\pgfpathlineto{\pgfqpoint{1.999781in}{1.107299in}}%
\pgfpathlineto{\pgfqpoint{2.022439in}{0.930124in}}%
\pgfpathlineto{\pgfqpoint{2.045098in}{1.047186in}}%
\pgfpathlineto{\pgfqpoint{2.090416in}{1.259164in}}%
\pgfpathlineto{\pgfqpoint{2.113074in}{1.322441in}}%
\pgfpathlineto{\pgfqpoint{2.135733in}{0.917469in}}%
\pgfpathlineto{\pgfqpoint{2.158392in}{1.426847in}}%
\pgfpathlineto{\pgfqpoint{2.181051in}{1.259164in}}%
\pgfpathlineto{\pgfqpoint{2.203709in}{1.069333in}}%
\pgfpathlineto{\pgfqpoint{2.226368in}{1.490124in}}%
\pgfpathlineto{\pgfqpoint{2.249027in}{2.012158in}}%
\pgfpathlineto{\pgfqpoint{2.271686in}{1.505944in}}%
\pgfpathlineto{\pgfqpoint{2.294344in}{1.844475in}}%
\pgfpathlineto{\pgfqpoint{2.317003in}{1.499616in}}%
\pgfpathlineto{\pgfqpoint{2.339662in}{1.708429in}}%
\pgfpathlineto{\pgfqpoint{2.362321in}{1.616678in}}%
\pgfpathlineto{\pgfqpoint{2.384979in}{1.388881in}}%
\pgfpathlineto{\pgfqpoint{2.407638in}{2.891706in}}%
\pgfpathlineto{\pgfqpoint{2.430297in}{2.948655in}}%
\pgfpathlineto{\pgfqpoint{2.452956in}{2.426621in}}%
\pgfpathlineto{\pgfqpoint{2.475614in}{2.312723in}}%
\pgfpathlineto{\pgfqpoint{2.498273in}{1.591367in}}%
\pgfpathlineto{\pgfqpoint{2.520932in}{1.812836in}}%
\pgfpathlineto{\pgfqpoint{2.543591in}{1.787525in}}%
\pgfpathlineto{\pgfqpoint{2.566249in}{1.119955in}}%
\pgfpathlineto{\pgfqpoint{2.588908in}{1.483797in}}%
\pgfpathlineto{\pgfqpoint{2.611567in}{2.097582in}}%
\pgfpathlineto{\pgfqpoint{2.634226in}{3.730124in}}%
\pgfpathlineto{\pgfqpoint{2.656884in}{2.860068in}}%
\pgfpathlineto{\pgfqpoint{2.679543in}{1.806508in}}%
\pgfpathlineto{\pgfqpoint{2.702202in}{2.451932in}}%
\pgfpathlineto{\pgfqpoint{2.724861in}{2.493062in}}%
\pgfpathlineto{\pgfqpoint{2.747519in}{2.518373in}}%
\pgfpathlineto{\pgfqpoint{2.770178in}{2.341198in}}%
\pgfpathlineto{\pgfqpoint{2.792837in}{2.486734in}}%
\pgfpathlineto{\pgfqpoint{2.815496in}{3.306169in}}%
\pgfpathlineto{\pgfqpoint{2.838154in}{2.262102in}}%
\pgfpathlineto{\pgfqpoint{2.860813in}{3.214418in}}%
\pgfpathlineto{\pgfqpoint{2.883472in}{2.394983in}}%
\pgfpathlineto{\pgfqpoint{2.906131in}{2.097582in}}%
\pgfpathlineto{\pgfqpoint{2.928789in}{2.765153in}}%
\pgfpathlineto{\pgfqpoint{2.951448in}{1.920407in}}%
\pgfpathlineto{\pgfqpoint{2.974107in}{1.841311in}}%
\pgfpathlineto{\pgfqpoint{2.996766in}{1.382554in}}%
\pgfpathlineto{\pgfqpoint{3.019424in}{1.436339in}}%
\pgfpathlineto{\pgfqpoint{3.042083in}{1.591367in}}%
\pgfpathlineto{\pgfqpoint{3.064742in}{1.806508in}}%
\pgfpathlineto{\pgfqpoint{3.087401in}{1.411028in}}%
\pgfpathlineto{\pgfqpoint{3.110059in}{2.281085in}}%
\pgfpathlineto{\pgfqpoint{3.132718in}{2.394983in}}%
\pgfpathlineto{\pgfqpoint{3.155377in}{2.005831in}}%
\pgfpathlineto{\pgfqpoint{3.178036in}{1.357243in}}%
\pgfpathlineto{\pgfqpoint{3.200694in}{0.822554in}}%
\pgfpathlineto{\pgfqpoint{3.223353in}{2.347525in}}%
\pgfpathlineto{\pgfqpoint{3.246012in}{2.122893in}}%
\pgfpathlineto{\pgfqpoint{3.268671in}{2.818938in}}%
\pgfpathlineto{\pgfqpoint{3.291329in}{2.474079in}}%
\pgfpathlineto{\pgfqpoint{3.313988in}{2.148203in}}%
\pgfpathlineto{\pgfqpoint{3.336647in}{3.182780in}}%
\pgfpathlineto{\pgfqpoint{3.359306in}{1.692610in}}%
\pgfpathlineto{\pgfqpoint{3.381964in}{1.572384in}}%
\pgfpathlineto{\pgfqpoint{3.404623in}{2.195661in}}%
\pgfpathlineto{\pgfqpoint{3.427282in}{2.065944in}}%
\pgfpathlineto{\pgfqpoint{3.449941in}{2.173514in}}%
\pgfpathlineto{\pgfqpoint{3.472599in}{1.430011in}}%
\pgfpathlineto{\pgfqpoint{3.495258in}{1.683119in}}%
\pgfpathlineto{\pgfqpoint{3.517917in}{1.445831in}}%
\pgfpathlineto{\pgfqpoint{3.540576in}{2.274757in}}%
\pgfpathlineto{\pgfqpoint{3.563234in}{2.075435in}}%
\pgfpathlineto{\pgfqpoint{3.585893in}{1.610350in}}%
\pgfpathlineto{\pgfqpoint{3.608552in}{2.448768in}}%
\pgfpathlineto{\pgfqpoint{3.631211in}{2.758825in}}%
\pgfpathlineto{\pgfqpoint{3.653869in}{2.084927in}}%
\pgfpathlineto{\pgfqpoint{3.676528in}{1.702102in}}%
\pgfpathlineto{\pgfqpoint{3.699187in}{2.366508in}}%
\pgfpathlineto{\pgfqpoint{3.721846in}{2.189333in}}%
\pgfpathlineto{\pgfqpoint{3.744504in}{1.455322in}}%
\pgfpathlineto{\pgfqpoint{3.767163in}{1.806508in}}%
\pgfpathlineto{\pgfqpoint{3.789822in}{2.385492in}}%
\pgfpathlineto{\pgfqpoint{3.812481in}{2.834757in}}%
\pgfpathlineto{\pgfqpoint{3.835139in}{1.727412in}}%
\pgfpathlineto{\pgfqpoint{3.857798in}{2.037469in}}%
\pgfpathlineto{\pgfqpoint{3.880457in}{1.518599in}}%
\pgfpathlineto{\pgfqpoint{3.903116in}{1.047186in}}%
\pgfpathlineto{\pgfqpoint{3.925774in}{1.673627in}}%
\pgfpathlineto{\pgfqpoint{3.948433in}{1.850802in}}%
\pgfpathlineto{\pgfqpoint{3.971092in}{1.882441in}}%
\pgfpathlineto{\pgfqpoint{3.993751in}{1.812836in}}%
\pgfpathlineto{\pgfqpoint{4.016409in}{1.955209in}}%
\pgfpathlineto{\pgfqpoint{4.039068in}{2.148203in}}%
\pgfpathlineto{\pgfqpoint{4.061727in}{2.353853in}}%
\pgfpathlineto{\pgfqpoint{4.084386in}{1.907751in}}%
\pgfpathlineto{\pgfqpoint{4.107044in}{1.616678in}}%
\pgfpathlineto{\pgfqpoint{4.129703in}{1.664136in}}%
\pgfpathlineto{\pgfqpoint{4.152362in}{2.167186in}}%
\pgfpathlineto{\pgfqpoint{4.175021in}{2.233627in}}%
\pgfpathlineto{\pgfqpoint{4.197679in}{1.999503in}}%
\pgfpathlineto{\pgfqpoint{4.220338in}{1.872949in}}%
\pgfpathlineto{\pgfqpoint{4.242997in}{2.572158in}}%
\pgfpathlineto{\pgfqpoint{4.265656in}{1.996339in}}%
\pgfpathlineto{\pgfqpoint{4.288314in}{1.556565in}}%
\pgfpathlineto{\pgfqpoint{4.310973in}{1.499616in}}%
\pgfpathlineto{\pgfqpoint{4.333632in}{1.869785in}}%
\pgfpathlineto{\pgfqpoint{4.356291in}{2.300068in}}%
\pgfpathlineto{\pgfqpoint{4.378949in}{2.591141in}}%
\pgfpathlineto{\pgfqpoint{4.401608in}{2.363345in}}%
\pgfpathlineto{\pgfqpoint{4.424267in}{1.398373in}}%
\pgfpathlineto{\pgfqpoint{4.446926in}{2.372836in}}%
\pgfpathlineto{\pgfqpoint{4.469584in}{2.198825in}}%
\pgfpathlineto{\pgfqpoint{4.492243in}{2.050124in}}%
\pgfpathlineto{\pgfqpoint{4.514902in}{1.274983in}}%
\pgfpathlineto{\pgfqpoint{4.537561in}{1.512271in}}%
\pgfpathlineto{\pgfqpoint{4.560219in}{1.205379in}}%
\pgfpathlineto{\pgfqpoint{4.582878in}{1.515435in}}%
\pgfpathlineto{\pgfqpoint{4.605537in}{1.882441in}}%
\pgfpathlineto{\pgfqpoint{4.628196in}{0.860520in}}%
\pgfpathlineto{\pgfqpoint{4.650854in}{2.053288in}}%
\pgfpathlineto{\pgfqpoint{4.673513in}{1.490124in}}%
\pgfpathlineto{\pgfqpoint{4.696172in}{1.708429in}}%
\pgfpathlineto{\pgfqpoint{4.718831in}{1.553401in}}%
\pgfpathlineto{\pgfqpoint{4.741489in}{1.246508in}}%
\pgfpathlineto{\pgfqpoint{4.764148in}{2.151367in}}%
\pgfpathlineto{\pgfqpoint{4.786807in}{1.945718in}}%
\pgfpathlineto{\pgfqpoint{4.809466in}{1.841311in}}%
\pgfpathlineto{\pgfqpoint{4.832124in}{2.151367in}}%
\pgfpathlineto{\pgfqpoint{4.854783in}{1.971028in}}%
\pgfpathlineto{\pgfqpoint{4.877442in}{1.471141in}}%
\pgfpathlineto{\pgfqpoint{4.900101in}{1.306621in}}%
\pgfpathlineto{\pgfqpoint{4.922759in}{2.053288in}}%
\pgfpathlineto{\pgfqpoint{4.945418in}{2.401311in}}%
\pgfpathlineto{\pgfqpoint{4.968077in}{1.850802in}}%
\pgfpathlineto{\pgfqpoint{4.990735in}{1.240181in}}%
\pgfpathlineto{\pgfqpoint{5.013394in}{1.161085in}}%
\pgfpathlineto{\pgfqpoint{5.036053in}{1.331932in}}%
\pgfpathlineto{\pgfqpoint{5.058712in}{1.926734in}}%
\pgfpathlineto{\pgfqpoint{5.081370in}{1.585040in}}%
\pgfpathlineto{\pgfqpoint{5.104029in}{2.758825in}}%
\pgfpathlineto{\pgfqpoint{5.126688in}{2.477243in}}%
\pgfpathlineto{\pgfqpoint{5.149347in}{1.828655in}}%
\pgfpathlineto{\pgfqpoint{5.172005in}{1.132610in}}%
\pgfpathlineto{\pgfqpoint{5.194664in}{2.081763in}}%
\pgfpathlineto{\pgfqpoint{5.217323in}{2.534192in}}%
\pgfpathlineto{\pgfqpoint{5.239982in}{2.667073in}}%
\pgfpathlineto{\pgfqpoint{5.262640in}{1.439503in}}%
\pgfpathlineto{\pgfqpoint{5.285299in}{1.173740in}}%
\pgfpathlineto{\pgfqpoint{5.307958in}{1.186395in}}%
\pgfpathlineto{\pgfqpoint{5.330617in}{1.395209in}}%
\pgfpathlineto{\pgfqpoint{5.353275in}{1.259164in}}%
\pgfpathlineto{\pgfqpoint{5.375934in}{1.398373in}}%
\pgfpathlineto{\pgfqpoint{5.398593in}{1.316113in}}%
\pgfpathlineto{\pgfqpoint{5.421252in}{1.819164in}}%
\pgfpathlineto{\pgfqpoint{5.443910in}{0.996565in}}%
\pgfpathlineto{\pgfqpoint{5.466569in}{1.059842in}}%
\pgfpathlineto{\pgfqpoint{5.489228in}{0.768768in}}%
\pgfpathlineto{\pgfqpoint{5.511887in}{1.531254in}}%
\pgfpathlineto{\pgfqpoint{5.534545in}{0.936452in}}%
\pgfpathlineto{\pgfqpoint{5.534545in}{0.936452in}}%
\pgfusepath{stroke}%
\end{pgfscope}%
\begin{pgfscope}%
\pgfpathrectangle{\pgfqpoint{0.800000in}{0.528000in}}{\pgfqpoint{4.960000in}{3.696000in}} %
\pgfusepath{clip}%
\pgfsetrectcap%
\pgfsetroundjoin%
\pgfsetlinewidth{1.505625pt}%
\definecolor{currentstroke}{rgb}{0.300000,0.300000,0.300000}%
\pgfsetstrokecolor{currentstroke}%
\pgfsetdash{}{0pt}%
\pgfpathmoveto{\pgfqpoint{1.025455in}{1.316113in}}%
\pgfpathlineto{\pgfqpoint{1.048113in}{0.724475in}}%
\pgfpathlineto{\pgfqpoint{1.070772in}{0.743458in}}%
\pgfpathlineto{\pgfqpoint{1.093431in}{0.759277in}}%
\pgfpathlineto{\pgfqpoint{1.116090in}{0.730802in}}%
\pgfpathlineto{\pgfqpoint{1.138748in}{0.885831in}}%
\pgfpathlineto{\pgfqpoint{1.161407in}{0.926960in}}%
\pgfpathlineto{\pgfqpoint{1.184066in}{0.980746in}}%
\pgfpathlineto{\pgfqpoint{1.206725in}{1.183232in}}%
\pgfpathlineto{\pgfqpoint{1.229383in}{1.882441in}}%
\pgfpathlineto{\pgfqpoint{1.252042in}{1.287638in}}%
\pgfpathlineto{\pgfqpoint{1.274701in}{1.100972in}}%
\pgfpathlineto{\pgfqpoint{1.297360in}{1.145266in}}%
\pgfpathlineto{\pgfqpoint{1.320018in}{1.426847in}}%
\pgfpathlineto{\pgfqpoint{1.342677in}{2.486734in}}%
\pgfpathlineto{\pgfqpoint{1.365336in}{1.467977in}}%
\pgfpathlineto{\pgfqpoint{1.387995in}{2.606960in}}%
\pgfpathlineto{\pgfqpoint{1.410653in}{1.493288in}}%
\pgfpathlineto{\pgfqpoint{1.433312in}{1.373062in}}%
\pgfpathlineto{\pgfqpoint{1.455971in}{1.879277in}}%
\pgfpathlineto{\pgfqpoint{1.478630in}{1.183232in}}%
\pgfpathlineto{\pgfqpoint{1.501288in}{2.622780in}}%
\pgfpathlineto{\pgfqpoint{1.523947in}{1.164249in}}%
\pgfpathlineto{\pgfqpoint{1.546606in}{2.841085in}}%
\pgfpathlineto{\pgfqpoint{1.569265in}{1.872949in}}%
\pgfpathlineto{\pgfqpoint{1.591923in}{1.844475in}}%
\pgfpathlineto{\pgfqpoint{1.614582in}{2.160859in}}%
\pgfpathlineto{\pgfqpoint{1.637241in}{2.724023in}}%
\pgfpathlineto{\pgfqpoint{1.659899in}{1.145266in}}%
\pgfpathlineto{\pgfqpoint{1.682558in}{1.341424in}}%
\pgfpathlineto{\pgfqpoint{1.705217in}{2.451932in}}%
\pgfpathlineto{\pgfqpoint{1.727876in}{2.189333in}}%
\pgfpathlineto{\pgfqpoint{1.750534in}{1.619842in}}%
\pgfpathlineto{\pgfqpoint{1.773193in}{1.834983in}}%
\pgfpathlineto{\pgfqpoint{1.795852in}{1.891932in}}%
\pgfpathlineto{\pgfqpoint{1.818511in}{2.790463in}}%
\pgfpathlineto{\pgfqpoint{1.841169in}{1.825492in}}%
\pgfpathlineto{\pgfqpoint{1.863828in}{1.439503in}}%
\pgfpathlineto{\pgfqpoint{1.886487in}{2.353853in}}%
\pgfpathlineto{\pgfqpoint{1.909146in}{3.106847in}}%
\pgfpathlineto{\pgfqpoint{1.931804in}{1.904588in}}%
\pgfpathlineto{\pgfqpoint{1.954463in}{1.686282in}}%
\pgfpathlineto{\pgfqpoint{1.977122in}{3.571932in}}%
\pgfpathlineto{\pgfqpoint{1.999781in}{4.056000in}}%
\pgfpathlineto{\pgfqpoint{2.022439in}{1.958373in}}%
\pgfpathlineto{\pgfqpoint{2.045098in}{2.252610in}}%
\pgfpathlineto{\pgfqpoint{2.067757in}{2.312723in}}%
\pgfpathlineto{\pgfqpoint{2.090416in}{2.705040in}}%
\pgfpathlineto{\pgfqpoint{2.113074in}{2.031141in}}%
\pgfpathlineto{\pgfqpoint{2.135733in}{1.952045in}}%
\pgfpathlineto{\pgfqpoint{2.158392in}{2.075435in}}%
\pgfpathlineto{\pgfqpoint{2.181051in}{2.331706in}}%
\pgfpathlineto{\pgfqpoint{2.203709in}{2.103910in}}%
\pgfpathlineto{\pgfqpoint{2.226368in}{1.853966in}}%
\pgfpathlineto{\pgfqpoint{2.249027in}{2.442441in}}%
\pgfpathlineto{\pgfqpoint{2.271686in}{2.401311in}}%
\pgfpathlineto{\pgfqpoint{2.294344in}{2.777808in}}%
\pgfpathlineto{\pgfqpoint{2.317003in}{1.882441in}}%
\pgfpathlineto{\pgfqpoint{2.339662in}{2.043797in}}%
\pgfpathlineto{\pgfqpoint{2.362321in}{1.749559in}}%
\pgfpathlineto{\pgfqpoint{2.384979in}{2.031141in}}%
\pgfpathlineto{\pgfqpoint{2.407638in}{1.550237in}}%
\pgfpathlineto{\pgfqpoint{2.430297in}{2.027977in}}%
\pgfpathlineto{\pgfqpoint{2.452956in}{0.949107in}}%
\pgfpathlineto{\pgfqpoint{2.475614in}{0.895322in}}%
\pgfpathlineto{\pgfqpoint{2.498273in}{2.138712in}}%
\pgfpathlineto{\pgfqpoint{2.520932in}{1.025040in}}%
\pgfpathlineto{\pgfqpoint{2.543591in}{1.869785in}}%
\pgfpathlineto{\pgfqpoint{2.566249in}{1.607186in}}%
\pgfpathlineto{\pgfqpoint{2.588908in}{1.328768in}}%
\pgfpathlineto{\pgfqpoint{2.611567in}{1.028203in}}%
\pgfpathlineto{\pgfqpoint{2.634226in}{1.632497in}}%
\pgfpathlineto{\pgfqpoint{2.656884in}{1.676791in}}%
\pgfpathlineto{\pgfqpoint{2.679543in}{2.021650in}}%
\pgfpathlineto{\pgfqpoint{2.702202in}{1.518599in}}%
\pgfpathlineto{\pgfqpoint{2.724861in}{2.315887in}}%
\pgfpathlineto{\pgfqpoint{2.747519in}{1.290802in}}%
\pgfpathlineto{\pgfqpoint{2.770178in}{1.028203in}}%
\pgfpathlineto{\pgfqpoint{2.792837in}{1.518599in}}%
\pgfpathlineto{\pgfqpoint{2.815496in}{1.559729in}}%
\pgfpathlineto{\pgfqpoint{2.838154in}{1.518599in}}%
\pgfpathlineto{\pgfqpoint{2.860813in}{1.186395in}}%
\pgfpathlineto{\pgfqpoint{2.883472in}{1.173740in}}%
\pgfpathlineto{\pgfqpoint{2.906131in}{1.227525in}}%
\pgfpathlineto{\pgfqpoint{2.951448in}{0.727638in}}%
\pgfpathlineto{\pgfqpoint{2.974107in}{1.161085in}}%
\pgfpathlineto{\pgfqpoint{2.996766in}{1.335096in}}%
\pgfpathlineto{\pgfqpoint{3.019424in}{0.762441in}}%
\pgfpathlineto{\pgfqpoint{3.042083in}{2.183006in}}%
\pgfpathlineto{\pgfqpoint{3.064742in}{1.360407in}}%
\pgfpathlineto{\pgfqpoint{3.087401in}{1.341424in}}%
\pgfpathlineto{\pgfqpoint{3.110059in}{0.813062in}}%
\pgfpathlineto{\pgfqpoint{3.132718in}{0.787751in}}%
\pgfpathlineto{\pgfqpoint{3.155377in}{1.123119in}}%
\pgfpathlineto{\pgfqpoint{3.178036in}{0.800407in}}%
\pgfpathlineto{\pgfqpoint{3.200694in}{1.581876in}}%
\pgfpathlineto{\pgfqpoint{3.223353in}{0.914305in}}%
\pgfpathlineto{\pgfqpoint{3.246012in}{1.246508in}}%
\pgfpathlineto{\pgfqpoint{3.268671in}{1.338260in}}%
\pgfpathlineto{\pgfqpoint{3.291329in}{1.366734in}}%
\pgfpathlineto{\pgfqpoint{3.313988in}{1.034531in}}%
\pgfpathlineto{\pgfqpoint{3.336647in}{1.119955in}}%
\pgfpathlineto{\pgfqpoint{3.359306in}{0.983910in}}%
\pgfpathlineto{\pgfqpoint{3.381964in}{1.338260in}}%
\pgfpathlineto{\pgfqpoint{3.404623in}{1.170576in}}%
\pgfpathlineto{\pgfqpoint{3.427282in}{1.091480in}}%
\pgfpathlineto{\pgfqpoint{3.449941in}{1.442667in}}%
\pgfpathlineto{\pgfqpoint{3.472599in}{1.857130in}}%
\pgfpathlineto{\pgfqpoint{3.495258in}{1.822328in}}%
\pgfpathlineto{\pgfqpoint{3.517917in}{1.129446in}}%
\pgfpathlineto{\pgfqpoint{3.540576in}{0.714983in}}%
\pgfpathlineto{\pgfqpoint{3.563234in}{0.895322in}}%
\pgfpathlineto{\pgfqpoint{3.585893in}{0.790915in}}%
\pgfpathlineto{\pgfqpoint{3.608552in}{1.401537in}}%
\pgfpathlineto{\pgfqpoint{3.631211in}{1.132610in}}%
\pgfpathlineto{\pgfqpoint{3.653869in}{1.123119in}}%
\pgfpathlineto{\pgfqpoint{3.676528in}{1.034531in}}%
\pgfpathlineto{\pgfqpoint{3.699187in}{1.028203in}}%
\pgfpathlineto{\pgfqpoint{3.721846in}{1.550237in}}%
\pgfpathlineto{\pgfqpoint{3.744504in}{0.964927in}}%
\pgfpathlineto{\pgfqpoint{3.767163in}{1.657808in}}%
\pgfpathlineto{\pgfqpoint{3.789822in}{0.784588in}}%
\pgfpathlineto{\pgfqpoint{3.812481in}{1.167412in}}%
\pgfpathlineto{\pgfqpoint{3.835139in}{1.142102in}}%
\pgfpathlineto{\pgfqpoint{3.857798in}{1.705266in}}%
\pgfpathlineto{\pgfqpoint{3.880457in}{0.847864in}}%
\pgfpathlineto{\pgfqpoint{3.903116in}{1.268655in}}%
\pgfpathlineto{\pgfqpoint{3.925774in}{1.268655in}}%
\pgfpathlineto{\pgfqpoint{3.948433in}{1.907751in}}%
\pgfpathlineto{\pgfqpoint{3.971092in}{0.844701in}}%
\pgfpathlineto{\pgfqpoint{3.993751in}{0.784588in}}%
\pgfpathlineto{\pgfqpoint{4.016409in}{0.907977in}}%
\pgfpathlineto{\pgfqpoint{4.039068in}{0.794079in}}%
\pgfpathlineto{\pgfqpoint{4.061727in}{1.278147in}}%
\pgfpathlineto{\pgfqpoint{4.084386in}{1.702102in}}%
\pgfpathlineto{\pgfqpoint{4.107044in}{1.195887in}}%
\pgfpathlineto{\pgfqpoint{4.129703in}{0.749785in}}%
\pgfpathlineto{\pgfqpoint{4.152362in}{1.578712in}}%
\pgfpathlineto{\pgfqpoint{4.175021in}{1.676791in}}%
\pgfpathlineto{\pgfqpoint{4.197679in}{1.679955in}}%
\pgfpathlineto{\pgfqpoint{4.220338in}{1.085153in}}%
\pgfpathlineto{\pgfqpoint{4.242997in}{1.841311in}}%
\pgfpathlineto{\pgfqpoint{4.265656in}{1.161085in}}%
\pgfpathlineto{\pgfqpoint{4.288314in}{1.129446in}}%
\pgfpathlineto{\pgfqpoint{4.310973in}{0.765605in}}%
\pgfpathlineto{\pgfqpoint{4.333632in}{1.091480in}}%
\pgfpathlineto{\pgfqpoint{4.356291in}{1.445831in}}%
\pgfpathlineto{\pgfqpoint{4.378949in}{1.056678in}}%
\pgfpathlineto{\pgfqpoint{4.401608in}{1.493288in}}%
\pgfpathlineto{\pgfqpoint{4.424267in}{2.088090in}}%
\pgfpathlineto{\pgfqpoint{4.446926in}{1.714757in}}%
\pgfpathlineto{\pgfqpoint{4.469584in}{1.363571in}}%
\pgfpathlineto{\pgfqpoint{4.492243in}{1.828655in}}%
\pgfpathlineto{\pgfqpoint{4.514902in}{1.483797in}}%
\pgfpathlineto{\pgfqpoint{4.537561in}{1.502780in}}%
\pgfpathlineto{\pgfqpoint{4.560219in}{0.790915in}}%
\pgfpathlineto{\pgfqpoint{4.582878in}{1.316113in}}%
\pgfpathlineto{\pgfqpoint{4.605537in}{1.537582in}}%
\pgfpathlineto{\pgfqpoint{4.628196in}{1.344588in}}%
\pgfpathlineto{\pgfqpoint{4.650854in}{2.239955in}}%
\pgfpathlineto{\pgfqpoint{4.673513in}{1.641989in}}%
\pgfpathlineto{\pgfqpoint{4.696172in}{0.955435in}}%
\pgfpathlineto{\pgfqpoint{4.718831in}{1.559729in}}%
\pgfpathlineto{\pgfqpoint{4.741489in}{1.581876in}}%
\pgfpathlineto{\pgfqpoint{4.764148in}{1.322441in}}%
\pgfpathlineto{\pgfqpoint{4.786807in}{1.129446in}}%
\pgfpathlineto{\pgfqpoint{4.809466in}{0.888994in}}%
\pgfpathlineto{\pgfqpoint{4.832124in}{1.186395in}}%
\pgfpathlineto{\pgfqpoint{4.854783in}{1.119955in}}%
\pgfpathlineto{\pgfqpoint{4.877442in}{1.230689in}}%
\pgfpathlineto{\pgfqpoint{4.900101in}{0.866847in}}%
\pgfpathlineto{\pgfqpoint{4.922759in}{1.009220in}}%
\pgfpathlineto{\pgfqpoint{4.945418in}{1.271819in}}%
\pgfpathlineto{\pgfqpoint{4.968077in}{1.161085in}}%
\pgfpathlineto{\pgfqpoint{4.990735in}{1.869785in}}%
\pgfpathlineto{\pgfqpoint{5.013394in}{1.645153in}}%
\pgfpathlineto{\pgfqpoint{5.036053in}{1.034531in}}%
\pgfpathlineto{\pgfqpoint{5.058712in}{1.774870in}}%
\pgfpathlineto{\pgfqpoint{5.081370in}{1.002893in}}%
\pgfpathlineto{\pgfqpoint{5.104029in}{0.832045in}}%
\pgfpathlineto{\pgfqpoint{5.126688in}{1.398373in}}%
\pgfpathlineto{\pgfqpoint{5.149347in}{1.274983in}}%
\pgfpathlineto{\pgfqpoint{5.172005in}{1.183232in}}%
\pgfpathlineto{\pgfqpoint{5.194664in}{1.078825in}}%
\pgfpathlineto{\pgfqpoint{5.217323in}{1.629333in}}%
\pgfpathlineto{\pgfqpoint{5.239982in}{1.392045in}}%
\pgfpathlineto{\pgfqpoint{5.262640in}{1.834983in}}%
\pgfpathlineto{\pgfqpoint{5.285299in}{1.107299in}}%
\pgfpathlineto{\pgfqpoint{5.307958in}{1.347751in}}%
\pgfpathlineto{\pgfqpoint{5.330617in}{0.907977in}}%
\pgfpathlineto{\pgfqpoint{5.353275in}{1.619842in}}%
\pgfpathlineto{\pgfqpoint{5.375934in}{1.581876in}}%
\pgfpathlineto{\pgfqpoint{5.398593in}{1.534418in}}%
\pgfpathlineto{\pgfqpoint{5.421252in}{2.129220in}}%
\pgfpathlineto{\pgfqpoint{5.443910in}{0.841537in}}%
\pgfpathlineto{\pgfqpoint{5.466569in}{1.578712in}}%
\pgfpathlineto{\pgfqpoint{5.489228in}{0.971254in}}%
\pgfpathlineto{\pgfqpoint{5.511887in}{1.170576in}}%
\pgfpathlineto{\pgfqpoint{5.534545in}{2.056452in}}%
\pgfpathlineto{\pgfqpoint{5.534545in}{2.056452in}}%
\pgfusepath{stroke}%
\end{pgfscope}%
\begin{pgfscope}%
\pgfpathrectangle{\pgfqpoint{0.800000in}{0.528000in}}{\pgfqpoint{4.960000in}{3.696000in}} %
\pgfusepath{clip}%
\pgfsetrectcap%
\pgfsetroundjoin%
\pgfsetlinewidth{1.505625pt}%
\definecolor{currentstroke}{rgb}{0.600000,0.600000,0.600000}%
\pgfsetstrokecolor{currentstroke}%
\pgfsetdash{}{0pt}%
\pgfpathmoveto{\pgfqpoint{1.025455in}{0.759277in}}%
\pgfpathlineto{\pgfqpoint{1.048113in}{0.809898in}}%
\pgfpathlineto{\pgfqpoint{1.070772in}{1.256000in}}%
\pgfpathlineto{\pgfqpoint{1.093431in}{1.502780in}}%
\pgfpathlineto{\pgfqpoint{1.116090in}{0.822554in}}%
\pgfpathlineto{\pgfqpoint{1.138748in}{1.278147in}}%
\pgfpathlineto{\pgfqpoint{1.161407in}{1.012384in}}%
\pgfpathlineto{\pgfqpoint{1.184066in}{0.825718in}}%
\pgfpathlineto{\pgfqpoint{1.206725in}{1.157921in}}%
\pgfpathlineto{\pgfqpoint{1.229383in}{0.879503in}}%
\pgfpathlineto{\pgfqpoint{1.252042in}{1.192723in}}%
\pgfpathlineto{\pgfqpoint{1.274701in}{0.860520in}}%
\pgfpathlineto{\pgfqpoint{1.297360in}{0.832045in}}%
\pgfpathlineto{\pgfqpoint{1.320018in}{1.490124in}}%
\pgfpathlineto{\pgfqpoint{1.342677in}{0.851028in}}%
\pgfpathlineto{\pgfqpoint{1.365336in}{1.461650in}}%
\pgfpathlineto{\pgfqpoint{1.387995in}{0.838373in}}%
\pgfpathlineto{\pgfqpoint{1.410653in}{0.787751in}}%
\pgfpathlineto{\pgfqpoint{1.433312in}{1.357243in}}%
\pgfpathlineto{\pgfqpoint{1.455971in}{0.876339in}}%
\pgfpathlineto{\pgfqpoint{1.478630in}{1.490124in}}%
\pgfpathlineto{\pgfqpoint{1.501288in}{0.838373in}}%
\pgfpathlineto{\pgfqpoint{1.523947in}{0.711819in}}%
\pgfpathlineto{\pgfqpoint{1.546606in}{0.907977in}}%
\pgfpathlineto{\pgfqpoint{1.569265in}{1.157921in}}%
\pgfpathlineto{\pgfqpoint{1.591923in}{0.923797in}}%
\pgfpathlineto{\pgfqpoint{1.637241in}{0.711819in}}%
\pgfpathlineto{\pgfqpoint{1.659899in}{0.790915in}}%
\pgfpathlineto{\pgfqpoint{1.682558in}{1.341424in}}%
\pgfpathlineto{\pgfqpoint{1.705217in}{0.771932in}}%
\pgfpathlineto{\pgfqpoint{1.727876in}{0.765605in}}%
\pgfpathlineto{\pgfqpoint{1.750534in}{0.714983in}}%
\pgfpathlineto{\pgfqpoint{1.773193in}{0.711819in}}%
\pgfpathlineto{\pgfqpoint{1.795852in}{0.819390in}}%
\pgfpathlineto{\pgfqpoint{1.818511in}{0.708655in}}%
\pgfpathlineto{\pgfqpoint{1.841169in}{0.714983in}}%
\pgfpathlineto{\pgfqpoint{1.863828in}{0.727638in}}%
\pgfpathlineto{\pgfqpoint{1.886487in}{0.711819in}}%
\pgfpathlineto{\pgfqpoint{1.909146in}{0.724475in}}%
\pgfpathlineto{\pgfqpoint{1.931804in}{0.847864in}}%
\pgfpathlineto{\pgfqpoint{1.954463in}{0.740294in}}%
\pgfpathlineto{\pgfqpoint{1.977122in}{1.458486in}}%
\pgfpathlineto{\pgfqpoint{1.999781in}{0.844701in}}%
\pgfpathlineto{\pgfqpoint{2.022439in}{0.832045in}}%
\pgfpathlineto{\pgfqpoint{2.045098in}{1.249672in}}%
\pgfpathlineto{\pgfqpoint{2.067757in}{1.297130in}}%
\pgfpathlineto{\pgfqpoint{2.090416in}{0.718147in}}%
\pgfpathlineto{\pgfqpoint{2.113074in}{0.816226in}}%
\pgfpathlineto{\pgfqpoint{2.135733in}{0.727638in}}%
\pgfpathlineto{\pgfqpoint{2.158392in}{0.794079in}}%
\pgfpathlineto{\pgfqpoint{2.181051in}{0.708655in}}%
\pgfpathlineto{\pgfqpoint{2.203709in}{0.854192in}}%
\pgfpathlineto{\pgfqpoint{2.226368in}{0.721311in}}%
\pgfpathlineto{\pgfqpoint{2.249027in}{0.718147in}}%
\pgfpathlineto{\pgfqpoint{2.271686in}{0.857356in}}%
\pgfpathlineto{\pgfqpoint{2.294344in}{0.749785in}}%
\pgfpathlineto{\pgfqpoint{2.317003in}{0.711819in}}%
\pgfpathlineto{\pgfqpoint{2.339662in}{0.724475in}}%
\pgfpathlineto{\pgfqpoint{2.362321in}{0.718147in}}%
\pgfpathlineto{\pgfqpoint{2.384979in}{0.708655in}}%
\pgfpathlineto{\pgfqpoint{2.407638in}{0.724475in}}%
\pgfpathlineto{\pgfqpoint{2.430297in}{0.718147in}}%
\pgfpathlineto{\pgfqpoint{2.452956in}{0.752949in}}%
\pgfpathlineto{\pgfqpoint{2.475614in}{0.711819in}}%
\pgfpathlineto{\pgfqpoint{2.498273in}{0.721311in}}%
\pgfpathlineto{\pgfqpoint{2.520932in}{0.714983in}}%
\pgfpathlineto{\pgfqpoint{2.543591in}{0.721311in}}%
\pgfpathlineto{\pgfqpoint{2.566249in}{0.705492in}}%
\pgfpathlineto{\pgfqpoint{2.588908in}{0.727638in}}%
\pgfpathlineto{\pgfqpoint{2.611567in}{0.721311in}}%
\pgfpathlineto{\pgfqpoint{2.634226in}{0.781424in}}%
\pgfpathlineto{\pgfqpoint{2.656884in}{0.714983in}}%
\pgfpathlineto{\pgfqpoint{2.679543in}{0.718147in}}%
\pgfpathlineto{\pgfqpoint{2.702202in}{0.702328in}}%
\pgfpathlineto{\pgfqpoint{2.747519in}{0.714983in}}%
\pgfpathlineto{\pgfqpoint{2.770178in}{0.702328in}}%
\pgfpathlineto{\pgfqpoint{2.792837in}{0.711819in}}%
\pgfpathlineto{\pgfqpoint{2.815496in}{0.711819in}}%
\pgfpathlineto{\pgfqpoint{2.838154in}{0.705492in}}%
\pgfpathlineto{\pgfqpoint{2.860813in}{0.721311in}}%
\pgfpathlineto{\pgfqpoint{2.883472in}{0.721311in}}%
\pgfpathlineto{\pgfqpoint{2.906131in}{0.705492in}}%
\pgfpathlineto{\pgfqpoint{2.928789in}{0.806734in}}%
\pgfpathlineto{\pgfqpoint{2.951448in}{0.718147in}}%
\pgfpathlineto{\pgfqpoint{2.974107in}{0.749785in}}%
\pgfpathlineto{\pgfqpoint{2.996766in}{0.718147in}}%
\pgfpathlineto{\pgfqpoint{3.019424in}{0.711819in}}%
\pgfpathlineto{\pgfqpoint{3.042083in}{0.696000in}}%
\pgfpathlineto{\pgfqpoint{3.064742in}{0.708655in}}%
\pgfpathlineto{\pgfqpoint{3.087401in}{0.718147in}}%
\pgfpathlineto{\pgfqpoint{3.110059in}{0.714983in}}%
\pgfpathlineto{\pgfqpoint{3.132718in}{0.705492in}}%
\pgfpathlineto{\pgfqpoint{3.155377in}{0.718147in}}%
\pgfpathlineto{\pgfqpoint{3.178036in}{0.724475in}}%
\pgfpathlineto{\pgfqpoint{3.200694in}{0.778260in}}%
\pgfpathlineto{\pgfqpoint{3.223353in}{0.705492in}}%
\pgfpathlineto{\pgfqpoint{3.246012in}{0.721311in}}%
\pgfpathlineto{\pgfqpoint{3.268671in}{0.714983in}}%
\pgfpathlineto{\pgfqpoint{3.291329in}{0.718147in}}%
\pgfpathlineto{\pgfqpoint{3.336647in}{0.711819in}}%
\pgfpathlineto{\pgfqpoint{3.359306in}{0.711819in}}%
\pgfpathlineto{\pgfqpoint{3.381964in}{0.705492in}}%
\pgfpathlineto{\pgfqpoint{3.404623in}{0.714983in}}%
\pgfpathlineto{\pgfqpoint{3.427282in}{0.718147in}}%
\pgfpathlineto{\pgfqpoint{3.449941in}{0.718147in}}%
\pgfpathlineto{\pgfqpoint{3.472599in}{0.708655in}}%
\pgfpathlineto{\pgfqpoint{3.495258in}{0.756113in}}%
\pgfpathlineto{\pgfqpoint{3.517917in}{0.835209in}}%
\pgfpathlineto{\pgfqpoint{3.540576in}{0.708655in}}%
\pgfpathlineto{\pgfqpoint{3.563234in}{0.696000in}}%
\pgfpathlineto{\pgfqpoint{3.585893in}{0.702328in}}%
\pgfpathlineto{\pgfqpoint{3.608552in}{0.724475in}}%
\pgfpathlineto{\pgfqpoint{3.631211in}{0.711819in}}%
\pgfpathlineto{\pgfqpoint{3.653869in}{0.756113in}}%
\pgfpathlineto{\pgfqpoint{3.676528in}{0.724475in}}%
\pgfpathlineto{\pgfqpoint{3.699187in}{0.714983in}}%
\pgfpathlineto{\pgfqpoint{3.721846in}{0.882667in}}%
\pgfpathlineto{\pgfqpoint{3.744504in}{0.718147in}}%
\pgfpathlineto{\pgfqpoint{3.767163in}{0.711819in}}%
\pgfpathlineto{\pgfqpoint{3.789822in}{0.809898in}}%
\pgfpathlineto{\pgfqpoint{3.812481in}{0.711819in}}%
\pgfpathlineto{\pgfqpoint{3.835139in}{0.901650in}}%
\pgfpathlineto{\pgfqpoint{3.857798in}{0.696000in}}%
\pgfpathlineto{\pgfqpoint{3.880457in}{0.714983in}}%
\pgfpathlineto{\pgfqpoint{3.903116in}{0.790915in}}%
\pgfpathlineto{\pgfqpoint{3.925774in}{0.711819in}}%
\pgfpathlineto{\pgfqpoint{3.948433in}{0.765605in}}%
\pgfpathlineto{\pgfqpoint{3.971092in}{0.711819in}}%
\pgfpathlineto{\pgfqpoint{3.993751in}{0.714983in}}%
\pgfpathlineto{\pgfqpoint{4.016409in}{0.733966in}}%
\pgfpathlineto{\pgfqpoint{4.039068in}{0.699164in}}%
\pgfpathlineto{\pgfqpoint{4.061727in}{0.749785in}}%
\pgfpathlineto{\pgfqpoint{4.084386in}{0.737130in}}%
\pgfpathlineto{\pgfqpoint{4.107044in}{0.749785in}}%
\pgfpathlineto{\pgfqpoint{4.129703in}{0.718147in}}%
\pgfpathlineto{\pgfqpoint{4.152362in}{0.756113in}}%
\pgfpathlineto{\pgfqpoint{4.175021in}{0.756113in}}%
\pgfpathlineto{\pgfqpoint{4.197679in}{0.705492in}}%
\pgfpathlineto{\pgfqpoint{4.220338in}{0.721311in}}%
\pgfpathlineto{\pgfqpoint{4.242997in}{0.740294in}}%
\pgfpathlineto{\pgfqpoint{4.265656in}{0.708655in}}%
\pgfpathlineto{\pgfqpoint{4.288314in}{0.721311in}}%
\pgfpathlineto{\pgfqpoint{4.310973in}{0.775096in}}%
\pgfpathlineto{\pgfqpoint{4.333632in}{0.711819in}}%
\pgfpathlineto{\pgfqpoint{4.356291in}{0.838373in}}%
\pgfpathlineto{\pgfqpoint{4.378949in}{1.034531in}}%
\pgfpathlineto{\pgfqpoint{4.401608in}{0.784588in}}%
\pgfpathlineto{\pgfqpoint{4.424267in}{0.718147in}}%
\pgfpathlineto{\pgfqpoint{4.446926in}{0.806734in}}%
\pgfpathlineto{\pgfqpoint{4.469584in}{0.746621in}}%
\pgfpathlineto{\pgfqpoint{4.492243in}{0.749785in}}%
\pgfpathlineto{\pgfqpoint{4.514902in}{0.787751in}}%
\pgfpathlineto{\pgfqpoint{4.537561in}{0.733966in}}%
\pgfpathlineto{\pgfqpoint{4.560219in}{0.743458in}}%
\pgfpathlineto{\pgfqpoint{4.582878in}{0.705492in}}%
\pgfpathlineto{\pgfqpoint{4.605537in}{0.711819in}}%
\pgfpathlineto{\pgfqpoint{4.628196in}{0.752949in}}%
\pgfpathlineto{\pgfqpoint{4.650854in}{0.724475in}}%
\pgfpathlineto{\pgfqpoint{4.673513in}{0.721311in}}%
\pgfpathlineto{\pgfqpoint{4.696172in}{0.740294in}}%
\pgfpathlineto{\pgfqpoint{4.718831in}{0.711819in}}%
\pgfpathlineto{\pgfqpoint{4.741489in}{0.714983in}}%
\pgfpathlineto{\pgfqpoint{4.764148in}{0.708655in}}%
\pgfpathlineto{\pgfqpoint{4.786807in}{0.749785in}}%
\pgfpathlineto{\pgfqpoint{4.809466in}{0.756113in}}%
\pgfpathlineto{\pgfqpoint{4.832124in}{0.724475in}}%
\pgfpathlineto{\pgfqpoint{4.877442in}{0.730802in}}%
\pgfpathlineto{\pgfqpoint{4.900101in}{0.708655in}}%
\pgfpathlineto{\pgfqpoint{4.922759in}{0.718147in}}%
\pgfpathlineto{\pgfqpoint{4.968077in}{0.705492in}}%
\pgfpathlineto{\pgfqpoint{4.990735in}{0.714983in}}%
\pgfpathlineto{\pgfqpoint{5.013394in}{0.711819in}}%
\pgfpathlineto{\pgfqpoint{5.036053in}{0.727638in}}%
\pgfpathlineto{\pgfqpoint{5.058712in}{0.718147in}}%
\pgfpathlineto{\pgfqpoint{5.081370in}{0.759277in}}%
\pgfpathlineto{\pgfqpoint{5.104029in}{0.714983in}}%
\pgfpathlineto{\pgfqpoint{5.126688in}{0.743458in}}%
\pgfpathlineto{\pgfqpoint{5.149347in}{0.718147in}}%
\pgfpathlineto{\pgfqpoint{5.172005in}{0.714983in}}%
\pgfpathlineto{\pgfqpoint{5.194664in}{0.737130in}}%
\pgfpathlineto{\pgfqpoint{5.217323in}{0.718147in}}%
\pgfpathlineto{\pgfqpoint{5.239982in}{0.721311in}}%
\pgfpathlineto{\pgfqpoint{5.262640in}{0.711819in}}%
\pgfpathlineto{\pgfqpoint{5.285299in}{0.714983in}}%
\pgfpathlineto{\pgfqpoint{5.307958in}{0.708655in}}%
\pgfpathlineto{\pgfqpoint{5.330617in}{0.711819in}}%
\pgfpathlineto{\pgfqpoint{5.353275in}{0.699164in}}%
\pgfpathlineto{\pgfqpoint{5.375934in}{0.705492in}}%
\pgfpathlineto{\pgfqpoint{5.398593in}{0.696000in}}%
\pgfpathlineto{\pgfqpoint{5.421252in}{0.724475in}}%
\pgfpathlineto{\pgfqpoint{5.443910in}{0.711819in}}%
\pgfpathlineto{\pgfqpoint{5.466569in}{0.718147in}}%
\pgfpathlineto{\pgfqpoint{5.489228in}{0.721311in}}%
\pgfpathlineto{\pgfqpoint{5.511887in}{0.699164in}}%
\pgfpathlineto{\pgfqpoint{5.534545in}{0.702328in}}%
\pgfpathlineto{\pgfqpoint{5.534545in}{0.702328in}}%
\pgfusepath{stroke}%
\end{pgfscope}%
\begin{pgfscope}%
\pgfpathrectangle{\pgfqpoint{0.800000in}{0.528000in}}{\pgfqpoint{4.960000in}{3.696000in}} %
\pgfusepath{clip}%
\pgfsetbuttcap%
\pgfsetmiterjoin%
\definecolor{currentfill}{rgb}{1.000000,1.000000,1.000000}%
\pgfsetfillcolor{currentfill}%
\pgfsetlinewidth{1.003750pt}%
\definecolor{currentstroke}{rgb}{1.000000,1.000000,1.000000}%
\pgfsetstrokecolor{currentstroke}%
\pgfsetdash{}{0pt}%
\pgfpathmoveto{\pgfqpoint{2.085572in}{0.610288in}}%
\pgfpathlineto{\pgfqpoint{2.235478in}{1.384749in}}%
\pgfpathlineto{\pgfqpoint{1.994428in}{1.431407in}}%
\pgfpathlineto{\pgfqpoint{1.844522in}{0.656946in}}%
\pgfpathclose%
\pgfusepath{stroke,fill}%
\end{pgfscope}%
\begin{pgfscope}%
\pgftext[x=2.013223in,y=0.680878in,left,base,rotate=79.045202]{\sffamily\fontsize{10.000000}{12.000000}\selectfont \(\displaystyle \alpha =\) 0.001}%
\end{pgfscope}%
\begin{pgfscope}%
\pgfpathrectangle{\pgfqpoint{0.800000in}{0.528000in}}{\pgfqpoint{4.960000in}{3.696000in}} %
\pgfusepath{clip}%
\pgfsetbuttcap%
\pgfsetmiterjoin%
\definecolor{currentfill}{rgb}{1.000000,1.000000,1.000000}%
\pgfsetfillcolor{currentfill}%
\pgfsetlinewidth{1.003750pt}%
\definecolor{currentstroke}{rgb}{1.000000,1.000000,1.000000}%
\pgfsetstrokecolor{currentstroke}%
\pgfsetdash{}{0pt}%
\pgfpathmoveto{\pgfqpoint{3.157980in}{1.002003in}}%
\pgfpathlineto{\pgfqpoint{3.594140in}{1.550112in}}%
\pgfpathlineto{\pgfqpoint{3.402020in}{1.702992in}}%
\pgfpathlineto{\pgfqpoint{2.965860in}{1.154882in}}%
\pgfpathclose%
\pgfusepath{stroke,fill}%
\end{pgfscope}%
\begin{pgfscope}%
\definecolor{textcolor}{rgb}{0.300000,0.300000,0.300000}%
\pgfsetstrokecolor{textcolor}%
\pgfsetfillcolor{textcolor}%
\pgftext[x=3.126495in,y=1.098056in,left,base,rotate=51.488851]{\color{textcolor}\sffamily\fontsize{10.000000}{12.000000}\selectfont \(\displaystyle \alpha =\) 0.01}%
\end{pgfscope}%
\begin{pgfscope}%
\pgfpathrectangle{\pgfqpoint{0.800000in}{0.528000in}}{\pgfqpoint{4.960000in}{3.696000in}} %
\pgfusepath{clip}%
\pgfsetbuttcap%
\pgfsetmiterjoin%
\definecolor{currentfill}{rgb}{1.000000,1.000000,1.000000}%
\pgfsetfillcolor{currentfill}%
\pgfsetlinewidth{1.003750pt}%
\definecolor{currentstroke}{rgb}{1.000000,1.000000,1.000000}%
\pgfsetstrokecolor{currentstroke}%
\pgfsetdash{}{0pt}%
\pgfpathmoveto{\pgfqpoint{4.288047in}{1.010035in}}%
\pgfpathlineto{\pgfqpoint{4.525688in}{0.445943in}}%
\pgfpathlineto{\pgfqpoint{4.751953in}{0.541264in}}%
\pgfpathlineto{\pgfqpoint{4.514312in}{1.105356in}}%
\pgfpathclose%
\pgfusepath{stroke,fill}%
\end{pgfscope}%
\begin{pgfscope}%
\definecolor{textcolor}{rgb}{0.600000,0.600000,0.600000}%
\pgfsetstrokecolor{textcolor}%
\pgfsetfillcolor{textcolor}%
\pgftext[x=4.387437in,y=0.991622in,left,base,rotate=292.844785]{\color{textcolor}\sffamily\fontsize{10.000000}{12.000000}\selectfont \(\displaystyle \alpha =\) 0.1}%
\end{pgfscope}%
\begin{pgfscope}%
\pgfsetrectcap%
\pgfsetmiterjoin%
\pgfsetlinewidth{0.803000pt}%
\definecolor{currentstroke}{rgb}{0.000000,0.000000,0.000000}%
\pgfsetstrokecolor{currentstroke}%
\pgfsetdash{}{0pt}%
\pgfpathmoveto{\pgfqpoint{0.800000in}{0.528000in}}%
\pgfpathlineto{\pgfqpoint{0.800000in}{4.224000in}}%
\pgfusepath{stroke}%
\end{pgfscope}%
\begin{pgfscope}%
\pgfsetrectcap%
\pgfsetmiterjoin%
\pgfsetlinewidth{0.803000pt}%
\definecolor{currentstroke}{rgb}{0.000000,0.000000,0.000000}%
\pgfsetstrokecolor{currentstroke}%
\pgfsetdash{}{0pt}%
\pgfpathmoveto{\pgfqpoint{5.760000in}{0.528000in}}%
\pgfpathlineto{\pgfqpoint{5.760000in}{4.224000in}}%
\pgfusepath{stroke}%
\end{pgfscope}%
\begin{pgfscope}%
\pgfsetrectcap%
\pgfsetmiterjoin%
\pgfsetlinewidth{0.803000pt}%
\definecolor{currentstroke}{rgb}{0.000000,0.000000,0.000000}%
\pgfsetstrokecolor{currentstroke}%
\pgfsetdash{}{0pt}%
\pgfpathmoveto{\pgfqpoint{0.800000in}{0.528000in}}%
\pgfpathlineto{\pgfqpoint{5.760000in}{0.528000in}}%
\pgfusepath{stroke}%
\end{pgfscope}%
\begin{pgfscope}%
\pgfsetrectcap%
\pgfsetmiterjoin%
\pgfsetlinewidth{0.803000pt}%
\definecolor{currentstroke}{rgb}{0.000000,0.000000,0.000000}%
\pgfsetstrokecolor{currentstroke}%
\pgfsetdash{}{0pt}%
\pgfpathmoveto{\pgfqpoint{0.800000in}{4.224000in}}%
\pgfpathlineto{\pgfqpoint{5.760000in}{4.224000in}}%
\pgfusepath{stroke}%
\end{pgfscope}%
\begin{pgfscope}%
\pgftext[x=3.280000in,y=4.307333in,,base]{\sffamily\fontsize{12.000000}{14.400000}\selectfont Q Learning Function Approximator Results}%
\end{pgfscope}%
\end{pgfpicture}%
\makeatother%
\endgroup%
} \\
\end{centering}
\begin{itemize}
    \item The results for different $\alpha$ can be summarized as follows:
    \begin{itemize}
        \item The largest $\alpha$ learned initially, but worsened and
            failed to find a policy that was capable of passing the episode at
            all. This suggests that, in starting with a larger $\alpha$, it
            could be useful to decay by a larger factor.
        \item The middle $\alpha$ learned very well initially, but 
            worsened and failed to find a policy that was capable of
            consistently passing the episode.  This also suggests that, in
            starting with a larger $\alpha$, it could be useful to decay by a
            larger factor.
        \item The small $\alpha$ learned slowly, but eventually leveled off at
            a better policy than those of the other $\alpha$s. This may mean
            that the decay rate was too large relative to this $\alpha$, causing
            it to cease improving after $\alpha$ became negligibly small.
    \end{itemize}
    \item These results suggest that this learner is feasable, and in order to
        improve this learner, it is necessary to more carefully control how
        $\alpha$ decays over time. It may also be useful to increase the number
        of hidden layers in the model. 
    \item To verify that this was the case,
        I move on to replacing my manual implementation with a Keras 
        implementation that handles training on its own and with which I can
        easily change the ANN parameters.
\end{itemize}
Keras Implementation

\begin{centering}
    \scalebox{0.6}{%% Creator: Matplotlib, PGF backend
%%
%% To include the figure in your LaTeX document, write
%%   \input{<filename>.pgf}
%%
%% Make sure the required packages are loaded in your preamble
%%   \usepackage{pgf}
%%
%% Figures using additional raster images can only be included by \input if
%% they are in the same directory as the main LaTeX file. For loading figures
%% from other directories you can use the `import` package
%%   \usepackage{import}
%% and then include the figures with
%%   \import{<path to file>}{<filename>.pgf}
%%
%% Matplotlib used the following preamble
%%   \usepackage{fontspec}
%%   \setmainfont{DejaVu Serif}
%%   \setsansfont{DejaVu Sans}
%%   \setmonofont{DejaVu Sans Mono}
%%
\begingroup%
\makeatletter%
\begin{pgfpicture}%
\pgfpathrectangle{\pgfpointorigin}{\pgfqpoint{6.400000in}{4.800000in}}%
\pgfusepath{use as bounding box, clip}%
\begin{pgfscope}%
\pgfsetbuttcap%
\pgfsetmiterjoin%
\definecolor{currentfill}{rgb}{1.000000,1.000000,1.000000}%
\pgfsetfillcolor{currentfill}%
\pgfsetlinewidth{0.000000pt}%
\definecolor{currentstroke}{rgb}{1.000000,1.000000,1.000000}%
\pgfsetstrokecolor{currentstroke}%
\pgfsetdash{}{0pt}%
\pgfpathmoveto{\pgfqpoint{0.000000in}{0.000000in}}%
\pgfpathlineto{\pgfqpoint{6.400000in}{0.000000in}}%
\pgfpathlineto{\pgfqpoint{6.400000in}{4.800000in}}%
\pgfpathlineto{\pgfqpoint{0.000000in}{4.800000in}}%
\pgfpathclose%
\pgfusepath{fill}%
\end{pgfscope}%
\begin{pgfscope}%
\pgfsetbuttcap%
\pgfsetmiterjoin%
\definecolor{currentfill}{rgb}{1.000000,1.000000,1.000000}%
\pgfsetfillcolor{currentfill}%
\pgfsetlinewidth{0.000000pt}%
\definecolor{currentstroke}{rgb}{0.000000,0.000000,0.000000}%
\pgfsetstrokecolor{currentstroke}%
\pgfsetstrokeopacity{0.000000}%
\pgfsetdash{}{0pt}%
\pgfpathmoveto{\pgfqpoint{0.800000in}{0.528000in}}%
\pgfpathlineto{\pgfqpoint{5.760000in}{0.528000in}}%
\pgfpathlineto{\pgfqpoint{5.760000in}{4.224000in}}%
\pgfpathlineto{\pgfqpoint{0.800000in}{4.224000in}}%
\pgfpathclose%
\pgfusepath{fill}%
\end{pgfscope}%
\begin{pgfscope}%
\pgfsetbuttcap%
\pgfsetroundjoin%
\definecolor{currentfill}{rgb}{0.000000,0.000000,0.000000}%
\pgfsetfillcolor{currentfill}%
\pgfsetlinewidth{0.803000pt}%
\definecolor{currentstroke}{rgb}{0.000000,0.000000,0.000000}%
\pgfsetstrokecolor{currentstroke}%
\pgfsetdash{}{0pt}%
\pgfsys@defobject{currentmarker}{\pgfqpoint{0.000000in}{-0.048611in}}{\pgfqpoint{0.000000in}{0.000000in}}{%
\pgfpathmoveto{\pgfqpoint{0.000000in}{0.000000in}}%
\pgfpathlineto{\pgfqpoint{0.000000in}{-0.048611in}}%
\pgfusepath{stroke,fill}%
}%
\begin{pgfscope}%
\pgfsys@transformshift{0.979908in}{0.528000in}%
\pgfsys@useobject{currentmarker}{}%
\end{pgfscope}%
\end{pgfscope}%
\begin{pgfscope}%
\pgftext[x=0.979908in,y=0.430778in,,top]{\sffamily\fontsize{10.000000}{12.000000}\selectfont \(\displaystyle 0\)}%
\end{pgfscope}%
\begin{pgfscope}%
\pgfsetbuttcap%
\pgfsetroundjoin%
\definecolor{currentfill}{rgb}{0.000000,0.000000,0.000000}%
\pgfsetfillcolor{currentfill}%
\pgfsetlinewidth{0.803000pt}%
\definecolor{currentstroke}{rgb}{0.000000,0.000000,0.000000}%
\pgfsetstrokecolor{currentstroke}%
\pgfsetdash{}{0pt}%
\pgfsys@defobject{currentmarker}{\pgfqpoint{0.000000in}{-0.048611in}}{\pgfqpoint{0.000000in}{0.000000in}}{%
\pgfpathmoveto{\pgfqpoint{0.000000in}{0.000000in}}%
\pgfpathlineto{\pgfqpoint{0.000000in}{-0.048611in}}%
\pgfusepath{stroke,fill}%
}%
\begin{pgfscope}%
\pgfsys@transformshift{1.890836in}{0.528000in}%
\pgfsys@useobject{currentmarker}{}%
\end{pgfscope}%
\end{pgfscope}%
\begin{pgfscope}%
\pgftext[x=1.890836in,y=0.430778in,,top]{\sffamily\fontsize{10.000000}{12.000000}\selectfont \(\displaystyle 20\)}%
\end{pgfscope}%
\begin{pgfscope}%
\pgfsetbuttcap%
\pgfsetroundjoin%
\definecolor{currentfill}{rgb}{0.000000,0.000000,0.000000}%
\pgfsetfillcolor{currentfill}%
\pgfsetlinewidth{0.803000pt}%
\definecolor{currentstroke}{rgb}{0.000000,0.000000,0.000000}%
\pgfsetstrokecolor{currentstroke}%
\pgfsetdash{}{0pt}%
\pgfsys@defobject{currentmarker}{\pgfqpoint{0.000000in}{-0.048611in}}{\pgfqpoint{0.000000in}{0.000000in}}{%
\pgfpathmoveto{\pgfqpoint{0.000000in}{0.000000in}}%
\pgfpathlineto{\pgfqpoint{0.000000in}{-0.048611in}}%
\pgfusepath{stroke,fill}%
}%
\begin{pgfscope}%
\pgfsys@transformshift{2.801763in}{0.528000in}%
\pgfsys@useobject{currentmarker}{}%
\end{pgfscope}%
\end{pgfscope}%
\begin{pgfscope}%
\pgftext[x=2.801763in,y=0.430778in,,top]{\sffamily\fontsize{10.000000}{12.000000}\selectfont \(\displaystyle 40\)}%
\end{pgfscope}%
\begin{pgfscope}%
\pgfsetbuttcap%
\pgfsetroundjoin%
\definecolor{currentfill}{rgb}{0.000000,0.000000,0.000000}%
\pgfsetfillcolor{currentfill}%
\pgfsetlinewidth{0.803000pt}%
\definecolor{currentstroke}{rgb}{0.000000,0.000000,0.000000}%
\pgfsetstrokecolor{currentstroke}%
\pgfsetdash{}{0pt}%
\pgfsys@defobject{currentmarker}{\pgfqpoint{0.000000in}{-0.048611in}}{\pgfqpoint{0.000000in}{0.000000in}}{%
\pgfpathmoveto{\pgfqpoint{0.000000in}{0.000000in}}%
\pgfpathlineto{\pgfqpoint{0.000000in}{-0.048611in}}%
\pgfusepath{stroke,fill}%
}%
\begin{pgfscope}%
\pgfsys@transformshift{3.712691in}{0.528000in}%
\pgfsys@useobject{currentmarker}{}%
\end{pgfscope}%
\end{pgfscope}%
\begin{pgfscope}%
\pgftext[x=3.712691in,y=0.430778in,,top]{\sffamily\fontsize{10.000000}{12.000000}\selectfont \(\displaystyle 60\)}%
\end{pgfscope}%
\begin{pgfscope}%
\pgfsetbuttcap%
\pgfsetroundjoin%
\definecolor{currentfill}{rgb}{0.000000,0.000000,0.000000}%
\pgfsetfillcolor{currentfill}%
\pgfsetlinewidth{0.803000pt}%
\definecolor{currentstroke}{rgb}{0.000000,0.000000,0.000000}%
\pgfsetstrokecolor{currentstroke}%
\pgfsetdash{}{0pt}%
\pgfsys@defobject{currentmarker}{\pgfqpoint{0.000000in}{-0.048611in}}{\pgfqpoint{0.000000in}{0.000000in}}{%
\pgfpathmoveto{\pgfqpoint{0.000000in}{0.000000in}}%
\pgfpathlineto{\pgfqpoint{0.000000in}{-0.048611in}}%
\pgfusepath{stroke,fill}%
}%
\begin{pgfscope}%
\pgfsys@transformshift{4.623618in}{0.528000in}%
\pgfsys@useobject{currentmarker}{}%
\end{pgfscope}%
\end{pgfscope}%
\begin{pgfscope}%
\pgftext[x=4.623618in,y=0.430778in,,top]{\sffamily\fontsize{10.000000}{12.000000}\selectfont \(\displaystyle 80\)}%
\end{pgfscope}%
\begin{pgfscope}%
\pgfsetbuttcap%
\pgfsetroundjoin%
\definecolor{currentfill}{rgb}{0.000000,0.000000,0.000000}%
\pgfsetfillcolor{currentfill}%
\pgfsetlinewidth{0.803000pt}%
\definecolor{currentstroke}{rgb}{0.000000,0.000000,0.000000}%
\pgfsetstrokecolor{currentstroke}%
\pgfsetdash{}{0pt}%
\pgfsys@defobject{currentmarker}{\pgfqpoint{0.000000in}{-0.048611in}}{\pgfqpoint{0.000000in}{0.000000in}}{%
\pgfpathmoveto{\pgfqpoint{0.000000in}{0.000000in}}%
\pgfpathlineto{\pgfqpoint{0.000000in}{-0.048611in}}%
\pgfusepath{stroke,fill}%
}%
\begin{pgfscope}%
\pgfsys@transformshift{5.534545in}{0.528000in}%
\pgfsys@useobject{currentmarker}{}%
\end{pgfscope}%
\end{pgfscope}%
\begin{pgfscope}%
\pgftext[x=5.534545in,y=0.430778in,,top]{\sffamily\fontsize{10.000000}{12.000000}\selectfont \(\displaystyle 100\)}%
\end{pgfscope}%
\begin{pgfscope}%
\pgftext[x=3.280000in,y=0.240809in,,top]{\sffamily\fontsize{10.000000}{12.000000}\selectfont Episode}%
\end{pgfscope}%
\begin{pgfscope}%
\pgfsetbuttcap%
\pgfsetroundjoin%
\definecolor{currentfill}{rgb}{0.000000,0.000000,0.000000}%
\pgfsetfillcolor{currentfill}%
\pgfsetlinewidth{0.803000pt}%
\definecolor{currentstroke}{rgb}{0.000000,0.000000,0.000000}%
\pgfsetstrokecolor{currentstroke}%
\pgfsetdash{}{0pt}%
\pgfsys@defobject{currentmarker}{\pgfqpoint{-0.048611in}{0.000000in}}{\pgfqpoint{0.000000in}{0.000000in}}{%
\pgfpathmoveto{\pgfqpoint{0.000000in}{0.000000in}}%
\pgfpathlineto{\pgfqpoint{-0.048611in}{0.000000in}}%
\pgfusepath{stroke,fill}%
}%
\begin{pgfscope}%
\pgfsys@transformshift{0.800000in}{0.556000in}%
\pgfsys@useobject{currentmarker}{}%
\end{pgfscope}%
\end{pgfscope}%
\begin{pgfscope}%
\pgftext[x=0.633333in,y=0.503238in,left,base]{\sffamily\fontsize{10.000000}{12.000000}\selectfont \(\displaystyle 0\)}%
\end{pgfscope}%
\begin{pgfscope}%
\pgfsetbuttcap%
\pgfsetroundjoin%
\definecolor{currentfill}{rgb}{0.000000,0.000000,0.000000}%
\pgfsetfillcolor{currentfill}%
\pgfsetlinewidth{0.803000pt}%
\definecolor{currentstroke}{rgb}{0.000000,0.000000,0.000000}%
\pgfsetstrokecolor{currentstroke}%
\pgfsetdash{}{0pt}%
\pgfsys@defobject{currentmarker}{\pgfqpoint{-0.048611in}{0.000000in}}{\pgfqpoint{0.000000in}{0.000000in}}{%
\pgfpathmoveto{\pgfqpoint{0.000000in}{0.000000in}}%
\pgfpathlineto{\pgfqpoint{-0.048611in}{0.000000in}}%
\pgfusepath{stroke,fill}%
}%
\begin{pgfscope}%
\pgfsys@transformshift{0.800000in}{0.993500in}%
\pgfsys@useobject{currentmarker}{}%
\end{pgfscope}%
\end{pgfscope}%
\begin{pgfscope}%
\pgftext[x=0.563888in,y=0.940738in,left,base]{\sffamily\fontsize{10.000000}{12.000000}\selectfont \(\displaystyle 25\)}%
\end{pgfscope}%
\begin{pgfscope}%
\pgfsetbuttcap%
\pgfsetroundjoin%
\definecolor{currentfill}{rgb}{0.000000,0.000000,0.000000}%
\pgfsetfillcolor{currentfill}%
\pgfsetlinewidth{0.803000pt}%
\definecolor{currentstroke}{rgb}{0.000000,0.000000,0.000000}%
\pgfsetstrokecolor{currentstroke}%
\pgfsetdash{}{0pt}%
\pgfsys@defobject{currentmarker}{\pgfqpoint{-0.048611in}{0.000000in}}{\pgfqpoint{0.000000in}{0.000000in}}{%
\pgfpathmoveto{\pgfqpoint{0.000000in}{0.000000in}}%
\pgfpathlineto{\pgfqpoint{-0.048611in}{0.000000in}}%
\pgfusepath{stroke,fill}%
}%
\begin{pgfscope}%
\pgfsys@transformshift{0.800000in}{1.431000in}%
\pgfsys@useobject{currentmarker}{}%
\end{pgfscope}%
\end{pgfscope}%
\begin{pgfscope}%
\pgftext[x=0.563888in,y=1.378238in,left,base]{\sffamily\fontsize{10.000000}{12.000000}\selectfont \(\displaystyle 50\)}%
\end{pgfscope}%
\begin{pgfscope}%
\pgfsetbuttcap%
\pgfsetroundjoin%
\definecolor{currentfill}{rgb}{0.000000,0.000000,0.000000}%
\pgfsetfillcolor{currentfill}%
\pgfsetlinewidth{0.803000pt}%
\definecolor{currentstroke}{rgb}{0.000000,0.000000,0.000000}%
\pgfsetstrokecolor{currentstroke}%
\pgfsetdash{}{0pt}%
\pgfsys@defobject{currentmarker}{\pgfqpoint{-0.048611in}{0.000000in}}{\pgfqpoint{0.000000in}{0.000000in}}{%
\pgfpathmoveto{\pgfqpoint{0.000000in}{0.000000in}}%
\pgfpathlineto{\pgfqpoint{-0.048611in}{0.000000in}}%
\pgfusepath{stroke,fill}%
}%
\begin{pgfscope}%
\pgfsys@transformshift{0.800000in}{1.868500in}%
\pgfsys@useobject{currentmarker}{}%
\end{pgfscope}%
\end{pgfscope}%
\begin{pgfscope}%
\pgftext[x=0.563888in,y=1.815738in,left,base]{\sffamily\fontsize{10.000000}{12.000000}\selectfont \(\displaystyle 75\)}%
\end{pgfscope}%
\begin{pgfscope}%
\pgfsetbuttcap%
\pgfsetroundjoin%
\definecolor{currentfill}{rgb}{0.000000,0.000000,0.000000}%
\pgfsetfillcolor{currentfill}%
\pgfsetlinewidth{0.803000pt}%
\definecolor{currentstroke}{rgb}{0.000000,0.000000,0.000000}%
\pgfsetstrokecolor{currentstroke}%
\pgfsetdash{}{0pt}%
\pgfsys@defobject{currentmarker}{\pgfqpoint{-0.048611in}{0.000000in}}{\pgfqpoint{0.000000in}{0.000000in}}{%
\pgfpathmoveto{\pgfqpoint{0.000000in}{0.000000in}}%
\pgfpathlineto{\pgfqpoint{-0.048611in}{0.000000in}}%
\pgfusepath{stroke,fill}%
}%
\begin{pgfscope}%
\pgfsys@transformshift{0.800000in}{2.306000in}%
\pgfsys@useobject{currentmarker}{}%
\end{pgfscope}%
\end{pgfscope}%
\begin{pgfscope}%
\pgftext[x=0.494444in,y=2.253238in,left,base]{\sffamily\fontsize{10.000000}{12.000000}\selectfont \(\displaystyle 100\)}%
\end{pgfscope}%
\begin{pgfscope}%
\pgfsetbuttcap%
\pgfsetroundjoin%
\definecolor{currentfill}{rgb}{0.000000,0.000000,0.000000}%
\pgfsetfillcolor{currentfill}%
\pgfsetlinewidth{0.803000pt}%
\definecolor{currentstroke}{rgb}{0.000000,0.000000,0.000000}%
\pgfsetstrokecolor{currentstroke}%
\pgfsetdash{}{0pt}%
\pgfsys@defobject{currentmarker}{\pgfqpoint{-0.048611in}{0.000000in}}{\pgfqpoint{0.000000in}{0.000000in}}{%
\pgfpathmoveto{\pgfqpoint{0.000000in}{0.000000in}}%
\pgfpathlineto{\pgfqpoint{-0.048611in}{0.000000in}}%
\pgfusepath{stroke,fill}%
}%
\begin{pgfscope}%
\pgfsys@transformshift{0.800000in}{2.743500in}%
\pgfsys@useobject{currentmarker}{}%
\end{pgfscope}%
\end{pgfscope}%
\begin{pgfscope}%
\pgftext[x=0.494444in,y=2.690738in,left,base]{\sffamily\fontsize{10.000000}{12.000000}\selectfont \(\displaystyle 125\)}%
\end{pgfscope}%
\begin{pgfscope}%
\pgfsetbuttcap%
\pgfsetroundjoin%
\definecolor{currentfill}{rgb}{0.000000,0.000000,0.000000}%
\pgfsetfillcolor{currentfill}%
\pgfsetlinewidth{0.803000pt}%
\definecolor{currentstroke}{rgb}{0.000000,0.000000,0.000000}%
\pgfsetstrokecolor{currentstroke}%
\pgfsetdash{}{0pt}%
\pgfsys@defobject{currentmarker}{\pgfqpoint{-0.048611in}{0.000000in}}{\pgfqpoint{0.000000in}{0.000000in}}{%
\pgfpathmoveto{\pgfqpoint{0.000000in}{0.000000in}}%
\pgfpathlineto{\pgfqpoint{-0.048611in}{0.000000in}}%
\pgfusepath{stroke,fill}%
}%
\begin{pgfscope}%
\pgfsys@transformshift{0.800000in}{3.181000in}%
\pgfsys@useobject{currentmarker}{}%
\end{pgfscope}%
\end{pgfscope}%
\begin{pgfscope}%
\pgftext[x=0.494444in,y=3.128238in,left,base]{\sffamily\fontsize{10.000000}{12.000000}\selectfont \(\displaystyle 150\)}%
\end{pgfscope}%
\begin{pgfscope}%
\pgfsetbuttcap%
\pgfsetroundjoin%
\definecolor{currentfill}{rgb}{0.000000,0.000000,0.000000}%
\pgfsetfillcolor{currentfill}%
\pgfsetlinewidth{0.803000pt}%
\definecolor{currentstroke}{rgb}{0.000000,0.000000,0.000000}%
\pgfsetstrokecolor{currentstroke}%
\pgfsetdash{}{0pt}%
\pgfsys@defobject{currentmarker}{\pgfqpoint{-0.048611in}{0.000000in}}{\pgfqpoint{0.000000in}{0.000000in}}{%
\pgfpathmoveto{\pgfqpoint{0.000000in}{0.000000in}}%
\pgfpathlineto{\pgfqpoint{-0.048611in}{0.000000in}}%
\pgfusepath{stroke,fill}%
}%
\begin{pgfscope}%
\pgfsys@transformshift{0.800000in}{3.618500in}%
\pgfsys@useobject{currentmarker}{}%
\end{pgfscope}%
\end{pgfscope}%
\begin{pgfscope}%
\pgftext[x=0.494444in,y=3.565738in,left,base]{\sffamily\fontsize{10.000000}{12.000000}\selectfont \(\displaystyle 175\)}%
\end{pgfscope}%
\begin{pgfscope}%
\pgfsetbuttcap%
\pgfsetroundjoin%
\definecolor{currentfill}{rgb}{0.000000,0.000000,0.000000}%
\pgfsetfillcolor{currentfill}%
\pgfsetlinewidth{0.803000pt}%
\definecolor{currentstroke}{rgb}{0.000000,0.000000,0.000000}%
\pgfsetstrokecolor{currentstroke}%
\pgfsetdash{}{0pt}%
\pgfsys@defobject{currentmarker}{\pgfqpoint{-0.048611in}{0.000000in}}{\pgfqpoint{0.000000in}{0.000000in}}{%
\pgfpathmoveto{\pgfqpoint{0.000000in}{0.000000in}}%
\pgfpathlineto{\pgfqpoint{-0.048611in}{0.000000in}}%
\pgfusepath{stroke,fill}%
}%
\begin{pgfscope}%
\pgfsys@transformshift{0.800000in}{4.056000in}%
\pgfsys@useobject{currentmarker}{}%
\end{pgfscope}%
\end{pgfscope}%
\begin{pgfscope}%
\pgftext[x=0.494444in,y=4.003238in,left,base]{\sffamily\fontsize{10.000000}{12.000000}\selectfont \(\displaystyle 200\)}%
\end{pgfscope}%
\begin{pgfscope}%
\pgftext[x=0.438888in,y=2.376000in,,bottom,rotate=90.000000]{\sffamily\fontsize{10.000000}{12.000000}\selectfont Average Target Policy Episode Length (1 Runs)}%
\end{pgfscope}%
\begin{pgfscope}%
\pgfpathrectangle{\pgfqpoint{0.800000in}{0.528000in}}{\pgfqpoint{4.960000in}{3.696000in}} %
\pgfusepath{clip}%
\pgfsetrectcap%
\pgfsetroundjoin%
\pgfsetlinewidth{1.505625pt}%
\definecolor{currentstroke}{rgb}{0.000000,0.000000,0.000000}%
\pgfsetstrokecolor{currentstroke}%
\pgfsetdash{}{0pt}%
\pgfpathmoveto{\pgfqpoint{1.025455in}{0.713500in}}%
\pgfpathlineto{\pgfqpoint{1.071001in}{0.696000in}}%
\pgfpathlineto{\pgfqpoint{1.116547in}{0.713500in}}%
\pgfpathlineto{\pgfqpoint{1.162094in}{0.731000in}}%
\pgfpathlineto{\pgfqpoint{1.207640in}{0.713500in}}%
\pgfpathlineto{\pgfqpoint{1.253186in}{0.696000in}}%
\pgfpathlineto{\pgfqpoint{1.298733in}{0.731000in}}%
\pgfpathlineto{\pgfqpoint{1.344279in}{0.731000in}}%
\pgfpathlineto{\pgfqpoint{1.389826in}{0.731000in}}%
\pgfpathlineto{\pgfqpoint{1.435372in}{0.713500in}}%
\pgfpathlineto{\pgfqpoint{1.480918in}{0.713500in}}%
\pgfpathlineto{\pgfqpoint{1.526465in}{0.713500in}}%
\pgfpathlineto{\pgfqpoint{1.572011in}{1.133500in}}%
\pgfpathlineto{\pgfqpoint{1.617557in}{0.853500in}}%
\pgfpathlineto{\pgfqpoint{1.663104in}{1.168500in}}%
\pgfpathlineto{\pgfqpoint{1.708650in}{1.413500in}}%
\pgfpathlineto{\pgfqpoint{1.754197in}{1.448500in}}%
\pgfpathlineto{\pgfqpoint{1.799743in}{4.056000in}}%
\pgfpathlineto{\pgfqpoint{1.845289in}{1.588500in}}%
\pgfpathlineto{\pgfqpoint{1.890836in}{0.976000in}}%
\pgfpathlineto{\pgfqpoint{1.936382in}{1.413500in}}%
\pgfpathlineto{\pgfqpoint{1.981928in}{1.063500in}}%
\pgfpathlineto{\pgfqpoint{2.027475in}{1.938500in}}%
\pgfpathlineto{\pgfqpoint{2.073021in}{2.306000in}}%
\pgfpathlineto{\pgfqpoint{2.118567in}{1.571000in}}%
\pgfpathlineto{\pgfqpoint{2.164114in}{2.306000in}}%
\pgfpathlineto{\pgfqpoint{2.209660in}{2.288500in}}%
\pgfpathlineto{\pgfqpoint{2.255207in}{3.566000in}}%
\pgfpathlineto{\pgfqpoint{2.300753in}{1.553500in}}%
\pgfpathlineto{\pgfqpoint{2.346299in}{2.936000in}}%
\pgfpathlineto{\pgfqpoint{2.391846in}{1.203500in}}%
\pgfpathlineto{\pgfqpoint{2.437392in}{1.413500in}}%
\pgfpathlineto{\pgfqpoint{2.482938in}{0.853500in}}%
\pgfpathlineto{\pgfqpoint{2.528485in}{1.728500in}}%
\pgfpathlineto{\pgfqpoint{2.574031in}{2.586000in}}%
\pgfpathlineto{\pgfqpoint{2.619578in}{2.096000in}}%
\pgfpathlineto{\pgfqpoint{2.665124in}{1.273500in}}%
\pgfpathlineto{\pgfqpoint{2.710670in}{1.571000in}}%
\pgfpathlineto{\pgfqpoint{2.756217in}{1.798500in}}%
\pgfpathlineto{\pgfqpoint{2.801763in}{1.833500in}}%
\pgfpathlineto{\pgfqpoint{2.847309in}{1.028500in}}%
\pgfpathlineto{\pgfqpoint{2.892856in}{2.131000in}}%
\pgfpathlineto{\pgfqpoint{2.938402in}{1.571000in}}%
\pgfpathlineto{\pgfqpoint{2.983949in}{0.871000in}}%
\pgfpathlineto{\pgfqpoint{3.029495in}{2.043500in}}%
\pgfpathlineto{\pgfqpoint{3.075041in}{3.321000in}}%
\pgfpathlineto{\pgfqpoint{3.120588in}{1.973500in}}%
\pgfpathlineto{\pgfqpoint{3.166134in}{1.483500in}}%
\pgfpathlineto{\pgfqpoint{3.211680in}{0.766000in}}%
\pgfpathlineto{\pgfqpoint{3.257227in}{1.203500in}}%
\pgfpathlineto{\pgfqpoint{3.302773in}{2.481000in}}%
\pgfpathlineto{\pgfqpoint{3.348320in}{2.043500in}}%
\pgfpathlineto{\pgfqpoint{3.393866in}{2.481000in}}%
\pgfpathlineto{\pgfqpoint{3.439412in}{2.883500in}}%
\pgfpathlineto{\pgfqpoint{3.484959in}{4.056000in}}%
\pgfpathlineto{\pgfqpoint{3.530505in}{1.431000in}}%
\pgfpathlineto{\pgfqpoint{3.576051in}{2.271000in}}%
\pgfpathlineto{\pgfqpoint{3.621598in}{1.641000in}}%
\pgfpathlineto{\pgfqpoint{3.667144in}{2.393500in}}%
\pgfpathlineto{\pgfqpoint{3.712691in}{3.706000in}}%
\pgfpathlineto{\pgfqpoint{3.758237in}{4.056000in}}%
\pgfpathlineto{\pgfqpoint{3.803783in}{2.376000in}}%
\pgfpathlineto{\pgfqpoint{3.849330in}{2.866000in}}%
\pgfpathlineto{\pgfqpoint{3.894876in}{3.513500in}}%
\pgfpathlineto{\pgfqpoint{3.940422in}{0.696000in}}%
\pgfpathlineto{\pgfqpoint{3.985969in}{0.713500in}}%
\pgfpathlineto{\pgfqpoint{4.031515in}{0.731000in}}%
\pgfpathlineto{\pgfqpoint{4.077062in}{0.731000in}}%
\pgfpathlineto{\pgfqpoint{4.122608in}{2.481000in}}%
\pgfpathlineto{\pgfqpoint{4.168154in}{2.953500in}}%
\pgfpathlineto{\pgfqpoint{4.213701in}{3.408500in}}%
\pgfpathlineto{\pgfqpoint{4.259247in}{3.303500in}}%
\pgfpathlineto{\pgfqpoint{4.304793in}{2.341000in}}%
\pgfpathlineto{\pgfqpoint{4.350340in}{2.218500in}}%
\pgfpathlineto{\pgfqpoint{4.395886in}{2.778500in}}%
\pgfpathlineto{\pgfqpoint{4.441433in}{2.481000in}}%
\pgfpathlineto{\pgfqpoint{4.486979in}{3.653500in}}%
\pgfpathlineto{\pgfqpoint{4.532525in}{2.201000in}}%
\pgfpathlineto{\pgfqpoint{4.578072in}{0.713500in}}%
\pgfpathlineto{\pgfqpoint{4.623618in}{0.696000in}}%
\pgfpathlineto{\pgfqpoint{4.669164in}{0.766000in}}%
\pgfpathlineto{\pgfqpoint{4.714711in}{2.446000in}}%
\pgfpathlineto{\pgfqpoint{4.760257in}{2.428500in}}%
\pgfpathlineto{\pgfqpoint{4.805803in}{2.831000in}}%
\pgfpathlineto{\pgfqpoint{4.851350in}{0.818500in}}%
\pgfpathlineto{\pgfqpoint{4.896896in}{2.813500in}}%
\pgfpathlineto{\pgfqpoint{4.942443in}{0.993500in}}%
\pgfpathlineto{\pgfqpoint{4.987989in}{1.448500in}}%
\pgfpathlineto{\pgfqpoint{5.033535in}{4.056000in}}%
\pgfpathlineto{\pgfqpoint{5.079082in}{4.056000in}}%
\pgfpathlineto{\pgfqpoint{5.124628in}{3.933500in}}%
\pgfpathlineto{\pgfqpoint{5.170174in}{4.056000in}}%
\pgfpathlineto{\pgfqpoint{5.215721in}{2.061000in}}%
\pgfpathlineto{\pgfqpoint{5.261267in}{4.056000in}}%
\pgfpathlineto{\pgfqpoint{5.306814in}{2.586000in}}%
\pgfpathlineto{\pgfqpoint{5.352360in}{2.411000in}}%
\pgfpathlineto{\pgfqpoint{5.397906in}{3.111000in}}%
\pgfpathlineto{\pgfqpoint{5.443453in}{4.056000in}}%
\pgfpathlineto{\pgfqpoint{5.488999in}{4.056000in}}%
\pgfpathlineto{\pgfqpoint{5.534545in}{2.866000in}}%
\pgfusepath{stroke}%
\end{pgfscope}%
\begin{pgfscope}%
\pgfpathrectangle{\pgfqpoint{0.800000in}{0.528000in}}{\pgfqpoint{4.960000in}{3.696000in}} %
\pgfusepath{clip}%
\pgfsetrectcap%
\pgfsetroundjoin%
\pgfsetlinewidth{1.505625pt}%
\definecolor{currentstroke}{rgb}{0.300000,0.300000,0.300000}%
\pgfsetstrokecolor{currentstroke}%
\pgfsetdash{}{0pt}%
\pgfpathmoveto{\pgfqpoint{1.025455in}{0.713500in}}%
\pgfpathlineto{\pgfqpoint{1.071001in}{0.731000in}}%
\pgfpathlineto{\pgfqpoint{1.116547in}{0.713500in}}%
\pgfpathlineto{\pgfqpoint{1.162094in}{2.306000in}}%
\pgfpathlineto{\pgfqpoint{1.207640in}{1.851000in}}%
\pgfpathlineto{\pgfqpoint{1.253186in}{1.588500in}}%
\pgfpathlineto{\pgfqpoint{1.298733in}{1.116000in}}%
\pgfpathlineto{\pgfqpoint{1.344279in}{1.571000in}}%
\pgfpathlineto{\pgfqpoint{1.389826in}{0.976000in}}%
\pgfpathlineto{\pgfqpoint{1.435372in}{0.906000in}}%
\pgfpathlineto{\pgfqpoint{1.480918in}{0.836000in}}%
\pgfpathlineto{\pgfqpoint{1.526465in}{0.766000in}}%
\pgfpathlineto{\pgfqpoint{1.572011in}{0.958500in}}%
\pgfpathlineto{\pgfqpoint{1.617557in}{1.133500in}}%
\pgfpathlineto{\pgfqpoint{1.663104in}{1.238500in}}%
\pgfpathlineto{\pgfqpoint{1.708650in}{1.046000in}}%
\pgfpathlineto{\pgfqpoint{1.754197in}{0.731000in}}%
\pgfpathlineto{\pgfqpoint{1.799743in}{1.011000in}}%
\pgfpathlineto{\pgfqpoint{1.845289in}{1.168500in}}%
\pgfpathlineto{\pgfqpoint{1.890836in}{1.081000in}}%
\pgfpathlineto{\pgfqpoint{1.936382in}{0.993500in}}%
\pgfpathlineto{\pgfqpoint{1.981928in}{1.518500in}}%
\pgfpathlineto{\pgfqpoint{2.027475in}{0.976000in}}%
\pgfpathlineto{\pgfqpoint{2.073021in}{1.291000in}}%
\pgfpathlineto{\pgfqpoint{2.118567in}{2.516000in}}%
\pgfpathlineto{\pgfqpoint{2.164114in}{3.968500in}}%
\pgfpathlineto{\pgfqpoint{2.209660in}{4.056000in}}%
\pgfpathlineto{\pgfqpoint{2.255207in}{4.056000in}}%
\pgfpathlineto{\pgfqpoint{2.300753in}{4.056000in}}%
\pgfpathlineto{\pgfqpoint{2.346299in}{1.413500in}}%
\pgfpathlineto{\pgfqpoint{2.391846in}{3.076000in}}%
\pgfpathlineto{\pgfqpoint{2.437392in}{4.056000in}}%
\pgfpathlineto{\pgfqpoint{2.482938in}{4.056000in}}%
\pgfpathlineto{\pgfqpoint{2.528485in}{2.603500in}}%
\pgfpathlineto{\pgfqpoint{2.574031in}{3.181000in}}%
\pgfpathlineto{\pgfqpoint{2.619578in}{2.428500in}}%
\pgfpathlineto{\pgfqpoint{2.665124in}{3.636000in}}%
\pgfpathlineto{\pgfqpoint{2.710670in}{1.238500in}}%
\pgfpathlineto{\pgfqpoint{2.756217in}{1.623500in}}%
\pgfpathlineto{\pgfqpoint{2.801763in}{1.553500in}}%
\pgfpathlineto{\pgfqpoint{2.847309in}{1.396000in}}%
\pgfpathlineto{\pgfqpoint{2.892856in}{1.063500in}}%
\pgfpathlineto{\pgfqpoint{2.938402in}{2.253500in}}%
\pgfpathlineto{\pgfqpoint{2.983949in}{1.886000in}}%
\pgfpathlineto{\pgfqpoint{3.029495in}{1.413500in}}%
\pgfpathlineto{\pgfqpoint{3.075041in}{3.636000in}}%
\pgfpathlineto{\pgfqpoint{3.120588in}{2.201000in}}%
\pgfpathlineto{\pgfqpoint{3.166134in}{1.676000in}}%
\pgfpathlineto{\pgfqpoint{3.211680in}{2.043500in}}%
\pgfpathlineto{\pgfqpoint{3.257227in}{1.798500in}}%
\pgfpathlineto{\pgfqpoint{3.302773in}{2.166000in}}%
\pgfpathlineto{\pgfqpoint{3.348320in}{3.163500in}}%
\pgfpathlineto{\pgfqpoint{3.393866in}{1.903500in}}%
\pgfpathlineto{\pgfqpoint{3.439412in}{2.358500in}}%
\pgfpathlineto{\pgfqpoint{3.484959in}{2.498500in}}%
\pgfpathlineto{\pgfqpoint{3.530505in}{2.988500in}}%
\pgfpathlineto{\pgfqpoint{3.576051in}{3.023500in}}%
\pgfpathlineto{\pgfqpoint{3.621598in}{2.516000in}}%
\pgfpathlineto{\pgfqpoint{3.667144in}{0.783500in}}%
\pgfpathlineto{\pgfqpoint{3.712691in}{3.321000in}}%
\pgfpathlineto{\pgfqpoint{3.758237in}{2.953500in}}%
\pgfpathlineto{\pgfqpoint{3.803783in}{3.513500in}}%
\pgfpathlineto{\pgfqpoint{3.849330in}{1.746000in}}%
\pgfpathlineto{\pgfqpoint{3.894876in}{2.183500in}}%
\pgfpathlineto{\pgfqpoint{3.940422in}{4.056000in}}%
\pgfpathlineto{\pgfqpoint{3.985969in}{1.256000in}}%
\pgfpathlineto{\pgfqpoint{4.031515in}{3.601000in}}%
\pgfpathlineto{\pgfqpoint{4.077062in}{1.746000in}}%
\pgfpathlineto{\pgfqpoint{4.122608in}{4.056000in}}%
\pgfpathlineto{\pgfqpoint{4.168154in}{1.238500in}}%
\pgfpathlineto{\pgfqpoint{4.213701in}{0.941000in}}%
\pgfpathlineto{\pgfqpoint{4.259247in}{3.058500in}}%
\pgfpathlineto{\pgfqpoint{4.304793in}{1.711000in}}%
\pgfpathlineto{\pgfqpoint{4.350340in}{1.343500in}}%
\pgfpathlineto{\pgfqpoint{4.395886in}{3.286000in}}%
\pgfpathlineto{\pgfqpoint{4.441433in}{1.851000in}}%
\pgfpathlineto{\pgfqpoint{4.486979in}{1.833500in}}%
\pgfpathlineto{\pgfqpoint{4.532525in}{3.233500in}}%
\pgfpathlineto{\pgfqpoint{4.578072in}{0.783500in}}%
\pgfpathlineto{\pgfqpoint{4.623618in}{0.818500in}}%
\pgfpathlineto{\pgfqpoint{4.669164in}{2.516000in}}%
\pgfpathlineto{\pgfqpoint{4.714711in}{3.023500in}}%
\pgfpathlineto{\pgfqpoint{4.760257in}{1.571000in}}%
\pgfpathlineto{\pgfqpoint{4.805803in}{0.801000in}}%
\pgfpathlineto{\pgfqpoint{4.851350in}{1.623500in}}%
\pgfpathlineto{\pgfqpoint{4.896896in}{1.396000in}}%
\pgfpathlineto{\pgfqpoint{4.942443in}{1.116000in}}%
\pgfpathlineto{\pgfqpoint{4.987989in}{1.361000in}}%
\pgfpathlineto{\pgfqpoint{5.033535in}{1.396000in}}%
\pgfpathlineto{\pgfqpoint{5.079082in}{1.763500in}}%
\pgfpathlineto{\pgfqpoint{5.124628in}{1.273500in}}%
\pgfpathlineto{\pgfqpoint{5.170174in}{1.466000in}}%
\pgfpathlineto{\pgfqpoint{5.215721in}{1.728500in}}%
\pgfpathlineto{\pgfqpoint{5.261267in}{1.781000in}}%
\pgfpathlineto{\pgfqpoint{5.306814in}{1.483500in}}%
\pgfpathlineto{\pgfqpoint{5.352360in}{1.606000in}}%
\pgfpathlineto{\pgfqpoint{5.397906in}{2.603500in}}%
\pgfpathlineto{\pgfqpoint{5.443453in}{2.568500in}}%
\pgfpathlineto{\pgfqpoint{5.488999in}{4.056000in}}%
\pgfpathlineto{\pgfqpoint{5.534545in}{3.531000in}}%
\pgfusepath{stroke}%
\end{pgfscope}%
\begin{pgfscope}%
\pgfpathrectangle{\pgfqpoint{0.800000in}{0.528000in}}{\pgfqpoint{4.960000in}{3.696000in}} %
\pgfusepath{clip}%
\pgfsetrectcap%
\pgfsetroundjoin%
\pgfsetlinewidth{1.505625pt}%
\definecolor{currentstroke}{rgb}{0.600000,0.600000,0.600000}%
\pgfsetstrokecolor{currentstroke}%
\pgfsetdash{}{0pt}%
\pgfpathmoveto{\pgfqpoint{1.025455in}{0.713500in}}%
\pgfpathlineto{\pgfqpoint{1.071001in}{0.713500in}}%
\pgfpathlineto{\pgfqpoint{1.116547in}{0.731000in}}%
\pgfpathlineto{\pgfqpoint{1.162094in}{0.731000in}}%
\pgfpathlineto{\pgfqpoint{1.207640in}{0.713500in}}%
\pgfpathlineto{\pgfqpoint{1.253186in}{0.731000in}}%
\pgfpathlineto{\pgfqpoint{1.298733in}{0.713500in}}%
\pgfpathlineto{\pgfqpoint{1.344279in}{0.713500in}}%
\pgfpathlineto{\pgfqpoint{1.389826in}{0.731000in}}%
\pgfpathlineto{\pgfqpoint{1.435372in}{0.696000in}}%
\pgfpathlineto{\pgfqpoint{1.480918in}{0.731000in}}%
\pgfpathlineto{\pgfqpoint{1.526465in}{0.696000in}}%
\pgfpathlineto{\pgfqpoint{1.572011in}{0.713500in}}%
\pgfpathlineto{\pgfqpoint{1.617557in}{0.731000in}}%
\pgfpathlineto{\pgfqpoint{1.663104in}{0.731000in}}%
\pgfpathlineto{\pgfqpoint{1.708650in}{0.713500in}}%
\pgfpathlineto{\pgfqpoint{1.754197in}{0.713500in}}%
\pgfpathlineto{\pgfqpoint{1.799743in}{0.713500in}}%
\pgfpathlineto{\pgfqpoint{1.845289in}{0.696000in}}%
\pgfpathlineto{\pgfqpoint{1.890836in}{0.731000in}}%
\pgfpathlineto{\pgfqpoint{1.936382in}{0.713500in}}%
\pgfpathlineto{\pgfqpoint{1.981928in}{0.713500in}}%
\pgfpathlineto{\pgfqpoint{2.027475in}{0.731000in}}%
\pgfpathlineto{\pgfqpoint{2.073021in}{0.713500in}}%
\pgfpathlineto{\pgfqpoint{2.118567in}{0.696000in}}%
\pgfpathlineto{\pgfqpoint{2.164114in}{0.731000in}}%
\pgfpathlineto{\pgfqpoint{2.209660in}{0.731000in}}%
\pgfpathlineto{\pgfqpoint{2.255207in}{0.731000in}}%
\pgfpathlineto{\pgfqpoint{2.300753in}{0.731000in}}%
\pgfpathlineto{\pgfqpoint{2.346299in}{0.748500in}}%
\pgfpathlineto{\pgfqpoint{2.391846in}{0.731000in}}%
\pgfpathlineto{\pgfqpoint{2.437392in}{0.731000in}}%
\pgfpathlineto{\pgfqpoint{2.482938in}{0.888500in}}%
\pgfpathlineto{\pgfqpoint{2.528485in}{0.766000in}}%
\pgfpathlineto{\pgfqpoint{2.574031in}{0.783500in}}%
\pgfpathlineto{\pgfqpoint{2.619578in}{1.501000in}}%
\pgfpathlineto{\pgfqpoint{2.665124in}{0.748500in}}%
\pgfpathlineto{\pgfqpoint{2.710670in}{2.008500in}}%
\pgfpathlineto{\pgfqpoint{2.756217in}{2.043500in}}%
\pgfpathlineto{\pgfqpoint{2.801763in}{1.011000in}}%
\pgfpathlineto{\pgfqpoint{2.847309in}{1.238500in}}%
\pgfpathlineto{\pgfqpoint{2.892856in}{3.776000in}}%
\pgfpathlineto{\pgfqpoint{2.938402in}{2.288500in}}%
\pgfpathlineto{\pgfqpoint{2.983949in}{2.218500in}}%
\pgfpathlineto{\pgfqpoint{3.029495in}{1.903500in}}%
\pgfpathlineto{\pgfqpoint{3.075041in}{2.306000in}}%
\pgfpathlineto{\pgfqpoint{3.120588in}{2.603500in}}%
\pgfpathlineto{\pgfqpoint{3.166134in}{3.111000in}}%
\pgfpathlineto{\pgfqpoint{3.211680in}{0.766000in}}%
\pgfpathlineto{\pgfqpoint{3.257227in}{0.766000in}}%
\pgfpathlineto{\pgfqpoint{3.302773in}{0.766000in}}%
\pgfpathlineto{\pgfqpoint{3.348320in}{2.988500in}}%
\pgfpathlineto{\pgfqpoint{3.393866in}{2.253500in}}%
\pgfpathlineto{\pgfqpoint{3.439412in}{1.711000in}}%
\pgfpathlineto{\pgfqpoint{3.484959in}{2.341000in}}%
\pgfpathlineto{\pgfqpoint{3.530505in}{3.496000in}}%
\pgfpathlineto{\pgfqpoint{3.576051in}{2.183500in}}%
\pgfpathlineto{\pgfqpoint{3.621598in}{2.463500in}}%
\pgfpathlineto{\pgfqpoint{3.667144in}{3.251000in}}%
\pgfpathlineto{\pgfqpoint{3.712691in}{2.883500in}}%
\pgfpathlineto{\pgfqpoint{3.758237in}{2.096000in}}%
\pgfpathlineto{\pgfqpoint{3.803783in}{2.201000in}}%
\pgfpathlineto{\pgfqpoint{3.849330in}{1.763500in}}%
\pgfpathlineto{\pgfqpoint{3.894876in}{2.131000in}}%
\pgfpathlineto{\pgfqpoint{3.940422in}{0.941000in}}%
\pgfpathlineto{\pgfqpoint{3.985969in}{0.696000in}}%
\pgfpathlineto{\pgfqpoint{4.031515in}{3.793500in}}%
\pgfpathlineto{\pgfqpoint{4.077062in}{1.728500in}}%
\pgfpathlineto{\pgfqpoint{4.122608in}{1.518500in}}%
\pgfpathlineto{\pgfqpoint{4.168154in}{1.868500in}}%
\pgfpathlineto{\pgfqpoint{4.213701in}{1.081000in}}%
\pgfpathlineto{\pgfqpoint{4.259247in}{2.253500in}}%
\pgfpathlineto{\pgfqpoint{4.304793in}{1.483500in}}%
\pgfpathlineto{\pgfqpoint{4.350340in}{1.921000in}}%
\pgfpathlineto{\pgfqpoint{4.395886in}{2.498500in}}%
\pgfpathlineto{\pgfqpoint{4.441433in}{2.008500in}}%
\pgfpathlineto{\pgfqpoint{4.486979in}{1.991000in}}%
\pgfpathlineto{\pgfqpoint{4.532525in}{4.056000in}}%
\pgfpathlineto{\pgfqpoint{4.578072in}{2.183500in}}%
\pgfpathlineto{\pgfqpoint{4.623618in}{4.056000in}}%
\pgfpathlineto{\pgfqpoint{4.669164in}{2.131000in}}%
\pgfpathlineto{\pgfqpoint{4.714711in}{1.991000in}}%
\pgfpathlineto{\pgfqpoint{4.760257in}{1.361000in}}%
\pgfpathlineto{\pgfqpoint{4.805803in}{0.748500in}}%
\pgfpathlineto{\pgfqpoint{4.851350in}{2.008500in}}%
\pgfpathlineto{\pgfqpoint{4.896896in}{2.323500in}}%
\pgfpathlineto{\pgfqpoint{4.942443in}{0.923500in}}%
\pgfpathlineto{\pgfqpoint{4.987989in}{0.801000in}}%
\pgfpathlineto{\pgfqpoint{5.033535in}{0.906000in}}%
\pgfpathlineto{\pgfqpoint{5.079082in}{1.833500in}}%
\pgfpathlineto{\pgfqpoint{5.124628in}{2.201000in}}%
\pgfpathlineto{\pgfqpoint{5.170174in}{0.713500in}}%
\pgfpathlineto{\pgfqpoint{5.215721in}{0.713500in}}%
\pgfpathlineto{\pgfqpoint{5.261267in}{1.501000in}}%
\pgfpathlineto{\pgfqpoint{5.306814in}{0.731000in}}%
\pgfpathlineto{\pgfqpoint{5.352360in}{0.731000in}}%
\pgfpathlineto{\pgfqpoint{5.397906in}{2.008500in}}%
\pgfpathlineto{\pgfqpoint{5.443453in}{0.958500in}}%
\pgfpathlineto{\pgfqpoint{5.488999in}{2.183500in}}%
\pgfpathlineto{\pgfqpoint{5.534545in}{0.713500in}}%
\pgfusepath{stroke}%
\end{pgfscope}%
\begin{pgfscope}%
\pgfpathrectangle{\pgfqpoint{0.800000in}{0.528000in}}{\pgfqpoint{4.960000in}{3.696000in}} %
\pgfusepath{clip}%
\pgfsetbuttcap%
\pgfsetmiterjoin%
\definecolor{currentfill}{rgb}{1.000000,1.000000,1.000000}%
\pgfsetfillcolor{currentfill}%
\pgfsetlinewidth{1.003750pt}%
\definecolor{currentstroke}{rgb}{1.000000,1.000000,1.000000}%
\pgfsetstrokecolor{currentstroke}%
\pgfsetdash{}{0pt}%
\pgfpathmoveto{\pgfqpoint{2.113319in}{1.633040in}}%
\pgfpathlineto{\pgfqpoint{2.210341in}{2.415887in}}%
\pgfpathlineto{\pgfqpoint{1.966681in}{2.446085in}}%
\pgfpathlineto{\pgfqpoint{1.869659in}{1.663238in}}%
\pgfpathclose%
\pgfusepath{stroke,fill}%
\end{pgfscope}%
\begin{pgfscope}%
\pgftext[x=2.036347in,y=1.698560in,left,base,rotate=82.935033]{\sffamily\fontsize{10.000000}{12.000000}\selectfont \(\displaystyle \alpha =\) 0.001}%
\end{pgfscope}%
\begin{pgfscope}%
\pgfpathrectangle{\pgfqpoint{0.800000in}{0.528000in}}{\pgfqpoint{4.960000in}{3.696000in}} %
\pgfusepath{clip}%
\pgfsetbuttcap%
\pgfsetmiterjoin%
\definecolor{currentfill}{rgb}{1.000000,1.000000,1.000000}%
\pgfsetfillcolor{currentfill}%
\pgfsetlinewidth{1.003750pt}%
\definecolor{currentstroke}{rgb}{1.000000,1.000000,1.000000}%
\pgfsetstrokecolor{currentstroke}%
\pgfsetdash{}{0pt}%
\pgfpathmoveto{\pgfqpoint{3.358753in}{1.619575in}}%
\pgfpathlineto{\pgfqpoint{3.444907in}{2.314727in}}%
\pgfpathlineto{\pgfqpoint{3.201247in}{2.344925in}}%
\pgfpathlineto{\pgfqpoint{3.115093in}{1.649773in}}%
\pgfpathclose%
\pgfusepath{stroke,fill}%
\end{pgfscope}%
\begin{pgfscope}%
\definecolor{textcolor}{rgb}{0.300000,0.300000,0.300000}%
\pgfsetstrokecolor{textcolor}%
\pgfsetfillcolor{textcolor}%
\pgftext[x=3.281782in,y=1.685095in,left,base,rotate=82.935033]{\color{textcolor}\sffamily\fontsize{10.000000}{12.000000}\selectfont \(\displaystyle \alpha =\) 0.01}%
\end{pgfscope}%
\begin{pgfscope}%
\pgfpathrectangle{\pgfqpoint{0.800000in}{0.528000in}}{\pgfqpoint{4.960000in}{3.696000in}} %
\pgfusepath{clip}%
\pgfsetbuttcap%
\pgfsetmiterjoin%
\definecolor{currentfill}{rgb}{1.000000,1.000000,1.000000}%
\pgfsetfillcolor{currentfill}%
\pgfsetlinewidth{1.003750pt}%
\definecolor{currentstroke}{rgb}{1.000000,1.000000,1.000000}%
\pgfsetstrokecolor{currentstroke}%
\pgfsetdash{}{0pt}%
\pgfpathmoveto{\pgfqpoint{4.635983in}{3.179440in}}%
\pgfpathlineto{\pgfqpoint{4.649481in}{3.791396in}}%
\pgfpathlineto{\pgfqpoint{4.404017in}{3.796810in}}%
\pgfpathlineto{\pgfqpoint{4.390519in}{3.184854in}}%
\pgfpathclose%
\pgfusepath{stroke,fill}%
\end{pgfscope}%
\begin{pgfscope}%
\definecolor{textcolor}{rgb}{0.600000,0.600000,0.600000}%
\pgfsetstrokecolor{textcolor}%
\pgfsetfillcolor{textcolor}%
\pgftext[x=4.552784in,y=3.236844in,left,base,rotate=88.736469]{\color{textcolor}\sffamily\fontsize{10.000000}{12.000000}\selectfont \(\displaystyle \alpha =\) 0.1}%
\end{pgfscope}%
\begin{pgfscope}%
\pgfsetrectcap%
\pgfsetmiterjoin%
\pgfsetlinewidth{0.803000pt}%
\definecolor{currentstroke}{rgb}{0.000000,0.000000,0.000000}%
\pgfsetstrokecolor{currentstroke}%
\pgfsetdash{}{0pt}%
\pgfpathmoveto{\pgfqpoint{0.800000in}{0.528000in}}%
\pgfpathlineto{\pgfqpoint{0.800000in}{4.224000in}}%
\pgfusepath{stroke}%
\end{pgfscope}%
\begin{pgfscope}%
\pgfsetrectcap%
\pgfsetmiterjoin%
\pgfsetlinewidth{0.803000pt}%
\definecolor{currentstroke}{rgb}{0.000000,0.000000,0.000000}%
\pgfsetstrokecolor{currentstroke}%
\pgfsetdash{}{0pt}%
\pgfpathmoveto{\pgfqpoint{5.760000in}{0.528000in}}%
\pgfpathlineto{\pgfqpoint{5.760000in}{4.224000in}}%
\pgfusepath{stroke}%
\end{pgfscope}%
\begin{pgfscope}%
\pgfsetrectcap%
\pgfsetmiterjoin%
\pgfsetlinewidth{0.803000pt}%
\definecolor{currentstroke}{rgb}{0.000000,0.000000,0.000000}%
\pgfsetstrokecolor{currentstroke}%
\pgfsetdash{}{0pt}%
\pgfpathmoveto{\pgfqpoint{0.800000in}{0.528000in}}%
\pgfpathlineto{\pgfqpoint{5.760000in}{0.528000in}}%
\pgfusepath{stroke}%
\end{pgfscope}%
\begin{pgfscope}%
\pgfsetrectcap%
\pgfsetmiterjoin%
\pgfsetlinewidth{0.803000pt}%
\definecolor{currentstroke}{rgb}{0.000000,0.000000,0.000000}%
\pgfsetstrokecolor{currentstroke}%
\pgfsetdash{}{0pt}%
\pgfpathmoveto{\pgfqpoint{0.800000in}{4.224000in}}%
\pgfpathlineto{\pgfqpoint{5.760000in}{4.224000in}}%
\pgfusepath{stroke}%
\end{pgfscope}%
\begin{pgfscope}%
\pgftext[x=3.280000in,y=4.307333in,,base]{\sffamily\fontsize{12.000000}{14.400000}\selectfont Q Learning Function Approximator Results}%
\end{pgfscope}%
\end{pgfpicture}%
\makeatother%
\endgroup%
} \\
\end{centering}
Note: Because Keras took significantly longer to run, I didn't collect enough
data to average, which explains the noise. However, I observed that the
smallest learning rate ($\alpha = 0.001$) consistently scored $> 150$ on
subsequent runs.
\begin{itemize}
    \item The results for different $\alpha$ can be summarized as follows:
    \begin{itemize}
        \item The largest $\alpha$ performed poorly initially, but learned a
            policy that generally performed worse than the other two $\alpha$s.
        \item The middle $\alpha$ Learned quickly, and performed averagely
            compared to the other two $\alpha$s.
        \item The smallest $\alpha$ learned quickly, and performed relatively
            well, converging to an almost optimal policy.
    \end{itemize}
    \item This clearly outperforms my manual implementation, but when I
        reduced the model to approximately the size of my manually constructed
        model, it performed similarly. This suggests that the size of the
        neural network and intelligent control of the learning rate can
        significantly affect the performance of the algorithm.
\end{itemize}
\end{document}
