\documentclass[a4paper]{article}
\setlength\parindent{0pt}

\usepackage{pgfplots}
\usepackage{amsthm, amsmath, amssymb, verbatim, enumerate, mathtools, algorithm}
\usepackage{pgf}
\usepackage{hyperref}
\def\labelitemi{--}
\pgfplotsset{compat=newest}

\pagestyle{empty}

\title{Policy Gradient Demo}
\author{Rishikesh Vaishnav}
\begin{document}
\maketitle
\section*{Monte Carlo Implementation}
\subsection*{Code}
\begin{itemize}
    \item The code for this project is available at: 
\url{https://github.com/rish987/Reinforcement-Learning/blob/master/demos/policy_gradient/code/policy_gradient.py}.
\end{itemize}
\subsection*{Implementation Details}
\begin{itemize}
    \item Each state-action pair is converted to the feature vector $x(s, a)$.
        Letting $S_{obs}$ and $S_{act}$ be the size of the observation and
        action spaces, respectively, the size of the vector is $S_{obs} \times
        S_{act}$, where all features are $0$ except for the $S_{obs}$ features
        starting at index $S_{obs} \times a$, which are set to the
        environment's parameterization of $s$.
    \begin{itemize}
        \item In this case, $S_{obs} = 4$ and $S_{act} = 2$.
    \end{itemize}
    \item The policy function $\pi(a | s, \theta)$ performs the softmax on a
        parameterized linear mapping of feature vectors:
        \begin{align*}
            \pi(a | s, \theta) &= \frac{e^{\theta^T x(s, a)}}
            {\sum_{b} e^{\theta^T x(s, b)}}\\
        \end{align*}
    \item The gradient of this policy function, used in the parameter update,
        was found to be:
        \begin{align*}
            \nabla \pi(a | s, \theta) &= 
            \frac{e^{\theta^T x(s, a)}}{(\sum_{b} e^{\theta^T x(s, b)}) ^2}
            \left(x(s, a) \sum_{b}
            e^{\theta^T x(s, b)} - \left(\sum_{b} e^{\theta^T x(s, b)} 
            x(s, b)\right)
            \right)
        \end{align*}
\end{itemize}
\subsection*{Results}
\begin{centering}
    \scalebox{0.6}{%% Creator: Matplotlib, PGF backend
%%
%% To include the figure in your LaTeX document, write
%%   \input{<filename>.pgf}
%%
%% Make sure the required packages are loaded in your preamble
%%   \usepackage{pgf}
%%
%% Figures using additional raster images can only be included by \input if
%% they are in the same directory as the main LaTeX file. For loading figures
%% from other directories you can use the `import` package
%%   \usepackage{import}
%% and then include the figures with
%%   \import{<path to file>}{<filename>.pgf}
%%
%% Matplotlib used the following preamble
%%   \usepackage{fontspec}
%%   \setmainfont{DejaVu Serif}
%%   \setsansfont{DejaVu Sans}
%%   \setmonofont{DejaVu Sans Mono}
%%
\begingroup%
\makeatletter%
\begin{pgfpicture}%
\pgfpathrectangle{\pgfpointorigin}{\pgfqpoint{6.400000in}{4.800000in}}%
\pgfusepath{use as bounding box, clip}%
\begin{pgfscope}%
\pgfsetbuttcap%
\pgfsetmiterjoin%
\definecolor{currentfill}{rgb}{1.000000,1.000000,1.000000}%
\pgfsetfillcolor{currentfill}%
\pgfsetlinewidth{0.000000pt}%
\definecolor{currentstroke}{rgb}{1.000000,1.000000,1.000000}%
\pgfsetstrokecolor{currentstroke}%
\pgfsetdash{}{0pt}%
\pgfpathmoveto{\pgfqpoint{0.000000in}{0.000000in}}%
\pgfpathlineto{\pgfqpoint{6.400000in}{0.000000in}}%
\pgfpathlineto{\pgfqpoint{6.400000in}{4.800000in}}%
\pgfpathlineto{\pgfqpoint{0.000000in}{4.800000in}}%
\pgfpathclose%
\pgfusepath{fill}%
\end{pgfscope}%
\begin{pgfscope}%
\pgfsetbuttcap%
\pgfsetmiterjoin%
\definecolor{currentfill}{rgb}{1.000000,1.000000,1.000000}%
\pgfsetfillcolor{currentfill}%
\pgfsetlinewidth{0.000000pt}%
\definecolor{currentstroke}{rgb}{0.000000,0.000000,0.000000}%
\pgfsetstrokecolor{currentstroke}%
\pgfsetstrokeopacity{0.000000}%
\pgfsetdash{}{0pt}%
\pgfpathmoveto{\pgfqpoint{0.800000in}{0.528000in}}%
\pgfpathlineto{\pgfqpoint{5.760000in}{0.528000in}}%
\pgfpathlineto{\pgfqpoint{5.760000in}{4.224000in}}%
\pgfpathlineto{\pgfqpoint{0.800000in}{4.224000in}}%
\pgfpathclose%
\pgfusepath{fill}%
\end{pgfscope}%
\begin{pgfscope}%
\pgfsetbuttcap%
\pgfsetroundjoin%
\definecolor{currentfill}{rgb}{0.000000,0.000000,0.000000}%
\pgfsetfillcolor{currentfill}%
\pgfsetlinewidth{0.803000pt}%
\definecolor{currentstroke}{rgb}{0.000000,0.000000,0.000000}%
\pgfsetstrokecolor{currentstroke}%
\pgfsetdash{}{0pt}%
\pgfsys@defobject{currentmarker}{\pgfqpoint{0.000000in}{-0.048611in}}{\pgfqpoint{0.000000in}{0.000000in}}{%
\pgfpathmoveto{\pgfqpoint{0.000000in}{0.000000in}}%
\pgfpathlineto{\pgfqpoint{0.000000in}{-0.048611in}}%
\pgfusepath{stroke,fill}%
}%
\begin{pgfscope}%
\pgfsys@transformshift{1.014154in}{0.528000in}%
\pgfsys@useobject{currentmarker}{}%
\end{pgfscope}%
\end{pgfscope}%
\begin{pgfscope}%
\pgftext[x=1.014154in,y=0.430778in,,top]{\sffamily\fontsize{10.000000}{12.000000}\selectfont \(\displaystyle 0\)}%
\end{pgfscope}%
\begin{pgfscope}%
\pgfsetbuttcap%
\pgfsetroundjoin%
\definecolor{currentfill}{rgb}{0.000000,0.000000,0.000000}%
\pgfsetfillcolor{currentfill}%
\pgfsetlinewidth{0.803000pt}%
\definecolor{currentstroke}{rgb}{0.000000,0.000000,0.000000}%
\pgfsetstrokecolor{currentstroke}%
\pgfsetdash{}{0pt}%
\pgfsys@defobject{currentmarker}{\pgfqpoint{0.000000in}{-0.048611in}}{\pgfqpoint{0.000000in}{0.000000in}}{%
\pgfpathmoveto{\pgfqpoint{0.000000in}{0.000000in}}%
\pgfpathlineto{\pgfqpoint{0.000000in}{-0.048611in}}%
\pgfusepath{stroke,fill}%
}%
\begin{pgfscope}%
\pgfsys@transformshift{1.579203in}{0.528000in}%
\pgfsys@useobject{currentmarker}{}%
\end{pgfscope}%
\end{pgfscope}%
\begin{pgfscope}%
\pgftext[x=1.579203in,y=0.430778in,,top]{\sffamily\fontsize{10.000000}{12.000000}\selectfont \(\displaystyle 50\)}%
\end{pgfscope}%
\begin{pgfscope}%
\pgfsetbuttcap%
\pgfsetroundjoin%
\definecolor{currentfill}{rgb}{0.000000,0.000000,0.000000}%
\pgfsetfillcolor{currentfill}%
\pgfsetlinewidth{0.803000pt}%
\definecolor{currentstroke}{rgb}{0.000000,0.000000,0.000000}%
\pgfsetstrokecolor{currentstroke}%
\pgfsetdash{}{0pt}%
\pgfsys@defobject{currentmarker}{\pgfqpoint{0.000000in}{-0.048611in}}{\pgfqpoint{0.000000in}{0.000000in}}{%
\pgfpathmoveto{\pgfqpoint{0.000000in}{0.000000in}}%
\pgfpathlineto{\pgfqpoint{0.000000in}{-0.048611in}}%
\pgfusepath{stroke,fill}%
}%
\begin{pgfscope}%
\pgfsys@transformshift{2.144252in}{0.528000in}%
\pgfsys@useobject{currentmarker}{}%
\end{pgfscope}%
\end{pgfscope}%
\begin{pgfscope}%
\pgftext[x=2.144252in,y=0.430778in,,top]{\sffamily\fontsize{10.000000}{12.000000}\selectfont \(\displaystyle 100\)}%
\end{pgfscope}%
\begin{pgfscope}%
\pgfsetbuttcap%
\pgfsetroundjoin%
\definecolor{currentfill}{rgb}{0.000000,0.000000,0.000000}%
\pgfsetfillcolor{currentfill}%
\pgfsetlinewidth{0.803000pt}%
\definecolor{currentstroke}{rgb}{0.000000,0.000000,0.000000}%
\pgfsetstrokecolor{currentstroke}%
\pgfsetdash{}{0pt}%
\pgfsys@defobject{currentmarker}{\pgfqpoint{0.000000in}{-0.048611in}}{\pgfqpoint{0.000000in}{0.000000in}}{%
\pgfpathmoveto{\pgfqpoint{0.000000in}{0.000000in}}%
\pgfpathlineto{\pgfqpoint{0.000000in}{-0.048611in}}%
\pgfusepath{stroke,fill}%
}%
\begin{pgfscope}%
\pgfsys@transformshift{2.709301in}{0.528000in}%
\pgfsys@useobject{currentmarker}{}%
\end{pgfscope}%
\end{pgfscope}%
\begin{pgfscope}%
\pgftext[x=2.709301in,y=0.430778in,,top]{\sffamily\fontsize{10.000000}{12.000000}\selectfont \(\displaystyle 150\)}%
\end{pgfscope}%
\begin{pgfscope}%
\pgfsetbuttcap%
\pgfsetroundjoin%
\definecolor{currentfill}{rgb}{0.000000,0.000000,0.000000}%
\pgfsetfillcolor{currentfill}%
\pgfsetlinewidth{0.803000pt}%
\definecolor{currentstroke}{rgb}{0.000000,0.000000,0.000000}%
\pgfsetstrokecolor{currentstroke}%
\pgfsetdash{}{0pt}%
\pgfsys@defobject{currentmarker}{\pgfqpoint{0.000000in}{-0.048611in}}{\pgfqpoint{0.000000in}{0.000000in}}{%
\pgfpathmoveto{\pgfqpoint{0.000000in}{0.000000in}}%
\pgfpathlineto{\pgfqpoint{0.000000in}{-0.048611in}}%
\pgfusepath{stroke,fill}%
}%
\begin{pgfscope}%
\pgfsys@transformshift{3.274350in}{0.528000in}%
\pgfsys@useobject{currentmarker}{}%
\end{pgfscope}%
\end{pgfscope}%
\begin{pgfscope}%
\pgftext[x=3.274350in,y=0.430778in,,top]{\sffamily\fontsize{10.000000}{12.000000}\selectfont \(\displaystyle 200\)}%
\end{pgfscope}%
\begin{pgfscope}%
\pgfsetbuttcap%
\pgfsetroundjoin%
\definecolor{currentfill}{rgb}{0.000000,0.000000,0.000000}%
\pgfsetfillcolor{currentfill}%
\pgfsetlinewidth{0.803000pt}%
\definecolor{currentstroke}{rgb}{0.000000,0.000000,0.000000}%
\pgfsetstrokecolor{currentstroke}%
\pgfsetdash{}{0pt}%
\pgfsys@defobject{currentmarker}{\pgfqpoint{0.000000in}{-0.048611in}}{\pgfqpoint{0.000000in}{0.000000in}}{%
\pgfpathmoveto{\pgfqpoint{0.000000in}{0.000000in}}%
\pgfpathlineto{\pgfqpoint{0.000000in}{-0.048611in}}%
\pgfusepath{stroke,fill}%
}%
\begin{pgfscope}%
\pgfsys@transformshift{3.839398in}{0.528000in}%
\pgfsys@useobject{currentmarker}{}%
\end{pgfscope}%
\end{pgfscope}%
\begin{pgfscope}%
\pgftext[x=3.839398in,y=0.430778in,,top]{\sffamily\fontsize{10.000000}{12.000000}\selectfont \(\displaystyle 250\)}%
\end{pgfscope}%
\begin{pgfscope}%
\pgfsetbuttcap%
\pgfsetroundjoin%
\definecolor{currentfill}{rgb}{0.000000,0.000000,0.000000}%
\pgfsetfillcolor{currentfill}%
\pgfsetlinewidth{0.803000pt}%
\definecolor{currentstroke}{rgb}{0.000000,0.000000,0.000000}%
\pgfsetstrokecolor{currentstroke}%
\pgfsetdash{}{0pt}%
\pgfsys@defobject{currentmarker}{\pgfqpoint{0.000000in}{-0.048611in}}{\pgfqpoint{0.000000in}{0.000000in}}{%
\pgfpathmoveto{\pgfqpoint{0.000000in}{0.000000in}}%
\pgfpathlineto{\pgfqpoint{0.000000in}{-0.048611in}}%
\pgfusepath{stroke,fill}%
}%
\begin{pgfscope}%
\pgfsys@transformshift{4.404447in}{0.528000in}%
\pgfsys@useobject{currentmarker}{}%
\end{pgfscope}%
\end{pgfscope}%
\begin{pgfscope}%
\pgftext[x=4.404447in,y=0.430778in,,top]{\sffamily\fontsize{10.000000}{12.000000}\selectfont \(\displaystyle 300\)}%
\end{pgfscope}%
\begin{pgfscope}%
\pgfsetbuttcap%
\pgfsetroundjoin%
\definecolor{currentfill}{rgb}{0.000000,0.000000,0.000000}%
\pgfsetfillcolor{currentfill}%
\pgfsetlinewidth{0.803000pt}%
\definecolor{currentstroke}{rgb}{0.000000,0.000000,0.000000}%
\pgfsetstrokecolor{currentstroke}%
\pgfsetdash{}{0pt}%
\pgfsys@defobject{currentmarker}{\pgfqpoint{0.000000in}{-0.048611in}}{\pgfqpoint{0.000000in}{0.000000in}}{%
\pgfpathmoveto{\pgfqpoint{0.000000in}{0.000000in}}%
\pgfpathlineto{\pgfqpoint{0.000000in}{-0.048611in}}%
\pgfusepath{stroke,fill}%
}%
\begin{pgfscope}%
\pgfsys@transformshift{4.969496in}{0.528000in}%
\pgfsys@useobject{currentmarker}{}%
\end{pgfscope}%
\end{pgfscope}%
\begin{pgfscope}%
\pgftext[x=4.969496in,y=0.430778in,,top]{\sffamily\fontsize{10.000000}{12.000000}\selectfont \(\displaystyle 350\)}%
\end{pgfscope}%
\begin{pgfscope}%
\pgfsetbuttcap%
\pgfsetroundjoin%
\definecolor{currentfill}{rgb}{0.000000,0.000000,0.000000}%
\pgfsetfillcolor{currentfill}%
\pgfsetlinewidth{0.803000pt}%
\definecolor{currentstroke}{rgb}{0.000000,0.000000,0.000000}%
\pgfsetstrokecolor{currentstroke}%
\pgfsetdash{}{0pt}%
\pgfsys@defobject{currentmarker}{\pgfqpoint{0.000000in}{-0.048611in}}{\pgfqpoint{0.000000in}{0.000000in}}{%
\pgfpathmoveto{\pgfqpoint{0.000000in}{0.000000in}}%
\pgfpathlineto{\pgfqpoint{0.000000in}{-0.048611in}}%
\pgfusepath{stroke,fill}%
}%
\begin{pgfscope}%
\pgfsys@transformshift{5.534545in}{0.528000in}%
\pgfsys@useobject{currentmarker}{}%
\end{pgfscope}%
\end{pgfscope}%
\begin{pgfscope}%
\pgftext[x=5.534545in,y=0.430778in,,top]{\sffamily\fontsize{10.000000}{12.000000}\selectfont \(\displaystyle 400\)}%
\end{pgfscope}%
\begin{pgfscope}%
\pgftext[x=3.280000in,y=0.240809in,,top]{\sffamily\fontsize{10.000000}{12.000000}\selectfont Episode}%
\end{pgfscope}%
\begin{pgfscope}%
\pgfsetbuttcap%
\pgfsetroundjoin%
\definecolor{currentfill}{rgb}{0.000000,0.000000,0.000000}%
\pgfsetfillcolor{currentfill}%
\pgfsetlinewidth{0.803000pt}%
\definecolor{currentstroke}{rgb}{0.000000,0.000000,0.000000}%
\pgfsetstrokecolor{currentstroke}%
\pgfsetdash{}{0pt}%
\pgfsys@defobject{currentmarker}{\pgfqpoint{-0.048611in}{0.000000in}}{\pgfqpoint{0.000000in}{0.000000in}}{%
\pgfpathmoveto{\pgfqpoint{0.000000in}{0.000000in}}%
\pgfpathlineto{\pgfqpoint{-0.048611in}{0.000000in}}%
\pgfusepath{stroke,fill}%
}%
\begin{pgfscope}%
\pgfsys@transformshift{0.800000in}{0.821677in}%
\pgfsys@useobject{currentmarker}{}%
\end{pgfscope}%
\end{pgfscope}%
\begin{pgfscope}%
\pgftext[x=0.563888in,y=0.768915in,left,base]{\sffamily\fontsize{10.000000}{12.000000}\selectfont \(\displaystyle 25\)}%
\end{pgfscope}%
\begin{pgfscope}%
\pgfsetbuttcap%
\pgfsetroundjoin%
\definecolor{currentfill}{rgb}{0.000000,0.000000,0.000000}%
\pgfsetfillcolor{currentfill}%
\pgfsetlinewidth{0.803000pt}%
\definecolor{currentstroke}{rgb}{0.000000,0.000000,0.000000}%
\pgfsetstrokecolor{currentstroke}%
\pgfsetdash{}{0pt}%
\pgfsys@defobject{currentmarker}{\pgfqpoint{-0.048611in}{0.000000in}}{\pgfqpoint{0.000000in}{0.000000in}}{%
\pgfpathmoveto{\pgfqpoint{0.000000in}{0.000000in}}%
\pgfpathlineto{\pgfqpoint{-0.048611in}{0.000000in}}%
\pgfusepath{stroke,fill}%
}%
\begin{pgfscope}%
\pgfsys@transformshift{0.800000in}{1.283723in}%
\pgfsys@useobject{currentmarker}{}%
\end{pgfscope}%
\end{pgfscope}%
\begin{pgfscope}%
\pgftext[x=0.563888in,y=1.230961in,left,base]{\sffamily\fontsize{10.000000}{12.000000}\selectfont \(\displaystyle 50\)}%
\end{pgfscope}%
\begin{pgfscope}%
\pgfsetbuttcap%
\pgfsetroundjoin%
\definecolor{currentfill}{rgb}{0.000000,0.000000,0.000000}%
\pgfsetfillcolor{currentfill}%
\pgfsetlinewidth{0.803000pt}%
\definecolor{currentstroke}{rgb}{0.000000,0.000000,0.000000}%
\pgfsetstrokecolor{currentstroke}%
\pgfsetdash{}{0pt}%
\pgfsys@defobject{currentmarker}{\pgfqpoint{-0.048611in}{0.000000in}}{\pgfqpoint{0.000000in}{0.000000in}}{%
\pgfpathmoveto{\pgfqpoint{0.000000in}{0.000000in}}%
\pgfpathlineto{\pgfqpoint{-0.048611in}{0.000000in}}%
\pgfusepath{stroke,fill}%
}%
\begin{pgfscope}%
\pgfsys@transformshift{0.800000in}{1.745769in}%
\pgfsys@useobject{currentmarker}{}%
\end{pgfscope}%
\end{pgfscope}%
\begin{pgfscope}%
\pgftext[x=0.563888in,y=1.693007in,left,base]{\sffamily\fontsize{10.000000}{12.000000}\selectfont \(\displaystyle 75\)}%
\end{pgfscope}%
\begin{pgfscope}%
\pgfsetbuttcap%
\pgfsetroundjoin%
\definecolor{currentfill}{rgb}{0.000000,0.000000,0.000000}%
\pgfsetfillcolor{currentfill}%
\pgfsetlinewidth{0.803000pt}%
\definecolor{currentstroke}{rgb}{0.000000,0.000000,0.000000}%
\pgfsetstrokecolor{currentstroke}%
\pgfsetdash{}{0pt}%
\pgfsys@defobject{currentmarker}{\pgfqpoint{-0.048611in}{0.000000in}}{\pgfqpoint{0.000000in}{0.000000in}}{%
\pgfpathmoveto{\pgfqpoint{0.000000in}{0.000000in}}%
\pgfpathlineto{\pgfqpoint{-0.048611in}{0.000000in}}%
\pgfusepath{stroke,fill}%
}%
\begin{pgfscope}%
\pgfsys@transformshift{0.800000in}{2.207815in}%
\pgfsys@useobject{currentmarker}{}%
\end{pgfscope}%
\end{pgfscope}%
\begin{pgfscope}%
\pgftext[x=0.494444in,y=2.155054in,left,base]{\sffamily\fontsize{10.000000}{12.000000}\selectfont \(\displaystyle 100\)}%
\end{pgfscope}%
\begin{pgfscope}%
\pgfsetbuttcap%
\pgfsetroundjoin%
\definecolor{currentfill}{rgb}{0.000000,0.000000,0.000000}%
\pgfsetfillcolor{currentfill}%
\pgfsetlinewidth{0.803000pt}%
\definecolor{currentstroke}{rgb}{0.000000,0.000000,0.000000}%
\pgfsetstrokecolor{currentstroke}%
\pgfsetdash{}{0pt}%
\pgfsys@defobject{currentmarker}{\pgfqpoint{-0.048611in}{0.000000in}}{\pgfqpoint{0.000000in}{0.000000in}}{%
\pgfpathmoveto{\pgfqpoint{0.000000in}{0.000000in}}%
\pgfpathlineto{\pgfqpoint{-0.048611in}{0.000000in}}%
\pgfusepath{stroke,fill}%
}%
\begin{pgfscope}%
\pgfsys@transformshift{0.800000in}{2.669861in}%
\pgfsys@useobject{currentmarker}{}%
\end{pgfscope}%
\end{pgfscope}%
\begin{pgfscope}%
\pgftext[x=0.494444in,y=2.617100in,left,base]{\sffamily\fontsize{10.000000}{12.000000}\selectfont \(\displaystyle 125\)}%
\end{pgfscope}%
\begin{pgfscope}%
\pgfsetbuttcap%
\pgfsetroundjoin%
\definecolor{currentfill}{rgb}{0.000000,0.000000,0.000000}%
\pgfsetfillcolor{currentfill}%
\pgfsetlinewidth{0.803000pt}%
\definecolor{currentstroke}{rgb}{0.000000,0.000000,0.000000}%
\pgfsetstrokecolor{currentstroke}%
\pgfsetdash{}{0pt}%
\pgfsys@defobject{currentmarker}{\pgfqpoint{-0.048611in}{0.000000in}}{\pgfqpoint{0.000000in}{0.000000in}}{%
\pgfpathmoveto{\pgfqpoint{0.000000in}{0.000000in}}%
\pgfpathlineto{\pgfqpoint{-0.048611in}{0.000000in}}%
\pgfusepath{stroke,fill}%
}%
\begin{pgfscope}%
\pgfsys@transformshift{0.800000in}{3.131908in}%
\pgfsys@useobject{currentmarker}{}%
\end{pgfscope}%
\end{pgfscope}%
\begin{pgfscope}%
\pgftext[x=0.494444in,y=3.079146in,left,base]{\sffamily\fontsize{10.000000}{12.000000}\selectfont \(\displaystyle 150\)}%
\end{pgfscope}%
\begin{pgfscope}%
\pgfsetbuttcap%
\pgfsetroundjoin%
\definecolor{currentfill}{rgb}{0.000000,0.000000,0.000000}%
\pgfsetfillcolor{currentfill}%
\pgfsetlinewidth{0.803000pt}%
\definecolor{currentstroke}{rgb}{0.000000,0.000000,0.000000}%
\pgfsetstrokecolor{currentstroke}%
\pgfsetdash{}{0pt}%
\pgfsys@defobject{currentmarker}{\pgfqpoint{-0.048611in}{0.000000in}}{\pgfqpoint{0.000000in}{0.000000in}}{%
\pgfpathmoveto{\pgfqpoint{0.000000in}{0.000000in}}%
\pgfpathlineto{\pgfqpoint{-0.048611in}{0.000000in}}%
\pgfusepath{stroke,fill}%
}%
\begin{pgfscope}%
\pgfsys@transformshift{0.800000in}{3.593954in}%
\pgfsys@useobject{currentmarker}{}%
\end{pgfscope}%
\end{pgfscope}%
\begin{pgfscope}%
\pgftext[x=0.494444in,y=3.541192in,left,base]{\sffamily\fontsize{10.000000}{12.000000}\selectfont \(\displaystyle 175\)}%
\end{pgfscope}%
\begin{pgfscope}%
\pgfsetbuttcap%
\pgfsetroundjoin%
\definecolor{currentfill}{rgb}{0.000000,0.000000,0.000000}%
\pgfsetfillcolor{currentfill}%
\pgfsetlinewidth{0.803000pt}%
\definecolor{currentstroke}{rgb}{0.000000,0.000000,0.000000}%
\pgfsetstrokecolor{currentstroke}%
\pgfsetdash{}{0pt}%
\pgfsys@defobject{currentmarker}{\pgfqpoint{-0.048611in}{0.000000in}}{\pgfqpoint{0.000000in}{0.000000in}}{%
\pgfpathmoveto{\pgfqpoint{0.000000in}{0.000000in}}%
\pgfpathlineto{\pgfqpoint{-0.048611in}{0.000000in}}%
\pgfusepath{stroke,fill}%
}%
\begin{pgfscope}%
\pgfsys@transformshift{0.800000in}{4.056000in}%
\pgfsys@useobject{currentmarker}{}%
\end{pgfscope}%
\end{pgfscope}%
\begin{pgfscope}%
\pgftext[x=0.494444in,y=4.003238in,left,base]{\sffamily\fontsize{10.000000}{12.000000}\selectfont \(\displaystyle 200\)}%
\end{pgfscope}%
\begin{pgfscope}%
\pgftext[x=0.438888in,y=2.376000in,,bottom,rotate=90.000000]{\sffamily\fontsize{10.000000}{12.000000}\selectfont Average Episode Length (10 Runs)}%
\end{pgfscope}%
\begin{pgfscope}%
\pgfpathrectangle{\pgfqpoint{0.800000in}{0.528000in}}{\pgfqpoint{4.960000in}{3.696000in}} %
\pgfusepath{clip}%
\pgfsetrectcap%
\pgfsetroundjoin%
\pgfsetlinewidth{1.505625pt}%
\definecolor{currentstroke}{rgb}{0.000000,0.000000,0.000000}%
\pgfsetstrokecolor{currentstroke}%
\pgfsetdash{}{0pt}%
\pgfpathmoveto{\pgfqpoint{1.025455in}{0.736660in}}%
\pgfpathlineto{\pgfqpoint{1.036756in}{0.793954in}}%
\pgfpathlineto{\pgfqpoint{1.048057in}{0.696000in}}%
\pgfpathlineto{\pgfqpoint{1.059357in}{0.758838in}}%
\pgfpathlineto{\pgfqpoint{1.070658in}{0.801347in}}%
\pgfpathlineto{\pgfqpoint{1.081959in}{0.938112in}}%
\pgfpathlineto{\pgfqpoint{1.093260in}{0.856792in}}%
\pgfpathlineto{\pgfqpoint{1.104561in}{0.745901in}}%
\pgfpathlineto{\pgfqpoint{1.115862in}{0.751446in}}%
\pgfpathlineto{\pgfqpoint{1.127163in}{0.788409in}}%
\pgfpathlineto{\pgfqpoint{1.138464in}{0.732964in}}%
\pgfpathlineto{\pgfqpoint{1.149765in}{0.864185in}}%
\pgfpathlineto{\pgfqpoint{1.161066in}{0.738508in}}%
\pgfpathlineto{\pgfqpoint{1.172367in}{0.821677in}}%
\pgfpathlineto{\pgfqpoint{1.183668in}{0.773624in}}%
\pgfpathlineto{\pgfqpoint{1.194969in}{0.880818in}}%
\pgfpathlineto{\pgfqpoint{1.206270in}{0.890059in}}%
\pgfpathlineto{\pgfqpoint{1.217571in}{0.912238in}}%
\pgfpathlineto{\pgfqpoint{1.228872in}{1.089663in}}%
\pgfpathlineto{\pgfqpoint{1.240173in}{1.023129in}}%
\pgfpathlineto{\pgfqpoint{1.251474in}{0.939960in}}%
\pgfpathlineto{\pgfqpoint{1.262775in}{0.936264in}}%
\pgfpathlineto{\pgfqpoint{1.274076in}{0.963987in}}%
\pgfpathlineto{\pgfqpoint{1.285377in}{0.845703in}}%
\pgfpathlineto{\pgfqpoint{1.296678in}{0.997254in}}%
\pgfpathlineto{\pgfqpoint{1.319280in}{0.869729in}}%
\pgfpathlineto{\pgfqpoint{1.330581in}{0.984317in}}%
\pgfpathlineto{\pgfqpoint{1.341882in}{0.934416in}}%
\pgfpathlineto{\pgfqpoint{1.364484in}{0.877122in}}%
\pgfpathlineto{\pgfqpoint{1.375785in}{1.052700in}}%
\pgfpathlineto{\pgfqpoint{1.387086in}{1.004647in}}%
\pgfpathlineto{\pgfqpoint{1.398387in}{1.156198in}}%
\pgfpathlineto{\pgfqpoint{1.409688in}{1.058244in}}%
\pgfpathlineto{\pgfqpoint{1.420989in}{1.276330in}}%
\pgfpathlineto{\pgfqpoint{1.432290in}{1.065637in}}%
\pgfpathlineto{\pgfqpoint{1.443591in}{1.228277in}}%
\pgfpathlineto{\pgfqpoint{1.454892in}{0.945505in}}%
\pgfpathlineto{\pgfqpoint{1.466193in}{0.980620in}}%
\pgfpathlineto{\pgfqpoint{1.477494in}{1.505505in}}%
\pgfpathlineto{\pgfqpoint{1.488795in}{1.389069in}}%
\pgfpathlineto{\pgfqpoint{1.500096in}{1.052700in}}%
\pgfpathlineto{\pgfqpoint{1.511397in}{1.342865in}}%
\pgfpathlineto{\pgfqpoint{1.522698in}{1.446363in}}%
\pgfpathlineto{\pgfqpoint{1.533999in}{1.634878in}}%
\pgfpathlineto{\pgfqpoint{1.545300in}{1.169135in}}%
\pgfpathlineto{\pgfqpoint{1.556601in}{1.167287in}}%
\pgfpathlineto{\pgfqpoint{1.567902in}{1.134020in}}%
\pgfpathlineto{\pgfqpoint{1.579203in}{1.379828in}}%
\pgfpathlineto{\pgfqpoint{1.590504in}{1.485175in}}%
\pgfpathlineto{\pgfqpoint{1.601805in}{1.420488in}}%
\pgfpathlineto{\pgfqpoint{1.613105in}{1.658904in}}%
\pgfpathlineto{\pgfqpoint{1.624406in}{1.797518in}}%
\pgfpathlineto{\pgfqpoint{1.635707in}{1.243063in}}%
\pgfpathlineto{\pgfqpoint{1.647008in}{1.448211in}}%
\pgfpathlineto{\pgfqpoint{1.658309in}{1.156198in}}%
\pgfpathlineto{\pgfqpoint{1.669610in}{1.498112in}}%
\pgfpathlineto{\pgfqpoint{1.680911in}{1.523987in}}%
\pgfpathlineto{\pgfqpoint{1.692212in}{1.801215in}}%
\pgfpathlineto{\pgfqpoint{1.703513in}{1.673690in}}%
\pgfpathlineto{\pgfqpoint{1.714814in}{1.668145in}}%
\pgfpathlineto{\pgfqpoint{1.726115in}{2.290983in}}%
\pgfpathlineto{\pgfqpoint{1.737416in}{1.588673in}}%
\pgfpathlineto{\pgfqpoint{1.748717in}{2.035934in}}%
\pgfpathlineto{\pgfqpoint{1.760018in}{1.572040in}}%
\pgfpathlineto{\pgfqpoint{1.771319in}{1.749465in}}%
\pgfpathlineto{\pgfqpoint{1.782620in}{2.128343in}}%
\pgfpathlineto{\pgfqpoint{1.793921in}{1.926891in}}%
\pgfpathlineto{\pgfqpoint{1.805222in}{2.119102in}}%
\pgfpathlineto{\pgfqpoint{1.816523in}{1.562799in}}%
\pgfpathlineto{\pgfqpoint{1.827824in}{1.762403in}}%
\pgfpathlineto{\pgfqpoint{1.839125in}{2.270653in}}%
\pgfpathlineto{\pgfqpoint{1.850426in}{2.043327in}}%
\pgfpathlineto{\pgfqpoint{1.861727in}{2.052568in}}%
\pgfpathlineto{\pgfqpoint{1.873028in}{2.021149in}}%
\pgfpathlineto{\pgfqpoint{1.884329in}{2.218904in}}%
\pgfpathlineto{\pgfqpoint{1.895630in}{1.967551in}}%
\pgfpathlineto{\pgfqpoint{1.906931in}{2.307617in}}%
\pgfpathlineto{\pgfqpoint{1.918232in}{2.825109in}}%
\pgfpathlineto{\pgfqpoint{1.929533in}{2.571908in}}%
\pgfpathlineto{\pgfqpoint{1.940834in}{2.228145in}}%
\pgfpathlineto{\pgfqpoint{1.952135in}{2.089531in}}%
\pgfpathlineto{\pgfqpoint{1.963436in}{2.444383in}}%
\pgfpathlineto{\pgfqpoint{1.974737in}{2.880554in}}%
\pgfpathlineto{\pgfqpoint{1.986038in}{2.847287in}}%
\pgfpathlineto{\pgfqpoint{1.997339in}{2.915670in}}%
\pgfpathlineto{\pgfqpoint{2.008640in}{2.582997in}}%
\pgfpathlineto{\pgfqpoint{2.019941in}{2.486891in}}%
\pgfpathlineto{\pgfqpoint{2.031242in}{2.923063in}}%
\pgfpathlineto{\pgfqpoint{2.042543in}{2.496132in}}%
\pgfpathlineto{\pgfqpoint{2.053844in}{2.224449in}}%
\pgfpathlineto{\pgfqpoint{2.065145in}{2.719762in}}%
\pgfpathlineto{\pgfqpoint{2.076446in}{2.185637in}}%
\pgfpathlineto{\pgfqpoint{2.087747in}{2.703129in}}%
\pgfpathlineto{\pgfqpoint{2.099048in}{2.418508in}}%
\pgfpathlineto{\pgfqpoint{2.110349in}{2.429597in}}%
\pgfpathlineto{\pgfqpoint{2.121650in}{2.477650in}}%
\pgfpathlineto{\pgfqpoint{2.132951in}{2.716066in}}%
\pgfpathlineto{\pgfqpoint{2.144252in}{2.666165in}}%
\pgfpathlineto{\pgfqpoint{2.155553in}{2.270653in}}%
\pgfpathlineto{\pgfqpoint{2.166853in}{2.777056in}}%
\pgfpathlineto{\pgfqpoint{2.178154in}{2.533096in}}%
\pgfpathlineto{\pgfqpoint{2.189455in}{2.747485in}}%
\pgfpathlineto{\pgfqpoint{2.200756in}{2.789993in}}%
\pgfpathlineto{\pgfqpoint{2.212057in}{2.671710in}}%
\pgfpathlineto{\pgfqpoint{2.223358in}{2.867617in}}%
\pgfpathlineto{\pgfqpoint{2.245960in}{3.152238in}}%
\pgfpathlineto{\pgfqpoint{2.257261in}{2.819564in}}%
\pgfpathlineto{\pgfqpoint{2.268562in}{2.995142in}}%
\pgfpathlineto{\pgfqpoint{2.279863in}{2.716066in}}%
\pgfpathlineto{\pgfqpoint{2.291164in}{2.956330in}}%
\pgfpathlineto{\pgfqpoint{2.302465in}{2.932304in}}%
\pgfpathlineto{\pgfqpoint{2.313766in}{3.172568in}}%
\pgfpathlineto{\pgfqpoint{2.325067in}{2.769663in}}%
\pgfpathlineto{\pgfqpoint{2.336368in}{3.185505in}}%
\pgfpathlineto{\pgfqpoint{2.358970in}{2.889795in}}%
\pgfpathlineto{\pgfqpoint{2.370271in}{2.649531in}}%
\pgfpathlineto{\pgfqpoint{2.381572in}{2.668013in}}%
\pgfpathlineto{\pgfqpoint{2.392873in}{2.911974in}}%
\pgfpathlineto{\pgfqpoint{2.404174in}{2.965571in}}%
\pgfpathlineto{\pgfqpoint{2.415475in}{2.856528in}}%
\pgfpathlineto{\pgfqpoint{2.426776in}{2.841743in}}%
\pgfpathlineto{\pgfqpoint{2.438077in}{2.899036in}}%
\pgfpathlineto{\pgfqpoint{2.449378in}{3.216924in}}%
\pgfpathlineto{\pgfqpoint{2.460679in}{2.878706in}}%
\pgfpathlineto{\pgfqpoint{2.483281in}{3.459036in}}%
\pgfpathlineto{\pgfqpoint{2.494582in}{3.216924in}}%
\pgfpathlineto{\pgfqpoint{2.505883in}{3.494152in}}%
\pgfpathlineto{\pgfqpoint{2.517184in}{3.608739in}}%
\pgfpathlineto{\pgfqpoint{2.528485in}{3.573624in}}%
\pgfpathlineto{\pgfqpoint{2.539786in}{3.052436in}}%
\pgfpathlineto{\pgfqpoint{2.551087in}{3.612436in}}%
\pgfpathlineto{\pgfqpoint{2.562388in}{3.174416in}}%
\pgfpathlineto{\pgfqpoint{2.573689in}{3.654944in}}%
\pgfpathlineto{\pgfqpoint{2.584990in}{2.993294in}}%
\pgfpathlineto{\pgfqpoint{2.596291in}{3.760290in}}%
\pgfpathlineto{\pgfqpoint{2.607592in}{3.671578in}}%
\pgfpathlineto{\pgfqpoint{2.618893in}{3.545901in}}%
\pgfpathlineto{\pgfqpoint{2.630194in}{3.130059in}}%
\pgfpathlineto{\pgfqpoint{2.641495in}{3.124515in}}%
\pgfpathlineto{\pgfqpoint{2.652796in}{3.632766in}}%
\pgfpathlineto{\pgfqpoint{2.664097in}{3.640158in}}%
\pgfpathlineto{\pgfqpoint{2.686699in}{3.593954in}}%
\pgfpathlineto{\pgfqpoint{2.698000in}{3.228013in}}%
\pgfpathlineto{\pgfqpoint{2.709301in}{3.666033in}}%
\pgfpathlineto{\pgfqpoint{2.720602in}{3.407287in}}%
\pgfpathlineto{\pgfqpoint{2.731902in}{3.374020in}}%
\pgfpathlineto{\pgfqpoint{2.743203in}{3.191050in}}%
\pgfpathlineto{\pgfqpoint{2.754504in}{3.124515in}}%
\pgfpathlineto{\pgfqpoint{2.765805in}{2.751182in}}%
\pgfpathlineto{\pgfqpoint{2.777106in}{3.259432in}}%
\pgfpathlineto{\pgfqpoint{2.788407in}{2.978508in}}%
\pgfpathlineto{\pgfqpoint{2.799708in}{3.002535in}}%
\pgfpathlineto{\pgfqpoint{2.811009in}{3.335208in}}%
\pgfpathlineto{\pgfqpoint{2.822310in}{3.525571in}}%
\pgfpathlineto{\pgfqpoint{2.833611in}{3.264977in}}%
\pgfpathlineto{\pgfqpoint{2.844912in}{3.192898in}}%
\pgfpathlineto{\pgfqpoint{2.856213in}{3.205835in}}%
\pgfpathlineto{\pgfqpoint{2.867514in}{2.594086in}}%
\pgfpathlineto{\pgfqpoint{2.878815in}{2.904581in}}%
\pgfpathlineto{\pgfqpoint{2.890116in}{3.174416in}}%
\pgfpathlineto{\pgfqpoint{2.901417in}{3.093096in}}%
\pgfpathlineto{\pgfqpoint{2.912718in}{2.546033in}}%
\pgfpathlineto{\pgfqpoint{2.924019in}{3.224317in}}%
\pgfpathlineto{\pgfqpoint{2.935320in}{3.263129in}}%
\pgfpathlineto{\pgfqpoint{2.946621in}{2.764119in}}%
\pgfpathlineto{\pgfqpoint{2.957922in}{3.305637in}}%
\pgfpathlineto{\pgfqpoint{2.969223in}{3.198442in}}%
\pgfpathlineto{\pgfqpoint{2.980524in}{3.233558in}}%
\pgfpathlineto{\pgfqpoint{2.991825in}{3.361083in}}%
\pgfpathlineto{\pgfqpoint{3.003126in}{3.250191in}}%
\pgfpathlineto{\pgfqpoint{3.014427in}{2.960026in}}%
\pgfpathlineto{\pgfqpoint{3.025728in}{3.239102in}}%
\pgfpathlineto{\pgfqpoint{3.037029in}{3.588409in}}%
\pgfpathlineto{\pgfqpoint{3.048330in}{3.499696in}}%
\pgfpathlineto{\pgfqpoint{3.070932in}{3.211380in}}%
\pgfpathlineto{\pgfqpoint{3.082233in}{3.460884in}}%
\pgfpathlineto{\pgfqpoint{3.093534in}{3.401743in}}%
\pgfpathlineto{\pgfqpoint{3.104835in}{3.252040in}}%
\pgfpathlineto{\pgfqpoint{3.116136in}{3.686363in}}%
\pgfpathlineto{\pgfqpoint{3.127437in}{3.364779in}}%
\pgfpathlineto{\pgfqpoint{3.138738in}{3.523723in}}%
\pgfpathlineto{\pgfqpoint{3.150039in}{3.325967in}}%
\pgfpathlineto{\pgfqpoint{3.161340in}{3.593954in}}%
\pgfpathlineto{\pgfqpoint{3.172641in}{3.455340in}}%
\pgfpathlineto{\pgfqpoint{3.183942in}{3.516330in}}%
\pgfpathlineto{\pgfqpoint{3.195243in}{3.538508in}}%
\pgfpathlineto{\pgfqpoint{3.206544in}{3.501545in}}%
\pgfpathlineto{\pgfqpoint{3.217845in}{3.359234in}}%
\pgfpathlineto{\pgfqpoint{3.229146in}{2.710521in}}%
\pgfpathlineto{\pgfqpoint{3.240447in}{2.967419in}}%
\pgfpathlineto{\pgfqpoint{3.263049in}{3.215076in}}%
\pgfpathlineto{\pgfqpoint{3.274350in}{3.728871in}}%
\pgfpathlineto{\pgfqpoint{3.285650in}{3.697452in}}%
\pgfpathlineto{\pgfqpoint{3.296951in}{3.320422in}}%
\pgfpathlineto{\pgfqpoint{3.308252in}{3.682667in}}%
\pgfpathlineto{\pgfqpoint{3.319553in}{3.309333in}}%
\pgfpathlineto{\pgfqpoint{3.330854in}{3.351842in}}%
\pgfpathlineto{\pgfqpoint{3.342155in}{3.083855in}}%
\pgfpathlineto{\pgfqpoint{3.353456in}{3.126363in}}%
\pgfpathlineto{\pgfqpoint{3.364757in}{3.386957in}}%
\pgfpathlineto{\pgfqpoint{3.376058in}{3.192898in}}%
\pgfpathlineto{\pgfqpoint{3.387359in}{2.972964in}}%
\pgfpathlineto{\pgfqpoint{3.398660in}{3.141149in}}%
\pgfpathlineto{\pgfqpoint{3.409961in}{3.109729in}}%
\pgfpathlineto{\pgfqpoint{3.421262in}{3.159630in}}%
\pgfpathlineto{\pgfqpoint{3.432563in}{3.163327in}}%
\pgfpathlineto{\pgfqpoint{3.443864in}{3.228013in}}%
\pgfpathlineto{\pgfqpoint{3.455165in}{3.270521in}}%
\pgfpathlineto{\pgfqpoint{3.466466in}{3.211380in}}%
\pgfpathlineto{\pgfqpoint{3.477767in}{3.560686in}}%
\pgfpathlineto{\pgfqpoint{3.489068in}{3.590257in}}%
\pgfpathlineto{\pgfqpoint{3.500369in}{3.473822in}}%
\pgfpathlineto{\pgfqpoint{3.511670in}{3.416528in}}%
\pgfpathlineto{\pgfqpoint{3.522971in}{3.512634in}}%
\pgfpathlineto{\pgfqpoint{3.534272in}{3.518178in}}%
\pgfpathlineto{\pgfqpoint{3.545573in}{3.244647in}}%
\pgfpathlineto{\pgfqpoint{3.556874in}{3.468277in}}%
\pgfpathlineto{\pgfqpoint{3.568175in}{3.152238in}}%
\pgfpathlineto{\pgfqpoint{3.579476in}{3.318574in}}%
\pgfpathlineto{\pgfqpoint{3.590777in}{3.808343in}}%
\pgfpathlineto{\pgfqpoint{3.602078in}{3.895208in}}%
\pgfpathlineto{\pgfqpoint{3.613379in}{3.333360in}}%
\pgfpathlineto{\pgfqpoint{3.624680in}{3.435010in}}%
\pgfpathlineto{\pgfqpoint{3.635981in}{3.823129in}}%
\pgfpathlineto{\pgfqpoint{3.647282in}{3.399894in}}%
\pgfpathlineto{\pgfqpoint{3.658583in}{3.704845in}}%
\pgfpathlineto{\pgfqpoint{3.669884in}{3.909993in}}%
\pgfpathlineto{\pgfqpoint{3.681185in}{3.695604in}}%
\pgfpathlineto{\pgfqpoint{3.692486in}{3.824977in}}%
\pgfpathlineto{\pgfqpoint{3.703787in}{3.481215in}}%
\pgfpathlineto{\pgfqpoint{3.715088in}{3.566231in}}%
\pgfpathlineto{\pgfqpoint{3.726389in}{3.287155in}}%
\pgfpathlineto{\pgfqpoint{3.737690in}{3.233558in}}%
\pgfpathlineto{\pgfqpoint{3.748991in}{3.497848in}}%
\pgfpathlineto{\pgfqpoint{3.760292in}{3.584713in}}%
\pgfpathlineto{\pgfqpoint{3.771593in}{3.320422in}}%
\pgfpathlineto{\pgfqpoint{3.782894in}{3.370323in}}%
\pgfpathlineto{\pgfqpoint{3.794195in}{3.325967in}}%
\pgfpathlineto{\pgfqpoint{3.805496in}{3.682667in}}%
\pgfpathlineto{\pgfqpoint{3.816797in}{3.309333in}}%
\pgfpathlineto{\pgfqpoint{3.828098in}{3.335208in}}%
\pgfpathlineto{\pgfqpoint{3.839398in}{2.769663in}}%
\pgfpathlineto{\pgfqpoint{3.850699in}{3.261281in}}%
\pgfpathlineto{\pgfqpoint{3.862000in}{3.313030in}}%
\pgfpathlineto{\pgfqpoint{3.873301in}{3.475670in}}%
\pgfpathlineto{\pgfqpoint{3.884602in}{3.093096in}}%
\pgfpathlineto{\pgfqpoint{3.895903in}{3.216924in}}%
\pgfpathlineto{\pgfqpoint{3.907204in}{3.305637in}}%
\pgfpathlineto{\pgfqpoint{3.918505in}{3.464581in}}%
\pgfpathlineto{\pgfqpoint{3.929806in}{3.198442in}}%
\pgfpathlineto{\pgfqpoint{3.941107in}{3.331512in}}%
\pgfpathlineto{\pgfqpoint{3.952408in}{3.538508in}}%
\pgfpathlineto{\pgfqpoint{3.963709in}{3.264977in}}%
\pgfpathlineto{\pgfqpoint{3.975010in}{3.492304in}}%
\pgfpathlineto{\pgfqpoint{3.986311in}{3.531116in}}%
\pgfpathlineto{\pgfqpoint{3.997612in}{3.660488in}}%
\pgfpathlineto{\pgfqpoint{4.008913in}{3.836066in}}%
\pgfpathlineto{\pgfqpoint{4.020214in}{3.771380in}}%
\pgfpathlineto{\pgfqpoint{4.031515in}{3.538508in}}%
\pgfpathlineto{\pgfqpoint{4.042816in}{3.555142in}}%
\pgfpathlineto{\pgfqpoint{4.054117in}{4.056000in}}%
\pgfpathlineto{\pgfqpoint{4.065418in}{3.913690in}}%
\pgfpathlineto{\pgfqpoint{4.076719in}{3.455340in}}%
\pgfpathlineto{\pgfqpoint{4.088020in}{3.926627in}}%
\pgfpathlineto{\pgfqpoint{4.099321in}{3.998706in}}%
\pgfpathlineto{\pgfqpoint{4.110622in}{3.702997in}}%
\pgfpathlineto{\pgfqpoint{4.121923in}{4.033822in}}%
\pgfpathlineto{\pgfqpoint{4.133224in}{4.015340in}}%
\pgfpathlineto{\pgfqpoint{4.144525in}{4.056000in}}%
\pgfpathlineto{\pgfqpoint{4.155826in}{3.636462in}}%
\pgfpathlineto{\pgfqpoint{4.167127in}{3.987617in}}%
\pgfpathlineto{\pgfqpoint{4.178428in}{4.024581in}}%
\pgfpathlineto{\pgfqpoint{4.189729in}{4.031974in}}%
\pgfpathlineto{\pgfqpoint{4.201030in}{3.976528in}}%
\pgfpathlineto{\pgfqpoint{4.212331in}{3.832370in}}%
\pgfpathlineto{\pgfqpoint{4.223632in}{3.802799in}}%
\pgfpathlineto{\pgfqpoint{4.234933in}{3.808343in}}%
\pgfpathlineto{\pgfqpoint{4.246234in}{3.765835in}}%
\pgfpathlineto{\pgfqpoint{4.257535in}{3.778772in}}%
\pgfpathlineto{\pgfqpoint{4.268836in}{3.970983in}}%
\pgfpathlineto{\pgfqpoint{4.280137in}{3.660488in}}%
\pgfpathlineto{\pgfqpoint{4.291438in}{3.751050in}}%
\pgfpathlineto{\pgfqpoint{4.302739in}{3.693756in}}%
\pgfpathlineto{\pgfqpoint{4.314040in}{3.799102in}}%
\pgfpathlineto{\pgfqpoint{4.325341in}{3.429465in}}%
\pgfpathlineto{\pgfqpoint{4.336642in}{3.909993in}}%
\pgfpathlineto{\pgfqpoint{4.347943in}{3.344449in}}%
\pgfpathlineto{\pgfqpoint{4.359244in}{3.793558in}}%
\pgfpathlineto{\pgfqpoint{4.381846in}{3.466429in}}%
\pgfpathlineto{\pgfqpoint{4.393147in}{3.727023in}}%
\pgfpathlineto{\pgfqpoint{4.404447in}{3.677122in}}%
\pgfpathlineto{\pgfqpoint{4.415748in}{3.730719in}}%
\pgfpathlineto{\pgfqpoint{4.427049in}{3.508937in}}%
\pgfpathlineto{\pgfqpoint{4.438350in}{3.876726in}}%
\pgfpathlineto{\pgfqpoint{4.449651in}{3.937716in}}%
\pgfpathlineto{\pgfqpoint{4.460952in}{3.697452in}}%
\pgfpathlineto{\pgfqpoint{4.472253in}{3.606891in}}%
\pgfpathlineto{\pgfqpoint{4.483554in}{3.643855in}}%
\pgfpathlineto{\pgfqpoint{4.494855in}{3.760290in}}%
\pgfpathlineto{\pgfqpoint{4.506156in}{3.913690in}}%
\pgfpathlineto{\pgfqpoint{4.517457in}{3.669729in}}%
\pgfpathlineto{\pgfqpoint{4.528758in}{3.553294in}}%
\pgfpathlineto{\pgfqpoint{4.540059in}{3.788013in}}%
\pgfpathlineto{\pgfqpoint{4.551360in}{3.717782in}}%
\pgfpathlineto{\pgfqpoint{4.562661in}{3.470125in}}%
\pgfpathlineto{\pgfqpoint{4.573962in}{3.721479in}}%
\pgfpathlineto{\pgfqpoint{4.585263in}{3.642007in}}%
\pgfpathlineto{\pgfqpoint{4.596564in}{3.675274in}}%
\pgfpathlineto{\pgfqpoint{4.607865in}{3.937716in}}%
\pgfpathlineto{\pgfqpoint{4.630467in}{3.739960in}}%
\pgfpathlineto{\pgfqpoint{4.641768in}{3.945109in}}%
\pgfpathlineto{\pgfqpoint{4.653069in}{3.629069in}}%
\pgfpathlineto{\pgfqpoint{4.664370in}{3.741809in}}%
\pgfpathlineto{\pgfqpoint{4.675671in}{3.416528in}}%
\pgfpathlineto{\pgfqpoint{4.686972in}{3.741809in}}%
\pgfpathlineto{\pgfqpoint{4.698273in}{3.847155in}}%
\pgfpathlineto{\pgfqpoint{4.709574in}{3.791710in}}%
\pgfpathlineto{\pgfqpoint{4.720875in}{3.806495in}}%
\pgfpathlineto{\pgfqpoint{4.732176in}{3.654944in}}%
\pgfpathlineto{\pgfqpoint{4.743477in}{3.776924in}}%
\pgfpathlineto{\pgfqpoint{4.754778in}{3.824977in}}%
\pgfpathlineto{\pgfqpoint{4.766079in}{3.477518in}}%
\pgfpathlineto{\pgfqpoint{4.777380in}{3.727023in}}%
\pgfpathlineto{\pgfqpoint{4.788681in}{3.819432in}}%
\pgfpathlineto{\pgfqpoint{4.799982in}{3.342601in}}%
\pgfpathlineto{\pgfqpoint{4.811283in}{3.545901in}}%
\pgfpathlineto{\pgfqpoint{4.822584in}{3.544053in}}%
\pgfpathlineto{\pgfqpoint{4.833885in}{3.520026in}}%
\pgfpathlineto{\pgfqpoint{4.845186in}{3.111578in}}%
\pgfpathlineto{\pgfqpoint{4.856487in}{3.338904in}}%
\pgfpathlineto{\pgfqpoint{4.867788in}{3.296396in}}%
\pgfpathlineto{\pgfqpoint{4.879089in}{3.290851in}}%
\pgfpathlineto{\pgfqpoint{4.890390in}{3.686363in}}%
\pgfpathlineto{\pgfqpoint{4.901691in}{3.697452in}}%
\pgfpathlineto{\pgfqpoint{4.912992in}{3.706693in}}%
\pgfpathlineto{\pgfqpoint{4.924293in}{3.575472in}}%
\pgfpathlineto{\pgfqpoint{4.935594in}{3.793558in}}%
\pgfpathlineto{\pgfqpoint{4.946895in}{3.468277in}}%
\pgfpathlineto{\pgfqpoint{4.958195in}{3.399894in}}%
\pgfpathlineto{\pgfqpoint{4.969496in}{3.747353in}}%
\pgfpathlineto{\pgfqpoint{4.980797in}{3.673426in}}%
\pgfpathlineto{\pgfqpoint{4.992098in}{3.477518in}}%
\pgfpathlineto{\pgfqpoint{5.003399in}{3.908145in}}%
\pgfpathlineto{\pgfqpoint{5.014700in}{3.608739in}}%
\pgfpathlineto{\pgfqpoint{5.026001in}{3.791710in}}%
\pgfpathlineto{\pgfqpoint{5.037302in}{3.895208in}}%
\pgfpathlineto{\pgfqpoint{5.048603in}{4.035670in}}%
\pgfpathlineto{\pgfqpoint{5.059904in}{3.804647in}}%
\pgfpathlineto{\pgfqpoint{5.071205in}{3.669729in}}%
\pgfpathlineto{\pgfqpoint{5.082506in}{3.950653in}}%
\pgfpathlineto{\pgfqpoint{5.093807in}{4.056000in}}%
\pgfpathlineto{\pgfqpoint{5.105108in}{3.952502in}}%
\pgfpathlineto{\pgfqpoint{5.116409in}{4.056000in}}%
\pgfpathlineto{\pgfqpoint{5.127710in}{3.669729in}}%
\pgfpathlineto{\pgfqpoint{5.139011in}{3.769531in}}%
\pgfpathlineto{\pgfqpoint{5.150312in}{3.980224in}}%
\pgfpathlineto{\pgfqpoint{5.161613in}{3.621677in}}%
\pgfpathlineto{\pgfqpoint{5.172914in}{3.924779in}}%
\pgfpathlineto{\pgfqpoint{5.184215in}{3.921083in}}%
\pgfpathlineto{\pgfqpoint{5.195516in}{3.812040in}}%
\pgfpathlineto{\pgfqpoint{5.206817in}{4.017188in}}%
\pgfpathlineto{\pgfqpoint{5.218118in}{3.721479in}}%
\pgfpathlineto{\pgfqpoint{5.229419in}{3.874878in}}%
\pgfpathlineto{\pgfqpoint{5.240720in}{3.839762in}}%
\pgfpathlineto{\pgfqpoint{5.252021in}{3.775076in}}%
\pgfpathlineto{\pgfqpoint{5.263322in}{3.928475in}}%
\pgfpathlineto{\pgfqpoint{5.274623in}{4.056000in}}%
\pgfpathlineto{\pgfqpoint{5.285924in}{3.684515in}}%
\pgfpathlineto{\pgfqpoint{5.297225in}{3.494152in}}%
\pgfpathlineto{\pgfqpoint{5.308526in}{3.963591in}}%
\pgfpathlineto{\pgfqpoint{5.319827in}{3.836066in}}%
\pgfpathlineto{\pgfqpoint{5.331128in}{4.044911in}}%
\pgfpathlineto{\pgfqpoint{5.342429in}{3.970983in}}%
\pgfpathlineto{\pgfqpoint{5.353730in}{4.044911in}}%
\pgfpathlineto{\pgfqpoint{5.376332in}{3.745505in}}%
\pgfpathlineto{\pgfqpoint{5.387633in}{3.728871in}}%
\pgfpathlineto{\pgfqpoint{5.398934in}{3.921083in}}%
\pgfpathlineto{\pgfqpoint{5.410235in}{3.941413in}}%
\pgfpathlineto{\pgfqpoint{5.421536in}{4.015340in}}%
\pgfpathlineto{\pgfqpoint{5.432837in}{3.775076in}}%
\pgfpathlineto{\pgfqpoint{5.444138in}{3.852700in}}%
\pgfpathlineto{\pgfqpoint{5.455439in}{3.608739in}}%
\pgfpathlineto{\pgfqpoint{5.466740in}{3.836066in}}%
\pgfpathlineto{\pgfqpoint{5.478041in}{3.710389in}}%
\pgfpathlineto{\pgfqpoint{5.489342in}{3.800950in}}%
\pgfpathlineto{\pgfqpoint{5.500643in}{3.651248in}}%
\pgfpathlineto{\pgfqpoint{5.511943in}{3.922931in}}%
\pgfpathlineto{\pgfqpoint{5.523244in}{3.462733in}}%
\pgfpathlineto{\pgfqpoint{5.534545in}{3.666033in}}%
\pgfpathlineto{\pgfqpoint{5.534545in}{3.666033in}}%
\pgfusepath{stroke}%
\end{pgfscope}%
\begin{pgfscope}%
\pgfpathrectangle{\pgfqpoint{0.800000in}{0.528000in}}{\pgfqpoint{4.960000in}{3.696000in}} %
\pgfusepath{clip}%
\pgfsetrectcap%
\pgfsetroundjoin%
\pgfsetlinewidth{1.505625pt}%
\definecolor{currentstroke}{rgb}{0.300000,0.300000,0.300000}%
\pgfsetstrokecolor{currentstroke}%
\pgfsetdash{}{0pt}%
\pgfpathmoveto{\pgfqpoint{1.025455in}{0.790257in}}%
\pgfpathlineto{\pgfqpoint{1.036756in}{0.786561in}}%
\pgfpathlineto{\pgfqpoint{1.048057in}{0.871578in}}%
\pgfpathlineto{\pgfqpoint{1.059357in}{1.049003in}}%
\pgfpathlineto{\pgfqpoint{1.070658in}{0.971380in}}%
\pgfpathlineto{\pgfqpoint{1.081959in}{1.281875in}}%
\pgfpathlineto{\pgfqpoint{1.093260in}{1.095208in}}%
\pgfpathlineto{\pgfqpoint{1.104561in}{1.135868in}}%
\pgfpathlineto{\pgfqpoint{1.115862in}{1.311446in}}%
\pgfpathlineto{\pgfqpoint{1.127163in}{1.372436in}}%
\pgfpathlineto{\pgfqpoint{1.138464in}{1.614548in}}%
\pgfpathlineto{\pgfqpoint{1.149765in}{1.346561in}}%
\pgfpathlineto{\pgfqpoint{1.161066in}{1.755010in}}%
\pgfpathlineto{\pgfqpoint{1.172367in}{1.479630in}}%
\pgfpathlineto{\pgfqpoint{1.183668in}{2.217056in}}%
\pgfpathlineto{\pgfqpoint{1.194969in}{2.032238in}}%
\pgfpathlineto{\pgfqpoint{1.206270in}{1.993426in}}%
\pgfpathlineto{\pgfqpoint{1.217571in}{1.673690in}}%
\pgfpathlineto{\pgfqpoint{1.228872in}{1.934284in}}%
\pgfpathlineto{\pgfqpoint{1.240173in}{2.000818in}}%
\pgfpathlineto{\pgfqpoint{1.251474in}{1.906561in}}%
\pgfpathlineto{\pgfqpoint{1.262775in}{1.755010in}}%
\pgfpathlineto{\pgfqpoint{1.274076in}{2.098772in}}%
\pgfpathlineto{\pgfqpoint{1.285377in}{1.875142in}}%
\pgfpathlineto{\pgfqpoint{1.296678in}{2.242931in}}%
\pgfpathlineto{\pgfqpoint{1.307979in}{2.357518in}}%
\pgfpathlineto{\pgfqpoint{1.319280in}{2.122799in}}%
\pgfpathlineto{\pgfqpoint{1.330581in}{2.651380in}}%
\pgfpathlineto{\pgfqpoint{1.341882in}{2.281743in}}%
\pgfpathlineto{\pgfqpoint{1.353183in}{2.069201in}}%
\pgfpathlineto{\pgfqpoint{1.364484in}{2.220752in}}%
\pgfpathlineto{\pgfqpoint{1.375785in}{2.533096in}}%
\pgfpathlineto{\pgfqpoint{1.387086in}{2.764119in}}%
\pgfpathlineto{\pgfqpoint{1.398387in}{1.906561in}}%
\pgfpathlineto{\pgfqpoint{1.409688in}{2.394482in}}%
\pgfpathlineto{\pgfqpoint{1.420989in}{2.283591in}}%
\pgfpathlineto{\pgfqpoint{1.432290in}{2.263261in}}%
\pgfpathlineto{\pgfqpoint{1.443591in}{2.675406in}}%
\pgfpathlineto{\pgfqpoint{1.454892in}{2.194878in}}%
\pgfpathlineto{\pgfqpoint{1.466193in}{2.392634in}}%
\pgfpathlineto{\pgfqpoint{1.477494in}{2.076594in}}%
\pgfpathlineto{\pgfqpoint{1.488795in}{2.259564in}}%
\pgfpathlineto{\pgfqpoint{1.500096in}{2.656924in}}%
\pgfpathlineto{\pgfqpoint{1.511397in}{2.370455in}}%
\pgfpathlineto{\pgfqpoint{1.522698in}{2.296528in}}%
\pgfpathlineto{\pgfqpoint{1.533999in}{2.582997in}}%
\pgfpathlineto{\pgfqpoint{1.545300in}{1.936132in}}%
\pgfpathlineto{\pgfqpoint{1.556601in}{1.827089in}}%
\pgfpathlineto{\pgfqpoint{1.567902in}{2.217056in}}%
\pgfpathlineto{\pgfqpoint{1.579203in}{2.311314in}}%
\pgfpathlineto{\pgfqpoint{1.590504in}{2.248475in}}%
\pgfpathlineto{\pgfqpoint{1.601805in}{2.542337in}}%
\pgfpathlineto{\pgfqpoint{1.613105in}{3.113426in}}%
\pgfpathlineto{\pgfqpoint{1.624406in}{3.094944in}}%
\pgfpathlineto{\pgfqpoint{1.635707in}{2.730851in}}%
\pgfpathlineto{\pgfqpoint{1.647008in}{2.760422in}}%
\pgfpathlineto{\pgfqpoint{1.658309in}{2.854680in}}%
\pgfpathlineto{\pgfqpoint{1.669610in}{3.037650in}}%
\pgfpathlineto{\pgfqpoint{1.680911in}{3.423921in}}%
\pgfpathlineto{\pgfqpoint{1.692212in}{3.666033in}}%
\pgfpathlineto{\pgfqpoint{1.703513in}{2.928607in}}%
\pgfpathlineto{\pgfqpoint{1.714814in}{3.037650in}}%
\pgfpathlineto{\pgfqpoint{1.726115in}{3.374020in}}%
\pgfpathlineto{\pgfqpoint{1.737416in}{3.425769in}}%
\pgfpathlineto{\pgfqpoint{1.748717in}{3.627221in}}%
\pgfpathlineto{\pgfqpoint{1.760018in}{3.296396in}}%
\pgfpathlineto{\pgfqpoint{1.771319in}{3.155934in}}%
\pgfpathlineto{\pgfqpoint{1.782620in}{3.520026in}}%
\pgfpathlineto{\pgfqpoint{1.793921in}{3.431314in}}%
\pgfpathlineto{\pgfqpoint{1.805222in}{3.438706in}}%
\pgfpathlineto{\pgfqpoint{1.816523in}{3.163327in}}%
\pgfpathlineto{\pgfqpoint{1.827824in}{3.316726in}}%
\pgfpathlineto{\pgfqpoint{1.839125in}{3.423921in}}%
\pgfpathlineto{\pgfqpoint{1.850426in}{3.442403in}}%
\pgfpathlineto{\pgfqpoint{1.861727in}{3.355538in}}%
\pgfpathlineto{\pgfqpoint{1.873028in}{3.281611in}}%
\pgfpathlineto{\pgfqpoint{1.884329in}{3.085703in}}%
\pgfpathlineto{\pgfqpoint{1.895630in}{3.257584in}}%
\pgfpathlineto{\pgfqpoint{1.906931in}{3.226165in}}%
\pgfpathlineto{\pgfqpoint{1.918232in}{3.242799in}}%
\pgfpathlineto{\pgfqpoint{1.929533in}{2.741941in}}%
\pgfpathlineto{\pgfqpoint{1.940834in}{2.880554in}}%
\pgfpathlineto{\pgfqpoint{1.952135in}{3.074614in}}%
\pgfpathlineto{\pgfqpoint{1.963436in}{3.026561in}}%
\pgfpathlineto{\pgfqpoint{1.974737in}{2.971116in}}%
\pgfpathlineto{\pgfqpoint{1.986038in}{2.856528in}}%
\pgfpathlineto{\pgfqpoint{1.997339in}{2.823261in}}%
\pgfpathlineto{\pgfqpoint{2.008640in}{2.311314in}}%
\pgfpathlineto{\pgfqpoint{2.031242in}{3.078310in}}%
\pgfpathlineto{\pgfqpoint{2.042543in}{2.581149in}}%
\pgfpathlineto{\pgfqpoint{2.053844in}{2.690191in}}%
\pgfpathlineto{\pgfqpoint{2.065145in}{3.059828in}}%
\pgfpathlineto{\pgfqpoint{2.076446in}{2.954482in}}%
\pgfpathlineto{\pgfqpoint{2.087747in}{3.057980in}}%
\pgfpathlineto{\pgfqpoint{2.099048in}{2.769663in}}%
\pgfpathlineto{\pgfqpoint{2.110349in}{2.850983in}}%
\pgfpathlineto{\pgfqpoint{2.121650in}{2.692040in}}%
\pgfpathlineto{\pgfqpoint{2.144252in}{2.865769in}}%
\pgfpathlineto{\pgfqpoint{2.155553in}{2.795538in}}%
\pgfpathlineto{\pgfqpoint{2.166853in}{3.054284in}}%
\pgfpathlineto{\pgfqpoint{2.178154in}{3.022865in}}%
\pgfpathlineto{\pgfqpoint{2.189455in}{3.074614in}}%
\pgfpathlineto{\pgfqpoint{2.200756in}{2.775208in}}%
\pgfpathlineto{\pgfqpoint{2.212057in}{3.183657in}}%
\pgfpathlineto{\pgfqpoint{2.223358in}{3.085703in}}%
\pgfpathlineto{\pgfqpoint{2.234659in}{2.699432in}}%
\pgfpathlineto{\pgfqpoint{2.245960in}{3.231710in}}%
\pgfpathlineto{\pgfqpoint{2.257261in}{3.102337in}}%
\pgfpathlineto{\pgfqpoint{2.268562in}{2.856528in}}%
\pgfpathlineto{\pgfqpoint{2.279863in}{3.157782in}}%
\pgfpathlineto{\pgfqpoint{2.291164in}{3.009927in}}%
\pgfpathlineto{\pgfqpoint{2.302465in}{3.009927in}}%
\pgfpathlineto{\pgfqpoint{2.313766in}{3.111578in}}%
\pgfpathlineto{\pgfqpoint{2.325067in}{2.911974in}}%
\pgfpathlineto{\pgfqpoint{2.336368in}{3.194746in}}%
\pgfpathlineto{\pgfqpoint{2.347669in}{3.035802in}}%
\pgfpathlineto{\pgfqpoint{2.358970in}{2.767815in}}%
\pgfpathlineto{\pgfqpoint{2.370271in}{2.943393in}}%
\pgfpathlineto{\pgfqpoint{2.381572in}{2.594086in}}%
\pgfpathlineto{\pgfqpoint{2.392873in}{2.802931in}}%
\pgfpathlineto{\pgfqpoint{2.404174in}{2.884251in}}%
\pgfpathlineto{\pgfqpoint{2.415475in}{2.838046in}}%
\pgfpathlineto{\pgfqpoint{2.426776in}{3.128211in}}%
\pgfpathlineto{\pgfqpoint{2.438077in}{2.876858in}}%
\pgfpathlineto{\pgfqpoint{2.449378in}{3.078310in}}%
\pgfpathlineto{\pgfqpoint{2.460679in}{3.024713in}}%
\pgfpathlineto{\pgfqpoint{2.471980in}{2.984053in}}%
\pgfpathlineto{\pgfqpoint{2.483281in}{2.688343in}}%
\pgfpathlineto{\pgfqpoint{2.494582in}{2.621809in}}%
\pgfpathlineto{\pgfqpoint{2.505883in}{2.899036in}}%
\pgfpathlineto{\pgfqpoint{2.517184in}{3.460884in}}%
\pgfpathlineto{\pgfqpoint{2.528485in}{2.911974in}}%
\pgfpathlineto{\pgfqpoint{2.539786in}{3.065373in}}%
\pgfpathlineto{\pgfqpoint{2.551087in}{2.982205in}}%
\pgfpathlineto{\pgfqpoint{2.562388in}{2.919366in}}%
\pgfpathlineto{\pgfqpoint{2.573689in}{3.569927in}}%
\pgfpathlineto{\pgfqpoint{2.584990in}{3.331512in}}%
\pgfpathlineto{\pgfqpoint{2.596291in}{3.436858in}}%
\pgfpathlineto{\pgfqpoint{2.607592in}{3.183657in}}%
\pgfpathlineto{\pgfqpoint{2.618893in}{3.608739in}}%
\pgfpathlineto{\pgfqpoint{2.630194in}{3.366627in}}%
\pgfpathlineto{\pgfqpoint{2.641495in}{3.610587in}}%
\pgfpathlineto{\pgfqpoint{2.652796in}{3.697452in}}%
\pgfpathlineto{\pgfqpoint{2.664097in}{3.510785in}}%
\pgfpathlineto{\pgfqpoint{2.675398in}{3.699300in}}%
\pgfpathlineto{\pgfqpoint{2.686699in}{3.128211in}}%
\pgfpathlineto{\pgfqpoint{2.709301in}{3.355538in}}%
\pgfpathlineto{\pgfqpoint{2.720602in}{3.381413in}}%
\pgfpathlineto{\pgfqpoint{2.731902in}{3.279762in}}%
\pgfpathlineto{\pgfqpoint{2.743203in}{3.314878in}}%
\pgfpathlineto{\pgfqpoint{2.754504in}{3.091248in}}%
\pgfpathlineto{\pgfqpoint{2.765805in}{3.435010in}}%
\pgfpathlineto{\pgfqpoint{2.777106in}{3.191050in}}%
\pgfpathlineto{\pgfqpoint{2.788407in}{3.431314in}}%
\pgfpathlineto{\pgfqpoint{2.799708in}{3.479366in}}%
\pgfpathlineto{\pgfqpoint{2.811009in}{3.250191in}}%
\pgfpathlineto{\pgfqpoint{2.822310in}{3.337056in}}%
\pgfpathlineto{\pgfqpoint{2.833611in}{3.057980in}}%
\pgfpathlineto{\pgfqpoint{2.844912in}{3.179960in}}%
\pgfpathlineto{\pgfqpoint{2.856213in}{3.407287in}}%
\pgfpathlineto{\pgfqpoint{2.867514in}{3.264977in}}%
\pgfpathlineto{\pgfqpoint{2.878815in}{3.876726in}}%
\pgfpathlineto{\pgfqpoint{2.890116in}{3.257584in}}%
\pgfpathlineto{\pgfqpoint{2.901417in}{3.222469in}}%
\pgfpathlineto{\pgfqpoint{2.912718in}{3.357386in}}%
\pgfpathlineto{\pgfqpoint{2.924019in}{3.301941in}}%
\pgfpathlineto{\pgfqpoint{2.935320in}{3.566231in}}%
\pgfpathlineto{\pgfqpoint{2.946621in}{3.318574in}}%
\pgfpathlineto{\pgfqpoint{2.957922in}{3.459036in}}%
\pgfpathlineto{\pgfqpoint{2.969223in}{3.231710in}}%
\pgfpathlineto{\pgfqpoint{2.980524in}{3.296396in}}%
\pgfpathlineto{\pgfqpoint{2.991825in}{3.436858in}}%
\pgfpathlineto{\pgfqpoint{3.003126in}{3.425769in}}%
\pgfpathlineto{\pgfqpoint{3.014427in}{3.338904in}}%
\pgfpathlineto{\pgfqpoint{3.025728in}{3.496000in}}%
\pgfpathlineto{\pgfqpoint{3.037029in}{3.401743in}}%
\pgfpathlineto{\pgfqpoint{3.048330in}{3.507089in}}%
\pgfpathlineto{\pgfqpoint{3.059631in}{3.032106in}}%
\pgfpathlineto{\pgfqpoint{3.070932in}{3.264977in}}%
\pgfpathlineto{\pgfqpoint{3.082233in}{3.460884in}}%
\pgfpathlineto{\pgfqpoint{3.093534in}{3.224317in}}%
\pgfpathlineto{\pgfqpoint{3.104835in}{3.270521in}}%
\pgfpathlineto{\pgfqpoint{3.116136in}{2.995142in}}%
\pgfpathlineto{\pgfqpoint{3.127437in}{3.617980in}}%
\pgfpathlineto{\pgfqpoint{3.138738in}{3.187353in}}%
\pgfpathlineto{\pgfqpoint{3.150039in}{3.043195in}}%
\pgfpathlineto{\pgfqpoint{3.161340in}{3.202139in}}%
\pgfpathlineto{\pgfqpoint{3.172641in}{3.008079in}}%
\pgfpathlineto{\pgfqpoint{3.183942in}{3.191050in}}%
\pgfpathlineto{\pgfqpoint{3.195243in}{3.039498in}}%
\pgfpathlineto{\pgfqpoint{3.206544in}{3.246495in}}%
\pgfpathlineto{\pgfqpoint{3.217845in}{2.886099in}}%
\pgfpathlineto{\pgfqpoint{3.229146in}{2.926759in}}%
\pgfpathlineto{\pgfqpoint{3.240447in}{3.296396in}}%
\pgfpathlineto{\pgfqpoint{3.251748in}{2.982205in}}%
\pgfpathlineto{\pgfqpoint{3.263049in}{3.154086in}}%
\pgfpathlineto{\pgfqpoint{3.274350in}{3.556990in}}%
\pgfpathlineto{\pgfqpoint{3.285650in}{3.109729in}}%
\pgfpathlineto{\pgfqpoint{3.296951in}{3.290851in}}%
\pgfpathlineto{\pgfqpoint{3.308252in}{3.240950in}}%
\pgfpathlineto{\pgfqpoint{3.319553in}{2.996990in}}%
\pgfpathlineto{\pgfqpoint{3.330854in}{3.327815in}}%
\pgfpathlineto{\pgfqpoint{3.342155in}{3.043195in}}%
\pgfpathlineto{\pgfqpoint{3.353456in}{3.107881in}}%
\pgfpathlineto{\pgfqpoint{3.364757in}{3.325967in}}%
\pgfpathlineto{\pgfqpoint{3.387359in}{3.176264in}}%
\pgfpathlineto{\pgfqpoint{3.398660in}{3.568079in}}%
\pgfpathlineto{\pgfqpoint{3.409961in}{3.616132in}}%
\pgfpathlineto{\pgfqpoint{3.421262in}{3.416528in}}%
\pgfpathlineto{\pgfqpoint{3.432563in}{3.388805in}}%
\pgfpathlineto{\pgfqpoint{3.443864in}{3.466429in}}%
\pgfpathlineto{\pgfqpoint{3.455165in}{3.447947in}}%
\pgfpathlineto{\pgfqpoint{3.466466in}{3.880422in}}%
\pgfpathlineto{\pgfqpoint{3.477767in}{3.292700in}}%
\pgfpathlineto{\pgfqpoint{3.489068in}{3.496000in}}%
\pgfpathlineto{\pgfqpoint{3.500369in}{3.468277in}}%
\pgfpathlineto{\pgfqpoint{3.511670in}{3.555142in}}%
\pgfpathlineto{\pgfqpoint{3.522971in}{3.435010in}}%
\pgfpathlineto{\pgfqpoint{3.534272in}{3.595802in}}%
\pgfpathlineto{\pgfqpoint{3.545573in}{3.479366in}}%
\pgfpathlineto{\pgfqpoint{3.556874in}{3.471974in}}%
\pgfpathlineto{\pgfqpoint{3.568175in}{3.410983in}}%
\pgfpathlineto{\pgfqpoint{3.579476in}{3.601347in}}%
\pgfpathlineto{\pgfqpoint{3.590777in}{3.285307in}}%
\pgfpathlineto{\pgfqpoint{3.602078in}{3.320422in}}%
\pgfpathlineto{\pgfqpoint{3.613379in}{3.723327in}}%
\pgfpathlineto{\pgfqpoint{3.624680in}{3.435010in}}%
\pgfpathlineto{\pgfqpoint{3.635981in}{3.634614in}}%
\pgfpathlineto{\pgfqpoint{3.647282in}{3.765835in}}%
\pgfpathlineto{\pgfqpoint{3.658583in}{3.599498in}}%
\pgfpathlineto{\pgfqpoint{3.669884in}{3.560686in}}%
\pgfpathlineto{\pgfqpoint{3.681185in}{3.301941in}}%
\pgfpathlineto{\pgfqpoint{3.692486in}{3.366627in}}%
\pgfpathlineto{\pgfqpoint{3.703787in}{3.566231in}}%
\pgfpathlineto{\pgfqpoint{3.715088in}{3.688211in}}%
\pgfpathlineto{\pgfqpoint{3.726389in}{3.836066in}}%
\pgfpathlineto{\pgfqpoint{3.737690in}{3.701149in}}%
\pgfpathlineto{\pgfqpoint{3.748991in}{3.625373in}}%
\pgfpathlineto{\pgfqpoint{3.760292in}{3.630917in}}%
\pgfpathlineto{\pgfqpoint{3.771593in}{3.812040in}}%
\pgfpathlineto{\pgfqpoint{3.782894in}{3.516330in}}%
\pgfpathlineto{\pgfqpoint{3.794195in}{3.664185in}}%
\pgfpathlineto{\pgfqpoint{3.805496in}{3.571776in}}%
\pgfpathlineto{\pgfqpoint{3.816797in}{3.603195in}}%
\pgfpathlineto{\pgfqpoint{3.828098in}{3.492304in}}%
\pgfpathlineto{\pgfqpoint{3.839398in}{3.830521in}}%
\pgfpathlineto{\pgfqpoint{3.850699in}{3.512634in}}%
\pgfpathlineto{\pgfqpoint{3.862000in}{3.684515in}}%
\pgfpathlineto{\pgfqpoint{3.873301in}{3.776924in}}%
\pgfpathlineto{\pgfqpoint{3.884602in}{3.575472in}}%
\pgfpathlineto{\pgfqpoint{3.895903in}{3.664185in}}%
\pgfpathlineto{\pgfqpoint{3.907204in}{3.449795in}}%
\pgfpathlineto{\pgfqpoint{3.918505in}{3.861941in}}%
\pgfpathlineto{\pgfqpoint{3.929806in}{3.566231in}}%
\pgfpathlineto{\pgfqpoint{3.941107in}{3.545901in}}%
\pgfpathlineto{\pgfqpoint{3.952408in}{3.514482in}}%
\pgfpathlineto{\pgfqpoint{3.963709in}{3.590257in}}%
\pgfpathlineto{\pgfqpoint{3.975010in}{3.327815in}}%
\pgfpathlineto{\pgfqpoint{3.986311in}{3.647551in}}%
\pgfpathlineto{\pgfqpoint{3.997612in}{3.863789in}}%
\pgfpathlineto{\pgfqpoint{4.008913in}{3.653096in}}%
\pgfpathlineto{\pgfqpoint{4.020214in}{3.621677in}}%
\pgfpathlineto{\pgfqpoint{4.031515in}{3.442403in}}%
\pgfpathlineto{\pgfqpoint{4.042816in}{3.521875in}}%
\pgfpathlineto{\pgfqpoint{4.054117in}{3.566231in}}%
\pgfpathlineto{\pgfqpoint{4.065418in}{3.782469in}}%
\pgfpathlineto{\pgfqpoint{4.076719in}{3.699300in}}%
\pgfpathlineto{\pgfqpoint{4.088020in}{3.727023in}}%
\pgfpathlineto{\pgfqpoint{4.110622in}{3.401743in}}%
\pgfpathlineto{\pgfqpoint{4.121923in}{3.627221in}}%
\pgfpathlineto{\pgfqpoint{4.133224in}{3.571776in}}%
\pgfpathlineto{\pgfqpoint{4.144525in}{3.621677in}}%
\pgfpathlineto{\pgfqpoint{4.155826in}{3.760290in}}%
\pgfpathlineto{\pgfqpoint{4.167127in}{3.662337in}}%
\pgfpathlineto{\pgfqpoint{4.178428in}{3.651248in}}%
\pgfpathlineto{\pgfqpoint{4.189729in}{3.492304in}}%
\pgfpathlineto{\pgfqpoint{4.201030in}{3.793558in}}%
\pgfpathlineto{\pgfqpoint{4.212331in}{3.573624in}}%
\pgfpathlineto{\pgfqpoint{4.223632in}{3.640158in}}%
\pgfpathlineto{\pgfqpoint{4.234933in}{3.547749in}}%
\pgfpathlineto{\pgfqpoint{4.246234in}{3.553294in}}%
\pgfpathlineto{\pgfqpoint{4.257535in}{3.542205in}}%
\pgfpathlineto{\pgfqpoint{4.268836in}{3.314878in}}%
\pgfpathlineto{\pgfqpoint{4.280137in}{3.481215in}}%
\pgfpathlineto{\pgfqpoint{4.291438in}{3.728871in}}%
\pgfpathlineto{\pgfqpoint{4.302739in}{3.510785in}}%
\pgfpathlineto{\pgfqpoint{4.314040in}{3.630917in}}%
\pgfpathlineto{\pgfqpoint{4.325341in}{3.475670in}}%
\pgfpathlineto{\pgfqpoint{4.336642in}{3.941413in}}%
\pgfpathlineto{\pgfqpoint{4.347943in}{3.335208in}}%
\pgfpathlineto{\pgfqpoint{4.359244in}{3.340752in}}%
\pgfpathlineto{\pgfqpoint{4.370545in}{3.534812in}}%
\pgfpathlineto{\pgfqpoint{4.381846in}{3.549597in}}%
\pgfpathlineto{\pgfqpoint{4.393147in}{3.823129in}}%
\pgfpathlineto{\pgfqpoint{4.404447in}{3.590257in}}%
\pgfpathlineto{\pgfqpoint{4.415748in}{3.754746in}}%
\pgfpathlineto{\pgfqpoint{4.427049in}{3.706693in}}%
\pgfpathlineto{\pgfqpoint{4.438350in}{3.536660in}}%
\pgfpathlineto{\pgfqpoint{4.449651in}{3.568079in}}%
\pgfpathlineto{\pgfqpoint{4.460952in}{3.592106in}}%
\pgfpathlineto{\pgfqpoint{4.472253in}{3.728871in}}%
\pgfpathlineto{\pgfqpoint{4.483554in}{3.799102in}}%
\pgfpathlineto{\pgfqpoint{4.494855in}{3.693756in}}%
\pgfpathlineto{\pgfqpoint{4.506156in}{3.647551in}}%
\pgfpathlineto{\pgfqpoint{4.517457in}{3.592106in}}%
\pgfpathlineto{\pgfqpoint{4.528758in}{3.634614in}}%
\pgfpathlineto{\pgfqpoint{4.540059in}{3.784317in}}%
\pgfpathlineto{\pgfqpoint{4.551360in}{3.841611in}}%
\pgfpathlineto{\pgfqpoint{4.562661in}{3.970983in}}%
\pgfpathlineto{\pgfqpoint{4.573962in}{3.680818in}}%
\pgfpathlineto{\pgfqpoint{4.585263in}{3.338904in}}%
\pgfpathlineto{\pgfqpoint{4.596564in}{3.100488in}}%
\pgfpathlineto{\pgfqpoint{4.607865in}{3.529267in}}%
\pgfpathlineto{\pgfqpoint{4.619166in}{3.717782in}}%
\pgfpathlineto{\pgfqpoint{4.630467in}{3.444251in}}%
\pgfpathlineto{\pgfqpoint{4.641768in}{3.486759in}}%
\pgfpathlineto{\pgfqpoint{4.653069in}{3.782469in}}%
\pgfpathlineto{\pgfqpoint{4.664370in}{3.841611in}}%
\pgfpathlineto{\pgfqpoint{4.675671in}{3.353690in}}%
\pgfpathlineto{\pgfqpoint{4.686972in}{3.457188in}}%
\pgfpathlineto{\pgfqpoint{4.698273in}{3.490455in}}%
\pgfpathlineto{\pgfqpoint{4.709574in}{3.640158in}}%
\pgfpathlineto{\pgfqpoint{4.720875in}{3.699300in}}%
\pgfpathlineto{\pgfqpoint{4.732176in}{3.366627in}}%
\pgfpathlineto{\pgfqpoint{4.743477in}{3.556990in}}%
\pgfpathlineto{\pgfqpoint{4.754778in}{3.492304in}}%
\pgfpathlineto{\pgfqpoint{4.766079in}{3.677122in}}%
\pgfpathlineto{\pgfqpoint{4.777380in}{3.483063in}}%
\pgfpathlineto{\pgfqpoint{4.788681in}{3.414680in}}%
\pgfpathlineto{\pgfqpoint{4.799982in}{3.849003in}}%
\pgfpathlineto{\pgfqpoint{4.811283in}{3.586561in}}%
\pgfpathlineto{\pgfqpoint{4.822584in}{3.619828in}}%
\pgfpathlineto{\pgfqpoint{4.833885in}{3.593954in}}%
\pgfpathlineto{\pgfqpoint{4.845186in}{3.325967in}}%
\pgfpathlineto{\pgfqpoint{4.856487in}{3.429465in}}%
\pgfpathlineto{\pgfqpoint{4.867788in}{3.385109in}}%
\pgfpathlineto{\pgfqpoint{4.879089in}{3.453492in}}%
\pgfpathlineto{\pgfqpoint{4.890390in}{3.688211in}}%
\pgfpathlineto{\pgfqpoint{4.901691in}{3.603195in}}%
\pgfpathlineto{\pgfqpoint{4.912992in}{3.532964in}}%
\pgfpathlineto{\pgfqpoint{4.924293in}{3.438706in}}%
\pgfpathlineto{\pgfqpoint{4.935594in}{3.496000in}}%
\pgfpathlineto{\pgfqpoint{4.946895in}{3.324119in}}%
\pgfpathlineto{\pgfqpoint{4.958195in}{3.431314in}}%
\pgfpathlineto{\pgfqpoint{4.969496in}{3.775076in}}%
\pgfpathlineto{\pgfqpoint{4.980797in}{3.466429in}}%
\pgfpathlineto{\pgfqpoint{4.992098in}{3.239102in}}%
\pgfpathlineto{\pgfqpoint{5.003399in}{3.154086in}}%
\pgfpathlineto{\pgfqpoint{5.014700in}{3.416528in}}%
\pgfpathlineto{\pgfqpoint{5.026001in}{3.179960in}}%
\pgfpathlineto{\pgfqpoint{5.037302in}{3.693756in}}%
\pgfpathlineto{\pgfqpoint{5.048603in}{3.325967in}}%
\pgfpathlineto{\pgfqpoint{5.059904in}{3.327815in}}%
\pgfpathlineto{\pgfqpoint{5.071205in}{3.749201in}}%
\pgfpathlineto{\pgfqpoint{5.082506in}{3.423921in}}%
\pgfpathlineto{\pgfqpoint{5.093807in}{3.385109in}}%
\pgfpathlineto{\pgfqpoint{5.105108in}{3.593954in}}%
\pgfpathlineto{\pgfqpoint{5.116409in}{3.629069in}}%
\pgfpathlineto{\pgfqpoint{5.127710in}{3.754746in}}%
\pgfpathlineto{\pgfqpoint{5.139011in}{3.584713in}}%
\pgfpathlineto{\pgfqpoint{5.150312in}{3.335208in}}%
\pgfpathlineto{\pgfqpoint{5.161613in}{3.462733in}}%
\pgfpathlineto{\pgfqpoint{5.172914in}{3.610587in}}%
\pgfpathlineto{\pgfqpoint{5.184215in}{3.538508in}}%
\pgfpathlineto{\pgfqpoint{5.195516in}{3.568079in}}%
\pgfpathlineto{\pgfqpoint{5.206817in}{3.516330in}}%
\pgfpathlineto{\pgfqpoint{5.218118in}{3.581017in}}%
\pgfpathlineto{\pgfqpoint{5.229419in}{3.503393in}}%
\pgfpathlineto{\pgfqpoint{5.240720in}{3.449795in}}%
\pgfpathlineto{\pgfqpoint{5.252021in}{3.579168in}}%
\pgfpathlineto{\pgfqpoint{5.263322in}{3.520026in}}%
\pgfpathlineto{\pgfqpoint{5.274623in}{3.322271in}}%
\pgfpathlineto{\pgfqpoint{5.285924in}{3.728871in}}%
\pgfpathlineto{\pgfqpoint{5.297225in}{3.494152in}}%
\pgfpathlineto{\pgfqpoint{5.308526in}{3.484911in}}%
\pgfpathlineto{\pgfqpoint{5.319827in}{3.664185in}}%
\pgfpathlineto{\pgfqpoint{5.331128in}{3.702997in}}%
\pgfpathlineto{\pgfqpoint{5.342429in}{3.556990in}}%
\pgfpathlineto{\pgfqpoint{5.353730in}{3.144845in}}%
\pgfpathlineto{\pgfqpoint{5.365031in}{3.564383in}}%
\pgfpathlineto{\pgfqpoint{5.376332in}{3.386957in}}%
\pgfpathlineto{\pgfqpoint{5.387633in}{3.610587in}}%
\pgfpathlineto{\pgfqpoint{5.398934in}{3.501545in}}%
\pgfpathlineto{\pgfqpoint{5.410235in}{3.216924in}}%
\pgfpathlineto{\pgfqpoint{5.421536in}{3.325967in}}%
\pgfpathlineto{\pgfqpoint{5.432837in}{3.189201in}}%
\pgfpathlineto{\pgfqpoint{5.444138in}{3.422073in}}%
\pgfpathlineto{\pgfqpoint{5.455439in}{3.758442in}}%
\pgfpathlineto{\pgfqpoint{5.466740in}{3.521875in}}%
\pgfpathlineto{\pgfqpoint{5.478041in}{3.157782in}}%
\pgfpathlineto{\pgfqpoint{5.489342in}{3.623525in}}%
\pgfpathlineto{\pgfqpoint{5.500643in}{3.712238in}}%
\pgfpathlineto{\pgfqpoint{5.511943in}{3.551446in}}%
\pgfpathlineto{\pgfqpoint{5.523244in}{3.248343in}}%
\pgfpathlineto{\pgfqpoint{5.534545in}{3.468277in}}%
\pgfpathlineto{\pgfqpoint{5.534545in}{3.468277in}}%
\pgfusepath{stroke}%
\end{pgfscope}%
\begin{pgfscope}%
\pgfpathrectangle{\pgfqpoint{0.800000in}{0.528000in}}{\pgfqpoint{4.960000in}{3.696000in}} %
\pgfusepath{clip}%
\pgfsetrectcap%
\pgfsetroundjoin%
\pgfsetlinewidth{1.505625pt}%
\definecolor{currentstroke}{rgb}{0.600000,0.600000,0.600000}%
\pgfsetstrokecolor{currentstroke}%
\pgfsetdash{}{0pt}%
\pgfpathmoveto{\pgfqpoint{1.025455in}{0.792106in}}%
\pgfpathlineto{\pgfqpoint{1.036756in}{0.838310in}}%
\pgfpathlineto{\pgfqpoint{1.048057in}{1.176528in}}%
\pgfpathlineto{\pgfqpoint{1.059357in}{0.930719in}}%
\pgfpathlineto{\pgfqpoint{1.070658in}{1.220884in}}%
\pgfpathlineto{\pgfqpoint{1.081959in}{1.167287in}}%
\pgfpathlineto{\pgfqpoint{1.093260in}{1.267089in}}%
\pgfpathlineto{\pgfqpoint{1.104561in}{1.202403in}}%
\pgfpathlineto{\pgfqpoint{1.115862in}{1.370587in}}%
\pgfpathlineto{\pgfqpoint{1.127163in}{1.207947in}}%
\pgfpathlineto{\pgfqpoint{1.138464in}{1.318838in}}%
\pgfpathlineto{\pgfqpoint{1.149765in}{1.307749in}}%
\pgfpathlineto{\pgfqpoint{1.161066in}{1.527683in}}%
\pgfpathlineto{\pgfqpoint{1.172367in}{1.296660in}}%
\pgfpathlineto{\pgfqpoint{1.183668in}{1.455604in}}%
\pgfpathlineto{\pgfqpoint{1.194969in}{1.170983in}}%
\pgfpathlineto{\pgfqpoint{1.217571in}{1.403855in}}%
\pgfpathlineto{\pgfqpoint{1.228872in}{1.294812in}}%
\pgfpathlineto{\pgfqpoint{1.240173in}{1.418640in}}%
\pgfpathlineto{\pgfqpoint{1.251474in}{1.366891in}}%
\pgfpathlineto{\pgfqpoint{1.262775in}{1.209795in}}%
\pgfpathlineto{\pgfqpoint{1.274076in}{1.599762in}}%
\pgfpathlineto{\pgfqpoint{1.285377in}{1.261545in}}%
\pgfpathlineto{\pgfqpoint{1.296678in}{1.137716in}}%
\pgfpathlineto{\pgfqpoint{1.307979in}{1.305901in}}%
\pgfpathlineto{\pgfqpoint{1.319280in}{1.200554in}}%
\pgfpathlineto{\pgfqpoint{1.330581in}{0.988013in}}%
\pgfpathlineto{\pgfqpoint{1.341882in}{1.137716in}}%
\pgfpathlineto{\pgfqpoint{1.353183in}{1.426033in}}%
\pgfpathlineto{\pgfqpoint{1.364484in}{1.518442in}}%
\pgfpathlineto{\pgfqpoint{1.375785in}{1.132172in}}%
\pgfpathlineto{\pgfqpoint{1.387086in}{1.365043in}}%
\pgfpathlineto{\pgfqpoint{1.398387in}{1.254152in}}%
\pgfpathlineto{\pgfqpoint{1.409688in}{1.416792in}}%
\pgfpathlineto{\pgfqpoint{1.420989in}{1.320686in}}%
\pgfpathlineto{\pgfqpoint{1.432290in}{1.475934in}}%
\pgfpathlineto{\pgfqpoint{1.443591in}{1.472238in}}%
\pgfpathlineto{\pgfqpoint{1.454892in}{1.536924in}}%
\pgfpathlineto{\pgfqpoint{1.466193in}{1.773492in}}%
\pgfpathlineto{\pgfqpoint{1.477494in}{1.518442in}}%
\pgfpathlineto{\pgfqpoint{1.488795in}{1.634878in}}%
\pgfpathlineto{\pgfqpoint{1.500096in}{1.250455in}}%
\pgfpathlineto{\pgfqpoint{1.511397in}{1.496264in}}%
\pgfpathlineto{\pgfqpoint{1.522698in}{1.603459in}}%
\pgfpathlineto{\pgfqpoint{1.533999in}{1.396462in}}%
\pgfpathlineto{\pgfqpoint{1.545300in}{1.261545in}}%
\pgfpathlineto{\pgfqpoint{1.556601in}{1.161743in}}%
\pgfpathlineto{\pgfqpoint{1.567902in}{1.402007in}}%
\pgfpathlineto{\pgfqpoint{1.579203in}{1.529531in}}%
\pgfpathlineto{\pgfqpoint{1.590504in}{1.174680in}}%
\pgfpathlineto{\pgfqpoint{1.601805in}{1.420488in}}%
\pgfpathlineto{\pgfqpoint{1.613105in}{1.540620in}}%
\pgfpathlineto{\pgfqpoint{1.624406in}{1.363195in}}%
\pgfpathlineto{\pgfqpoint{1.635707in}{1.527683in}}%
\pgfpathlineto{\pgfqpoint{1.647008in}{1.374284in}}%
\pgfpathlineto{\pgfqpoint{1.658309in}{1.461149in}}%
\pgfpathlineto{\pgfqpoint{1.669610in}{1.629333in}}%
\pgfpathlineto{\pgfqpoint{1.680911in}{1.331776in}}%
\pgfpathlineto{\pgfqpoint{1.692212in}{1.396462in}}%
\pgfpathlineto{\pgfqpoint{1.703513in}{1.350257in}}%
\pgfpathlineto{\pgfqpoint{1.714814in}{1.187617in}}%
\pgfpathlineto{\pgfqpoint{1.726115in}{1.329927in}}%
\pgfpathlineto{\pgfqpoint{1.737416in}{1.523987in}}%
\pgfpathlineto{\pgfqpoint{1.748717in}{1.464845in}}%
\pgfpathlineto{\pgfqpoint{1.760018in}{1.292964in}}%
\pgfpathlineto{\pgfqpoint{1.771319in}{1.664449in}}%
\pgfpathlineto{\pgfqpoint{1.782620in}{1.383525in}}%
\pgfpathlineto{\pgfqpoint{1.793921in}{1.527683in}}%
\pgfpathlineto{\pgfqpoint{1.805222in}{1.265241in}}%
\pgfpathlineto{\pgfqpoint{1.816523in}{1.233822in}}%
\pgfpathlineto{\pgfqpoint{1.827824in}{1.246759in}}%
\pgfpathlineto{\pgfqpoint{1.839125in}{1.394614in}}%
\pgfpathlineto{\pgfqpoint{1.850426in}{1.268937in}}%
\pgfpathlineto{\pgfqpoint{1.861727in}{1.721743in}}%
\pgfpathlineto{\pgfqpoint{1.873028in}{1.618244in}}%
\pgfpathlineto{\pgfqpoint{1.884329in}{1.185769in}}%
\pgfpathlineto{\pgfqpoint{1.895630in}{1.675538in}}%
\pgfpathlineto{\pgfqpoint{1.906931in}{1.548013in}}%
\pgfpathlineto{\pgfqpoint{1.918232in}{1.609003in}}%
\pgfpathlineto{\pgfqpoint{1.929533in}{1.854812in}}%
\pgfpathlineto{\pgfqpoint{1.940834in}{1.636726in}}%
\pgfpathlineto{\pgfqpoint{1.952135in}{1.466693in}}%
\pgfpathlineto{\pgfqpoint{1.963436in}{1.499960in}}%
\pgfpathlineto{\pgfqpoint{1.974737in}{1.366891in}}%
\pgfpathlineto{\pgfqpoint{1.986038in}{1.455604in}}%
\pgfpathlineto{\pgfqpoint{1.997339in}{1.618244in}}%
\pgfpathlineto{\pgfqpoint{2.008640in}{1.629333in}}%
\pgfpathlineto{\pgfqpoint{2.019941in}{1.294812in}}%
\pgfpathlineto{\pgfqpoint{2.031242in}{1.446363in}}%
\pgfpathlineto{\pgfqpoint{2.042543in}{1.767947in}}%
\pgfpathlineto{\pgfqpoint{2.053844in}{1.305901in}}%
\pgfpathlineto{\pgfqpoint{2.065145in}{1.838178in}}%
\pgfpathlineto{\pgfqpoint{2.076446in}{1.420488in}}%
\pgfpathlineto{\pgfqpoint{2.087747in}{1.590521in}}%
\pgfpathlineto{\pgfqpoint{2.099048in}{1.638574in}}%
\pgfpathlineto{\pgfqpoint{2.110349in}{1.450059in}}%
\pgfpathlineto{\pgfqpoint{2.121650in}{1.791974in}}%
\pgfpathlineto{\pgfqpoint{2.132951in}{1.790125in}}%
\pgfpathlineto{\pgfqpoint{2.144252in}{1.457452in}}%
\pgfpathlineto{\pgfqpoint{2.155553in}{1.499960in}}%
\pgfpathlineto{\pgfqpoint{2.166853in}{1.475934in}}%
\pgfpathlineto{\pgfqpoint{2.178154in}{1.791974in}}%
\pgfpathlineto{\pgfqpoint{2.189455in}{1.705109in}}%
\pgfpathlineto{\pgfqpoint{2.200756in}{1.901017in}}%
\pgfpathlineto{\pgfqpoint{2.212057in}{1.906561in}}%
\pgfpathlineto{\pgfqpoint{2.223358in}{1.341017in}}%
\pgfpathlineto{\pgfqpoint{2.234659in}{1.864053in}}%
\pgfpathlineto{\pgfqpoint{2.245960in}{1.692172in}}%
\pgfpathlineto{\pgfqpoint{2.257261in}{1.888079in}}%
\pgfpathlineto{\pgfqpoint{2.268562in}{1.499960in}}%
\pgfpathlineto{\pgfqpoint{2.279863in}{1.503657in}}%
\pgfpathlineto{\pgfqpoint{2.291164in}{1.481479in}}%
\pgfpathlineto{\pgfqpoint{2.302465in}{1.638574in}}%
\pgfpathlineto{\pgfqpoint{2.313766in}{1.581281in}}%
\pgfpathlineto{\pgfqpoint{2.325067in}{1.448211in}}%
\pgfpathlineto{\pgfqpoint{2.336368in}{1.353954in}}%
\pgfpathlineto{\pgfqpoint{2.347669in}{1.596066in}}%
\pgfpathlineto{\pgfqpoint{2.358970in}{1.965703in}}%
\pgfpathlineto{\pgfqpoint{2.370271in}{1.311446in}}%
\pgfpathlineto{\pgfqpoint{2.381572in}{1.875142in}}%
\pgfpathlineto{\pgfqpoint{2.392873in}{2.122799in}}%
\pgfpathlineto{\pgfqpoint{2.404174in}{2.178244in}}%
\pgfpathlineto{\pgfqpoint{2.415475in}{1.991578in}}%
\pgfpathlineto{\pgfqpoint{2.426776in}{2.100620in}}%
\pgfpathlineto{\pgfqpoint{2.438077in}{2.172700in}}%
\pgfpathlineto{\pgfqpoint{2.449378in}{2.108013in}}%
\pgfpathlineto{\pgfqpoint{2.460679in}{1.902865in}}%
\pgfpathlineto{\pgfqpoint{2.471980in}{2.095076in}}%
\pgfpathlineto{\pgfqpoint{2.483281in}{1.884383in}}%
\pgfpathlineto{\pgfqpoint{2.494582in}{2.218904in}}%
\pgfpathlineto{\pgfqpoint{2.505883in}{2.193030in}}%
\pgfpathlineto{\pgfqpoint{2.517184in}{2.222601in}}%
\pgfpathlineto{\pgfqpoint{2.528485in}{2.357518in}}%
\pgfpathlineto{\pgfqpoint{2.539786in}{2.248475in}}%
\pgfpathlineto{\pgfqpoint{2.551087in}{2.076594in}}%
\pgfpathlineto{\pgfqpoint{2.562388in}{1.998970in}}%
\pgfpathlineto{\pgfqpoint{2.573689in}{1.949069in}}%
\pgfpathlineto{\pgfqpoint{2.584990in}{2.087683in}}%
\pgfpathlineto{\pgfqpoint{2.596291in}{2.161611in}}%
\pgfpathlineto{\pgfqpoint{2.607592in}{2.213360in}}%
\pgfpathlineto{\pgfqpoint{2.618893in}{1.801215in}}%
\pgfpathlineto{\pgfqpoint{2.630194in}{2.015604in}}%
\pgfpathlineto{\pgfqpoint{2.641495in}{2.146825in}}%
\pgfpathlineto{\pgfqpoint{2.652796in}{1.708805in}}%
\pgfpathlineto{\pgfqpoint{2.664097in}{1.734680in}}%
\pgfpathlineto{\pgfqpoint{2.675398in}{2.059960in}}%
\pgfpathlineto{\pgfqpoint{2.686699in}{1.932436in}}%
\pgfpathlineto{\pgfqpoint{2.698000in}{1.738376in}}%
\pgfpathlineto{\pgfqpoint{2.709301in}{1.827089in}}%
\pgfpathlineto{\pgfqpoint{2.720602in}{1.897320in}}%
\pgfpathlineto{\pgfqpoint{2.731902in}{1.867749in}}%
\pgfpathlineto{\pgfqpoint{2.743203in}{1.817848in}}%
\pgfpathlineto{\pgfqpoint{2.754504in}{2.095076in}}%
\pgfpathlineto{\pgfqpoint{2.765805in}{1.974944in}}%
\pgfpathlineto{\pgfqpoint{2.777106in}{1.825241in}}%
\pgfpathlineto{\pgfqpoint{2.788407in}{2.065505in}}%
\pgfpathlineto{\pgfqpoint{2.799708in}{1.786429in}}%
\pgfpathlineto{\pgfqpoint{2.811009in}{1.899168in}}%
\pgfpathlineto{\pgfqpoint{2.822310in}{1.636726in}}%
\pgfpathlineto{\pgfqpoint{2.833611in}{1.901017in}}%
\pgfpathlineto{\pgfqpoint{2.844912in}{1.684779in}}%
\pgfpathlineto{\pgfqpoint{2.856213in}{2.043327in}}%
\pgfpathlineto{\pgfqpoint{2.867514in}{2.039630in}}%
\pgfpathlineto{\pgfqpoint{2.878815in}{1.823393in}}%
\pgfpathlineto{\pgfqpoint{2.890116in}{1.993426in}}%
\pgfpathlineto{\pgfqpoint{2.901417in}{1.998970in}}%
\pgfpathlineto{\pgfqpoint{2.912718in}{2.274350in}}%
\pgfpathlineto{\pgfqpoint{2.924019in}{2.034086in}}%
\pgfpathlineto{\pgfqpoint{2.935320in}{2.229993in}}%
\pgfpathlineto{\pgfqpoint{2.946621in}{2.283591in}}%
\pgfpathlineto{\pgfqpoint{2.957922in}{1.950917in}}%
\pgfpathlineto{\pgfqpoint{2.969223in}{2.083987in}}%
\pgfpathlineto{\pgfqpoint{2.980524in}{2.144977in}}%
\pgfpathlineto{\pgfqpoint{2.991825in}{2.242931in}}%
\pgfpathlineto{\pgfqpoint{3.003126in}{2.222601in}}%
\pgfpathlineto{\pgfqpoint{3.014427in}{2.283591in}}%
\pgfpathlineto{\pgfqpoint{3.025728in}{2.466561in}}%
\pgfpathlineto{\pgfqpoint{3.037029in}{2.178244in}}%
\pgfpathlineto{\pgfqpoint{3.048330in}{2.102469in}}%
\pgfpathlineto{\pgfqpoint{3.059631in}{2.181941in}}%
\pgfpathlineto{\pgfqpoint{3.082233in}{2.518310in}}%
\pgfpathlineto{\pgfqpoint{3.093534in}{2.536792in}}%
\pgfpathlineto{\pgfqpoint{3.104835in}{2.281743in}}%
\pgfpathlineto{\pgfqpoint{3.116136in}{2.198574in}}%
\pgfpathlineto{\pgfqpoint{3.127437in}{2.516462in}}%
\pgfpathlineto{\pgfqpoint{3.138738in}{2.183789in}}%
\pgfpathlineto{\pgfqpoint{3.150039in}{2.420356in}}%
\pgfpathlineto{\pgfqpoint{3.161340in}{2.316858in}}%
\pgfpathlineto{\pgfqpoint{3.172641in}{2.414812in}}%
\pgfpathlineto{\pgfqpoint{3.183942in}{2.276198in}}%
\pgfpathlineto{\pgfqpoint{3.195243in}{2.300224in}}%
\pgfpathlineto{\pgfqpoint{3.206544in}{2.331644in}}%
\pgfpathlineto{\pgfqpoint{3.217845in}{2.568211in}}%
\pgfpathlineto{\pgfqpoint{3.229146in}{2.601479in}}%
\pgfpathlineto{\pgfqpoint{3.240447in}{2.682799in}}%
\pgfpathlineto{\pgfqpoint{3.251748in}{2.069201in}}%
\pgfpathlineto{\pgfqpoint{3.263049in}{2.453624in}}%
\pgfpathlineto{\pgfqpoint{3.274350in}{2.311314in}}%
\pgfpathlineto{\pgfqpoint{3.285650in}{2.246627in}}%
\pgfpathlineto{\pgfqpoint{3.296951in}{2.211512in}}%
\pgfpathlineto{\pgfqpoint{3.308252in}{2.229993in}}%
\pgfpathlineto{\pgfqpoint{3.319553in}{2.424053in}}%
\pgfpathlineto{\pgfqpoint{3.330854in}{2.712370in}}%
\pgfpathlineto{\pgfqpoint{3.342155in}{2.472106in}}%
\pgfpathlineto{\pgfqpoint{3.353456in}{2.196726in}}%
\pgfpathlineto{\pgfqpoint{3.364757in}{2.490587in}}%
\pgfpathlineto{\pgfqpoint{3.376058in}{2.494284in}}%
\pgfpathlineto{\pgfqpoint{3.387359in}{2.379696in}}%
\pgfpathlineto{\pgfqpoint{3.398660in}{2.472106in}}%
\pgfpathlineto{\pgfqpoint{3.409961in}{2.246627in}}%
\pgfpathlineto{\pgfqpoint{3.421262in}{2.259564in}}%
\pgfpathlineto{\pgfqpoint{3.432563in}{2.339036in}}%
\pgfpathlineto{\pgfqpoint{3.443864in}{2.220752in}}%
\pgfpathlineto{\pgfqpoint{3.455165in}{2.281743in}}%
\pgfpathlineto{\pgfqpoint{3.466466in}{2.392634in}}%
\pgfpathlineto{\pgfqpoint{3.477767in}{2.472106in}}%
\pgfpathlineto{\pgfqpoint{3.489068in}{2.296528in}}%
\pgfpathlineto{\pgfqpoint{3.500369in}{2.146825in}}%
\pgfpathlineto{\pgfqpoint{3.511670in}{2.187485in}}%
\pgfpathlineto{\pgfqpoint{3.522971in}{2.296528in}}%
\pgfpathlineto{\pgfqpoint{3.534272in}{2.119102in}}%
\pgfpathlineto{\pgfqpoint{3.545573in}{2.198574in}}%
\pgfpathlineto{\pgfqpoint{3.556874in}{2.172700in}}%
\pgfpathlineto{\pgfqpoint{3.568175in}{2.106165in}}%
\pgfpathlineto{\pgfqpoint{3.579476in}{2.257716in}}%
\pgfpathlineto{\pgfqpoint{3.590777in}{2.363063in}}%
\pgfpathlineto{\pgfqpoint{3.602078in}{2.420356in}}%
\pgfpathlineto{\pgfqpoint{3.613379in}{2.082139in}}%
\pgfpathlineto{\pgfqpoint{3.624680in}{2.313162in}}%
\pgfpathlineto{\pgfqpoint{3.635981in}{2.451776in}}%
\pgfpathlineto{\pgfqpoint{3.647282in}{2.250323in}}%
\pgfpathlineto{\pgfqpoint{3.658583in}{2.178244in}}%
\pgfpathlineto{\pgfqpoint{3.669884in}{2.207815in}}%
\pgfpathlineto{\pgfqpoint{3.681185in}{2.409267in}}%
\pgfpathlineto{\pgfqpoint{3.692486in}{2.442535in}}%
\pgfpathlineto{\pgfqpoint{3.703787in}{2.252172in}}%
\pgfpathlineto{\pgfqpoint{3.726389in}{2.492436in}}%
\pgfpathlineto{\pgfqpoint{3.737690in}{2.313162in}}%
\pgfpathlineto{\pgfqpoint{3.748991in}{2.176396in}}%
\pgfpathlineto{\pgfqpoint{3.760292in}{2.237386in}}%
\pgfpathlineto{\pgfqpoint{3.771593in}{2.412964in}}%
\pgfpathlineto{\pgfqpoint{3.782894in}{2.627353in}}%
\pgfpathlineto{\pgfqpoint{3.794195in}{2.536792in}}%
\pgfpathlineto{\pgfqpoint{3.805496in}{2.710521in}}%
\pgfpathlineto{\pgfqpoint{3.816797in}{2.575604in}}%
\pgfpathlineto{\pgfqpoint{3.828098in}{2.509069in}}%
\pgfpathlineto{\pgfqpoint{3.839398in}{2.525703in}}%
\pgfpathlineto{\pgfqpoint{3.850699in}{2.575604in}}%
\pgfpathlineto{\pgfqpoint{3.862000in}{2.462865in}}%
\pgfpathlineto{\pgfqpoint{3.873301in}{2.662469in}}%
\pgfpathlineto{\pgfqpoint{3.884602in}{2.461017in}}%
\pgfpathlineto{\pgfqpoint{3.895903in}{2.324251in}}%
\pgfpathlineto{\pgfqpoint{3.907204in}{2.642139in}}%
\pgfpathlineto{\pgfqpoint{3.918505in}{2.740092in}}%
\pgfpathlineto{\pgfqpoint{3.929806in}{2.703129in}}%
\pgfpathlineto{\pgfqpoint{3.941107in}{2.704977in}}%
\pgfpathlineto{\pgfqpoint{3.952408in}{2.656924in}}%
\pgfpathlineto{\pgfqpoint{3.963709in}{2.431446in}}%
\pgfpathlineto{\pgfqpoint{3.975010in}{2.509069in}}%
\pgfpathlineto{\pgfqpoint{3.986311in}{2.370455in}}%
\pgfpathlineto{\pgfqpoint{3.997612in}{2.442535in}}%
\pgfpathlineto{\pgfqpoint{4.008913in}{2.503525in}}%
\pgfpathlineto{\pgfqpoint{4.020214in}{2.490587in}}%
\pgfpathlineto{\pgfqpoint{4.031515in}{2.189333in}}%
\pgfpathlineto{\pgfqpoint{4.042816in}{2.424053in}}%
\pgfpathlineto{\pgfqpoint{4.054117in}{2.865769in}}%
\pgfpathlineto{\pgfqpoint{4.065418in}{2.701281in}}%
\pgfpathlineto{\pgfqpoint{4.076719in}{2.743789in}}%
\pgfpathlineto{\pgfqpoint{4.088020in}{2.704977in}}%
\pgfpathlineto{\pgfqpoint{4.099321in}{2.196726in}}%
\pgfpathlineto{\pgfqpoint{4.110622in}{2.588541in}}%
\pgfpathlineto{\pgfqpoint{4.121923in}{2.618112in}}%
\pgfpathlineto{\pgfqpoint{4.133224in}{2.473954in}}%
\pgfpathlineto{\pgfqpoint{4.144525in}{2.385241in}}%
\pgfpathlineto{\pgfqpoint{4.155826in}{2.475802in}}%
\pgfpathlineto{\pgfqpoint{4.167127in}{2.307617in}}%
\pgfpathlineto{\pgfqpoint{4.178428in}{2.433294in}}%
\pgfpathlineto{\pgfqpoint{4.189729in}{2.632898in}}%
\pgfpathlineto{\pgfqpoint{4.201030in}{2.533096in}}%
\pgfpathlineto{\pgfqpoint{4.212331in}{2.706825in}}%
\pgfpathlineto{\pgfqpoint{4.223632in}{2.649531in}}%
\pgfpathlineto{\pgfqpoint{4.234933in}{2.503525in}}%
\pgfpathlineto{\pgfqpoint{4.257535in}{2.427749in}}%
\pgfpathlineto{\pgfqpoint{4.268836in}{2.688343in}}%
\pgfpathlineto{\pgfqpoint{4.280137in}{2.897188in}}%
\pgfpathlineto{\pgfqpoint{4.291438in}{2.344581in}}%
\pgfpathlineto{\pgfqpoint{4.302739in}{2.634746in}}%
\pgfpathlineto{\pgfqpoint{4.314040in}{2.810323in}}%
\pgfpathlineto{\pgfqpoint{4.325341in}{2.534944in}}%
\pgfpathlineto{\pgfqpoint{4.336642in}{2.401875in}}%
\pgfpathlineto{\pgfqpoint{4.347943in}{2.638442in}}%
\pgfpathlineto{\pgfqpoint{4.359244in}{2.376000in}}%
\pgfpathlineto{\pgfqpoint{4.370545in}{2.529399in}}%
\pgfpathlineto{\pgfqpoint{4.381846in}{2.420356in}}%
\pgfpathlineto{\pgfqpoint{4.393147in}{2.135736in}}%
\pgfpathlineto{\pgfqpoint{4.404447in}{2.801083in}}%
\pgfpathlineto{\pgfqpoint{4.415748in}{2.747485in}}%
\pgfpathlineto{\pgfqpoint{4.427049in}{2.468409in}}%
\pgfpathlineto{\pgfqpoint{4.438350in}{2.732700in}}%
\pgfpathlineto{\pgfqpoint{4.449651in}{2.505373in}}%
\pgfpathlineto{\pgfqpoint{4.460952in}{2.339036in}}%
\pgfpathlineto{\pgfqpoint{4.472253in}{2.575604in}}%
\pgfpathlineto{\pgfqpoint{4.483554in}{2.581149in}}%
\pgfpathlineto{\pgfqpoint{4.494855in}{2.653228in}}%
\pgfpathlineto{\pgfqpoint{4.506156in}{2.775208in}}%
\pgfpathlineto{\pgfqpoint{4.517457in}{2.462865in}}%
\pgfpathlineto{\pgfqpoint{4.528758in}{2.546033in}}%
\pgfpathlineto{\pgfqpoint{4.540059in}{2.464713in}}%
\pgfpathlineto{\pgfqpoint{4.551360in}{2.313162in}}%
\pgfpathlineto{\pgfqpoint{4.562661in}{2.387089in}}%
\pgfpathlineto{\pgfqpoint{4.573962in}{2.664317in}}%
\pgfpathlineto{\pgfqpoint{4.585263in}{2.756726in}}%
\pgfpathlineto{\pgfqpoint{4.607865in}{2.536792in}}%
\pgfpathlineto{\pgfqpoint{4.619166in}{2.751182in}}%
\pgfpathlineto{\pgfqpoint{4.630467in}{2.570059in}}%
\pgfpathlineto{\pgfqpoint{4.641768in}{2.544185in}}%
\pgfpathlineto{\pgfqpoint{4.653069in}{2.509069in}}%
\pgfpathlineto{\pgfqpoint{4.664370in}{2.653228in}}%
\pgfpathlineto{\pgfqpoint{4.675671in}{2.503525in}}%
\pgfpathlineto{\pgfqpoint{4.686972in}{2.509069in}}%
\pgfpathlineto{\pgfqpoint{4.698273in}{2.782601in}}%
\pgfpathlineto{\pgfqpoint{4.709574in}{2.736396in}}%
\pgfpathlineto{\pgfqpoint{4.720875in}{2.594086in}}%
\pgfpathlineto{\pgfqpoint{4.732176in}{2.601479in}}%
\pgfpathlineto{\pgfqpoint{4.743477in}{2.777056in}}%
\pgfpathlineto{\pgfqpoint{4.754778in}{2.536792in}}%
\pgfpathlineto{\pgfqpoint{4.766079in}{2.819564in}}%
\pgfpathlineto{\pgfqpoint{4.777380in}{2.828805in}}%
\pgfpathlineto{\pgfqpoint{4.788681in}{2.878706in}}%
\pgfpathlineto{\pgfqpoint{4.799982in}{2.642139in}}%
\pgfpathlineto{\pgfqpoint{4.811283in}{2.662469in}}%
\pgfpathlineto{\pgfqpoint{4.822584in}{2.538640in}}%
\pgfpathlineto{\pgfqpoint{4.833885in}{2.745637in}}%
\pgfpathlineto{\pgfqpoint{4.845186in}{2.357518in}}%
\pgfpathlineto{\pgfqpoint{4.856487in}{2.418508in}}%
\pgfpathlineto{\pgfqpoint{4.867788in}{2.512766in}}%
\pgfpathlineto{\pgfqpoint{4.879089in}{2.760422in}}%
\pgfpathlineto{\pgfqpoint{4.890390in}{2.656924in}}%
\pgfpathlineto{\pgfqpoint{4.901691in}{2.758574in}}%
\pgfpathlineto{\pgfqpoint{4.912992in}{2.826957in}}%
\pgfpathlineto{\pgfqpoint{4.924293in}{2.424053in}}%
\pgfpathlineto{\pgfqpoint{4.935594in}{2.331644in}}%
\pgfpathlineto{\pgfqpoint{4.946895in}{2.836198in}}%
\pgfpathlineto{\pgfqpoint{4.958195in}{2.525703in}}%
\pgfpathlineto{\pgfqpoint{4.969496in}{2.603327in}}%
\pgfpathlineto{\pgfqpoint{4.980797in}{2.836198in}}%
\pgfpathlineto{\pgfqpoint{4.992098in}{2.647683in}}%
\pgfpathlineto{\pgfqpoint{5.003399in}{2.686495in}}%
\pgfpathlineto{\pgfqpoint{5.014700in}{2.592238in}}%
\pgfpathlineto{\pgfqpoint{5.026001in}{2.882403in}}%
\pgfpathlineto{\pgfqpoint{5.037302in}{2.780752in}}%
\pgfpathlineto{\pgfqpoint{5.048603in}{2.850983in}}%
\pgfpathlineto{\pgfqpoint{5.059904in}{2.710521in}}%
\pgfpathlineto{\pgfqpoint{5.082506in}{2.610719in}}%
\pgfpathlineto{\pgfqpoint{5.093807in}{2.601479in}}%
\pgfpathlineto{\pgfqpoint{5.105108in}{2.721611in}}%
\pgfpathlineto{\pgfqpoint{5.116409in}{2.631050in}}%
\pgfpathlineto{\pgfqpoint{5.127710in}{2.714218in}}%
\pgfpathlineto{\pgfqpoint{5.139011in}{2.895340in}}%
\pgfpathlineto{\pgfqpoint{5.150312in}{2.673558in}}%
\pgfpathlineto{\pgfqpoint{5.161613in}{2.631050in}}%
\pgfpathlineto{\pgfqpoint{5.172914in}{2.712370in}}%
\pgfpathlineto{\pgfqpoint{5.184215in}{2.557122in}}%
\pgfpathlineto{\pgfqpoint{5.195516in}{2.754878in}}%
\pgfpathlineto{\pgfqpoint{5.206817in}{2.488739in}}%
\pgfpathlineto{\pgfqpoint{5.218118in}{2.601479in}}%
\pgfpathlineto{\pgfqpoint{5.229419in}{2.734548in}}%
\pgfpathlineto{\pgfqpoint{5.240720in}{2.259564in}}%
\pgfpathlineto{\pgfqpoint{5.274623in}{2.756726in}}%
\pgfpathlineto{\pgfqpoint{5.285924in}{2.699432in}}%
\pgfpathlineto{\pgfqpoint{5.297225in}{2.501677in}}%
\pgfpathlineto{\pgfqpoint{5.308526in}{2.446231in}}%
\pgfpathlineto{\pgfqpoint{5.319827in}{2.566363in}}%
\pgfpathlineto{\pgfqpoint{5.331128in}{2.736396in}}%
\pgfpathlineto{\pgfqpoint{5.342429in}{2.582997in}}%
\pgfpathlineto{\pgfqpoint{5.353730in}{2.730851in}}%
\pgfpathlineto{\pgfqpoint{5.365031in}{2.984053in}}%
\pgfpathlineto{\pgfqpoint{5.376332in}{2.595934in}}%
\pgfpathlineto{\pgfqpoint{5.387633in}{2.817716in}}%
\pgfpathlineto{\pgfqpoint{5.398934in}{2.875010in}}%
\pgfpathlineto{\pgfqpoint{5.410235in}{2.492436in}}%
\pgfpathlineto{\pgfqpoint{5.421536in}{2.812172in}}%
\pgfpathlineto{\pgfqpoint{5.432837in}{2.924911in}}%
\pgfpathlineto{\pgfqpoint{5.444138in}{2.595934in}}%
\pgfpathlineto{\pgfqpoint{5.455439in}{2.420356in}}%
\pgfpathlineto{\pgfqpoint{5.466740in}{2.608871in}}%
\pgfpathlineto{\pgfqpoint{5.478041in}{2.850983in}}%
\pgfpathlineto{\pgfqpoint{5.489342in}{2.621809in}}%
\pgfpathlineto{\pgfqpoint{5.500643in}{2.732700in}}%
\pgfpathlineto{\pgfqpoint{5.511943in}{2.603327in}}%
\pgfpathlineto{\pgfqpoint{5.523244in}{2.817716in}}%
\pgfpathlineto{\pgfqpoint{5.534545in}{2.483195in}}%
\pgfpathlineto{\pgfqpoint{5.534545in}{2.483195in}}%
\pgfusepath{stroke}%
\end{pgfscope}%
\begin{pgfscope}%
\pgfpathrectangle{\pgfqpoint{0.800000in}{0.528000in}}{\pgfqpoint{4.960000in}{3.696000in}} %
\pgfusepath{clip}%
\pgfsetbuttcap%
\pgfsetmiterjoin%
\definecolor{currentfill}{rgb}{1.000000,1.000000,1.000000}%
\pgfsetfillcolor{currentfill}%
\pgfsetlinewidth{1.003750pt}%
\definecolor{currentstroke}{rgb}{1.000000,1.000000,1.000000}%
\pgfsetstrokecolor{currentstroke}%
\pgfsetdash{}{0pt}%
\pgfpathmoveto{\pgfqpoint{1.906844in}{2.983223in}}%
\pgfpathlineto{\pgfqpoint{1.927718in}{2.194663in}}%
\pgfpathlineto{\pgfqpoint{2.173156in}{2.201160in}}%
\pgfpathlineto{\pgfqpoint{2.152282in}{2.989720in}}%
\pgfpathclose%
\pgfusepath{stroke,fill}%
\end{pgfscope}%
\begin{pgfscope}%
\pgftext[x=1.992730in,y=2.929921in,left,base,rotate=271.516282]{\sffamily\fontsize{10.000000}{12.000000}\selectfont \(\displaystyle \alpha =\) 0.001}%
\end{pgfscope}%
\begin{pgfscope}%
\pgfpathrectangle{\pgfqpoint{0.800000in}{0.528000in}}{\pgfqpoint{4.960000in}{3.696000in}} %
\pgfusepath{clip}%
\pgfsetbuttcap%
\pgfsetmiterjoin%
\definecolor{currentfill}{rgb}{1.000000,1.000000,1.000000}%
\pgfsetfillcolor{currentfill}%
\pgfsetlinewidth{1.003750pt}%
\definecolor{currentstroke}{rgb}{1.000000,1.000000,1.000000}%
\pgfsetstrokecolor{currentstroke}%
\pgfsetdash{}{0pt}%
\pgfpathmoveto{\pgfqpoint{3.148431in}{3.680382in}}%
\pgfpathlineto{\pgfqpoint{3.166124in}{2.980135in}}%
\pgfpathlineto{\pgfqpoint{3.411569in}{2.986337in}}%
\pgfpathlineto{\pgfqpoint{3.393876in}{3.686584in}}%
\pgfpathclose%
\pgfusepath{stroke,fill}%
\end{pgfscope}%
\begin{pgfscope}%
\definecolor{textcolor}{rgb}{0.300000,0.300000,0.300000}%
\pgfsetstrokecolor{textcolor}%
\pgfsetfillcolor{textcolor}%
\pgftext[x=3.234252in,y=3.626978in,left,base,rotate=271.447390]{\color{textcolor}\sffamily\fontsize{10.000000}{12.000000}\selectfont \(\displaystyle \alpha =\) 0.01}%
\end{pgfscope}%
\begin{pgfscope}%
\pgfpathrectangle{\pgfqpoint{0.800000in}{0.528000in}}{\pgfqpoint{4.960000in}{3.696000in}} %
\pgfusepath{clip}%
\pgfsetbuttcap%
\pgfsetmiterjoin%
\definecolor{currentfill}{rgb}{1.000000,1.000000,1.000000}%
\pgfsetfillcolor{currentfill}%
\pgfsetlinewidth{1.003750pt}%
\definecolor{currentstroke}{rgb}{1.000000,1.000000,1.000000}%
\pgfsetstrokecolor{currentstroke}%
\pgfsetdash{}{0pt}%
\pgfpathmoveto{\pgfqpoint{4.600436in}{2.161783in}}%
\pgfpathlineto{\pgfqpoint{4.682852in}{2.768314in}}%
\pgfpathlineto{\pgfqpoint{4.439564in}{2.801372in}}%
\pgfpathlineto{\pgfqpoint{4.357148in}{2.194841in}}%
\pgfpathclose%
\pgfusepath{stroke,fill}%
\end{pgfscope}%
\begin{pgfscope}%
\definecolor{textcolor}{rgb}{0.600000,0.600000,0.600000}%
\pgfsetstrokecolor{textcolor}%
\pgfsetfillcolor{textcolor}%
\pgftext[x=4.524240in,y=2.228202in,left,base,rotate=82.261994]{\color{textcolor}\sffamily\fontsize{10.000000}{12.000000}\selectfont \(\displaystyle \alpha =\) 0.1}%
\end{pgfscope}%
\begin{pgfscope}%
\pgfsetrectcap%
\pgfsetmiterjoin%
\pgfsetlinewidth{0.803000pt}%
\definecolor{currentstroke}{rgb}{0.000000,0.000000,0.000000}%
\pgfsetstrokecolor{currentstroke}%
\pgfsetdash{}{0pt}%
\pgfpathmoveto{\pgfqpoint{0.800000in}{0.528000in}}%
\pgfpathlineto{\pgfqpoint{0.800000in}{4.224000in}}%
\pgfusepath{stroke}%
\end{pgfscope}%
\begin{pgfscope}%
\pgfsetrectcap%
\pgfsetmiterjoin%
\pgfsetlinewidth{0.803000pt}%
\definecolor{currentstroke}{rgb}{0.000000,0.000000,0.000000}%
\pgfsetstrokecolor{currentstroke}%
\pgfsetdash{}{0pt}%
\pgfpathmoveto{\pgfqpoint{5.760000in}{0.528000in}}%
\pgfpathlineto{\pgfqpoint{5.760000in}{4.224000in}}%
\pgfusepath{stroke}%
\end{pgfscope}%
\begin{pgfscope}%
\pgfsetrectcap%
\pgfsetmiterjoin%
\pgfsetlinewidth{0.803000pt}%
\definecolor{currentstroke}{rgb}{0.000000,0.000000,0.000000}%
\pgfsetstrokecolor{currentstroke}%
\pgfsetdash{}{0pt}%
\pgfpathmoveto{\pgfqpoint{0.800000in}{0.528000in}}%
\pgfpathlineto{\pgfqpoint{5.760000in}{0.528000in}}%
\pgfusepath{stroke}%
\end{pgfscope}%
\begin{pgfscope}%
\pgfsetrectcap%
\pgfsetmiterjoin%
\pgfsetlinewidth{0.803000pt}%
\definecolor{currentstroke}{rgb}{0.000000,0.000000,0.000000}%
\pgfsetstrokecolor{currentstroke}%
\pgfsetdash{}{0pt}%
\pgfpathmoveto{\pgfqpoint{0.800000in}{4.224000in}}%
\pgfpathlineto{\pgfqpoint{5.760000in}{4.224000in}}%
\pgfusepath{stroke}%
\end{pgfscope}%
\begin{pgfscope}%
\pgftext[x=3.280000in,y=4.307333in,,base]{\sffamily\fontsize{12.000000}{14.400000}\selectfont Monte Carlo Policy Gradient Results}%
\end{pgfscope}%
\end{pgfpicture}%
\makeatother%
\endgroup%
} \\
\end{centering}
\begin{itemize}
    \item The results can be summarized as follows:
    \begin{itemize}
        \item The largest learning rate initiates learning quickly
            but fails to converge to an optimal policy, likely because it
            overshoots the mark at each parameter update.
        \item The smallest learning rate learns the policy slowly because of
            its smaller updates but does reach a near-optimal policy.
        \item The middle learning rate finds a near-optimal policy relatively
            quickly.
    \end{itemize}
\end{itemize}
\end{document}
