\documentclass[letterpaper,twocolumn,10pt]{article}
\usepackage{epsfig,endnotes}
\usepackage{pgfplots}
\usepackage{pgf}
\usepackage{hyperref}
\usepackage{amsthm, amsmath, amssymb, verbatim, enumerate, mathtools, algorithm}
\usepackage{algorithm}
\usepackage{algpseudocode}
\usepackage{relsize}
\usepackage{usenix}
\setlength{\parindent}{0cm}
\makeatletter
\newcommand{\algmargin}{\the\ALG@thistlm}
\makeatother
\algnewcommand{\parState}[1]{\State%
    \parbox[t]{\dimexpr\linewidth-\algmargin} {\strut\hangindent=\algorithmicindent \hangafter=1 #1\strut}}

\begin{document}

%don't want date printed
\date{}

%make title bold and 14 pt font (Latex default is non-bold, 16 pt)
\title{\Large \bf Proximal Policy Optimization with Dynamic Clipping}

\author{
{\rm Student: Rishikesh Vaishnav}\\
University of California, San Diego
\and
{\rm Mentor: Sicun Gao, Ph.D.}\\
University of California, San Diego
}

\maketitle

% Use the following at camera-ready time to suppress page numbers.
% Comment it out when you first submit the paper for review.
\thispagestyle{empty}

\subsection*{Abstract}
Proximal Policy Optimization (PPO), a policy gradient algorithm drawing closely
from the theory supporting Trust Region Policy Optimization (TRPO), has emerged
as one of the most effective tools in reinforcement learning (RL) problems. PPO
makes use of a loss function with a clipped importance sampling ratio using a
single parameter $\epsilon$. Although PPO shows promising empirical
performance, it is vulnerable to problem-specific imbalances in its handling of
positive and negative advantages. We investigate one such imbalance, addressing
a discrepancy in expected penalty contributions of positive and negative
estimators. By precisely calculating this discrepancy and minimizing it before
each model update, we empirically demonstrate that eliminating this discrepancy
can improve the overall performance of PPO.

\section{Background}

Note: This paper assumes knowledge of the common terms and concepts in
reinforcement learning (see \cite{DBLP:books/lib/SuttonB98}).

Among current reinforcement learning (RL) algorithms, Policy Gradient methods
have seen significant success at a wide range of tasks. These methods seek to
learn performant policy distributions directly, rather than learning a value
function to indirectly guide the policy. Using a policy $\pi_{\theta}(a)$ parameterized by $\theta$, these
algorithms have the general form: 
\begin{algorithm}[H]
    \caption{Generic Policy Gradient}
    \begin{algorithmic}
        \State Initialize $\theta$ arbitrarily
        \While{True}\Comment{loop forever}
            \State $\theta_{old} \gets \theta$
            \parState{$rollout \gets (s, a, r)$ from multiple\\
            episodes following $\pi_{\theta}$}
			\parState{Set $\theta$ to maximize the loss function $L(rollout,
				\theta, \theta_{old})$}
        \EndWhile
    \end{algorithmic}
\end{algorithm}
Within this framework, the choice of a loss function is the key to an effective
algorithm, and current research in reinforcement learning focuses particularly
on finding new ways to represent this loss. A recent innovation in policy
gradient methods is Trust Region Policy Optimization (TRPO) 
\cite{DBLP:journals/corr/SchulmanLMJA15}, which approximates
the updates perfomed by an agent that uses the following loss function to 
guarantee monotonic improvement in the policy's performance:
\begin{equation}
    L_{\theta_{old}}(\theta) 
    - CD_{KL}^{max}(\theta, \theta_{old}),
    \label{eq:1}
\end{equation}
where $C$ is a constant,
\begin{equation}
L_{\theta_{old}}(\theta) =
    \mathbb{E}_t \left[ 
        \frac
        {\pi_{\theta}(a_t | s_t)}
        {\pi_{\theta_{old}} (a_t | s_t)}
        A_t
    \right],
    \label{eq:2}
\end{equation}
and $A_t$ is the advantage at time $t$.
In TRPO implementations, this loss function results in relatively
small step sizes, so in practice the penalty term $CD_{KL}^{max}(\theta,
\theta_{old})$ is removed from equation \eqref{eq:1}, and we instead maximize
\eqref{eq:2} with a constraint on the KL divergence.

Implementations of TRPO generally have significant computational complexity
relative to other RL algorithms, largely due to the need to calculate KL
divergences and perform constraint optimization. To address this, a new
algorithm based on the theory behind TRPO, Proximal Policy Optimization (PPO)
\cite{DBLP:journals/corr/SchulmanWDRK17}, was developed.

Rather than maximizing \eqref{eq:2} subject to a constraint, PPO maximizes the
following ``clipped'' loss function:
\small
\begin{equation}
    L^{CLIP}(\theta) = \\
    \mathbb{E}_t\left[ 
    \min\left(r_tA_t, \text{clip}
    (r_t, 1 - \epsilon, 1 + \epsilon)A_t\right)
    \right],
    \label{eq:3}
\end{equation}
\normalsize
where $r_t(\theta) = 
    \frac
    {\pi_{\theta}(a_t | s_t)}
    {\pi_{\theta_{old}} (a_t | s_t)}$.

This loss function can be maximized using standard optimization techniques and
does not require constraint optimization, significantly simplifying
implementation relative to TRPO. This loss function can be thought of as a
heuristic approximation to the loss function from \eqref{eq:1}, where the
minimization substitutes the penalty term.

PPO performs well in practice, in many cases outperforming TRPO. However, its
simple equation does not allow for precise control over what penalties are
actually introduced, and, as we will show, can exhibit problem-specific
imbalances in the way positive and negative advantages are handled. 

\section{Expected Penalty Contributions}

\eqref{eq:3} can be split into positive and negative advantage cases as follows:
\scriptsize
\begin{equation}
    L^{CLIP}(\theta) 
    = \mathbb{E}_t\left[ 
    \begin{cases}
        \min\left(r_t, \text{clip}
        (r_t, 1 + \epsilon)\right)A_t & A_t > 0\\
        \max\left(r_t, \text{clip}
        (r_t, 1 - \epsilon)\right)A_t & A_t < 0
    \end{cases}\right]
    \label{eq:4}.
\end{equation}
\normalsize

Now, let
\begin{align*}
    r_{t, CLIP}^+ &= 
    \min\left(r_t, \text{clip}
    (r_t, 1 + \epsilon)\right)\\
    r_{t, CLIP}^- &= 
    \max\left(r_t, \text{clip}
    (r_t, 1 - \epsilon)\right).
\end{align*}

It follows mathematically that $\mathbb{E}_t\left[ r_t \right] = 1$, because
ratios are sampled according to the old policy distribution. Therefore, it must
be the case that $\mathbb{E}_t\left[ r_{t, CLIP}^+ \right] < 1$ and 
$\mathbb{E}_t\left[ r_{t, CLIP}^- \right] > 1$. As the new policy changes
relative to the old policy, clipping becomes more frequent, generally causing
these expectations to become smaller and larger, respectively. Assuming
independence between ratios and advantages, the effect of this is to make
positive advantages less positive and make negative advatages more negative.
This is a reflection of how penalization occurs when a new policy is learned.

We can define the ``expected penalty contributions'':
$1 - \mathbb{E}_t[r_{t, CLIP}^+]$ and $\mathbb{E}_t[r_{t, CLIP}^-] - 1$. We
would expect both of these values to become more positive as learning
progresses in a single iteration.

In practice, the assumption of independence between ratios and advantages is
incorrect, because if these were independent, we would expect to see a
monotonically decreasing loss at each model update. In reality, however, an
optimizer will specifically work around this by assigning ratios to advantages
such that the loss tends to increase at each model update. Therefore, these
expected ratios do not suggest the actual proportional contributions of
positive and negative advantages to the overall loss.

However, it seems likely that these expectations have a significant influence
on how positive and negative advatages are considered in calculating the
overall loss. Addressing these expectations and controlling their relative
values will affect the loss that is calculated, and could have an influence on
overall performance.

There is no reason to assume that the expected penalty contributions will agree
with one another. We found that the both the rate at which these increased and
the discrepancy between these two values was dependent on the choice of
distribution to represent the policy.

\section{Problem-Specific Expected Penalty Contribution Discrepancy}

In order to investigate the effect of controlling the discrepancy between
positive and negative expected penalty contributions, we focus our attention to
a specific problem setting involving a continuous action space, with actions
chosen according to a gaussian distribution with fixed standard deviation. From
a simulated run, we observed that, for a single state, as the learned gaussian
diverged from the original gaussian, the expected penalty contributions
increased at different rates, resulting in a growing discrepancy between them
(see simulation here:
\href{https://github.com/rish987/Reinforcement-Learning/blob/master/projects/dynamic_clipping/report/discrepancy.gif}{[link]}).
This discrepancy also appears empirically, as shown in figure \ref{fig:1}.
\begin{figure}
    \centering
    \scalebox{0.4}{%% Creator: Matplotlib, PGF backend
%%
%% To include the figure in your LaTeX document, write
%%   \input{<filename>.pgf}
%%
%% Make sure the required packages are loaded in your preamble
%%   \usepackage{pgf}
%%
%% Figures using additional raster images can only be included by \input if
%% they are in the same directory as the main LaTeX file. For loading figures
%% from other directories you can use the `import` package
%%   \usepackage{import}
%% and then include the figures with
%%   \import{<path to file>}{<filename>.pgf}
%%
%% Matplotlib used the following preamble
%%   \usepackage{fontspec}
%%   \setmainfont{DejaVu Serif}
%%   \setsansfont{DejaVu Sans}
%%   \setmonofont{DejaVu Sans Mono}
%%
\begingroup%
\makeatletter%
\begin{pgfpicture}%
\pgfpathrectangle{\pgfpointorigin}{\pgfqpoint{6.400000in}{4.800000in}}%
\pgfusepath{use as bounding box, clip}%
\begin{pgfscope}%
\pgfsetbuttcap%
\pgfsetmiterjoin%
\definecolor{currentfill}{rgb}{1.000000,1.000000,1.000000}%
\pgfsetfillcolor{currentfill}%
\pgfsetlinewidth{0.000000pt}%
\definecolor{currentstroke}{rgb}{1.000000,1.000000,1.000000}%
\pgfsetstrokecolor{currentstroke}%
\pgfsetdash{}{0pt}%
\pgfpathmoveto{\pgfqpoint{0.000000in}{0.000000in}}%
\pgfpathlineto{\pgfqpoint{6.400000in}{0.000000in}}%
\pgfpathlineto{\pgfqpoint{6.400000in}{4.800000in}}%
\pgfpathlineto{\pgfqpoint{0.000000in}{4.800000in}}%
\pgfpathclose%
\pgfusepath{fill}%
\end{pgfscope}%
\begin{pgfscope}%
\pgfsetbuttcap%
\pgfsetmiterjoin%
\definecolor{currentfill}{rgb}{1.000000,1.000000,1.000000}%
\pgfsetfillcolor{currentfill}%
\pgfsetlinewidth{0.000000pt}%
\definecolor{currentstroke}{rgb}{0.000000,0.000000,0.000000}%
\pgfsetstrokecolor{currentstroke}%
\pgfsetstrokeopacity{0.000000}%
\pgfsetdash{}{0pt}%
\pgfpathmoveto{\pgfqpoint{0.757955in}{0.582778in}}%
\pgfpathlineto{\pgfqpoint{6.215000in}{0.582778in}}%
\pgfpathlineto{\pgfqpoint{6.215000in}{4.426667in}}%
\pgfpathlineto{\pgfqpoint{0.757955in}{4.426667in}}%
\pgfpathclose%
\pgfusepath{fill}%
\end{pgfscope}%
\begin{pgfscope}%
\pgfsetbuttcap%
\pgfsetroundjoin%
\definecolor{currentfill}{rgb}{0.000000,0.000000,0.000000}%
\pgfsetfillcolor{currentfill}%
\pgfsetlinewidth{0.803000pt}%
\definecolor{currentstroke}{rgb}{0.000000,0.000000,0.000000}%
\pgfsetstrokecolor{currentstroke}%
\pgfsetdash{}{0pt}%
\pgfsys@defobject{currentmarker}{\pgfqpoint{0.000000in}{-0.048611in}}{\pgfqpoint{0.000000in}{0.000000in}}{%
\pgfpathmoveto{\pgfqpoint{0.000000in}{0.000000in}}%
\pgfpathlineto{\pgfqpoint{0.000000in}{-0.048611in}}%
\pgfusepath{stroke,fill}%
}%
\begin{pgfscope}%
\pgfsys@transformshift{1.316062in}{0.582778in}%
\pgfsys@useobject{currentmarker}{}%
\end{pgfscope}%
\end{pgfscope}%
\begin{pgfscope}%
\pgftext[x=1.316062in,y=0.485556in,,top]{\sffamily\fontsize{10.000000}{12.000000}\selectfont \(\displaystyle 2\)}%
\end{pgfscope}%
\begin{pgfscope}%
\pgfsetbuttcap%
\pgfsetroundjoin%
\definecolor{currentfill}{rgb}{0.000000,0.000000,0.000000}%
\pgfsetfillcolor{currentfill}%
\pgfsetlinewidth{0.803000pt}%
\definecolor{currentstroke}{rgb}{0.000000,0.000000,0.000000}%
\pgfsetstrokecolor{currentstroke}%
\pgfsetdash{}{0pt}%
\pgfsys@defobject{currentmarker}{\pgfqpoint{0.000000in}{-0.048611in}}{\pgfqpoint{0.000000in}{0.000000in}}{%
\pgfpathmoveto{\pgfqpoint{0.000000in}{0.000000in}}%
\pgfpathlineto{\pgfqpoint{0.000000in}{-0.048611in}}%
\pgfusepath{stroke,fill}%
}%
\begin{pgfscope}%
\pgfsys@transformshift{1.936181in}{0.582778in}%
\pgfsys@useobject{currentmarker}{}%
\end{pgfscope}%
\end{pgfscope}%
\begin{pgfscope}%
\pgftext[x=1.936181in,y=0.485556in,,top]{\sffamily\fontsize{10.000000}{12.000000}\selectfont \(\displaystyle 4\)}%
\end{pgfscope}%
\begin{pgfscope}%
\pgfsetbuttcap%
\pgfsetroundjoin%
\definecolor{currentfill}{rgb}{0.000000,0.000000,0.000000}%
\pgfsetfillcolor{currentfill}%
\pgfsetlinewidth{0.803000pt}%
\definecolor{currentstroke}{rgb}{0.000000,0.000000,0.000000}%
\pgfsetstrokecolor{currentstroke}%
\pgfsetdash{}{0pt}%
\pgfsys@defobject{currentmarker}{\pgfqpoint{0.000000in}{-0.048611in}}{\pgfqpoint{0.000000in}{0.000000in}}{%
\pgfpathmoveto{\pgfqpoint{0.000000in}{0.000000in}}%
\pgfpathlineto{\pgfqpoint{0.000000in}{-0.048611in}}%
\pgfusepath{stroke,fill}%
}%
\begin{pgfscope}%
\pgfsys@transformshift{2.556299in}{0.582778in}%
\pgfsys@useobject{currentmarker}{}%
\end{pgfscope}%
\end{pgfscope}%
\begin{pgfscope}%
\pgftext[x=2.556299in,y=0.485556in,,top]{\sffamily\fontsize{10.000000}{12.000000}\selectfont \(\displaystyle 6\)}%
\end{pgfscope}%
\begin{pgfscope}%
\pgfsetbuttcap%
\pgfsetroundjoin%
\definecolor{currentfill}{rgb}{0.000000,0.000000,0.000000}%
\pgfsetfillcolor{currentfill}%
\pgfsetlinewidth{0.803000pt}%
\definecolor{currentstroke}{rgb}{0.000000,0.000000,0.000000}%
\pgfsetstrokecolor{currentstroke}%
\pgfsetdash{}{0pt}%
\pgfsys@defobject{currentmarker}{\pgfqpoint{0.000000in}{-0.048611in}}{\pgfqpoint{0.000000in}{0.000000in}}{%
\pgfpathmoveto{\pgfqpoint{0.000000in}{0.000000in}}%
\pgfpathlineto{\pgfqpoint{0.000000in}{-0.048611in}}%
\pgfusepath{stroke,fill}%
}%
\begin{pgfscope}%
\pgfsys@transformshift{3.176418in}{0.582778in}%
\pgfsys@useobject{currentmarker}{}%
\end{pgfscope}%
\end{pgfscope}%
\begin{pgfscope}%
\pgftext[x=3.176418in,y=0.485556in,,top]{\sffamily\fontsize{10.000000}{12.000000}\selectfont \(\displaystyle 8\)}%
\end{pgfscope}%
\begin{pgfscope}%
\pgfsetbuttcap%
\pgfsetroundjoin%
\definecolor{currentfill}{rgb}{0.000000,0.000000,0.000000}%
\pgfsetfillcolor{currentfill}%
\pgfsetlinewidth{0.803000pt}%
\definecolor{currentstroke}{rgb}{0.000000,0.000000,0.000000}%
\pgfsetstrokecolor{currentstroke}%
\pgfsetdash{}{0pt}%
\pgfsys@defobject{currentmarker}{\pgfqpoint{0.000000in}{-0.048611in}}{\pgfqpoint{0.000000in}{0.000000in}}{%
\pgfpathmoveto{\pgfqpoint{0.000000in}{0.000000in}}%
\pgfpathlineto{\pgfqpoint{0.000000in}{-0.048611in}}%
\pgfusepath{stroke,fill}%
}%
\begin{pgfscope}%
\pgfsys@transformshift{3.796537in}{0.582778in}%
\pgfsys@useobject{currentmarker}{}%
\end{pgfscope}%
\end{pgfscope}%
\begin{pgfscope}%
\pgftext[x=3.796537in,y=0.485556in,,top]{\sffamily\fontsize{10.000000}{12.000000}\selectfont \(\displaystyle 10\)}%
\end{pgfscope}%
\begin{pgfscope}%
\pgfsetbuttcap%
\pgfsetroundjoin%
\definecolor{currentfill}{rgb}{0.000000,0.000000,0.000000}%
\pgfsetfillcolor{currentfill}%
\pgfsetlinewidth{0.803000pt}%
\definecolor{currentstroke}{rgb}{0.000000,0.000000,0.000000}%
\pgfsetstrokecolor{currentstroke}%
\pgfsetdash{}{0pt}%
\pgfsys@defobject{currentmarker}{\pgfqpoint{0.000000in}{-0.048611in}}{\pgfqpoint{0.000000in}{0.000000in}}{%
\pgfpathmoveto{\pgfqpoint{0.000000in}{0.000000in}}%
\pgfpathlineto{\pgfqpoint{0.000000in}{-0.048611in}}%
\pgfusepath{stroke,fill}%
}%
\begin{pgfscope}%
\pgfsys@transformshift{4.416656in}{0.582778in}%
\pgfsys@useobject{currentmarker}{}%
\end{pgfscope}%
\end{pgfscope}%
\begin{pgfscope}%
\pgftext[x=4.416656in,y=0.485556in,,top]{\sffamily\fontsize{10.000000}{12.000000}\selectfont \(\displaystyle 12\)}%
\end{pgfscope}%
\begin{pgfscope}%
\pgfsetbuttcap%
\pgfsetroundjoin%
\definecolor{currentfill}{rgb}{0.000000,0.000000,0.000000}%
\pgfsetfillcolor{currentfill}%
\pgfsetlinewidth{0.803000pt}%
\definecolor{currentstroke}{rgb}{0.000000,0.000000,0.000000}%
\pgfsetstrokecolor{currentstroke}%
\pgfsetdash{}{0pt}%
\pgfsys@defobject{currentmarker}{\pgfqpoint{0.000000in}{-0.048611in}}{\pgfqpoint{0.000000in}{0.000000in}}{%
\pgfpathmoveto{\pgfqpoint{0.000000in}{0.000000in}}%
\pgfpathlineto{\pgfqpoint{0.000000in}{-0.048611in}}%
\pgfusepath{stroke,fill}%
}%
\begin{pgfscope}%
\pgfsys@transformshift{5.036774in}{0.582778in}%
\pgfsys@useobject{currentmarker}{}%
\end{pgfscope}%
\end{pgfscope}%
\begin{pgfscope}%
\pgftext[x=5.036774in,y=0.485556in,,top]{\sffamily\fontsize{10.000000}{12.000000}\selectfont \(\displaystyle 14\)}%
\end{pgfscope}%
\begin{pgfscope}%
\pgfsetbuttcap%
\pgfsetroundjoin%
\definecolor{currentfill}{rgb}{0.000000,0.000000,0.000000}%
\pgfsetfillcolor{currentfill}%
\pgfsetlinewidth{0.803000pt}%
\definecolor{currentstroke}{rgb}{0.000000,0.000000,0.000000}%
\pgfsetstrokecolor{currentstroke}%
\pgfsetdash{}{0pt}%
\pgfsys@defobject{currentmarker}{\pgfqpoint{0.000000in}{-0.048611in}}{\pgfqpoint{0.000000in}{0.000000in}}{%
\pgfpathmoveto{\pgfqpoint{0.000000in}{0.000000in}}%
\pgfpathlineto{\pgfqpoint{0.000000in}{-0.048611in}}%
\pgfusepath{stroke,fill}%
}%
\begin{pgfscope}%
\pgfsys@transformshift{5.656893in}{0.582778in}%
\pgfsys@useobject{currentmarker}{}%
\end{pgfscope}%
\end{pgfscope}%
\begin{pgfscope}%
\pgftext[x=5.656893in,y=0.485556in,,top]{\sffamily\fontsize{10.000000}{12.000000}\selectfont \(\displaystyle 16\)}%
\end{pgfscope}%
\begin{pgfscope}%
\pgftext[x=3.486477in,y=0.295587in,,top]{\sffamily\fontsize{10.000000}{12.000000}\selectfont Number of Iterations}%
\end{pgfscope}%
\begin{pgfscope}%
\pgfsetbuttcap%
\pgfsetroundjoin%
\definecolor{currentfill}{rgb}{0.000000,0.000000,0.000000}%
\pgfsetfillcolor{currentfill}%
\pgfsetlinewidth{0.803000pt}%
\definecolor{currentstroke}{rgb}{0.000000,0.000000,0.000000}%
\pgfsetstrokecolor{currentstroke}%
\pgfsetdash{}{0pt}%
\pgfsys@defobject{currentmarker}{\pgfqpoint{-0.048611in}{0.000000in}}{\pgfqpoint{0.000000in}{0.000000in}}{%
\pgfpathmoveto{\pgfqpoint{0.000000in}{0.000000in}}%
\pgfpathlineto{\pgfqpoint{-0.048611in}{0.000000in}}%
\pgfusepath{stroke,fill}%
}%
\begin{pgfscope}%
\pgfsys@transformshift{0.757955in}{0.711952in}%
\pgfsys@useobject{currentmarker}{}%
\end{pgfscope}%
\end{pgfscope}%
\begin{pgfscope}%
\pgftext[x=0.344374in,y=0.659191in,left,base]{\sffamily\fontsize{10.000000}{12.000000}\selectfont \(\displaystyle 0.000\)}%
\end{pgfscope}%
\begin{pgfscope}%
\pgfsetbuttcap%
\pgfsetroundjoin%
\definecolor{currentfill}{rgb}{0.000000,0.000000,0.000000}%
\pgfsetfillcolor{currentfill}%
\pgfsetlinewidth{0.803000pt}%
\definecolor{currentstroke}{rgb}{0.000000,0.000000,0.000000}%
\pgfsetstrokecolor{currentstroke}%
\pgfsetdash{}{0pt}%
\pgfsys@defobject{currentmarker}{\pgfqpoint{-0.048611in}{0.000000in}}{\pgfqpoint{0.000000in}{0.000000in}}{%
\pgfpathmoveto{\pgfqpoint{0.000000in}{0.000000in}}%
\pgfpathlineto{\pgfqpoint{-0.048611in}{0.000000in}}%
\pgfusepath{stroke,fill}%
}%
\begin{pgfscope}%
\pgfsys@transformshift{0.757955in}{1.532636in}%
\pgfsys@useobject{currentmarker}{}%
\end{pgfscope}%
\end{pgfscope}%
\begin{pgfscope}%
\pgftext[x=0.344374in,y=1.479874in,left,base]{\sffamily\fontsize{10.000000}{12.000000}\selectfont \(\displaystyle 0.005\)}%
\end{pgfscope}%
\begin{pgfscope}%
\pgfsetbuttcap%
\pgfsetroundjoin%
\definecolor{currentfill}{rgb}{0.000000,0.000000,0.000000}%
\pgfsetfillcolor{currentfill}%
\pgfsetlinewidth{0.803000pt}%
\definecolor{currentstroke}{rgb}{0.000000,0.000000,0.000000}%
\pgfsetstrokecolor{currentstroke}%
\pgfsetdash{}{0pt}%
\pgfsys@defobject{currentmarker}{\pgfqpoint{-0.048611in}{0.000000in}}{\pgfqpoint{0.000000in}{0.000000in}}{%
\pgfpathmoveto{\pgfqpoint{0.000000in}{0.000000in}}%
\pgfpathlineto{\pgfqpoint{-0.048611in}{0.000000in}}%
\pgfusepath{stroke,fill}%
}%
\begin{pgfscope}%
\pgfsys@transformshift{0.757955in}{2.353319in}%
\pgfsys@useobject{currentmarker}{}%
\end{pgfscope}%
\end{pgfscope}%
\begin{pgfscope}%
\pgftext[x=0.344374in,y=2.300557in,left,base]{\sffamily\fontsize{10.000000}{12.000000}\selectfont \(\displaystyle 0.010\)}%
\end{pgfscope}%
\begin{pgfscope}%
\pgfsetbuttcap%
\pgfsetroundjoin%
\definecolor{currentfill}{rgb}{0.000000,0.000000,0.000000}%
\pgfsetfillcolor{currentfill}%
\pgfsetlinewidth{0.803000pt}%
\definecolor{currentstroke}{rgb}{0.000000,0.000000,0.000000}%
\pgfsetstrokecolor{currentstroke}%
\pgfsetdash{}{0pt}%
\pgfsys@defobject{currentmarker}{\pgfqpoint{-0.048611in}{0.000000in}}{\pgfqpoint{0.000000in}{0.000000in}}{%
\pgfpathmoveto{\pgfqpoint{0.000000in}{0.000000in}}%
\pgfpathlineto{\pgfqpoint{-0.048611in}{0.000000in}}%
\pgfusepath{stroke,fill}%
}%
\begin{pgfscope}%
\pgfsys@transformshift{0.757955in}{3.174002in}%
\pgfsys@useobject{currentmarker}{}%
\end{pgfscope}%
\end{pgfscope}%
\begin{pgfscope}%
\pgftext[x=0.344374in,y=3.121241in,left,base]{\sffamily\fontsize{10.000000}{12.000000}\selectfont \(\displaystyle 0.015\)}%
\end{pgfscope}%
\begin{pgfscope}%
\pgfsetbuttcap%
\pgfsetroundjoin%
\definecolor{currentfill}{rgb}{0.000000,0.000000,0.000000}%
\pgfsetfillcolor{currentfill}%
\pgfsetlinewidth{0.803000pt}%
\definecolor{currentstroke}{rgb}{0.000000,0.000000,0.000000}%
\pgfsetstrokecolor{currentstroke}%
\pgfsetdash{}{0pt}%
\pgfsys@defobject{currentmarker}{\pgfqpoint{-0.048611in}{0.000000in}}{\pgfqpoint{0.000000in}{0.000000in}}{%
\pgfpathmoveto{\pgfqpoint{0.000000in}{0.000000in}}%
\pgfpathlineto{\pgfqpoint{-0.048611in}{0.000000in}}%
\pgfusepath{stroke,fill}%
}%
\begin{pgfscope}%
\pgfsys@transformshift{0.757955in}{3.994686in}%
\pgfsys@useobject{currentmarker}{}%
\end{pgfscope}%
\end{pgfscope}%
\begin{pgfscope}%
\pgftext[x=0.344374in,y=3.941924in,left,base]{\sffamily\fontsize{10.000000}{12.000000}\selectfont \(\displaystyle 0.020\)}%
\end{pgfscope}%
\begin{pgfscope}%
\pgftext[x=0.288818in,y=2.504722in,,bottom,rotate=90.000000]{\sffamily\fontsize{10.000000}{12.000000}\selectfont Proportional Contribution}%
\end{pgfscope}%
\begin{pgfscope}%
\pgfpathrectangle{\pgfqpoint{0.757955in}{0.582778in}}{\pgfqpoint{5.457045in}{3.843889in}}%
\pgfusepath{clip}%
\pgfsetrectcap%
\pgfsetroundjoin%
\pgfsetlinewidth{1.505625pt}%
\definecolor{currentstroke}{rgb}{0.000000,0.000000,0.000000}%
\pgfsetstrokecolor{currentstroke}%
\pgfsetdash{}{0pt}%
\pgfpathmoveto{\pgfqpoint{1.006002in}{3.533677in}}%
\pgfpathlineto{\pgfqpoint{1.316062in}{4.251944in}}%
\pgfpathlineto{\pgfqpoint{1.626121in}{3.569340in}}%
\pgfpathlineto{\pgfqpoint{1.936181in}{2.765420in}}%
\pgfpathlineto{\pgfqpoint{2.246240in}{1.871555in}}%
\pgfpathlineto{\pgfqpoint{2.556299in}{1.553503in}}%
\pgfpathlineto{\pgfqpoint{2.866359in}{1.431250in}}%
\pgfpathlineto{\pgfqpoint{3.176418in}{1.416363in}}%
\pgfpathlineto{\pgfqpoint{3.486477in}{1.508206in}}%
\pgfpathlineto{\pgfqpoint{3.796537in}{1.204933in}}%
\pgfpathlineto{\pgfqpoint{4.106596in}{1.108082in}}%
\pgfpathlineto{\pgfqpoint{4.416656in}{0.865045in}}%
\pgfpathlineto{\pgfqpoint{4.726715in}{0.938093in}}%
\pgfpathlineto{\pgfqpoint{5.036774in}{0.893178in}}%
\pgfpathlineto{\pgfqpoint{5.346834in}{0.949233in}}%
\pgfpathlineto{\pgfqpoint{5.656893in}{0.924012in}}%
\pgfpathlineto{\pgfqpoint{5.966952in}{0.822725in}}%
\pgfusepath{stroke}%
\end{pgfscope}%
\begin{pgfscope}%
\pgfpathrectangle{\pgfqpoint{0.757955in}{0.582778in}}{\pgfqpoint{5.457045in}{3.843889in}}%
\pgfusepath{clip}%
\pgfsetrectcap%
\pgfsetroundjoin%
\pgfsetlinewidth{1.505625pt}%
\definecolor{currentstroke}{rgb}{0.5,0.5,0.5}%
\pgfsetstrokecolor{currentstroke}%
\pgfsetdash{}{0pt}%
\pgfpathmoveto{\pgfqpoint{1.006002in}{2.308621in}}%
\pgfpathlineto{\pgfqpoint{1.316062in}{2.810948in}}%
\pgfpathlineto{\pgfqpoint{1.626121in}{2.537687in}}%
\pgfpathlineto{\pgfqpoint{1.936181in}{1.963244in}}%
\pgfpathlineto{\pgfqpoint{2.246240in}{1.341394in}}%
\pgfpathlineto{\pgfqpoint{2.556299in}{1.120764in}}%
\pgfpathlineto{\pgfqpoint{2.866359in}{1.080235in}}%
\pgfpathlineto{\pgfqpoint{3.176418in}{1.067171in}}%
\pgfpathlineto{\pgfqpoint{3.486477in}{1.170607in}}%
\pgfpathlineto{\pgfqpoint{3.796537in}{0.954125in}}%
\pgfpathlineto{\pgfqpoint{4.106596in}{0.909487in}}%
\pgfpathlineto{\pgfqpoint{4.416656in}{0.773956in}}%
\pgfpathlineto{\pgfqpoint{4.726715in}{0.804567in}}%
\pgfpathlineto{\pgfqpoint{5.036774in}{0.817436in}}%
\pgfpathlineto{\pgfqpoint{5.346834in}{0.809981in}}%
\pgfpathlineto{\pgfqpoint{5.656893in}{0.815870in}}%
\pgfpathlineto{\pgfqpoint{5.966952in}{0.757500in}}%
\pgfusepath{stroke}%
\end{pgfscope}%
\begin{pgfscope}%
\pgfsetrectcap%
\pgfsetmiterjoin%
\pgfsetlinewidth{0.803000pt}%
\definecolor{currentstroke}{rgb}{0.000000,0.000000,0.000000}%
\pgfsetstrokecolor{currentstroke}%
\pgfsetdash{}{0pt}%
\pgfpathmoveto{\pgfqpoint{0.757955in}{0.582778in}}%
\pgfpathlineto{\pgfqpoint{0.757955in}{4.426667in}}%
\pgfusepath{stroke}%
\end{pgfscope}%
\begin{pgfscope}%
\pgfsetrectcap%
\pgfsetmiterjoin%
\pgfsetlinewidth{0.803000pt}%
\definecolor{currentstroke}{rgb}{0.000000,0.000000,0.000000}%
\pgfsetstrokecolor{currentstroke}%
\pgfsetdash{}{0pt}%
\pgfpathmoveto{\pgfqpoint{6.215000in}{0.582778in}}%
\pgfpathlineto{\pgfqpoint{6.215000in}{4.426667in}}%
\pgfusepath{stroke}%
\end{pgfscope}%
\begin{pgfscope}%
\pgfsetrectcap%
\pgfsetmiterjoin%
\pgfsetlinewidth{0.803000pt}%
\definecolor{currentstroke}{rgb}{0.000000,0.000000,0.000000}%
\pgfsetstrokecolor{currentstroke}%
\pgfsetdash{}{0pt}%
\pgfpathmoveto{\pgfqpoint{0.757955in}{0.582778in}}%
\pgfpathlineto{\pgfqpoint{6.215000in}{0.582778in}}%
\pgfusepath{stroke}%
\end{pgfscope}%
\begin{pgfscope}%
\pgfsetrectcap%
\pgfsetmiterjoin%
\pgfsetlinewidth{0.803000pt}%
\definecolor{currentstroke}{rgb}{0.000000,0.000000,0.000000}%
\pgfsetstrokecolor{currentstroke}%
\pgfsetdash{}{0pt}%
\pgfpathmoveto{\pgfqpoint{0.757955in}{4.426667in}}%
\pgfpathlineto{\pgfqpoint{6.215000in}{4.426667in}}%
\pgfusepath{stroke}%
\end{pgfscope}%
\begin{pgfscope}%
\pgftext[x=3.486477in,y=4.510000in,,base]{\sffamily\fontsize{12.000000}{14.400000}\selectfont Expected Penalty Contributions}%
\end{pgfscope}%
\begin{pgfscope}%
\pgfsetbuttcap%
\pgfsetmiterjoin%
\definecolor{currentfill}{rgb}{1.000000,1.000000,1.000000}%
\pgfsetfillcolor{currentfill}%
\pgfsetfillopacity{0.800000}%
\pgfsetlinewidth{1.003750pt}%
\definecolor{currentstroke}{rgb}{0.800000,0.800000,0.800000}%
\pgfsetstrokecolor{currentstroke}%
\pgfsetstrokeopacity{0.800000}%
\pgfsetdash{}{0pt}%
\pgfpathmoveto{\pgfqpoint{4.796785in}{3.818027in}}%
\pgfpathlineto{\pgfqpoint{6.117778in}{3.818027in}}%
\pgfpathquadraticcurveto{\pgfqpoint{6.145556in}{3.818027in}}{\pgfqpoint{6.145556in}{3.845804in}}%
\pgfpathlineto{\pgfqpoint{6.145556in}{4.329444in}}%
\pgfpathquadraticcurveto{\pgfqpoint{6.145556in}{4.357222in}}{\pgfqpoint{6.117778in}{4.357222in}}%
\pgfpathlineto{\pgfqpoint{4.796785in}{4.357222in}}%
\pgfpathquadraticcurveto{\pgfqpoint{4.769007in}{4.357222in}}{\pgfqpoint{4.769007in}{4.329444in}}%
\pgfpathlineto{\pgfqpoint{4.769007in}{3.845804in}}%
\pgfpathquadraticcurveto{\pgfqpoint{4.769007in}{3.818027in}}{\pgfqpoint{4.796785in}{3.818027in}}%
\pgfpathclose%
\pgfusepath{stroke,fill}%
\end{pgfscope}%
\begin{pgfscope}%
\pgfsetrectcap%
\pgfsetroundjoin%
\pgfsetlinewidth{1.505625pt}%
\definecolor{currentstroke}{rgb}{0.000000,0.000000,0.000000}%
\pgfsetstrokecolor{currentstroke}%
\pgfsetdash{}{0pt}%
\pgfpathmoveto{\pgfqpoint{4.824562in}{4.230679in}}%
\pgfpathlineto{\pgfqpoint{5.102340in}{4.230679in}}%
\pgfusepath{stroke}%
\end{pgfscope}%
\begin{pgfscope}%
\pgftext[x=5.213451in,y=4.182068in,left,base]{\sffamily\fontsize{10.000000}{12.000000}\selectfont \(\displaystyle 1 - E[r_{t, CLIP}^+]\)}%
\end{pgfscope}%
\begin{pgfscope}%
\pgfsetrectcap%
\pgfsetroundjoin%
\pgfsetlinewidth{1.505625pt}%
\definecolor{currentstroke}{rgb}{0.5,0.5,0.5}%
\pgfsetstrokecolor{currentstroke}%
\pgfsetdash{}{0pt}%
\pgfpathmoveto{\pgfqpoint{4.824562in}{3.981915in}}%
\pgfpathlineto{\pgfqpoint{5.102340in}{3.981915in}}%
\pgfusepath{stroke}%
\end{pgfscope}%
\begin{pgfscope}%
\pgftext[x=5.213451in,y=3.933303in,left,base]{\sffamily\fontsize{10.000000}{12.000000}\selectfont \(\displaystyle E[r_{t, CLIP}^-] - 1\)}%
\end{pgfscope}%
\end{pgfpicture}%
\makeatother%
\endgroup%
}
    \caption{Empirical expected penalty contribution discrepancies. Expeceted
    values were calculated after the final model update in each iteration. }
    \label{fig:1}
\end{figure}
As suggested in the simulation, the empirical data shows that initial,
high-learning phases are marked by a larger discrepancy and expected total
penalty contribution than later, slower learning phases.

\section{Addressing the Discrepancy}

Using the aforementioned gaussian policy model, we can precisely calculate the
discrepancy at a certain state using the following equation:
\begin{align*}
    (1 - E[r_{t, CLIP}^{+}]) - (E[r_{t, CLIP}^{-}] - 1)\\
    =
    \epsilon
    + 
    (1 - \epsilon)
    \int_{x^{-}}^{x^{+}}p(\mu_{old}, x)dx
    \\
    -\left(\int_{x^{-}}^{x^{+}}p(\mu, x)dx 
    +2\epsilon
    \int_{x^{+}}^{\infty}p(\mu_{old}, x)dx\right)
\end{align*}
where
\begin{align*}
    x^{+} &=
    \frac{(\mu^2 - \mu_{old}^2)
        + 2\sigma^2\ln\left(1 + \epsilon\right)}
        {2(\mu - \mu_{old})},
\end{align*}
\begin{align*}
    x^{-} &=
    \frac{(\mu^2 - \mu_{old}^2)
        + 2\sigma^2\ln\left(1 - \epsilon\right)}
        {2(\mu - \mu_{old})},
\end{align*}
and $\mu$ and $\mu_{old}$ are the gaussian means predicted for this state for
the new and old policies, respectively.

On its own, this equation does not allow for close control over the
discrepancy. Consider allowing for distinct upper and lower $\epsilon$,
replacing the loss function with
\small
\begin{equation}
    L^{CLIP}(\theta) = \\
    \mathbb{E}_t\left[ 
    \min\left(r_tA_t, \text{clip}
    (r_t, 1 - \epsilon^-, 1 + \epsilon^+)A_t\right)
    \right].
    \label{eq:5}
\end{equation}
\normalsize
We can now write the discrepancy as:
\begin{equation}
    \begin{aligned}
        (1 - E[r_{t, CLIP}^{+}]) - (E[r_{t, CLIP}^{-}] - 1)\\
        =
        \epsilon^-
        + 
        (1 - \epsilon^-)
        \int_{x^{-}}^{x^{+}}p(\mu_{old}, x)dx
        \\
        -\left(\int_{x^{-}}^{x^{+}}p(\mu, x)dx 
        +(\epsilon^+ + \epsilon^-)
        \int_{x^{+}}^{\infty}p(\mu_{old}, x)dx\right)
    \end{aligned}
    \label{eq:6}
\end{equation}

Changing the subtraction between expected penalty contributions to an addition,
we can also calculate the total expected penalty contribution:
\begin{equation}
    \begin{aligned}
        (1 - E[r_{t, CLIP}^{+}]) + (E[r_{t, CLIP}^{-}] - 1)\\
        =
        2\int_{x^{+}}^{\infty}
        p(\mu, x)
        dx -
        (2 + \epsilon^+ - \epsilon^-)
        \int_{x^{+}}^{\infty}p(\mu_{old}, x)dx\\
        -
        \epsilon^-
        - 
        (1 - \epsilon^-)
        \int_{x^{-}}^{x^{+}}p(\mu_{old}, x)dx + 
        \int_{x^{-}}^{x^{+}}p(\mu, x)dx
    \end{aligned}
    \label{eq:7}
\end{equation}

From \eqref{eq:6}, we can see that increasing $\epsilon^+$ and $\epsilon^-$ will
generally have the effect of decreasing and increasing the discrepancy,
respectively. We can define an algorithm that makes use of these equations
in optimizing the clip parameters at every iteration. Given set initializations
of $\epsilon^+$ and $\epsilon^-$, before calculating the loss function at each
batch in each iteration, we can optimize these parameters to minimize
this discrepancy while maintaining the total expected penalty contributions.

\begin{algorithm}[H]
    \caption{PPO with Dynamic Clipping}
    \begin{algorithmic}
        \State Initialize $\theta$ arbitrarily
        \State Initialize $\epsilon^+$ and $\epsilon^-$
        \State $\mu_{old} \gets 0$
        \While{True}\Comment{loop forever}
            \State $\theta_{old} \gets \theta$
            \parState{$rollout \gets (s, a, r)$ from multiple episodes
                following $\pi_{\theta}$}
            \For{$batch$ from $rollout$}
                \If{not first $batch$ in $rollout$}
                    \parState {$\mu \gets$ the average absolute distance of the
                    new policy's actions from the old policy's actions}

                    \parState {Minimize \eqref{eq:6} while keeping \eqref{eq:7}
                    constant}
                \EndIf
                \parState{Perform an optimization step on \eqref{eq:5}}
            \EndFor
        \EndWhile
    \end{algorithmic}
\end{algorithm}
\section{Results}
Our tests were run on standard MuJoCo environments from OpenAI Gym. We ran
tests with the original PPO environment using $\epsilon$ values of $0.1, 0.2,
0.3, 0.4$. We then ran tests with the new algorithm using these same $\epsilon$
values as initializers for both $\epsilon^+$ and $\epsilon^-$. 

Overall, looking at individual environments and comparing the best runs
(as defined by end-of-training performance) between the control (original
PPO) and experimental algorithms, we saw that, compared to the control
algorithm, the experimental algorithm either performed approximately the same
as the control or slightly better. This suggests an improvement in
performance. In all cases, the algorithm successfully reduced the empirical
expected discrepancies, however this did not have a consistent effect on the
actual discrepancies when calculating the loss.

Figures \ref{fig:2}, \ref{fig:3}, and
\ref{fig:4} show the results for some environments. In these runs, the
experimental and control algorithms were initialized with the same value of
$\epsilon$.
\begin{figure}
    \centering
    \scalebox{0.45}{%% Creator: Matplotlib, PGF backend
%%
%% To include the figure in your LaTeX document, write
%%   \input{<filename>.pgf}
%%
%% Make sure the required packages are loaded in your preamble
%%   \usepackage{pgf}
%%
%% Figures using additional raster images can only be included by \input if
%% they are in the same directory as the main LaTeX file. For loading figures
%% from other directories you can use the `import` package
%%   \usepackage{import}
%% and then include the figures with
%%   \import{<path to file>}{<filename>.pgf}
%%
%% Matplotlib used the following preamble
%%   \usepackage{fontspec}
%%   \setmainfont{DejaVu Serif}
%%   \setsansfont{DejaVu Sans}
%%   \setmonofont{DejaVu Sans Mono}
%%
\begingroup%
\makeatletter%
\begin{pgfpicture}%
\pgfpathrectangle{\pgfpointorigin}{\pgfqpoint{6.400000in}{4.800000in}}%
\pgfusepath{use as bounding box, clip}%
\begin{pgfscope}%
\pgfsetbuttcap%
\pgfsetmiterjoin%
\definecolor{currentfill}{rgb}{1.000000,1.000000,1.000000}%
\pgfsetfillcolor{currentfill}%
\pgfsetlinewidth{0.000000pt}%
\definecolor{currentstroke}{rgb}{1.000000,1.000000,1.000000}%
\pgfsetstrokecolor{currentstroke}%
\pgfsetdash{}{0pt}%
\pgfpathmoveto{\pgfqpoint{0.000000in}{0.000000in}}%
\pgfpathlineto{\pgfqpoint{6.400000in}{0.000000in}}%
\pgfpathlineto{\pgfqpoint{6.400000in}{4.800000in}}%
\pgfpathlineto{\pgfqpoint{0.000000in}{4.800000in}}%
\pgfpathclose%
\pgfusepath{fill}%
\end{pgfscope}%
\begin{pgfscope}%
\pgfsetbuttcap%
\pgfsetmiterjoin%
\definecolor{currentfill}{rgb}{1.000000,1.000000,1.000000}%
\pgfsetfillcolor{currentfill}%
\pgfsetlinewidth{0.000000pt}%
\definecolor{currentstroke}{rgb}{0.000000,0.000000,0.000000}%
\pgfsetstrokecolor{currentstroke}%
\pgfsetstrokeopacity{0.000000}%
\pgfsetdash{}{0pt}%
\pgfpathmoveto{\pgfqpoint{0.757955in}{2.907778in}}%
\pgfpathlineto{\pgfqpoint{6.215000in}{2.907778in}}%
\pgfpathlineto{\pgfqpoint{6.215000in}{4.426667in}}%
\pgfpathlineto{\pgfqpoint{0.757955in}{4.426667in}}%
\pgfpathclose%
\pgfusepath{fill}%
\end{pgfscope}%
\begin{pgfscope}%
\pgfsetbuttcap%
\pgfsetroundjoin%
\definecolor{currentfill}{rgb}{0.000000,0.000000,0.000000}%
\pgfsetfillcolor{currentfill}%
\pgfsetlinewidth{0.803000pt}%
\definecolor{currentstroke}{rgb}{0.000000,0.000000,0.000000}%
\pgfsetstrokecolor{currentstroke}%
\pgfsetdash{}{0pt}%
\pgfsys@defobject{currentmarker}{\pgfqpoint{0.000000in}{-0.048611in}}{\pgfqpoint{0.000000in}{0.000000in}}{%
\pgfpathmoveto{\pgfqpoint{0.000000in}{0.000000in}}%
\pgfpathlineto{\pgfqpoint{0.000000in}{-0.048611in}}%
\pgfusepath{stroke,fill}%
}%
\begin{pgfscope}%
\pgfsys@transformshift{0.955198in}{2.907778in}%
\pgfsys@useobject{currentmarker}{}%
\end{pgfscope}%
\end{pgfscope}%
\begin{pgfscope}%
\pgftext[x=0.955198in,y=2.810556in,,top]{\sffamily\fontsize{10.000000}{12.000000}\selectfont \(\displaystyle 0\)}%
\end{pgfscope}%
\begin{pgfscope}%
\pgfsetbuttcap%
\pgfsetroundjoin%
\definecolor{currentfill}{rgb}{0.000000,0.000000,0.000000}%
\pgfsetfillcolor{currentfill}%
\pgfsetlinewidth{0.803000pt}%
\definecolor{currentstroke}{rgb}{0.000000,0.000000,0.000000}%
\pgfsetstrokecolor{currentstroke}%
\pgfsetdash{}{0pt}%
\pgfsys@defobject{currentmarker}{\pgfqpoint{0.000000in}{-0.048611in}}{\pgfqpoint{0.000000in}{0.000000in}}{%
\pgfpathmoveto{\pgfqpoint{0.000000in}{0.000000in}}%
\pgfpathlineto{\pgfqpoint{0.000000in}{-0.048611in}}%
\pgfusepath{stroke,fill}%
}%
\begin{pgfscope}%
\pgfsys@transformshift{1.579743in}{2.907778in}%
\pgfsys@useobject{currentmarker}{}%
\end{pgfscope}%
\end{pgfscope}%
\begin{pgfscope}%
\pgftext[x=1.579743in,y=2.810556in,,top]{\sffamily\fontsize{10.000000}{12.000000}\selectfont \(\displaystyle 25000\)}%
\end{pgfscope}%
\begin{pgfscope}%
\pgfsetbuttcap%
\pgfsetroundjoin%
\definecolor{currentfill}{rgb}{0.000000,0.000000,0.000000}%
\pgfsetfillcolor{currentfill}%
\pgfsetlinewidth{0.803000pt}%
\definecolor{currentstroke}{rgb}{0.000000,0.000000,0.000000}%
\pgfsetstrokecolor{currentstroke}%
\pgfsetdash{}{0pt}%
\pgfsys@defobject{currentmarker}{\pgfqpoint{0.000000in}{-0.048611in}}{\pgfqpoint{0.000000in}{0.000000in}}{%
\pgfpathmoveto{\pgfqpoint{0.000000in}{0.000000in}}%
\pgfpathlineto{\pgfqpoint{0.000000in}{-0.048611in}}%
\pgfusepath{stroke,fill}%
}%
\begin{pgfscope}%
\pgfsys@transformshift{2.204289in}{2.907778in}%
\pgfsys@useobject{currentmarker}{}%
\end{pgfscope}%
\end{pgfscope}%
\begin{pgfscope}%
\pgftext[x=2.204289in,y=2.810556in,,top]{\sffamily\fontsize{10.000000}{12.000000}\selectfont \(\displaystyle 50000\)}%
\end{pgfscope}%
\begin{pgfscope}%
\pgfsetbuttcap%
\pgfsetroundjoin%
\definecolor{currentfill}{rgb}{0.000000,0.000000,0.000000}%
\pgfsetfillcolor{currentfill}%
\pgfsetlinewidth{0.803000pt}%
\definecolor{currentstroke}{rgb}{0.000000,0.000000,0.000000}%
\pgfsetstrokecolor{currentstroke}%
\pgfsetdash{}{0pt}%
\pgfsys@defobject{currentmarker}{\pgfqpoint{0.000000in}{-0.048611in}}{\pgfqpoint{0.000000in}{0.000000in}}{%
\pgfpathmoveto{\pgfqpoint{0.000000in}{0.000000in}}%
\pgfpathlineto{\pgfqpoint{0.000000in}{-0.048611in}}%
\pgfusepath{stroke,fill}%
}%
\begin{pgfscope}%
\pgfsys@transformshift{2.828835in}{2.907778in}%
\pgfsys@useobject{currentmarker}{}%
\end{pgfscope}%
\end{pgfscope}%
\begin{pgfscope}%
\pgftext[x=2.828835in,y=2.810556in,,top]{\sffamily\fontsize{10.000000}{12.000000}\selectfont \(\displaystyle 75000\)}%
\end{pgfscope}%
\begin{pgfscope}%
\pgfsetbuttcap%
\pgfsetroundjoin%
\definecolor{currentfill}{rgb}{0.000000,0.000000,0.000000}%
\pgfsetfillcolor{currentfill}%
\pgfsetlinewidth{0.803000pt}%
\definecolor{currentstroke}{rgb}{0.000000,0.000000,0.000000}%
\pgfsetstrokecolor{currentstroke}%
\pgfsetdash{}{0pt}%
\pgfsys@defobject{currentmarker}{\pgfqpoint{0.000000in}{-0.048611in}}{\pgfqpoint{0.000000in}{0.000000in}}{%
\pgfpathmoveto{\pgfqpoint{0.000000in}{0.000000in}}%
\pgfpathlineto{\pgfqpoint{0.000000in}{-0.048611in}}%
\pgfusepath{stroke,fill}%
}%
\begin{pgfscope}%
\pgfsys@transformshift{3.453381in}{2.907778in}%
\pgfsys@useobject{currentmarker}{}%
\end{pgfscope}%
\end{pgfscope}%
\begin{pgfscope}%
\pgftext[x=3.453381in,y=2.810556in,,top]{\sffamily\fontsize{10.000000}{12.000000}\selectfont \(\displaystyle 100000\)}%
\end{pgfscope}%
\begin{pgfscope}%
\pgfsetbuttcap%
\pgfsetroundjoin%
\definecolor{currentfill}{rgb}{0.000000,0.000000,0.000000}%
\pgfsetfillcolor{currentfill}%
\pgfsetlinewidth{0.803000pt}%
\definecolor{currentstroke}{rgb}{0.000000,0.000000,0.000000}%
\pgfsetstrokecolor{currentstroke}%
\pgfsetdash{}{0pt}%
\pgfsys@defobject{currentmarker}{\pgfqpoint{0.000000in}{-0.048611in}}{\pgfqpoint{0.000000in}{0.000000in}}{%
\pgfpathmoveto{\pgfqpoint{0.000000in}{0.000000in}}%
\pgfpathlineto{\pgfqpoint{0.000000in}{-0.048611in}}%
\pgfusepath{stroke,fill}%
}%
\begin{pgfscope}%
\pgfsys@transformshift{4.077926in}{2.907778in}%
\pgfsys@useobject{currentmarker}{}%
\end{pgfscope}%
\end{pgfscope}%
\begin{pgfscope}%
\pgftext[x=4.077926in,y=2.810556in,,top]{\sffamily\fontsize{10.000000}{12.000000}\selectfont \(\displaystyle 125000\)}%
\end{pgfscope}%
\begin{pgfscope}%
\pgfsetbuttcap%
\pgfsetroundjoin%
\definecolor{currentfill}{rgb}{0.000000,0.000000,0.000000}%
\pgfsetfillcolor{currentfill}%
\pgfsetlinewidth{0.803000pt}%
\definecolor{currentstroke}{rgb}{0.000000,0.000000,0.000000}%
\pgfsetstrokecolor{currentstroke}%
\pgfsetdash{}{0pt}%
\pgfsys@defobject{currentmarker}{\pgfqpoint{0.000000in}{-0.048611in}}{\pgfqpoint{0.000000in}{0.000000in}}{%
\pgfpathmoveto{\pgfqpoint{0.000000in}{0.000000in}}%
\pgfpathlineto{\pgfqpoint{0.000000in}{-0.048611in}}%
\pgfusepath{stroke,fill}%
}%
\begin{pgfscope}%
\pgfsys@transformshift{4.702472in}{2.907778in}%
\pgfsys@useobject{currentmarker}{}%
\end{pgfscope}%
\end{pgfscope}%
\begin{pgfscope}%
\pgftext[x=4.702472in,y=2.810556in,,top]{\sffamily\fontsize{10.000000}{12.000000}\selectfont \(\displaystyle 150000\)}%
\end{pgfscope}%
\begin{pgfscope}%
\pgfsetbuttcap%
\pgfsetroundjoin%
\definecolor{currentfill}{rgb}{0.000000,0.000000,0.000000}%
\pgfsetfillcolor{currentfill}%
\pgfsetlinewidth{0.803000pt}%
\definecolor{currentstroke}{rgb}{0.000000,0.000000,0.000000}%
\pgfsetstrokecolor{currentstroke}%
\pgfsetdash{}{0pt}%
\pgfsys@defobject{currentmarker}{\pgfqpoint{0.000000in}{-0.048611in}}{\pgfqpoint{0.000000in}{0.000000in}}{%
\pgfpathmoveto{\pgfqpoint{0.000000in}{0.000000in}}%
\pgfpathlineto{\pgfqpoint{0.000000in}{-0.048611in}}%
\pgfusepath{stroke,fill}%
}%
\begin{pgfscope}%
\pgfsys@transformshift{5.327018in}{2.907778in}%
\pgfsys@useobject{currentmarker}{}%
\end{pgfscope}%
\end{pgfscope}%
\begin{pgfscope}%
\pgftext[x=5.327018in,y=2.810556in,,top]{\sffamily\fontsize{10.000000}{12.000000}\selectfont \(\displaystyle 175000\)}%
\end{pgfscope}%
\begin{pgfscope}%
\pgfsetbuttcap%
\pgfsetroundjoin%
\definecolor{currentfill}{rgb}{0.000000,0.000000,0.000000}%
\pgfsetfillcolor{currentfill}%
\pgfsetlinewidth{0.803000pt}%
\definecolor{currentstroke}{rgb}{0.000000,0.000000,0.000000}%
\pgfsetstrokecolor{currentstroke}%
\pgfsetdash{}{0pt}%
\pgfsys@defobject{currentmarker}{\pgfqpoint{0.000000in}{-0.048611in}}{\pgfqpoint{0.000000in}{0.000000in}}{%
\pgfpathmoveto{\pgfqpoint{0.000000in}{0.000000in}}%
\pgfpathlineto{\pgfqpoint{0.000000in}{-0.048611in}}%
\pgfusepath{stroke,fill}%
}%
\begin{pgfscope}%
\pgfsys@transformshift{5.951564in}{2.907778in}%
\pgfsys@useobject{currentmarker}{}%
\end{pgfscope}%
\end{pgfscope}%
\begin{pgfscope}%
\pgftext[x=5.951564in,y=2.810556in,,top]{\sffamily\fontsize{10.000000}{12.000000}\selectfont \(\displaystyle 200000\)}%
\end{pgfscope}%
\begin{pgfscope}%
\pgftext[x=3.486477in,y=2.620587in,,top]{\sffamily\fontsize{10.000000}{12.000000}\selectfont Timestep}%
\end{pgfscope}%
\begin{pgfscope}%
\pgfsetbuttcap%
\pgfsetroundjoin%
\definecolor{currentfill}{rgb}{0.000000,0.000000,0.000000}%
\pgfsetfillcolor{currentfill}%
\pgfsetlinewidth{0.803000pt}%
\definecolor{currentstroke}{rgb}{0.000000,0.000000,0.000000}%
\pgfsetstrokecolor{currentstroke}%
\pgfsetdash{}{0pt}%
\pgfsys@defobject{currentmarker}{\pgfqpoint{-0.048611in}{0.000000in}}{\pgfqpoint{0.000000in}{0.000000in}}{%
\pgfpathmoveto{\pgfqpoint{0.000000in}{0.000000in}}%
\pgfpathlineto{\pgfqpoint{-0.048611in}{0.000000in}}%
\pgfusepath{stroke,fill}%
}%
\begin{pgfscope}%
\pgfsys@transformshift{0.757955in}{2.978323in}%
\pgfsys@useobject{currentmarker}{}%
\end{pgfscope}%
\end{pgfscope}%
\begin{pgfscope}%
\pgftext[x=0.591288in,y=2.925562in,left,base]{\sffamily\fontsize{10.000000}{12.000000}\selectfont \(\displaystyle 0\)}%
\end{pgfscope}%
\begin{pgfscope}%
\pgfsetbuttcap%
\pgfsetroundjoin%
\definecolor{currentfill}{rgb}{0.000000,0.000000,0.000000}%
\pgfsetfillcolor{currentfill}%
\pgfsetlinewidth{0.803000pt}%
\definecolor{currentstroke}{rgb}{0.000000,0.000000,0.000000}%
\pgfsetstrokecolor{currentstroke}%
\pgfsetdash{}{0pt}%
\pgfsys@defobject{currentmarker}{\pgfqpoint{-0.048611in}{0.000000in}}{\pgfqpoint{0.000000in}{0.000000in}}{%
\pgfpathmoveto{\pgfqpoint{0.000000in}{0.000000in}}%
\pgfpathlineto{\pgfqpoint{-0.048611in}{0.000000in}}%
\pgfusepath{stroke,fill}%
}%
\begin{pgfscope}%
\pgfsys@transformshift{0.757955in}{3.321338in}%
\pgfsys@useobject{currentmarker}{}%
\end{pgfscope}%
\end{pgfscope}%
\begin{pgfscope}%
\pgftext[x=0.452399in,y=3.268577in,left,base]{\sffamily\fontsize{10.000000}{12.000000}\selectfont \(\displaystyle 100\)}%
\end{pgfscope}%
\begin{pgfscope}%
\pgfsetbuttcap%
\pgfsetroundjoin%
\definecolor{currentfill}{rgb}{0.000000,0.000000,0.000000}%
\pgfsetfillcolor{currentfill}%
\pgfsetlinewidth{0.803000pt}%
\definecolor{currentstroke}{rgb}{0.000000,0.000000,0.000000}%
\pgfsetstrokecolor{currentstroke}%
\pgfsetdash{}{0pt}%
\pgfsys@defobject{currentmarker}{\pgfqpoint{-0.048611in}{0.000000in}}{\pgfqpoint{0.000000in}{0.000000in}}{%
\pgfpathmoveto{\pgfqpoint{0.000000in}{0.000000in}}%
\pgfpathlineto{\pgfqpoint{-0.048611in}{0.000000in}}%
\pgfusepath{stroke,fill}%
}%
\begin{pgfscope}%
\pgfsys@transformshift{0.757955in}{3.664353in}%
\pgfsys@useobject{currentmarker}{}%
\end{pgfscope}%
\end{pgfscope}%
\begin{pgfscope}%
\pgftext[x=0.452399in,y=3.611592in,left,base]{\sffamily\fontsize{10.000000}{12.000000}\selectfont \(\displaystyle 200\)}%
\end{pgfscope}%
\begin{pgfscope}%
\pgfsetbuttcap%
\pgfsetroundjoin%
\definecolor{currentfill}{rgb}{0.000000,0.000000,0.000000}%
\pgfsetfillcolor{currentfill}%
\pgfsetlinewidth{0.803000pt}%
\definecolor{currentstroke}{rgb}{0.000000,0.000000,0.000000}%
\pgfsetstrokecolor{currentstroke}%
\pgfsetdash{}{0pt}%
\pgfsys@defobject{currentmarker}{\pgfqpoint{-0.048611in}{0.000000in}}{\pgfqpoint{0.000000in}{0.000000in}}{%
\pgfpathmoveto{\pgfqpoint{0.000000in}{0.000000in}}%
\pgfpathlineto{\pgfqpoint{-0.048611in}{0.000000in}}%
\pgfusepath{stroke,fill}%
}%
\begin{pgfscope}%
\pgfsys@transformshift{0.757955in}{4.007368in}%
\pgfsys@useobject{currentmarker}{}%
\end{pgfscope}%
\end{pgfscope}%
\begin{pgfscope}%
\pgftext[x=0.452399in,y=3.954607in,left,base]{\sffamily\fontsize{10.000000}{12.000000}\selectfont \(\displaystyle 300\)}%
\end{pgfscope}%
\begin{pgfscope}%
\pgfsetbuttcap%
\pgfsetroundjoin%
\definecolor{currentfill}{rgb}{0.000000,0.000000,0.000000}%
\pgfsetfillcolor{currentfill}%
\pgfsetlinewidth{0.803000pt}%
\definecolor{currentstroke}{rgb}{0.000000,0.000000,0.000000}%
\pgfsetstrokecolor{currentstroke}%
\pgfsetdash{}{0pt}%
\pgfsys@defobject{currentmarker}{\pgfqpoint{-0.048611in}{0.000000in}}{\pgfqpoint{0.000000in}{0.000000in}}{%
\pgfpathmoveto{\pgfqpoint{0.000000in}{0.000000in}}%
\pgfpathlineto{\pgfqpoint{-0.048611in}{0.000000in}}%
\pgfusepath{stroke,fill}%
}%
\begin{pgfscope}%
\pgfsys@transformshift{0.757955in}{4.350383in}%
\pgfsys@useobject{currentmarker}{}%
\end{pgfscope}%
\end{pgfscope}%
\begin{pgfscope}%
\pgftext[x=0.452399in,y=4.297622in,left,base]{\sffamily\fontsize{10.000000}{12.000000}\selectfont \(\displaystyle 400\)}%
\end{pgfscope}%
\begin{pgfscope}%
\pgftext[x=0.396843in,y=3.667222in,,bottom,rotate=90.000000]{\sffamily\fontsize{10.000000}{12.000000}\selectfont Average Return}%
\end{pgfscope}%
\begin{pgfscope}%
\pgfpathrectangle{\pgfqpoint{0.757955in}{2.907778in}}{\pgfqpoint{5.457045in}{1.518889in}}%
\pgfusepath{clip}%
\pgfsetrectcap%
\pgfsetroundjoin%
\pgfsetlinewidth{1.505625pt}%
\definecolor{currentstroke}{rgb}{0.000000,0.000000,0.000000}%
\pgfsetstrokecolor{currentstroke}%
\pgfsetdash{}{0pt}%
\pgfpathmoveto{\pgfqpoint{1.006002in}{2.976818in}}%
\pgfpathlineto{\pgfqpoint{1.057407in}{2.980723in}}%
\pgfpathlineto{\pgfqpoint{1.108486in}{2.986590in}}%
\pgfpathlineto{\pgfqpoint{1.159691in}{2.995456in}}%
\pgfpathlineto{\pgfqpoint{1.210196in}{3.003373in}}%
\pgfpathlineto{\pgfqpoint{1.261308in}{3.015132in}}%
\pgfpathlineto{\pgfqpoint{1.312946in}{3.030083in}}%
\pgfpathlineto{\pgfqpoint{1.363434in}{3.071651in}}%
\pgfpathlineto{\pgfqpoint{1.414355in}{3.090841in}}%
\pgfpathlineto{\pgfqpoint{1.466251in}{3.136722in}}%
\pgfpathlineto{\pgfqpoint{1.517597in}{3.147620in}}%
\pgfpathlineto{\pgfqpoint{1.564938in}{3.191078in}}%
\pgfpathlineto{\pgfqpoint{1.617100in}{3.231227in}}%
\pgfpathlineto{\pgfqpoint{1.669570in}{3.251098in}}%
\pgfpathlineto{\pgfqpoint{1.721740in}{3.265828in}}%
\pgfpathlineto{\pgfqpoint{1.771862in}{3.269634in}}%
\pgfpathlineto{\pgfqpoint{1.823325in}{3.365867in}}%
\pgfpathlineto{\pgfqpoint{1.874104in}{3.361130in}}%
\pgfpathlineto{\pgfqpoint{1.925758in}{3.423346in}}%
\pgfpathlineto{\pgfqpoint{1.976463in}{3.402277in}}%
\pgfpathlineto{\pgfqpoint{2.029033in}{3.504600in}}%
\pgfpathlineto{\pgfqpoint{2.076424in}{3.569851in}}%
\pgfpathlineto{\pgfqpoint{2.128528in}{3.619586in}}%
\pgfpathlineto{\pgfqpoint{2.180540in}{3.594650in}}%
\pgfpathlineto{\pgfqpoint{2.231802in}{3.819148in}}%
\pgfpathlineto{\pgfqpoint{2.277977in}{3.559672in}}%
\pgfpathlineto{\pgfqpoint{2.333595in}{3.705422in}}%
\pgfpathlineto{\pgfqpoint{2.383859in}{3.650218in}}%
\pgfpathlineto{\pgfqpoint{2.436770in}{3.719049in}}%
\pgfpathlineto{\pgfqpoint{2.488158in}{3.926601in}}%
\pgfpathlineto{\pgfqpoint{2.535923in}{3.818990in}}%
\pgfpathlineto{\pgfqpoint{2.589817in}{3.701996in}}%
\pgfpathlineto{\pgfqpoint{2.641946in}{3.697634in}}%
\pgfpathlineto{\pgfqpoint{2.692143in}{3.789082in}}%
\pgfpathlineto{\pgfqpoint{2.743622in}{3.764877in}}%
\pgfpathlineto{\pgfqpoint{2.793319in}{3.811896in}}%
\pgfpathlineto{\pgfqpoint{2.840168in}{3.706425in}}%
\pgfpathlineto{\pgfqpoint{2.896611in}{3.990314in}}%
\pgfpathlineto{\pgfqpoint{2.949447in}{3.777459in}}%
\pgfpathlineto{\pgfqpoint{2.995805in}{3.941922in}}%
\pgfpathlineto{\pgfqpoint{3.050291in}{3.980929in}}%
\pgfpathlineto{\pgfqpoint{3.097681in}{4.063834in}}%
\pgfpathlineto{\pgfqpoint{3.153082in}{4.035639in}}%
\pgfpathlineto{\pgfqpoint{3.202188in}{3.943288in}}%
\pgfpathlineto{\pgfqpoint{3.254267in}{4.017307in}}%
\pgfpathlineto{\pgfqpoint{3.303898in}{3.974623in}}%
\pgfpathlineto{\pgfqpoint{3.350247in}{3.984970in}}%
\pgfpathlineto{\pgfqpoint{3.408164in}{4.101052in}}%
\pgfpathlineto{\pgfqpoint{3.456603in}{3.913899in}}%
\pgfpathlineto{\pgfqpoint{3.507899in}{4.104906in}}%
\pgfpathlineto{\pgfqpoint{3.556481in}{4.018275in}}%
\pgfpathlineto{\pgfqpoint{3.608776in}{3.952643in}}%
\pgfpathlineto{\pgfqpoint{3.664794in}{4.019518in}}%
\pgfpathlineto{\pgfqpoint{3.715940in}{3.877387in}}%
\pgfpathlineto{\pgfqpoint{3.765970in}{4.045718in}}%
\pgfpathlineto{\pgfqpoint{3.815126in}{4.067888in}}%
\pgfpathlineto{\pgfqpoint{3.867913in}{4.036866in}}%
\pgfpathlineto{\pgfqpoint{3.920000in}{4.357626in}}%
\pgfpathlineto{\pgfqpoint{3.971021in}{3.921289in}}%
\pgfpathlineto{\pgfqpoint{4.023533in}{4.037316in}}%
\pgfpathlineto{\pgfqpoint{4.071223in}{3.961714in}}%
\pgfpathlineto{\pgfqpoint{4.125642in}{3.962882in}}%
\pgfpathlineto{\pgfqpoint{4.173623in}{4.070499in}}%
\pgfpathlineto{\pgfqpoint{4.225752in}{4.228738in}}%
\pgfpathlineto{\pgfqpoint{4.279005in}{4.165838in}}%
\pgfpathlineto{\pgfqpoint{4.326962in}{3.941701in}}%
\pgfpathlineto{\pgfqpoint{4.378841in}{4.147728in}}%
\pgfpathlineto{\pgfqpoint{4.432077in}{4.080465in}}%
\pgfpathlineto{\pgfqpoint{4.478002in}{4.214216in}}%
\pgfpathlineto{\pgfqpoint{4.534028in}{4.131256in}}%
\pgfpathlineto{\pgfqpoint{4.583767in}{4.084297in}}%
\pgfpathlineto{\pgfqpoint{4.631199in}{4.178175in}}%
\pgfpathlineto{\pgfqpoint{4.687874in}{4.246210in}}%
\pgfpathlineto{\pgfqpoint{4.737913in}{4.291156in}}%
\pgfpathlineto{\pgfqpoint{4.786161in}{4.197412in}}%
\pgfpathlineto{\pgfqpoint{4.839631in}{4.230034in}}%
\pgfpathlineto{\pgfqpoint{4.891918in}{4.141774in}}%
\pgfpathlineto{\pgfqpoint{4.940233in}{4.196198in}}%
\pgfpathlineto{\pgfqpoint{4.994818in}{4.080516in}}%
\pgfpathlineto{\pgfqpoint{5.044757in}{4.137335in}}%
\pgfpathlineto{\pgfqpoint{5.096502in}{4.061320in}}%
\pgfpathlineto{\pgfqpoint{5.148273in}{4.186129in}}%
\pgfpathlineto{\pgfqpoint{5.199569in}{4.187512in}}%
\pgfpathlineto{\pgfqpoint{5.248067in}{4.113202in}}%
\pgfpathlineto{\pgfqpoint{5.300654in}{4.177514in}}%
\pgfpathlineto{\pgfqpoint{5.350717in}{4.196151in}}%
\pgfpathlineto{\pgfqpoint{5.403579in}{4.168070in}}%
\pgfpathlineto{\pgfqpoint{5.454026in}{4.172189in}}%
\pgfpathlineto{\pgfqpoint{5.506654in}{4.281681in}}%
\pgfpathlineto{\pgfqpoint{5.556251in}{4.144511in}}%
\pgfpathlineto{\pgfqpoint{5.608946in}{4.224333in}}%
\pgfpathlineto{\pgfqpoint{5.659626in}{4.219335in}}%
\pgfpathlineto{\pgfqpoint{5.711338in}{4.097088in}}%
\pgfpathlineto{\pgfqpoint{5.761594in}{4.148737in}}%
\pgfpathlineto{\pgfqpoint{5.812115in}{4.237859in}}%
\pgfpathlineto{\pgfqpoint{5.863827in}{4.136567in}}%
\pgfpathlineto{\pgfqpoint{5.916847in}{4.231776in}}%
\pgfpathlineto{\pgfqpoint{5.966952in}{4.291953in}}%
\pgfusepath{stroke}%
\end{pgfscope}%
\begin{pgfscope}%
\pgfpathrectangle{\pgfqpoint{0.757955in}{2.907778in}}{\pgfqpoint{5.457045in}{1.518889in}}%
\pgfusepath{clip}%
\pgfsetbuttcap%
\pgfsetroundjoin%
\pgfsetlinewidth{1.505625pt}%
\definecolor{currentstroke}{rgb}{0.000000,0.000000,0.000000}%
\pgfsetstrokecolor{currentstroke}%
\pgfsetdash{{5.550000pt}{2.400000pt}}{0.000000pt}%
\pgfpathmoveto{\pgfqpoint{1.006002in}{2.976818in}}%
\pgfpathlineto{\pgfqpoint{1.057415in}{2.980825in}}%
\pgfpathlineto{\pgfqpoint{1.108453in}{2.988216in}}%
\pgfpathlineto{\pgfqpoint{1.159532in}{2.999321in}}%
\pgfpathlineto{\pgfqpoint{1.210820in}{3.003461in}}%
\pgfpathlineto{\pgfqpoint{1.261208in}{3.009610in}}%
\pgfpathlineto{\pgfqpoint{1.312954in}{3.036661in}}%
\pgfpathlineto{\pgfqpoint{1.362868in}{3.066219in}}%
\pgfpathlineto{\pgfqpoint{1.415296in}{3.103142in}}%
\pgfpathlineto{\pgfqpoint{1.466168in}{3.123492in}}%
\pgfpathlineto{\pgfqpoint{1.517214in}{3.164792in}}%
\pgfpathlineto{\pgfqpoint{1.567594in}{3.178335in}}%
\pgfpathlineto{\pgfqpoint{1.618973in}{3.260845in}}%
\pgfpathlineto{\pgfqpoint{1.669919in}{3.247892in}}%
\pgfpathlineto{\pgfqpoint{1.721757in}{3.318665in}}%
\pgfpathlineto{\pgfqpoint{1.772187in}{3.428322in}}%
\pgfpathlineto{\pgfqpoint{1.820960in}{3.511791in}}%
\pgfpathlineto{\pgfqpoint{1.874296in}{3.446356in}}%
\pgfpathlineto{\pgfqpoint{1.925101in}{3.476138in}}%
\pgfpathlineto{\pgfqpoint{1.976996in}{3.513269in}}%
\pgfpathlineto{\pgfqpoint{2.027085in}{3.483558in}}%
\pgfpathlineto{\pgfqpoint{2.076865in}{3.612500in}}%
\pgfpathlineto{\pgfqpoint{2.128586in}{3.625555in}}%
\pgfpathlineto{\pgfqpoint{2.179432in}{3.675222in}}%
\pgfpathlineto{\pgfqpoint{2.232377in}{3.720638in}}%
\pgfpathlineto{\pgfqpoint{2.281550in}{3.502870in}}%
\pgfpathlineto{\pgfqpoint{2.333745in}{3.700354in}}%
\pgfpathlineto{\pgfqpoint{2.385724in}{3.725424in}}%
\pgfpathlineto{\pgfqpoint{2.436703in}{3.644411in}}%
\pgfpathlineto{\pgfqpoint{2.487816in}{3.891434in}}%
\pgfpathlineto{\pgfqpoint{2.537547in}{3.896172in}}%
\pgfpathlineto{\pgfqpoint{2.588793in}{3.730299in}}%
\pgfpathlineto{\pgfqpoint{2.639839in}{3.692798in}}%
\pgfpathlineto{\pgfqpoint{2.692950in}{3.816642in}}%
\pgfpathlineto{\pgfqpoint{2.742423in}{3.744348in}}%
\pgfpathlineto{\pgfqpoint{2.795259in}{3.732405in}}%
\pgfpathlineto{\pgfqpoint{2.845631in}{3.896425in}}%
\pgfpathlineto{\pgfqpoint{2.896036in}{3.835525in}}%
\pgfpathlineto{\pgfqpoint{2.945700in}{3.900073in}}%
\pgfpathlineto{\pgfqpoint{2.999752in}{3.829202in}}%
\pgfpathlineto{\pgfqpoint{3.049508in}{3.844843in}}%
\pgfpathlineto{\pgfqpoint{3.095141in}{3.775572in}}%
\pgfpathlineto{\pgfqpoint{3.152216in}{3.759908in}}%
\pgfpathlineto{\pgfqpoint{3.201464in}{3.959600in}}%
\pgfpathlineto{\pgfqpoint{3.256424in}{4.022704in}}%
\pgfpathlineto{\pgfqpoint{3.304073in}{4.021319in}}%
\pgfpathlineto{\pgfqpoint{3.357042in}{4.072285in}}%
\pgfpathlineto{\pgfqpoint{3.406873in}{3.931154in}}%
\pgfpathlineto{\pgfqpoint{3.458785in}{3.800448in}}%
\pgfpathlineto{\pgfqpoint{3.510822in}{3.993236in}}%
\pgfpathlineto{\pgfqpoint{3.562443in}{4.143828in}}%
\pgfpathlineto{\pgfqpoint{3.613281in}{3.944083in}}%
\pgfpathlineto{\pgfqpoint{3.661946in}{4.029449in}}%
\pgfpathlineto{\pgfqpoint{3.712309in}{3.983472in}}%
\pgfpathlineto{\pgfqpoint{3.764979in}{4.052945in}}%
\pgfpathlineto{\pgfqpoint{3.819231in}{4.113737in}}%
\pgfpathlineto{\pgfqpoint{3.868279in}{4.021591in}}%
\pgfpathlineto{\pgfqpoint{3.918909in}{4.042379in}}%
\pgfpathlineto{\pgfqpoint{3.971296in}{4.152526in}}%
\pgfpathlineto{\pgfqpoint{4.021268in}{4.019167in}}%
\pgfpathlineto{\pgfqpoint{4.073388in}{4.070897in}}%
\pgfpathlineto{\pgfqpoint{4.123360in}{4.002750in}}%
\pgfpathlineto{\pgfqpoint{4.175114in}{4.101757in}}%
\pgfpathlineto{\pgfqpoint{4.226876in}{3.980680in}}%
\pgfpathlineto{\pgfqpoint{4.275916in}{4.093776in}}%
\pgfpathlineto{\pgfqpoint{4.330934in}{4.063096in}}%
\pgfpathlineto{\pgfqpoint{4.380365in}{4.112193in}}%
\pgfpathlineto{\pgfqpoint{4.431561in}{4.122413in}}%
\pgfpathlineto{\pgfqpoint{4.482790in}{4.047724in}}%
\pgfpathlineto{\pgfqpoint{4.532429in}{4.104083in}}%
\pgfpathlineto{\pgfqpoint{4.584150in}{4.176782in}}%
\pgfpathlineto{\pgfqpoint{4.636770in}{4.064739in}}%
\pgfpathlineto{\pgfqpoint{4.687575in}{4.062942in}}%
\pgfpathlineto{\pgfqpoint{4.735407in}{4.077473in}}%
\pgfpathlineto{\pgfqpoint{4.789634in}{4.155576in}}%
\pgfpathlineto{\pgfqpoint{4.841230in}{4.196053in}}%
\pgfpathlineto{\pgfqpoint{4.889086in}{4.159978in}}%
\pgfpathlineto{\pgfqpoint{4.940124in}{4.037716in}}%
\pgfpathlineto{\pgfqpoint{4.992503in}{4.163107in}}%
\pgfpathlineto{\pgfqpoint{5.045681in}{4.166875in}}%
\pgfpathlineto{\pgfqpoint{5.097177in}{4.176662in}}%
\pgfpathlineto{\pgfqpoint{5.148514in}{4.101937in}}%
\pgfpathlineto{\pgfqpoint{5.200243in}{4.158865in}}%
\pgfpathlineto{\pgfqpoint{5.247584in}{4.117803in}}%
\pgfpathlineto{\pgfqpoint{5.299804in}{4.208619in}}%
\pgfpathlineto{\pgfqpoint{5.352108in}{4.104866in}}%
\pgfpathlineto{\pgfqpoint{5.403770in}{4.221657in}}%
\pgfpathlineto{\pgfqpoint{5.455225in}{4.147319in}}%
\pgfpathlineto{\pgfqpoint{5.506254in}{4.074571in}}%
\pgfpathlineto{\pgfqpoint{5.557317in}{4.239152in}}%
\pgfpathlineto{\pgfqpoint{5.607705in}{4.162944in}}%
\pgfpathlineto{\pgfqpoint{5.658768in}{4.193569in}}%
\pgfpathlineto{\pgfqpoint{5.710081in}{4.201147in}}%
\pgfpathlineto{\pgfqpoint{5.761185in}{4.167767in}}%
\pgfpathlineto{\pgfqpoint{5.810192in}{4.209195in}}%
\pgfpathlineto{\pgfqpoint{5.861796in}{4.112566in}}%
\pgfpathlineto{\pgfqpoint{5.912584in}{4.134255in}}%
\pgfpathlineto{\pgfqpoint{5.965728in}{4.205450in}}%
\pgfusepath{stroke}%
\end{pgfscope}%
\begin{pgfscope}%
\pgfsetrectcap%
\pgfsetmiterjoin%
\pgfsetlinewidth{0.803000pt}%
\definecolor{currentstroke}{rgb}{0.000000,0.000000,0.000000}%
\pgfsetstrokecolor{currentstroke}%
\pgfsetdash{}{0pt}%
\pgfpathmoveto{\pgfqpoint{0.757955in}{2.907778in}}%
\pgfpathlineto{\pgfqpoint{0.757955in}{4.426667in}}%
\pgfusepath{stroke}%
\end{pgfscope}%
\begin{pgfscope}%
\pgfsetrectcap%
\pgfsetmiterjoin%
\pgfsetlinewidth{0.803000pt}%
\definecolor{currentstroke}{rgb}{0.000000,0.000000,0.000000}%
\pgfsetstrokecolor{currentstroke}%
\pgfsetdash{}{0pt}%
\pgfpathmoveto{\pgfqpoint{6.215000in}{2.907778in}}%
\pgfpathlineto{\pgfqpoint{6.215000in}{4.426667in}}%
\pgfusepath{stroke}%
\end{pgfscope}%
\begin{pgfscope}%
\pgfsetrectcap%
\pgfsetmiterjoin%
\pgfsetlinewidth{0.803000pt}%
\definecolor{currentstroke}{rgb}{0.000000,0.000000,0.000000}%
\pgfsetstrokecolor{currentstroke}%
\pgfsetdash{}{0pt}%
\pgfpathmoveto{\pgfqpoint{0.757955in}{2.907778in}}%
\pgfpathlineto{\pgfqpoint{6.215000in}{2.907778in}}%
\pgfusepath{stroke}%
\end{pgfscope}%
\begin{pgfscope}%
\pgfsetrectcap%
\pgfsetmiterjoin%
\pgfsetlinewidth{0.803000pt}%
\definecolor{currentstroke}{rgb}{0.000000,0.000000,0.000000}%
\pgfsetstrokecolor{currentstroke}%
\pgfsetdash{}{0pt}%
\pgfpathmoveto{\pgfqpoint{0.757955in}{4.426667in}}%
\pgfpathlineto{\pgfqpoint{6.215000in}{4.426667in}}%
\pgfusepath{stroke}%
\end{pgfscope}%
\begin{pgfscope}%
\pgftext[x=3.486477in,y=4.510000in,,base]{\sffamily\fontsize{12.000000}{14.400000}\selectfont Performance}%
\end{pgfscope}%
\begin{pgfscope}%
\pgfsetbuttcap%
\pgfsetmiterjoin%
\definecolor{currentfill}{rgb}{1.000000,1.000000,1.000000}%
\pgfsetfillcolor{currentfill}%
\pgfsetfillopacity{0.800000}%
\pgfsetlinewidth{1.003750pt}%
\definecolor{currentstroke}{rgb}{0.800000,0.800000,0.800000}%
\pgfsetstrokecolor{currentstroke}%
\pgfsetstrokeopacity{0.800000}%
\pgfsetdash{}{0pt}%
\pgfpathmoveto{\pgfqpoint{0.855177in}{3.907841in}}%
\pgfpathlineto{\pgfqpoint{2.221049in}{3.907841in}}%
\pgfpathquadraticcurveto{\pgfqpoint{2.248827in}{3.907841in}}{\pgfqpoint{2.248827in}{3.935619in}}%
\pgfpathlineto{\pgfqpoint{2.248827in}{4.329444in}}%
\pgfpathquadraticcurveto{\pgfqpoint{2.248827in}{4.357222in}}{\pgfqpoint{2.221049in}{4.357222in}}%
\pgfpathlineto{\pgfqpoint{0.855177in}{4.357222in}}%
\pgfpathquadraticcurveto{\pgfqpoint{0.827399in}{4.357222in}}{\pgfqpoint{0.827399in}{4.329444in}}%
\pgfpathlineto{\pgfqpoint{0.827399in}{3.935619in}}%
\pgfpathquadraticcurveto{\pgfqpoint{0.827399in}{3.907841in}}{\pgfqpoint{0.855177in}{3.907841in}}%
\pgfpathclose%
\pgfusepath{stroke,fill}%
\end{pgfscope}%
\begin{pgfscope}%
\pgfsetrectcap%
\pgfsetroundjoin%
\pgfsetlinewidth{1.505625pt}%
\definecolor{currentstroke}{rgb}{0.000000,0.000000,0.000000}%
\pgfsetstrokecolor{currentstroke}%
\pgfsetdash{}{0pt}%
\pgfpathmoveto{\pgfqpoint{0.882955in}{4.244755in}}%
\pgfpathlineto{\pgfqpoint{1.160733in}{4.244755in}}%
\pgfusepath{stroke}%
\end{pgfscope}%
\begin{pgfscope}%
\pgftext[x=1.271844in,y=4.196144in,left,base]{\sffamily\fontsize{10.000000}{12.000000}\selectfont control}%
\end{pgfscope}%
\begin{pgfscope}%
\pgfsetbuttcap%
\pgfsetroundjoin%
\pgfsetlinewidth{1.505625pt}%
\definecolor{currentstroke}{rgb}{0.000000,0.000000,0.000000}%
\pgfsetstrokecolor{currentstroke}%
\pgfsetdash{{5.550000pt}{2.400000pt}}{0.000000pt}%
\pgfpathmoveto{\pgfqpoint{0.882955in}{4.040897in}}%
\pgfpathlineto{\pgfqpoint{1.160733in}{4.040897in}}%
\pgfusepath{stroke}%
\end{pgfscope}%
\begin{pgfscope}%
\pgftext[x=1.271844in,y=3.992286in,left,base]{\sffamily\fontsize{10.000000}{12.000000}\selectfont experimental}%
\end{pgfscope}%
\begin{pgfscope}%
\pgfsetbuttcap%
\pgfsetmiterjoin%
\definecolor{currentfill}{rgb}{1.000000,1.000000,1.000000}%
\pgfsetfillcolor{currentfill}%
\pgfsetlinewidth{0.000000pt}%
\definecolor{currentstroke}{rgb}{0.000000,0.000000,0.000000}%
\pgfsetstrokecolor{currentstroke}%
\pgfsetstrokeopacity{0.000000}%
\pgfsetdash{}{0pt}%
\pgfpathmoveto{\pgfqpoint{0.757955in}{0.582778in}}%
\pgfpathlineto{\pgfqpoint{6.215000in}{0.582778in}}%
\pgfpathlineto{\pgfqpoint{6.215000in}{2.101667in}}%
\pgfpathlineto{\pgfqpoint{0.757955in}{2.101667in}}%
\pgfpathclose%
\pgfusepath{fill}%
\end{pgfscope}%
\begin{pgfscope}%
\pgfsetbuttcap%
\pgfsetroundjoin%
\definecolor{currentfill}{rgb}{0.000000,0.000000,0.000000}%
\pgfsetfillcolor{currentfill}%
\pgfsetlinewidth{0.803000pt}%
\definecolor{currentstroke}{rgb}{0.000000,0.000000,0.000000}%
\pgfsetstrokecolor{currentstroke}%
\pgfsetdash{}{0pt}%
\pgfsys@defobject{currentmarker}{\pgfqpoint{0.000000in}{-0.048611in}}{\pgfqpoint{0.000000in}{0.000000in}}{%
\pgfpathmoveto{\pgfqpoint{0.000000in}{0.000000in}}%
\pgfpathlineto{\pgfqpoint{0.000000in}{-0.048611in}}%
\pgfusepath{stroke,fill}%
}%
\begin{pgfscope}%
\pgfsys@transformshift{0.954859in}{0.582778in}%
\pgfsys@useobject{currentmarker}{}%
\end{pgfscope}%
\end{pgfscope}%
\begin{pgfscope}%
\pgftext[x=0.954859in,y=0.485556in,,top]{\sffamily\fontsize{10.000000}{12.000000}\selectfont \(\displaystyle 0\)}%
\end{pgfscope}%
\begin{pgfscope}%
\pgfsetbuttcap%
\pgfsetroundjoin%
\definecolor{currentfill}{rgb}{0.000000,0.000000,0.000000}%
\pgfsetfillcolor{currentfill}%
\pgfsetlinewidth{0.803000pt}%
\definecolor{currentstroke}{rgb}{0.000000,0.000000,0.000000}%
\pgfsetstrokecolor{currentstroke}%
\pgfsetdash{}{0pt}%
\pgfsys@defobject{currentmarker}{\pgfqpoint{0.000000in}{-0.048611in}}{\pgfqpoint{0.000000in}{0.000000in}}{%
\pgfpathmoveto{\pgfqpoint{0.000000in}{0.000000in}}%
\pgfpathlineto{\pgfqpoint{0.000000in}{-0.048611in}}%
\pgfusepath{stroke,fill}%
}%
\begin{pgfscope}%
\pgfsys@transformshift{1.977735in}{0.582778in}%
\pgfsys@useobject{currentmarker}{}%
\end{pgfscope}%
\end{pgfscope}%
\begin{pgfscope}%
\pgftext[x=1.977735in,y=0.485556in,,top]{\sffamily\fontsize{10.000000}{12.000000}\selectfont \(\displaystyle 20\)}%
\end{pgfscope}%
\begin{pgfscope}%
\pgfsetbuttcap%
\pgfsetroundjoin%
\definecolor{currentfill}{rgb}{0.000000,0.000000,0.000000}%
\pgfsetfillcolor{currentfill}%
\pgfsetlinewidth{0.803000pt}%
\definecolor{currentstroke}{rgb}{0.000000,0.000000,0.000000}%
\pgfsetstrokecolor{currentstroke}%
\pgfsetdash{}{0pt}%
\pgfsys@defobject{currentmarker}{\pgfqpoint{0.000000in}{-0.048611in}}{\pgfqpoint{0.000000in}{0.000000in}}{%
\pgfpathmoveto{\pgfqpoint{0.000000in}{0.000000in}}%
\pgfpathlineto{\pgfqpoint{0.000000in}{-0.048611in}}%
\pgfusepath{stroke,fill}%
}%
\begin{pgfscope}%
\pgfsys@transformshift{3.000611in}{0.582778in}%
\pgfsys@useobject{currentmarker}{}%
\end{pgfscope}%
\end{pgfscope}%
\begin{pgfscope}%
\pgftext[x=3.000611in,y=0.485556in,,top]{\sffamily\fontsize{10.000000}{12.000000}\selectfont \(\displaystyle 40\)}%
\end{pgfscope}%
\begin{pgfscope}%
\pgfsetbuttcap%
\pgfsetroundjoin%
\definecolor{currentfill}{rgb}{0.000000,0.000000,0.000000}%
\pgfsetfillcolor{currentfill}%
\pgfsetlinewidth{0.803000pt}%
\definecolor{currentstroke}{rgb}{0.000000,0.000000,0.000000}%
\pgfsetstrokecolor{currentstroke}%
\pgfsetdash{}{0pt}%
\pgfsys@defobject{currentmarker}{\pgfqpoint{0.000000in}{-0.048611in}}{\pgfqpoint{0.000000in}{0.000000in}}{%
\pgfpathmoveto{\pgfqpoint{0.000000in}{0.000000in}}%
\pgfpathlineto{\pgfqpoint{0.000000in}{-0.048611in}}%
\pgfusepath{stroke,fill}%
}%
\begin{pgfscope}%
\pgfsys@transformshift{4.023487in}{0.582778in}%
\pgfsys@useobject{currentmarker}{}%
\end{pgfscope}%
\end{pgfscope}%
\begin{pgfscope}%
\pgftext[x=4.023487in,y=0.485556in,,top]{\sffamily\fontsize{10.000000}{12.000000}\selectfont \(\displaystyle 60\)}%
\end{pgfscope}%
\begin{pgfscope}%
\pgfsetbuttcap%
\pgfsetroundjoin%
\definecolor{currentfill}{rgb}{0.000000,0.000000,0.000000}%
\pgfsetfillcolor{currentfill}%
\pgfsetlinewidth{0.803000pt}%
\definecolor{currentstroke}{rgb}{0.000000,0.000000,0.000000}%
\pgfsetstrokecolor{currentstroke}%
\pgfsetdash{}{0pt}%
\pgfsys@defobject{currentmarker}{\pgfqpoint{0.000000in}{-0.048611in}}{\pgfqpoint{0.000000in}{0.000000in}}{%
\pgfpathmoveto{\pgfqpoint{0.000000in}{0.000000in}}%
\pgfpathlineto{\pgfqpoint{0.000000in}{-0.048611in}}%
\pgfusepath{stroke,fill}%
}%
\begin{pgfscope}%
\pgfsys@transformshift{5.046364in}{0.582778in}%
\pgfsys@useobject{currentmarker}{}%
\end{pgfscope}%
\end{pgfscope}%
\begin{pgfscope}%
\pgftext[x=5.046364in,y=0.485556in,,top]{\sffamily\fontsize{10.000000}{12.000000}\selectfont \(\displaystyle 80\)}%
\end{pgfscope}%
\begin{pgfscope}%
\pgfsetbuttcap%
\pgfsetroundjoin%
\definecolor{currentfill}{rgb}{0.000000,0.000000,0.000000}%
\pgfsetfillcolor{currentfill}%
\pgfsetlinewidth{0.803000pt}%
\definecolor{currentstroke}{rgb}{0.000000,0.000000,0.000000}%
\pgfsetstrokecolor{currentstroke}%
\pgfsetdash{}{0pt}%
\pgfsys@defobject{currentmarker}{\pgfqpoint{0.000000in}{-0.048611in}}{\pgfqpoint{0.000000in}{0.000000in}}{%
\pgfpathmoveto{\pgfqpoint{0.000000in}{0.000000in}}%
\pgfpathlineto{\pgfqpoint{0.000000in}{-0.048611in}}%
\pgfusepath{stroke,fill}%
}%
\begin{pgfscope}%
\pgfsys@transformshift{6.069240in}{0.582778in}%
\pgfsys@useobject{currentmarker}{}%
\end{pgfscope}%
\end{pgfscope}%
\begin{pgfscope}%
\pgftext[x=6.069240in,y=0.485556in,,top]{\sffamily\fontsize{10.000000}{12.000000}\selectfont \(\displaystyle 100\)}%
\end{pgfscope}%
\begin{pgfscope}%
\pgftext[x=3.486477in,y=0.295587in,,top]{\sffamily\fontsize{10.000000}{12.000000}\selectfont Number of Iterations}%
\end{pgfscope}%
\begin{pgfscope}%
\pgfsetbuttcap%
\pgfsetroundjoin%
\definecolor{currentfill}{rgb}{0.000000,0.000000,0.000000}%
\pgfsetfillcolor{currentfill}%
\pgfsetlinewidth{0.803000pt}%
\definecolor{currentstroke}{rgb}{0.000000,0.000000,0.000000}%
\pgfsetstrokecolor{currentstroke}%
\pgfsetdash{}{0pt}%
\pgfsys@defobject{currentmarker}{\pgfqpoint{-0.048611in}{0.000000in}}{\pgfqpoint{0.000000in}{0.000000in}}{%
\pgfpathmoveto{\pgfqpoint{0.000000in}{0.000000in}}%
\pgfpathlineto{\pgfqpoint{-0.048611in}{0.000000in}}%
\pgfusepath{stroke,fill}%
}%
\begin{pgfscope}%
\pgfsys@transformshift{0.757955in}{0.882824in}%
\pgfsys@useobject{currentmarker}{}%
\end{pgfscope}%
\end{pgfscope}%
\begin{pgfscope}%
\pgftext[x=0.274929in,y=0.830063in,left,base]{\sffamily\fontsize{10.000000}{12.000000}\selectfont \(\displaystyle 0.0100\)}%
\end{pgfscope}%
\begin{pgfscope}%
\pgfsetbuttcap%
\pgfsetroundjoin%
\definecolor{currentfill}{rgb}{0.000000,0.000000,0.000000}%
\pgfsetfillcolor{currentfill}%
\pgfsetlinewidth{0.803000pt}%
\definecolor{currentstroke}{rgb}{0.000000,0.000000,0.000000}%
\pgfsetstrokecolor{currentstroke}%
\pgfsetdash{}{0pt}%
\pgfsys@defobject{currentmarker}{\pgfqpoint{-0.048611in}{0.000000in}}{\pgfqpoint{0.000000in}{0.000000in}}{%
\pgfpathmoveto{\pgfqpoint{0.000000in}{0.000000in}}%
\pgfpathlineto{\pgfqpoint{-0.048611in}{0.000000in}}%
\pgfusepath{stroke,fill}%
}%
\begin{pgfscope}%
\pgfsys@transformshift{0.757955in}{1.201306in}%
\pgfsys@useobject{currentmarker}{}%
\end{pgfscope}%
\end{pgfscope}%
\begin{pgfscope}%
\pgftext[x=0.274929in,y=1.148544in,left,base]{\sffamily\fontsize{10.000000}{12.000000}\selectfont \(\displaystyle 0.0125\)}%
\end{pgfscope}%
\begin{pgfscope}%
\pgfsetbuttcap%
\pgfsetroundjoin%
\definecolor{currentfill}{rgb}{0.000000,0.000000,0.000000}%
\pgfsetfillcolor{currentfill}%
\pgfsetlinewidth{0.803000pt}%
\definecolor{currentstroke}{rgb}{0.000000,0.000000,0.000000}%
\pgfsetstrokecolor{currentstroke}%
\pgfsetdash{}{0pt}%
\pgfsys@defobject{currentmarker}{\pgfqpoint{-0.048611in}{0.000000in}}{\pgfqpoint{0.000000in}{0.000000in}}{%
\pgfpathmoveto{\pgfqpoint{0.000000in}{0.000000in}}%
\pgfpathlineto{\pgfqpoint{-0.048611in}{0.000000in}}%
\pgfusepath{stroke,fill}%
}%
\begin{pgfscope}%
\pgfsys@transformshift{0.757955in}{1.519787in}%
\pgfsys@useobject{currentmarker}{}%
\end{pgfscope}%
\end{pgfscope}%
\begin{pgfscope}%
\pgftext[x=0.274929in,y=1.467025in,left,base]{\sffamily\fontsize{10.000000}{12.000000}\selectfont \(\displaystyle 0.0150\)}%
\end{pgfscope}%
\begin{pgfscope}%
\pgfsetbuttcap%
\pgfsetroundjoin%
\definecolor{currentfill}{rgb}{0.000000,0.000000,0.000000}%
\pgfsetfillcolor{currentfill}%
\pgfsetlinewidth{0.803000pt}%
\definecolor{currentstroke}{rgb}{0.000000,0.000000,0.000000}%
\pgfsetstrokecolor{currentstroke}%
\pgfsetdash{}{0pt}%
\pgfsys@defobject{currentmarker}{\pgfqpoint{-0.048611in}{0.000000in}}{\pgfqpoint{0.000000in}{0.000000in}}{%
\pgfpathmoveto{\pgfqpoint{0.000000in}{0.000000in}}%
\pgfpathlineto{\pgfqpoint{-0.048611in}{0.000000in}}%
\pgfusepath{stroke,fill}%
}%
\begin{pgfscope}%
\pgfsys@transformshift{0.757955in}{1.838268in}%
\pgfsys@useobject{currentmarker}{}%
\end{pgfscope}%
\end{pgfscope}%
\begin{pgfscope}%
\pgftext[x=0.274929in,y=1.785507in,left,base]{\sffamily\fontsize{10.000000}{12.000000}\selectfont \(\displaystyle 0.0175\)}%
\end{pgfscope}%
\begin{pgfscope}%
\pgftext[x=0.219373in,y=1.342222in,,bottom,rotate=90.000000]{\sffamily\fontsize{10.000000}{12.000000}\selectfont Proportional Contribution}%
\end{pgfscope}%
\begin{pgfscope}%
\pgfpathrectangle{\pgfqpoint{0.757955in}{0.582778in}}{\pgfqpoint{5.457045in}{1.518889in}}%
\pgfusepath{clip}%
\pgfsetrectcap%
\pgfsetroundjoin%
\pgfsetlinewidth{1.505625pt}%
\definecolor{currentstroke}{rgb}{0.000000,0.000000,0.000000}%
\pgfsetstrokecolor{currentstroke}%
\pgfsetdash{}{0pt}%
\pgfpathmoveto{\pgfqpoint{1.006002in}{1.760375in}}%
\pgfpathlineto{\pgfqpoint{1.057146in}{1.675347in}}%
\pgfpathlineto{\pgfqpoint{1.108290in}{1.938185in}}%
\pgfpathlineto{\pgfqpoint{1.159434in}{1.925955in}}%
\pgfpathlineto{\pgfqpoint{1.210578in}{1.970820in}}%
\pgfpathlineto{\pgfqpoint{1.261721in}{1.933677in}}%
\pgfpathlineto{\pgfqpoint{1.312865in}{1.603267in}}%
\pgfpathlineto{\pgfqpoint{1.364009in}{1.682573in}}%
\pgfpathlineto{\pgfqpoint{1.415153in}{1.898027in}}%
\pgfpathlineto{\pgfqpoint{1.466297in}{1.552562in}}%
\pgfpathlineto{\pgfqpoint{1.517441in}{1.587881in}}%
\pgfpathlineto{\pgfqpoint{1.568584in}{1.293212in}}%
\pgfpathlineto{\pgfqpoint{1.619728in}{1.589652in}}%
\pgfpathlineto{\pgfqpoint{1.670872in}{1.339660in}}%
\pgfpathlineto{\pgfqpoint{1.722016in}{1.300377in}}%
\pgfpathlineto{\pgfqpoint{1.773160in}{1.463104in}}%
\pgfpathlineto{\pgfqpoint{1.824303in}{1.479744in}}%
\pgfpathlineto{\pgfqpoint{1.875447in}{1.194460in}}%
\pgfpathlineto{\pgfqpoint{1.926591in}{0.964794in}}%
\pgfpathlineto{\pgfqpoint{1.977735in}{1.087690in}}%
\pgfpathlineto{\pgfqpoint{2.028879in}{1.069664in}}%
\pgfpathlineto{\pgfqpoint{2.080022in}{1.191390in}}%
\pgfpathlineto{\pgfqpoint{2.131166in}{0.965705in}}%
\pgfpathlineto{\pgfqpoint{2.182310in}{1.062478in}}%
\pgfpathlineto{\pgfqpoint{2.233454in}{1.218632in}}%
\pgfpathlineto{\pgfqpoint{2.284598in}{1.165330in}}%
\pgfpathlineto{\pgfqpoint{2.335742in}{1.153194in}}%
\pgfpathlineto{\pgfqpoint{2.386885in}{0.949901in}}%
\pgfpathlineto{\pgfqpoint{2.438029in}{0.855700in}}%
\pgfpathlineto{\pgfqpoint{2.489173in}{1.030176in}}%
\pgfpathlineto{\pgfqpoint{2.540317in}{0.991975in}}%
\pgfpathlineto{\pgfqpoint{2.591461in}{1.038215in}}%
\pgfpathlineto{\pgfqpoint{2.642604in}{1.100960in}}%
\pgfpathlineto{\pgfqpoint{2.693748in}{1.079201in}}%
\pgfpathlineto{\pgfqpoint{2.744892in}{1.197647in}}%
\pgfpathlineto{\pgfqpoint{2.796036in}{1.236947in}}%
\pgfpathlineto{\pgfqpoint{2.847180in}{1.395943in}}%
\pgfpathlineto{\pgfqpoint{2.898324in}{1.079476in}}%
\pgfpathlineto{\pgfqpoint{2.949467in}{1.241426in}}%
\pgfpathlineto{\pgfqpoint{3.000611in}{1.169780in}}%
\pgfpathlineto{\pgfqpoint{3.051755in}{1.358189in}}%
\pgfpathlineto{\pgfqpoint{3.102899in}{0.882646in}}%
\pgfpathlineto{\pgfqpoint{3.154043in}{1.022409in}}%
\pgfpathlineto{\pgfqpoint{3.205186in}{1.342887in}}%
\pgfpathlineto{\pgfqpoint{3.256330in}{1.334063in}}%
\pgfpathlineto{\pgfqpoint{3.307474in}{1.299542in}}%
\pgfpathlineto{\pgfqpoint{3.358618in}{1.216857in}}%
\pgfpathlineto{\pgfqpoint{3.409762in}{1.275540in}}%
\pgfpathlineto{\pgfqpoint{3.460906in}{1.215754in}}%
\pgfpathlineto{\pgfqpoint{3.512049in}{1.096392in}}%
\pgfpathlineto{\pgfqpoint{3.563193in}{1.368169in}}%
\pgfpathlineto{\pgfqpoint{3.614337in}{1.549378in}}%
\pgfpathlineto{\pgfqpoint{3.665481in}{1.136314in}}%
\pgfpathlineto{\pgfqpoint{3.716625in}{1.293296in}}%
\pgfpathlineto{\pgfqpoint{3.767768in}{1.466547in}}%
\pgfpathlineto{\pgfqpoint{3.818912in}{1.712943in}}%
\pgfpathlineto{\pgfqpoint{3.870056in}{1.607205in}}%
\pgfpathlineto{\pgfqpoint{3.921200in}{1.148969in}}%
\pgfpathlineto{\pgfqpoint{3.972344in}{1.233456in}}%
\pgfpathlineto{\pgfqpoint{4.023487in}{1.350067in}}%
\pgfpathlineto{\pgfqpoint{4.074631in}{1.020267in}}%
\pgfpathlineto{\pgfqpoint{4.125775in}{1.620306in}}%
\pgfpathlineto{\pgfqpoint{4.176919in}{1.171116in}}%
\pgfpathlineto{\pgfqpoint{4.228063in}{1.329887in}}%
\pgfpathlineto{\pgfqpoint{4.279207in}{1.409648in}}%
\pgfpathlineto{\pgfqpoint{4.330350in}{1.515090in}}%
\pgfpathlineto{\pgfqpoint{4.381494in}{1.541608in}}%
\pgfpathlineto{\pgfqpoint{4.432638in}{1.311155in}}%
\pgfpathlineto{\pgfqpoint{4.483782in}{1.597137in}}%
\pgfpathlineto{\pgfqpoint{4.534926in}{1.301767in}}%
\pgfpathlineto{\pgfqpoint{4.586069in}{1.101790in}}%
\pgfpathlineto{\pgfqpoint{4.637213in}{1.165970in}}%
\pgfpathlineto{\pgfqpoint{4.688357in}{0.998720in}}%
\pgfpathlineto{\pgfqpoint{4.739501in}{1.412592in}}%
\pgfpathlineto{\pgfqpoint{4.790645in}{1.709324in}}%
\pgfpathlineto{\pgfqpoint{4.841789in}{1.493733in}}%
\pgfpathlineto{\pgfqpoint{4.892932in}{1.459153in}}%
\pgfpathlineto{\pgfqpoint{4.944076in}{1.425027in}}%
\pgfpathlineto{\pgfqpoint{4.995220in}{1.526540in}}%
\pgfpathlineto{\pgfqpoint{5.046364in}{1.582330in}}%
\pgfpathlineto{\pgfqpoint{5.097508in}{1.626897in}}%
\pgfpathlineto{\pgfqpoint{5.148651in}{1.691814in}}%
\pgfpathlineto{\pgfqpoint{5.199795in}{1.199505in}}%
\pgfpathlineto{\pgfqpoint{5.250939in}{1.706294in}}%
\pgfpathlineto{\pgfqpoint{5.302083in}{1.183005in}}%
\pgfpathlineto{\pgfqpoint{5.353227in}{1.440208in}}%
\pgfpathlineto{\pgfqpoint{5.404371in}{1.963037in}}%
\pgfpathlineto{\pgfqpoint{5.455514in}{1.383085in}}%
\pgfpathlineto{\pgfqpoint{5.506658in}{1.538933in}}%
\pgfpathlineto{\pgfqpoint{5.557802in}{1.340723in}}%
\pgfpathlineto{\pgfqpoint{5.608946in}{1.375241in}}%
\pgfpathlineto{\pgfqpoint{5.660090in}{1.242426in}}%
\pgfpathlineto{\pgfqpoint{5.711233in}{1.690219in}}%
\pgfpathlineto{\pgfqpoint{5.762377in}{1.577837in}}%
\pgfpathlineto{\pgfqpoint{5.813521in}{1.202959in}}%
\pgfpathlineto{\pgfqpoint{5.864665in}{1.582879in}}%
\pgfpathlineto{\pgfqpoint{5.915809in}{1.375353in}}%
\pgfpathlineto{\pgfqpoint{5.966952in}{1.318318in}}%
\pgfusepath{stroke}%
\end{pgfscope}%
\begin{pgfscope}%
\pgfpathrectangle{\pgfqpoint{0.757955in}{0.582778in}}{\pgfqpoint{5.457045in}{1.518889in}}%
\pgfusepath{clip}%
\pgfsetrectcap%
\pgfsetroundjoin%
\pgfsetlinewidth{1.505625pt}%
\definecolor{currentstroke}{rgb}{0.500000,0.500000,0.500000}%
\pgfsetstrokecolor{currentstroke}%
\pgfsetdash{}{0pt}%
\pgfpathmoveto{\pgfqpoint{1.006002in}{1.504857in}}%
\pgfpathlineto{\pgfqpoint{1.057146in}{1.535953in}}%
\pgfpathlineto{\pgfqpoint{1.108290in}{1.733138in}}%
\pgfpathlineto{\pgfqpoint{1.159434in}{1.675549in}}%
\pgfpathlineto{\pgfqpoint{1.210578in}{1.791950in}}%
\pgfpathlineto{\pgfqpoint{1.261721in}{1.649606in}}%
\pgfpathlineto{\pgfqpoint{1.312865in}{1.444980in}}%
\pgfpathlineto{\pgfqpoint{1.364009in}{1.346534in}}%
\pgfpathlineto{\pgfqpoint{1.415153in}{1.587914in}}%
\pgfpathlineto{\pgfqpoint{1.466297in}{1.259749in}}%
\pgfpathlineto{\pgfqpoint{1.517441in}{1.402744in}}%
\pgfpathlineto{\pgfqpoint{1.568584in}{1.041779in}}%
\pgfpathlineto{\pgfqpoint{1.619728in}{1.250100in}}%
\pgfpathlineto{\pgfqpoint{1.670872in}{1.072612in}}%
\pgfpathlineto{\pgfqpoint{1.722016in}{0.926059in}}%
\pgfpathlineto{\pgfqpoint{1.773160in}{0.949942in}}%
\pgfpathlineto{\pgfqpoint{1.824303in}{1.054508in}}%
\pgfpathlineto{\pgfqpoint{1.875447in}{0.792545in}}%
\pgfpathlineto{\pgfqpoint{1.926591in}{0.864076in}}%
\pgfpathlineto{\pgfqpoint{1.977735in}{0.860517in}}%
\pgfpathlineto{\pgfqpoint{2.028879in}{0.776481in}}%
\pgfpathlineto{\pgfqpoint{2.080022in}{0.929835in}}%
\pgfpathlineto{\pgfqpoint{2.131166in}{0.694649in}}%
\pgfpathlineto{\pgfqpoint{2.182310in}{0.728082in}}%
\pgfpathlineto{\pgfqpoint{2.233454in}{0.763200in}}%
\pgfpathlineto{\pgfqpoint{2.284598in}{0.905053in}}%
\pgfpathlineto{\pgfqpoint{2.335742in}{0.735386in}}%
\pgfpathlineto{\pgfqpoint{2.386885in}{0.720319in}}%
\pgfpathlineto{\pgfqpoint{2.438029in}{0.651818in}}%
\pgfpathlineto{\pgfqpoint{2.489173in}{0.684099in}}%
\pgfpathlineto{\pgfqpoint{2.540317in}{0.796759in}}%
\pgfpathlineto{\pgfqpoint{2.591461in}{0.785656in}}%
\pgfpathlineto{\pgfqpoint{2.642604in}{0.735465in}}%
\pgfpathlineto{\pgfqpoint{2.693748in}{0.817636in}}%
\pgfpathlineto{\pgfqpoint{2.744892in}{0.791158in}}%
\pgfpathlineto{\pgfqpoint{2.796036in}{0.804831in}}%
\pgfpathlineto{\pgfqpoint{2.847180in}{0.948555in}}%
\pgfpathlineto{\pgfqpoint{2.898324in}{0.750221in}}%
\pgfpathlineto{\pgfqpoint{2.949467in}{0.854865in}}%
\pgfpathlineto{\pgfqpoint{3.000611in}{0.898986in}}%
\pgfpathlineto{\pgfqpoint{3.051755in}{0.940169in}}%
\pgfpathlineto{\pgfqpoint{3.102899in}{0.795626in}}%
\pgfpathlineto{\pgfqpoint{3.154043in}{0.822741in}}%
\pgfpathlineto{\pgfqpoint{3.205186in}{1.005777in}}%
\pgfpathlineto{\pgfqpoint{3.256330in}{0.951053in}}%
\pgfpathlineto{\pgfqpoint{3.307474in}{0.994633in}}%
\pgfpathlineto{\pgfqpoint{3.358618in}{0.773438in}}%
\pgfpathlineto{\pgfqpoint{3.409762in}{0.945237in}}%
\pgfpathlineto{\pgfqpoint{3.460906in}{0.856508in}}%
\pgfpathlineto{\pgfqpoint{3.512049in}{0.741190in}}%
\pgfpathlineto{\pgfqpoint{3.563193in}{0.996820in}}%
\pgfpathlineto{\pgfqpoint{3.614337in}{1.040043in}}%
\pgfpathlineto{\pgfqpoint{3.665481in}{0.754589in}}%
\pgfpathlineto{\pgfqpoint{3.716625in}{1.139650in}}%
\pgfpathlineto{\pgfqpoint{3.767768in}{1.133591in}}%
\pgfpathlineto{\pgfqpoint{3.818912in}{1.361199in}}%
\pgfpathlineto{\pgfqpoint{3.870056in}{1.221854in}}%
\pgfpathlineto{\pgfqpoint{3.921200in}{0.812589in}}%
\pgfpathlineto{\pgfqpoint{3.972344in}{0.914803in}}%
\pgfpathlineto{\pgfqpoint{4.023487in}{1.033064in}}%
\pgfpathlineto{\pgfqpoint{4.074631in}{0.718955in}}%
\pgfpathlineto{\pgfqpoint{4.125775in}{1.121442in}}%
\pgfpathlineto{\pgfqpoint{4.176919in}{0.774122in}}%
\pgfpathlineto{\pgfqpoint{4.228063in}{0.921685in}}%
\pgfpathlineto{\pgfqpoint{4.279207in}{0.888103in}}%
\pgfpathlineto{\pgfqpoint{4.330350in}{0.948656in}}%
\pgfpathlineto{\pgfqpoint{4.381494in}{0.969320in}}%
\pgfpathlineto{\pgfqpoint{4.432638in}{0.953956in}}%
\pgfpathlineto{\pgfqpoint{4.483782in}{1.148395in}}%
\pgfpathlineto{\pgfqpoint{4.534926in}{0.936254in}}%
\pgfpathlineto{\pgfqpoint{4.586069in}{0.761033in}}%
\pgfpathlineto{\pgfqpoint{4.637213in}{0.896048in}}%
\pgfpathlineto{\pgfqpoint{4.688357in}{0.712268in}}%
\pgfpathlineto{\pgfqpoint{4.739501in}{1.012882in}}%
\pgfpathlineto{\pgfqpoint{4.790645in}{1.014686in}}%
\pgfpathlineto{\pgfqpoint{4.841789in}{0.961787in}}%
\pgfpathlineto{\pgfqpoint{4.892932in}{1.072301in}}%
\pgfpathlineto{\pgfqpoint{4.944076in}{1.015527in}}%
\pgfpathlineto{\pgfqpoint{4.995220in}{1.116549in}}%
\pgfpathlineto{\pgfqpoint{5.046364in}{1.269258in}}%
\pgfpathlineto{\pgfqpoint{5.097508in}{1.192261in}}%
\pgfpathlineto{\pgfqpoint{5.148651in}{1.095468in}}%
\pgfpathlineto{\pgfqpoint{5.199795in}{0.925056in}}%
\pgfpathlineto{\pgfqpoint{5.250939in}{1.318335in}}%
\pgfpathlineto{\pgfqpoint{5.302083in}{0.967654in}}%
\pgfpathlineto{\pgfqpoint{5.353227in}{1.097336in}}%
\pgfpathlineto{\pgfqpoint{5.404371in}{1.344134in}}%
\pgfpathlineto{\pgfqpoint{5.455514in}{1.010163in}}%
\pgfpathlineto{\pgfqpoint{5.506658in}{1.127962in}}%
\pgfpathlineto{\pgfqpoint{5.557802in}{1.005478in}}%
\pgfpathlineto{\pgfqpoint{5.608946in}{1.077730in}}%
\pgfpathlineto{\pgfqpoint{5.660090in}{0.974868in}}%
\pgfpathlineto{\pgfqpoint{5.711233in}{1.228951in}}%
\pgfpathlineto{\pgfqpoint{5.762377in}{1.046477in}}%
\pgfpathlineto{\pgfqpoint{5.813521in}{0.963488in}}%
\pgfpathlineto{\pgfqpoint{5.864665in}{1.178945in}}%
\pgfpathlineto{\pgfqpoint{5.915809in}{1.011872in}}%
\pgfpathlineto{\pgfqpoint{5.966952in}{1.040250in}}%
\pgfusepath{stroke}%
\end{pgfscope}%
\begin{pgfscope}%
\pgfpathrectangle{\pgfqpoint{0.757955in}{0.582778in}}{\pgfqpoint{5.457045in}{1.518889in}}%
\pgfusepath{clip}%
\pgfsetbuttcap%
\pgfsetroundjoin%
\pgfsetlinewidth{1.505625pt}%
\definecolor{currentstroke}{rgb}{0.000000,0.000000,0.000000}%
\pgfsetstrokecolor{currentstroke}%
\pgfsetdash{{5.550000pt}{2.400000pt}}{0.000000pt}%
\pgfpathmoveto{\pgfqpoint{1.006002in}{1.574241in}}%
\pgfpathlineto{\pgfqpoint{1.057146in}{1.682659in}}%
\pgfpathlineto{\pgfqpoint{1.108290in}{1.762739in}}%
\pgfpathlineto{\pgfqpoint{1.159434in}{1.768616in}}%
\pgfpathlineto{\pgfqpoint{1.210578in}{1.685964in}}%
\pgfpathlineto{\pgfqpoint{1.261721in}{1.722014in}}%
\pgfpathlineto{\pgfqpoint{1.312865in}{1.923697in}}%
\pgfpathlineto{\pgfqpoint{1.364009in}{1.484887in}}%
\pgfpathlineto{\pgfqpoint{1.415153in}{1.229657in}}%
\pgfpathlineto{\pgfqpoint{1.466297in}{1.430317in}}%
\pgfpathlineto{\pgfqpoint{1.517441in}{1.499172in}}%
\pgfpathlineto{\pgfqpoint{1.568584in}{1.231368in}}%
\pgfpathlineto{\pgfqpoint{1.619728in}{1.392835in}}%
\pgfpathlineto{\pgfqpoint{1.670872in}{1.309231in}}%
\pgfpathlineto{\pgfqpoint{1.722016in}{1.153971in}}%
\pgfpathlineto{\pgfqpoint{1.773160in}{1.277411in}}%
\pgfpathlineto{\pgfqpoint{1.824303in}{0.966670in}}%
\pgfpathlineto{\pgfqpoint{1.875447in}{1.217579in}}%
\pgfpathlineto{\pgfqpoint{1.926591in}{1.198649in}}%
\pgfpathlineto{\pgfqpoint{1.977735in}{1.106604in}}%
\pgfpathlineto{\pgfqpoint{2.028879in}{0.993630in}}%
\pgfpathlineto{\pgfqpoint{2.080022in}{0.871765in}}%
\pgfpathlineto{\pgfqpoint{2.131166in}{0.864108in}}%
\pgfpathlineto{\pgfqpoint{2.182310in}{0.959137in}}%
\pgfpathlineto{\pgfqpoint{2.233454in}{1.002325in}}%
\pgfpathlineto{\pgfqpoint{2.284598in}{1.284045in}}%
\pgfpathlineto{\pgfqpoint{2.335742in}{1.001368in}}%
\pgfpathlineto{\pgfqpoint{2.386885in}{0.873856in}}%
\pgfpathlineto{\pgfqpoint{2.438029in}{1.091416in}}%
\pgfpathlineto{\pgfqpoint{2.489173in}{0.740474in}}%
\pgfpathlineto{\pgfqpoint{2.540317in}{1.026319in}}%
\pgfpathlineto{\pgfqpoint{2.591461in}{1.084093in}}%
\pgfpathlineto{\pgfqpoint{2.642604in}{1.232922in}}%
\pgfpathlineto{\pgfqpoint{2.693748in}{1.424442in}}%
\pgfpathlineto{\pgfqpoint{2.744892in}{1.003459in}}%
\pgfpathlineto{\pgfqpoint{2.796036in}{1.039433in}}%
\pgfpathlineto{\pgfqpoint{2.847180in}{0.995820in}}%
\pgfpathlineto{\pgfqpoint{2.898324in}{0.902507in}}%
\pgfpathlineto{\pgfqpoint{2.949467in}{1.309173in}}%
\pgfpathlineto{\pgfqpoint{3.000611in}{1.042419in}}%
\pgfpathlineto{\pgfqpoint{3.051755in}{1.197576in}}%
\pgfpathlineto{\pgfqpoint{3.102899in}{1.233016in}}%
\pgfpathlineto{\pgfqpoint{3.154043in}{1.204946in}}%
\pgfpathlineto{\pgfqpoint{3.205186in}{1.108364in}}%
\pgfpathlineto{\pgfqpoint{3.256330in}{1.191370in}}%
\pgfpathlineto{\pgfqpoint{3.307474in}{0.900282in}}%
\pgfpathlineto{\pgfqpoint{3.358618in}{0.984359in}}%
\pgfpathlineto{\pgfqpoint{3.409762in}{0.804148in}}%
\pgfpathlineto{\pgfqpoint{3.460906in}{1.261855in}}%
\pgfpathlineto{\pgfqpoint{3.512049in}{1.111434in}}%
\pgfpathlineto{\pgfqpoint{3.563193in}{1.269623in}}%
\pgfpathlineto{\pgfqpoint{3.614337in}{1.313040in}}%
\pgfpathlineto{\pgfqpoint{3.665481in}{0.837049in}}%
\pgfpathlineto{\pgfqpoint{3.716625in}{1.295546in}}%
\pgfpathlineto{\pgfqpoint{3.767768in}{1.382984in}}%
\pgfpathlineto{\pgfqpoint{3.818912in}{1.394905in}}%
\pgfpathlineto{\pgfqpoint{3.870056in}{1.165090in}}%
\pgfpathlineto{\pgfqpoint{3.921200in}{1.350589in}}%
\pgfpathlineto{\pgfqpoint{3.972344in}{1.103896in}}%
\pgfpathlineto{\pgfqpoint{4.023487in}{1.392435in}}%
\pgfpathlineto{\pgfqpoint{4.074631in}{1.267329in}}%
\pgfpathlineto{\pgfqpoint{4.125775in}{1.456691in}}%
\pgfpathlineto{\pgfqpoint{4.176919in}{1.122649in}}%
\pgfpathlineto{\pgfqpoint{4.228063in}{1.448432in}}%
\pgfpathlineto{\pgfqpoint{4.279207in}{1.076968in}}%
\pgfpathlineto{\pgfqpoint{4.330350in}{1.377175in}}%
\pgfpathlineto{\pgfqpoint{4.381494in}{1.426865in}}%
\pgfpathlineto{\pgfqpoint{4.432638in}{1.407335in}}%
\pgfpathlineto{\pgfqpoint{4.483782in}{1.494652in}}%
\pgfpathlineto{\pgfqpoint{4.534926in}{1.043839in}}%
\pgfpathlineto{\pgfqpoint{4.586069in}{1.179793in}}%
\pgfpathlineto{\pgfqpoint{4.637213in}{0.970653in}}%
\pgfpathlineto{\pgfqpoint{4.688357in}{1.602138in}}%
\pgfpathlineto{\pgfqpoint{4.739501in}{1.368212in}}%
\pgfpathlineto{\pgfqpoint{4.790645in}{1.632992in}}%
\pgfpathlineto{\pgfqpoint{4.841789in}{1.530843in}}%
\pgfpathlineto{\pgfqpoint{4.892932in}{1.201059in}}%
\pgfpathlineto{\pgfqpoint{4.944076in}{1.793350in}}%
\pgfpathlineto{\pgfqpoint{4.995220in}{1.285915in}}%
\pgfpathlineto{\pgfqpoint{5.046364in}{1.156239in}}%
\pgfpathlineto{\pgfqpoint{5.097508in}{1.295363in}}%
\pgfpathlineto{\pgfqpoint{5.148651in}{1.278770in}}%
\pgfpathlineto{\pgfqpoint{5.199795in}{1.529251in}}%
\pgfpathlineto{\pgfqpoint{5.250939in}{1.461634in}}%
\pgfpathlineto{\pgfqpoint{5.302083in}{1.503867in}}%
\pgfpathlineto{\pgfqpoint{5.353227in}{1.729316in}}%
\pgfpathlineto{\pgfqpoint{5.404371in}{1.402068in}}%
\pgfpathlineto{\pgfqpoint{5.455514in}{1.672907in}}%
\pgfpathlineto{\pgfqpoint{5.506658in}{1.388519in}}%
\pgfpathlineto{\pgfqpoint{5.557802in}{1.202830in}}%
\pgfpathlineto{\pgfqpoint{5.608946in}{1.530567in}}%
\pgfpathlineto{\pgfqpoint{5.660090in}{1.306768in}}%
\pgfpathlineto{\pgfqpoint{5.711233in}{1.470169in}}%
\pgfpathlineto{\pgfqpoint{5.762377in}{1.423908in}}%
\pgfpathlineto{\pgfqpoint{5.813521in}{1.214896in}}%
\pgfpathlineto{\pgfqpoint{5.864665in}{1.507646in}}%
\pgfpathlineto{\pgfqpoint{5.915809in}{2.032626in}}%
\pgfpathlineto{\pgfqpoint{5.966952in}{1.185941in}}%
\pgfusepath{stroke}%
\end{pgfscope}%
\begin{pgfscope}%
\pgfpathrectangle{\pgfqpoint{0.757955in}{0.582778in}}{\pgfqpoint{5.457045in}{1.518889in}}%
\pgfusepath{clip}%
\pgfsetbuttcap%
\pgfsetroundjoin%
\pgfsetlinewidth{1.505625pt}%
\definecolor{currentstroke}{rgb}{0.500000,0.500000,0.500000}%
\pgfsetstrokecolor{currentstroke}%
\pgfsetdash{{5.550000pt}{2.400000pt}}{0.000000pt}%
\pgfpathmoveto{\pgfqpoint{1.006002in}{1.536895in}}%
\pgfpathlineto{\pgfqpoint{1.057146in}{1.616256in}}%
\pgfpathlineto{\pgfqpoint{1.108290in}{1.827775in}}%
\pgfpathlineto{\pgfqpoint{1.159434in}{1.817327in}}%
\pgfpathlineto{\pgfqpoint{1.210578in}{1.675635in}}%
\pgfpathlineto{\pgfqpoint{1.261721in}{1.699336in}}%
\pgfpathlineto{\pgfqpoint{1.312865in}{1.738461in}}%
\pgfpathlineto{\pgfqpoint{1.364009in}{1.482055in}}%
\pgfpathlineto{\pgfqpoint{1.415153in}{1.205794in}}%
\pgfpathlineto{\pgfqpoint{1.466297in}{1.417105in}}%
\pgfpathlineto{\pgfqpoint{1.517441in}{1.114906in}}%
\pgfpathlineto{\pgfqpoint{1.568584in}{1.285340in}}%
\pgfpathlineto{\pgfqpoint{1.619728in}{1.249711in}}%
\pgfpathlineto{\pgfqpoint{1.670872in}{1.270202in}}%
\pgfpathlineto{\pgfqpoint{1.722016in}{1.128994in}}%
\pgfpathlineto{\pgfqpoint{1.773160in}{0.967674in}}%
\pgfpathlineto{\pgfqpoint{1.824303in}{0.891117in}}%
\pgfpathlineto{\pgfqpoint{1.875447in}{1.055695in}}%
\pgfpathlineto{\pgfqpoint{1.926591in}{0.980466in}}%
\pgfpathlineto{\pgfqpoint{1.977735in}{0.974119in}}%
\pgfpathlineto{\pgfqpoint{2.028879in}{0.911768in}}%
\pgfpathlineto{\pgfqpoint{2.080022in}{0.815834in}}%
\pgfpathlineto{\pgfqpoint{2.131166in}{0.830430in}}%
\pgfpathlineto{\pgfqpoint{2.182310in}{0.814634in}}%
\pgfpathlineto{\pgfqpoint{2.233454in}{0.897721in}}%
\pgfpathlineto{\pgfqpoint{2.284598in}{1.086758in}}%
\pgfpathlineto{\pgfqpoint{2.335742in}{0.894955in}}%
\pgfpathlineto{\pgfqpoint{2.386885in}{0.788096in}}%
\pgfpathlineto{\pgfqpoint{2.438029in}{0.937524in}}%
\pgfpathlineto{\pgfqpoint{2.489173in}{0.740347in}}%
\pgfpathlineto{\pgfqpoint{2.540317in}{0.991026in}}%
\pgfpathlineto{\pgfqpoint{2.591461in}{0.922935in}}%
\pgfpathlineto{\pgfqpoint{2.642604in}{1.196872in}}%
\pgfpathlineto{\pgfqpoint{2.693748in}{1.189656in}}%
\pgfpathlineto{\pgfqpoint{2.744892in}{0.957543in}}%
\pgfpathlineto{\pgfqpoint{2.796036in}{0.863207in}}%
\pgfpathlineto{\pgfqpoint{2.847180in}{0.996445in}}%
\pgfpathlineto{\pgfqpoint{2.898324in}{0.719499in}}%
\pgfpathlineto{\pgfqpoint{2.949467in}{1.082815in}}%
\pgfpathlineto{\pgfqpoint{3.000611in}{0.847115in}}%
\pgfpathlineto{\pgfqpoint{3.051755in}{1.065100in}}%
\pgfpathlineto{\pgfqpoint{3.102899in}{1.051184in}}%
\pgfpathlineto{\pgfqpoint{3.154043in}{1.052910in}}%
\pgfpathlineto{\pgfqpoint{3.205186in}{1.024279in}}%
\pgfpathlineto{\pgfqpoint{3.256330in}{1.023548in}}%
\pgfpathlineto{\pgfqpoint{3.307474in}{0.917744in}}%
\pgfpathlineto{\pgfqpoint{3.358618in}{0.841220in}}%
\pgfpathlineto{\pgfqpoint{3.409762in}{0.728537in}}%
\pgfpathlineto{\pgfqpoint{3.460906in}{1.075675in}}%
\pgfpathlineto{\pgfqpoint{3.512049in}{0.986103in}}%
\pgfpathlineto{\pgfqpoint{3.563193in}{1.070127in}}%
\pgfpathlineto{\pgfqpoint{3.614337in}{0.996088in}}%
\pgfpathlineto{\pgfqpoint{3.665481in}{0.700901in}}%
\pgfpathlineto{\pgfqpoint{3.716625in}{1.144945in}}%
\pgfpathlineto{\pgfqpoint{3.767768in}{1.229108in}}%
\pgfpathlineto{\pgfqpoint{3.818912in}{1.236774in}}%
\pgfpathlineto{\pgfqpoint{3.870056in}{1.094134in}}%
\pgfpathlineto{\pgfqpoint{3.921200in}{1.231770in}}%
\pgfpathlineto{\pgfqpoint{3.972344in}{0.957338in}}%
\pgfpathlineto{\pgfqpoint{4.023487in}{1.221968in}}%
\pgfpathlineto{\pgfqpoint{4.074631in}{1.122335in}}%
\pgfpathlineto{\pgfqpoint{4.125775in}{1.053796in}}%
\pgfpathlineto{\pgfqpoint{4.176919in}{0.848654in}}%
\pgfpathlineto{\pgfqpoint{4.228063in}{1.113484in}}%
\pgfpathlineto{\pgfqpoint{4.279207in}{0.966080in}}%
\pgfpathlineto{\pgfqpoint{4.330350in}{1.146403in}}%
\pgfpathlineto{\pgfqpoint{4.381494in}{1.261521in}}%
\pgfpathlineto{\pgfqpoint{4.432638in}{1.237270in}}%
\pgfpathlineto{\pgfqpoint{4.483782in}{1.214509in}}%
\pgfpathlineto{\pgfqpoint{4.534926in}{0.866614in}}%
\pgfpathlineto{\pgfqpoint{4.586069in}{1.085222in}}%
\pgfpathlineto{\pgfqpoint{4.637213in}{0.995994in}}%
\pgfpathlineto{\pgfqpoint{4.688357in}{1.283814in}}%
\pgfpathlineto{\pgfqpoint{4.739501in}{1.157562in}}%
\pgfpathlineto{\pgfqpoint{4.790645in}{1.494254in}}%
\pgfpathlineto{\pgfqpoint{4.841789in}{1.253224in}}%
\pgfpathlineto{\pgfqpoint{4.892932in}{1.015598in}}%
\pgfpathlineto{\pgfqpoint{4.944076in}{1.344914in}}%
\pgfpathlineto{\pgfqpoint{4.995220in}{0.984225in}}%
\pgfpathlineto{\pgfqpoint{5.046364in}{0.923188in}}%
\pgfpathlineto{\pgfqpoint{5.097508in}{1.161050in}}%
\pgfpathlineto{\pgfqpoint{5.148651in}{1.053199in}}%
\pgfpathlineto{\pgfqpoint{5.199795in}{1.363125in}}%
\pgfpathlineto{\pgfqpoint{5.250939in}{1.275064in}}%
\pgfpathlineto{\pgfqpoint{5.302083in}{1.195207in}}%
\pgfpathlineto{\pgfqpoint{5.353227in}{1.421307in}}%
\pgfpathlineto{\pgfqpoint{5.404371in}{1.165502in}}%
\pgfpathlineto{\pgfqpoint{5.455514in}{1.279777in}}%
\pgfpathlineto{\pgfqpoint{5.506658in}{1.092861in}}%
\pgfpathlineto{\pgfqpoint{5.557802in}{1.062982in}}%
\pgfpathlineto{\pgfqpoint{5.608946in}{1.274034in}}%
\pgfpathlineto{\pgfqpoint{5.660090in}{1.291013in}}%
\pgfpathlineto{\pgfqpoint{5.711233in}{1.384183in}}%
\pgfpathlineto{\pgfqpoint{5.762377in}{1.161749in}}%
\pgfpathlineto{\pgfqpoint{5.813521in}{1.130275in}}%
\pgfpathlineto{\pgfqpoint{5.864665in}{1.154566in}}%
\pgfpathlineto{\pgfqpoint{5.915809in}{1.532992in}}%
\pgfpathlineto{\pgfqpoint{5.966952in}{1.071630in}}%
\pgfusepath{stroke}%
\end{pgfscope}%
\begin{pgfscope}%
\pgfsetrectcap%
\pgfsetmiterjoin%
\pgfsetlinewidth{0.803000pt}%
\definecolor{currentstroke}{rgb}{0.000000,0.000000,0.000000}%
\pgfsetstrokecolor{currentstroke}%
\pgfsetdash{}{0pt}%
\pgfpathmoveto{\pgfqpoint{0.757955in}{0.582778in}}%
\pgfpathlineto{\pgfqpoint{0.757955in}{2.101667in}}%
\pgfusepath{stroke}%
\end{pgfscope}%
\begin{pgfscope}%
\pgfsetrectcap%
\pgfsetmiterjoin%
\pgfsetlinewidth{0.803000pt}%
\definecolor{currentstroke}{rgb}{0.000000,0.000000,0.000000}%
\pgfsetstrokecolor{currentstroke}%
\pgfsetdash{}{0pt}%
\pgfpathmoveto{\pgfqpoint{6.215000in}{0.582778in}}%
\pgfpathlineto{\pgfqpoint{6.215000in}{2.101667in}}%
\pgfusepath{stroke}%
\end{pgfscope}%
\begin{pgfscope}%
\pgfsetrectcap%
\pgfsetmiterjoin%
\pgfsetlinewidth{0.803000pt}%
\definecolor{currentstroke}{rgb}{0.000000,0.000000,0.000000}%
\pgfsetstrokecolor{currentstroke}%
\pgfsetdash{}{0pt}%
\pgfpathmoveto{\pgfqpoint{0.757955in}{0.582778in}}%
\pgfpathlineto{\pgfqpoint{6.215000in}{0.582778in}}%
\pgfusepath{stroke}%
\end{pgfscope}%
\begin{pgfscope}%
\pgfsetrectcap%
\pgfsetmiterjoin%
\pgfsetlinewidth{0.803000pt}%
\definecolor{currentstroke}{rgb}{0.000000,0.000000,0.000000}%
\pgfsetstrokecolor{currentstroke}%
\pgfsetdash{}{0pt}%
\pgfpathmoveto{\pgfqpoint{0.757955in}{2.101667in}}%
\pgfpathlineto{\pgfqpoint{6.215000in}{2.101667in}}%
\pgfusepath{stroke}%
\end{pgfscope}%
\begin{pgfscope}%
\pgftext[x=3.486477in,y=2.185000in,,base]{\sffamily\fontsize{12.000000}{14.400000}\selectfont Expected Penalty Contributions}%
\end{pgfscope}%
\begin{pgfscope}%
\pgfsetbuttcap%
\pgfsetmiterjoin%
\definecolor{currentfill}{rgb}{1.000000,1.000000,1.000000}%
\pgfsetfillcolor{currentfill}%
\pgfsetfillopacity{0.800000}%
\pgfsetlinewidth{1.003750pt}%
\definecolor{currentstroke}{rgb}{0.800000,0.800000,0.800000}%
\pgfsetstrokecolor{currentstroke}%
\pgfsetstrokeopacity{0.800000}%
\pgfsetdash{}{0pt}%
\pgfpathmoveto{\pgfqpoint{0.855177in}{0.995498in}}%
\pgfpathlineto{\pgfqpoint{3.185895in}{0.995498in}}%
\pgfpathquadraticcurveto{\pgfqpoint{3.213673in}{0.995498in}}{\pgfqpoint{3.213673in}{1.023276in}}%
\pgfpathlineto{\pgfqpoint{3.213673in}{2.004444in}}%
\pgfpathquadraticcurveto{\pgfqpoint{3.213673in}{2.032222in}}{\pgfqpoint{3.185895in}{2.032222in}}%
\pgfpathlineto{\pgfqpoint{0.855177in}{2.032222in}}%
\pgfpathquadraticcurveto{\pgfqpoint{0.827399in}{2.032222in}}{\pgfqpoint{0.827399in}{2.004444in}}%
\pgfpathlineto{\pgfqpoint{0.827399in}{1.023276in}}%
\pgfpathquadraticcurveto{\pgfqpoint{0.827399in}{0.995498in}}{\pgfqpoint{0.855177in}{0.995498in}}%
\pgfpathclose%
\pgfusepath{stroke,fill}%
\end{pgfscope}%
\begin{pgfscope}%
\pgfsetrectcap%
\pgfsetroundjoin%
\pgfsetlinewidth{1.505625pt}%
\definecolor{currentstroke}{rgb}{0.000000,0.000000,0.000000}%
\pgfsetstrokecolor{currentstroke}%
\pgfsetdash{}{0pt}%
\pgfpathmoveto{\pgfqpoint{0.882955in}{1.905679in}}%
\pgfpathlineto{\pgfqpoint{1.160733in}{1.905679in}}%
\pgfusepath{stroke}%
\end{pgfscope}%
\begin{pgfscope}%
\pgftext[x=1.271844in,y=1.857068in,left,base]{\sffamily\fontsize{10.000000}{12.000000}\selectfont \(\displaystyle 1 - E[r_{t, CLIP}^+]\), control}%
\end{pgfscope}%
\begin{pgfscope}%
\pgfsetrectcap%
\pgfsetroundjoin%
\pgfsetlinewidth{1.505625pt}%
\definecolor{currentstroke}{rgb}{0.500000,0.500000,0.500000}%
\pgfsetstrokecolor{currentstroke}%
\pgfsetdash{}{0pt}%
\pgfpathmoveto{\pgfqpoint{0.882955in}{1.656915in}}%
\pgfpathlineto{\pgfqpoint{1.160733in}{1.656915in}}%
\pgfusepath{stroke}%
\end{pgfscope}%
\begin{pgfscope}%
\pgftext[x=1.271844in,y=1.608303in,left,base]{\sffamily\fontsize{10.000000}{12.000000}\selectfont \(\displaystyle E[r_{t, CLIP}^-] - 1\), control}%
\end{pgfscope}%
\begin{pgfscope}%
\pgfsetbuttcap%
\pgfsetroundjoin%
\pgfsetlinewidth{1.505625pt}%
\definecolor{currentstroke}{rgb}{0.000000,0.000000,0.000000}%
\pgfsetstrokecolor{currentstroke}%
\pgfsetdash{{5.550000pt}{2.400000pt}}{0.000000pt}%
\pgfpathmoveto{\pgfqpoint{0.882955in}{1.408150in}}%
\pgfpathlineto{\pgfqpoint{1.160733in}{1.408150in}}%
\pgfusepath{stroke}%
\end{pgfscope}%
\begin{pgfscope}%
\pgftext[x=1.271844in,y=1.359539in,left,base]{\sffamily\fontsize{10.000000}{12.000000}\selectfont \(\displaystyle 1 - E[r_{t, CLIP}^+]\), experimental}%
\end{pgfscope}%
\begin{pgfscope}%
\pgfsetbuttcap%
\pgfsetroundjoin%
\pgfsetlinewidth{1.505625pt}%
\definecolor{currentstroke}{rgb}{0.500000,0.500000,0.500000}%
\pgfsetstrokecolor{currentstroke}%
\pgfsetdash{{5.550000pt}{2.400000pt}}{0.000000pt}%
\pgfpathmoveto{\pgfqpoint{0.882955in}{1.159386in}}%
\pgfpathlineto{\pgfqpoint{1.160733in}{1.159386in}}%
\pgfusepath{stroke}%
\end{pgfscope}%
\begin{pgfscope}%
\pgftext[x=1.271844in,y=1.110775in,left,base]{\sffamily\fontsize{10.000000}{12.000000}\selectfont \(\displaystyle E[r_{t, CLIP}^-] - 1 \), experimental}%
\end{pgfscope}%
\end{pgfpicture}%
\makeatother%
\endgroup%
}\\
    \caption{Results: Walker2d-v2 environment}
    \label{fig:2}
\end{figure}
\begin{figure}
    \centering
    \scalebox{0.45}{%% Creator: Matplotlib, PGF backend
%%
%% To include the figure in your LaTeX document, write
%%   \input{<filename>.pgf}
%%
%% Make sure the required packages are loaded in your preamble
%%   \usepackage{pgf}
%%
%% Figures using additional raster images can only be included by \input if
%% they are in the same directory as the main LaTeX file. For loading figures
%% from other directories you can use the `import` package
%%   \usepackage{import}
%% and then include the figures with
%%   \import{<path to file>}{<filename>.pgf}
%%
%% Matplotlib used the following preamble
%%   \usepackage{fontspec}
%%   \setmainfont{DejaVu Serif}
%%   \setsansfont{DejaVu Sans}
%%   \setmonofont{DejaVu Sans Mono}
%%
\begingroup%
\makeatletter%
\begin{pgfpicture}%
\pgfpathrectangle{\pgfpointorigin}{\pgfqpoint{5.000000in}{10.000000in}}%
\pgfusepath{use as bounding box, clip}%
\begin{pgfscope}%
\pgfsetbuttcap%
\pgfsetmiterjoin%
\definecolor{currentfill}{rgb}{1.000000,1.000000,1.000000}%
\pgfsetfillcolor{currentfill}%
\pgfsetlinewidth{0.000000pt}%
\definecolor{currentstroke}{rgb}{1.000000,1.000000,1.000000}%
\pgfsetstrokecolor{currentstroke}%
\pgfsetdash{}{0pt}%
\pgfpathmoveto{\pgfqpoint{0.000000in}{0.000000in}}%
\pgfpathlineto{\pgfqpoint{5.000000in}{0.000000in}}%
\pgfpathlineto{\pgfqpoint{5.000000in}{10.000000in}}%
\pgfpathlineto{\pgfqpoint{0.000000in}{10.000000in}}%
\pgfpathclose%
\pgfusepath{fill}%
\end{pgfscope}%
\begin{pgfscope}%
\pgfsetbuttcap%
\pgfsetmiterjoin%
\definecolor{currentfill}{rgb}{1.000000,1.000000,1.000000}%
\pgfsetfillcolor{currentfill}%
\pgfsetlinewidth{0.000000pt}%
\definecolor{currentstroke}{rgb}{0.000000,0.000000,0.000000}%
\pgfsetstrokecolor{currentstroke}%
\pgfsetstrokeopacity{0.000000}%
\pgfsetdash{}{0pt}%
\pgfpathmoveto{\pgfqpoint{0.757955in}{7.149444in}}%
\pgfpathlineto{\pgfqpoint{4.815000in}{7.149444in}}%
\pgfpathlineto{\pgfqpoint{4.815000in}{9.626667in}}%
\pgfpathlineto{\pgfqpoint{0.757955in}{9.626667in}}%
\pgfpathclose%
\pgfusepath{fill}%
\end{pgfscope}%
\begin{pgfscope}%
\pgfsetbuttcap%
\pgfsetroundjoin%
\definecolor{currentfill}{rgb}{0.000000,0.000000,0.000000}%
\pgfsetfillcolor{currentfill}%
\pgfsetlinewidth{0.803000pt}%
\definecolor{currentstroke}{rgb}{0.000000,0.000000,0.000000}%
\pgfsetstrokecolor{currentstroke}%
\pgfsetdash{}{0pt}%
\pgfsys@defobject{currentmarker}{\pgfqpoint{0.000000in}{-0.048611in}}{\pgfqpoint{0.000000in}{0.000000in}}{%
\pgfpathmoveto{\pgfqpoint{0.000000in}{0.000000in}}%
\pgfpathlineto{\pgfqpoint{0.000000in}{-0.048611in}}%
\pgfusepath{stroke,fill}%
}%
\begin{pgfscope}%
\pgfsys@transformshift{1.181260in}{7.149444in}%
\pgfsys@useobject{currentmarker}{}%
\end{pgfscope}%
\end{pgfscope}%
\begin{pgfscope}%
\pgftext[x=1.181260in,y=7.052222in,,top]{\sffamily\fontsize{10.000000}{12.000000}\selectfont \(\displaystyle 25000\)}%
\end{pgfscope}%
\begin{pgfscope}%
\pgfsetbuttcap%
\pgfsetroundjoin%
\definecolor{currentfill}{rgb}{0.000000,0.000000,0.000000}%
\pgfsetfillcolor{currentfill}%
\pgfsetlinewidth{0.803000pt}%
\definecolor{currentstroke}{rgb}{0.000000,0.000000,0.000000}%
\pgfsetstrokecolor{currentstroke}%
\pgfsetdash{}{0pt}%
\pgfsys@defobject{currentmarker}{\pgfqpoint{0.000000in}{-0.048611in}}{\pgfqpoint{0.000000in}{0.000000in}}{%
\pgfpathmoveto{\pgfqpoint{0.000000in}{0.000000in}}%
\pgfpathlineto{\pgfqpoint{0.000000in}{-0.048611in}}%
\pgfusepath{stroke,fill}%
}%
\begin{pgfscope}%
\pgfsys@transformshift{1.650255in}{7.149444in}%
\pgfsys@useobject{currentmarker}{}%
\end{pgfscope}%
\end{pgfscope}%
\begin{pgfscope}%
\pgftext[x=1.650255in,y=7.052222in,,top]{\sffamily\fontsize{10.000000}{12.000000}\selectfont \(\displaystyle 50000\)}%
\end{pgfscope}%
\begin{pgfscope}%
\pgfsetbuttcap%
\pgfsetroundjoin%
\definecolor{currentfill}{rgb}{0.000000,0.000000,0.000000}%
\pgfsetfillcolor{currentfill}%
\pgfsetlinewidth{0.803000pt}%
\definecolor{currentstroke}{rgb}{0.000000,0.000000,0.000000}%
\pgfsetstrokecolor{currentstroke}%
\pgfsetdash{}{0pt}%
\pgfsys@defobject{currentmarker}{\pgfqpoint{0.000000in}{-0.048611in}}{\pgfqpoint{0.000000in}{0.000000in}}{%
\pgfpathmoveto{\pgfqpoint{0.000000in}{0.000000in}}%
\pgfpathlineto{\pgfqpoint{0.000000in}{-0.048611in}}%
\pgfusepath{stroke,fill}%
}%
\begin{pgfscope}%
\pgfsys@transformshift{2.119250in}{7.149444in}%
\pgfsys@useobject{currentmarker}{}%
\end{pgfscope}%
\end{pgfscope}%
\begin{pgfscope}%
\pgftext[x=2.119250in,y=7.052222in,,top]{\sffamily\fontsize{10.000000}{12.000000}\selectfont \(\displaystyle 75000\)}%
\end{pgfscope}%
\begin{pgfscope}%
\pgfsetbuttcap%
\pgfsetroundjoin%
\definecolor{currentfill}{rgb}{0.000000,0.000000,0.000000}%
\pgfsetfillcolor{currentfill}%
\pgfsetlinewidth{0.803000pt}%
\definecolor{currentstroke}{rgb}{0.000000,0.000000,0.000000}%
\pgfsetstrokecolor{currentstroke}%
\pgfsetdash{}{0pt}%
\pgfsys@defobject{currentmarker}{\pgfqpoint{0.000000in}{-0.048611in}}{\pgfqpoint{0.000000in}{0.000000in}}{%
\pgfpathmoveto{\pgfqpoint{0.000000in}{0.000000in}}%
\pgfpathlineto{\pgfqpoint{0.000000in}{-0.048611in}}%
\pgfusepath{stroke,fill}%
}%
\begin{pgfscope}%
\pgfsys@transformshift{2.588246in}{7.149444in}%
\pgfsys@useobject{currentmarker}{}%
\end{pgfscope}%
\end{pgfscope}%
\begin{pgfscope}%
\pgftext[x=2.588246in,y=7.052222in,,top]{\sffamily\fontsize{10.000000}{12.000000}\selectfont \(\displaystyle 100000\)}%
\end{pgfscope}%
\begin{pgfscope}%
\pgfsetbuttcap%
\pgfsetroundjoin%
\definecolor{currentfill}{rgb}{0.000000,0.000000,0.000000}%
\pgfsetfillcolor{currentfill}%
\pgfsetlinewidth{0.803000pt}%
\definecolor{currentstroke}{rgb}{0.000000,0.000000,0.000000}%
\pgfsetstrokecolor{currentstroke}%
\pgfsetdash{}{0pt}%
\pgfsys@defobject{currentmarker}{\pgfqpoint{0.000000in}{-0.048611in}}{\pgfqpoint{0.000000in}{0.000000in}}{%
\pgfpathmoveto{\pgfqpoint{0.000000in}{0.000000in}}%
\pgfpathlineto{\pgfqpoint{0.000000in}{-0.048611in}}%
\pgfusepath{stroke,fill}%
}%
\begin{pgfscope}%
\pgfsys@transformshift{3.057241in}{7.149444in}%
\pgfsys@useobject{currentmarker}{}%
\end{pgfscope}%
\end{pgfscope}%
\begin{pgfscope}%
\pgftext[x=3.057241in,y=7.052222in,,top]{\sffamily\fontsize{10.000000}{12.000000}\selectfont \(\displaystyle 125000\)}%
\end{pgfscope}%
\begin{pgfscope}%
\pgfsetbuttcap%
\pgfsetroundjoin%
\definecolor{currentfill}{rgb}{0.000000,0.000000,0.000000}%
\pgfsetfillcolor{currentfill}%
\pgfsetlinewidth{0.803000pt}%
\definecolor{currentstroke}{rgb}{0.000000,0.000000,0.000000}%
\pgfsetstrokecolor{currentstroke}%
\pgfsetdash{}{0pt}%
\pgfsys@defobject{currentmarker}{\pgfqpoint{0.000000in}{-0.048611in}}{\pgfqpoint{0.000000in}{0.000000in}}{%
\pgfpathmoveto{\pgfqpoint{0.000000in}{0.000000in}}%
\pgfpathlineto{\pgfqpoint{0.000000in}{-0.048611in}}%
\pgfusepath{stroke,fill}%
}%
\begin{pgfscope}%
\pgfsys@transformshift{3.526236in}{7.149444in}%
\pgfsys@useobject{currentmarker}{}%
\end{pgfscope}%
\end{pgfscope}%
\begin{pgfscope}%
\pgftext[x=3.526236in,y=7.052222in,,top]{\sffamily\fontsize{10.000000}{12.000000}\selectfont \(\displaystyle 150000\)}%
\end{pgfscope}%
\begin{pgfscope}%
\pgfsetbuttcap%
\pgfsetroundjoin%
\definecolor{currentfill}{rgb}{0.000000,0.000000,0.000000}%
\pgfsetfillcolor{currentfill}%
\pgfsetlinewidth{0.803000pt}%
\definecolor{currentstroke}{rgb}{0.000000,0.000000,0.000000}%
\pgfsetstrokecolor{currentstroke}%
\pgfsetdash{}{0pt}%
\pgfsys@defobject{currentmarker}{\pgfqpoint{0.000000in}{-0.048611in}}{\pgfqpoint{0.000000in}{0.000000in}}{%
\pgfpathmoveto{\pgfqpoint{0.000000in}{0.000000in}}%
\pgfpathlineto{\pgfqpoint{0.000000in}{-0.048611in}}%
\pgfusepath{stroke,fill}%
}%
\begin{pgfscope}%
\pgfsys@transformshift{3.995232in}{7.149444in}%
\pgfsys@useobject{currentmarker}{}%
\end{pgfscope}%
\end{pgfscope}%
\begin{pgfscope}%
\pgftext[x=3.995232in,y=7.052222in,,top]{\sffamily\fontsize{10.000000}{12.000000}\selectfont \(\displaystyle 175000\)}%
\end{pgfscope}%
\begin{pgfscope}%
\pgfsetbuttcap%
\pgfsetroundjoin%
\definecolor{currentfill}{rgb}{0.000000,0.000000,0.000000}%
\pgfsetfillcolor{currentfill}%
\pgfsetlinewidth{0.803000pt}%
\definecolor{currentstroke}{rgb}{0.000000,0.000000,0.000000}%
\pgfsetstrokecolor{currentstroke}%
\pgfsetdash{}{0pt}%
\pgfsys@defobject{currentmarker}{\pgfqpoint{0.000000in}{-0.048611in}}{\pgfqpoint{0.000000in}{0.000000in}}{%
\pgfpathmoveto{\pgfqpoint{0.000000in}{0.000000in}}%
\pgfpathlineto{\pgfqpoint{0.000000in}{-0.048611in}}%
\pgfusepath{stroke,fill}%
}%
\begin{pgfscope}%
\pgfsys@transformshift{4.464227in}{7.149444in}%
\pgfsys@useobject{currentmarker}{}%
\end{pgfscope}%
\end{pgfscope}%
\begin{pgfscope}%
\pgftext[x=4.464227in,y=7.052222in,,top]{\sffamily\fontsize{10.000000}{12.000000}\selectfont \(\displaystyle 200000\)}%
\end{pgfscope}%
\begin{pgfscope}%
\pgftext[x=2.786477in,y=6.862254in,,top]{\sffamily\fontsize{10.000000}{12.000000}\selectfont Timestep}%
\end{pgfscope}%
\begin{pgfscope}%
\pgfsetbuttcap%
\pgfsetroundjoin%
\definecolor{currentfill}{rgb}{0.000000,0.000000,0.000000}%
\pgfsetfillcolor{currentfill}%
\pgfsetlinewidth{0.803000pt}%
\definecolor{currentstroke}{rgb}{0.000000,0.000000,0.000000}%
\pgfsetstrokecolor{currentstroke}%
\pgfsetdash{}{0pt}%
\pgfsys@defobject{currentmarker}{\pgfqpoint{-0.048611in}{0.000000in}}{\pgfqpoint{0.000000in}{0.000000in}}{%
\pgfpathmoveto{\pgfqpoint{0.000000in}{0.000000in}}%
\pgfpathlineto{\pgfqpoint{-0.048611in}{0.000000in}}%
\pgfusepath{stroke,fill}%
}%
\begin{pgfscope}%
\pgfsys@transformshift{0.757955in}{7.149444in}%
\pgfsys@useobject{currentmarker}{}%
\end{pgfscope}%
\end{pgfscope}%
\begin{pgfscope}%
\pgftext[x=0.591288in,y=7.096683in,left,base]{\sffamily\fontsize{10.000000}{12.000000}\selectfont \(\displaystyle 0\)}%
\end{pgfscope}%
\begin{pgfscope}%
\pgfsetbuttcap%
\pgfsetroundjoin%
\definecolor{currentfill}{rgb}{0.000000,0.000000,0.000000}%
\pgfsetfillcolor{currentfill}%
\pgfsetlinewidth{0.803000pt}%
\definecolor{currentstroke}{rgb}{0.000000,0.000000,0.000000}%
\pgfsetstrokecolor{currentstroke}%
\pgfsetdash{}{0pt}%
\pgfsys@defobject{currentmarker}{\pgfqpoint{-0.048611in}{0.000000in}}{\pgfqpoint{0.000000in}{0.000000in}}{%
\pgfpathmoveto{\pgfqpoint{0.000000in}{0.000000in}}%
\pgfpathlineto{\pgfqpoint{-0.048611in}{0.000000in}}%
\pgfusepath{stroke,fill}%
}%
\begin{pgfscope}%
\pgfsys@transformshift{0.757955in}{7.530556in}%
\pgfsys@useobject{currentmarker}{}%
\end{pgfscope}%
\end{pgfscope}%
\begin{pgfscope}%
\pgftext[x=0.452399in,y=7.477794in,left,base]{\sffamily\fontsize{10.000000}{12.000000}\selectfont \(\displaystyle 200\)}%
\end{pgfscope}%
\begin{pgfscope}%
\pgfsetbuttcap%
\pgfsetroundjoin%
\definecolor{currentfill}{rgb}{0.000000,0.000000,0.000000}%
\pgfsetfillcolor{currentfill}%
\pgfsetlinewidth{0.803000pt}%
\definecolor{currentstroke}{rgb}{0.000000,0.000000,0.000000}%
\pgfsetstrokecolor{currentstroke}%
\pgfsetdash{}{0pt}%
\pgfsys@defobject{currentmarker}{\pgfqpoint{-0.048611in}{0.000000in}}{\pgfqpoint{0.000000in}{0.000000in}}{%
\pgfpathmoveto{\pgfqpoint{0.000000in}{0.000000in}}%
\pgfpathlineto{\pgfqpoint{-0.048611in}{0.000000in}}%
\pgfusepath{stroke,fill}%
}%
\begin{pgfscope}%
\pgfsys@transformshift{0.757955in}{7.911667in}%
\pgfsys@useobject{currentmarker}{}%
\end{pgfscope}%
\end{pgfscope}%
\begin{pgfscope}%
\pgftext[x=0.452399in,y=7.858905in,left,base]{\sffamily\fontsize{10.000000}{12.000000}\selectfont \(\displaystyle 400\)}%
\end{pgfscope}%
\begin{pgfscope}%
\pgfsetbuttcap%
\pgfsetroundjoin%
\definecolor{currentfill}{rgb}{0.000000,0.000000,0.000000}%
\pgfsetfillcolor{currentfill}%
\pgfsetlinewidth{0.803000pt}%
\definecolor{currentstroke}{rgb}{0.000000,0.000000,0.000000}%
\pgfsetstrokecolor{currentstroke}%
\pgfsetdash{}{0pt}%
\pgfsys@defobject{currentmarker}{\pgfqpoint{-0.048611in}{0.000000in}}{\pgfqpoint{0.000000in}{0.000000in}}{%
\pgfpathmoveto{\pgfqpoint{0.000000in}{0.000000in}}%
\pgfpathlineto{\pgfqpoint{-0.048611in}{0.000000in}}%
\pgfusepath{stroke,fill}%
}%
\begin{pgfscope}%
\pgfsys@transformshift{0.757955in}{8.292778in}%
\pgfsys@useobject{currentmarker}{}%
\end{pgfscope}%
\end{pgfscope}%
\begin{pgfscope}%
\pgftext[x=0.452399in,y=8.240016in,left,base]{\sffamily\fontsize{10.000000}{12.000000}\selectfont \(\displaystyle 600\)}%
\end{pgfscope}%
\begin{pgfscope}%
\pgfsetbuttcap%
\pgfsetroundjoin%
\definecolor{currentfill}{rgb}{0.000000,0.000000,0.000000}%
\pgfsetfillcolor{currentfill}%
\pgfsetlinewidth{0.803000pt}%
\definecolor{currentstroke}{rgb}{0.000000,0.000000,0.000000}%
\pgfsetstrokecolor{currentstroke}%
\pgfsetdash{}{0pt}%
\pgfsys@defobject{currentmarker}{\pgfqpoint{-0.048611in}{0.000000in}}{\pgfqpoint{0.000000in}{0.000000in}}{%
\pgfpathmoveto{\pgfqpoint{0.000000in}{0.000000in}}%
\pgfpathlineto{\pgfqpoint{-0.048611in}{0.000000in}}%
\pgfusepath{stroke,fill}%
}%
\begin{pgfscope}%
\pgfsys@transformshift{0.757955in}{8.673889in}%
\pgfsys@useobject{currentmarker}{}%
\end{pgfscope}%
\end{pgfscope}%
\begin{pgfscope}%
\pgftext[x=0.452399in,y=8.621127in,left,base]{\sffamily\fontsize{10.000000}{12.000000}\selectfont \(\displaystyle 800\)}%
\end{pgfscope}%
\begin{pgfscope}%
\pgfsetbuttcap%
\pgfsetroundjoin%
\definecolor{currentfill}{rgb}{0.000000,0.000000,0.000000}%
\pgfsetfillcolor{currentfill}%
\pgfsetlinewidth{0.803000pt}%
\definecolor{currentstroke}{rgb}{0.000000,0.000000,0.000000}%
\pgfsetstrokecolor{currentstroke}%
\pgfsetdash{}{0pt}%
\pgfsys@defobject{currentmarker}{\pgfqpoint{-0.048611in}{0.000000in}}{\pgfqpoint{0.000000in}{0.000000in}}{%
\pgfpathmoveto{\pgfqpoint{0.000000in}{0.000000in}}%
\pgfpathlineto{\pgfqpoint{-0.048611in}{0.000000in}}%
\pgfusepath{stroke,fill}%
}%
\begin{pgfscope}%
\pgfsys@transformshift{0.757955in}{9.055000in}%
\pgfsys@useobject{currentmarker}{}%
\end{pgfscope}%
\end{pgfscope}%
\begin{pgfscope}%
\pgftext[x=0.382954in,y=9.002238in,left,base]{\sffamily\fontsize{10.000000}{12.000000}\selectfont \(\displaystyle 1000\)}%
\end{pgfscope}%
\begin{pgfscope}%
\pgfsetbuttcap%
\pgfsetroundjoin%
\definecolor{currentfill}{rgb}{0.000000,0.000000,0.000000}%
\pgfsetfillcolor{currentfill}%
\pgfsetlinewidth{0.803000pt}%
\definecolor{currentstroke}{rgb}{0.000000,0.000000,0.000000}%
\pgfsetstrokecolor{currentstroke}%
\pgfsetdash{}{0pt}%
\pgfsys@defobject{currentmarker}{\pgfqpoint{-0.048611in}{0.000000in}}{\pgfqpoint{0.000000in}{0.000000in}}{%
\pgfpathmoveto{\pgfqpoint{0.000000in}{0.000000in}}%
\pgfpathlineto{\pgfqpoint{-0.048611in}{0.000000in}}%
\pgfusepath{stroke,fill}%
}%
\begin{pgfscope}%
\pgfsys@transformshift{0.757955in}{9.436111in}%
\pgfsys@useobject{currentmarker}{}%
\end{pgfscope}%
\end{pgfscope}%
\begin{pgfscope}%
\pgftext[x=0.382954in,y=9.383350in,left,base]{\sffamily\fontsize{10.000000}{12.000000}\selectfont \(\displaystyle 1200\)}%
\end{pgfscope}%
\begin{pgfscope}%
\pgftext[x=0.327398in,y=8.388056in,,bottom,rotate=90.000000]{\sffamily\fontsize{10.000000}{12.000000}\selectfont Average Return}%
\end{pgfscope}%
\begin{pgfscope}%
\pgfpathrectangle{\pgfqpoint{0.757955in}{7.149444in}}{\pgfqpoint{4.057045in}{2.477222in}}%
\pgfusepath{clip}%
\pgfsetrectcap%
\pgfsetroundjoin%
\pgfsetlinewidth{1.505625pt}%
\definecolor{currentstroke}{rgb}{0.000000,0.000000,0.000000}%
\pgfsetstrokecolor{currentstroke}%
\pgfsetdash{}{0pt}%
\pgfpathmoveto{\pgfqpoint{0.942366in}{7.176108in}}%
\pgfpathlineto{\pgfqpoint{1.173249in}{7.183918in}}%
\pgfpathlineto{\pgfqpoint{1.403626in}{7.197621in}}%
\pgfpathlineto{\pgfqpoint{1.633846in}{7.209147in}}%
\pgfpathlineto{\pgfqpoint{1.864561in}{7.230574in}}%
\pgfpathlineto{\pgfqpoint{2.094956in}{7.281432in}}%
\pgfpathlineto{\pgfqpoint{2.325258in}{7.316843in}}%
\pgfpathlineto{\pgfqpoint{2.556273in}{7.365715in}}%
\pgfpathlineto{\pgfqpoint{2.785955in}{7.394868in}}%
\pgfpathlineto{\pgfqpoint{3.016388in}{7.439189in}}%
\pgfpathlineto{\pgfqpoint{3.246971in}{7.510949in}}%
\pgfpathlineto{\pgfqpoint{3.477354in}{7.531478in}}%
\pgfpathlineto{\pgfqpoint{3.708419in}{7.558311in}}%
\pgfpathlineto{\pgfqpoint{3.938070in}{7.585945in}}%
\pgfpathlineto{\pgfqpoint{4.169135in}{7.615549in}}%
\pgfpathlineto{\pgfqpoint{4.399693in}{7.632538in}}%
\pgfpathlineto{\pgfqpoint{4.629788in}{7.650279in}}%
\pgfusepath{stroke}%
\end{pgfscope}%
\begin{pgfscope}%
\pgfpathrectangle{\pgfqpoint{0.757955in}{7.149444in}}{\pgfqpoint{4.057045in}{2.477222in}}%
\pgfusepath{clip}%
\pgfsetbuttcap%
\pgfsetroundjoin%
\pgfsetlinewidth{1.505625pt}%
\definecolor{currentstroke}{rgb}{0.000000,0.000000,0.000000}%
\pgfsetstrokecolor{currentstroke}%
\pgfsetdash{{5.550000pt}{2.400000pt}}{0.000000pt}%
\pgfpathmoveto{\pgfqpoint{0.942366in}{7.176108in}}%
\pgfpathlineto{\pgfqpoint{1.173212in}{7.184132in}}%
\pgfpathlineto{\pgfqpoint{1.403451in}{7.198676in}}%
\pgfpathlineto{\pgfqpoint{1.633859in}{7.208201in}}%
\pgfpathlineto{\pgfqpoint{1.864561in}{7.228965in}}%
\pgfpathlineto{\pgfqpoint{2.094944in}{7.267231in}}%
\pgfpathlineto{\pgfqpoint{2.325027in}{7.318366in}}%
\pgfpathlineto{\pgfqpoint{2.556054in}{7.367852in}}%
\pgfpathlineto{\pgfqpoint{2.785799in}{7.385282in}}%
\pgfpathlineto{\pgfqpoint{3.016251in}{7.437983in}}%
\pgfpathlineto{\pgfqpoint{3.247222in}{7.488530in}}%
\pgfpathlineto{\pgfqpoint{3.478142in}{7.513354in}}%
\pgfpathlineto{\pgfqpoint{3.707875in}{7.547498in}}%
\pgfpathlineto{\pgfqpoint{3.938589in}{7.564482in}}%
\pgfpathlineto{\pgfqpoint{4.169429in}{7.590806in}}%
\pgfpathlineto{\pgfqpoint{4.399693in}{7.611783in}}%
\pgfpathlineto{\pgfqpoint{4.630589in}{7.626450in}}%
\pgfusepath{stroke}%
\end{pgfscope}%
\begin{pgfscope}%
\pgfsetrectcap%
\pgfsetmiterjoin%
\pgfsetlinewidth{0.803000pt}%
\definecolor{currentstroke}{rgb}{0.000000,0.000000,0.000000}%
\pgfsetstrokecolor{currentstroke}%
\pgfsetdash{}{0pt}%
\pgfpathmoveto{\pgfqpoint{0.757955in}{7.149444in}}%
\pgfpathlineto{\pgfqpoint{0.757955in}{9.626667in}}%
\pgfusepath{stroke}%
\end{pgfscope}%
\begin{pgfscope}%
\pgfsetrectcap%
\pgfsetmiterjoin%
\pgfsetlinewidth{0.803000pt}%
\definecolor{currentstroke}{rgb}{0.000000,0.000000,0.000000}%
\pgfsetstrokecolor{currentstroke}%
\pgfsetdash{}{0pt}%
\pgfpathmoveto{\pgfqpoint{4.815000in}{7.149444in}}%
\pgfpathlineto{\pgfqpoint{4.815000in}{9.626667in}}%
\pgfusepath{stroke}%
\end{pgfscope}%
\begin{pgfscope}%
\pgfsetrectcap%
\pgfsetmiterjoin%
\pgfsetlinewidth{0.803000pt}%
\definecolor{currentstroke}{rgb}{0.000000,0.000000,0.000000}%
\pgfsetstrokecolor{currentstroke}%
\pgfsetdash{}{0pt}%
\pgfpathmoveto{\pgfqpoint{0.757955in}{7.149444in}}%
\pgfpathlineto{\pgfqpoint{4.815000in}{7.149444in}}%
\pgfusepath{stroke}%
\end{pgfscope}%
\begin{pgfscope}%
\pgfsetrectcap%
\pgfsetmiterjoin%
\pgfsetlinewidth{0.803000pt}%
\definecolor{currentstroke}{rgb}{0.000000,0.000000,0.000000}%
\pgfsetstrokecolor{currentstroke}%
\pgfsetdash{}{0pt}%
\pgfpathmoveto{\pgfqpoint{0.757955in}{9.626667in}}%
\pgfpathlineto{\pgfqpoint{4.815000in}{9.626667in}}%
\pgfusepath{stroke}%
\end{pgfscope}%
\begin{pgfscope}%
\pgftext[x=2.786477in,y=9.710000in,,base]{\sffamily\fontsize{12.000000}{14.400000}\selectfont Performance}%
\end{pgfscope}%
\begin{pgfscope}%
\pgfsetbuttcap%
\pgfsetmiterjoin%
\definecolor{currentfill}{rgb}{1.000000,1.000000,1.000000}%
\pgfsetfillcolor{currentfill}%
\pgfsetfillopacity{0.800000}%
\pgfsetlinewidth{1.003750pt}%
\definecolor{currentstroke}{rgb}{0.800000,0.800000,0.800000}%
\pgfsetstrokecolor{currentstroke}%
\pgfsetstrokeopacity{0.800000}%
\pgfsetdash{}{0pt}%
\pgfpathmoveto{\pgfqpoint{3.351906in}{9.107841in}}%
\pgfpathlineto{\pgfqpoint{4.717778in}{9.107841in}}%
\pgfpathquadraticcurveto{\pgfqpoint{4.745556in}{9.107841in}}{\pgfqpoint{4.745556in}{9.135619in}}%
\pgfpathlineto{\pgfqpoint{4.745556in}{9.529444in}}%
\pgfpathquadraticcurveto{\pgfqpoint{4.745556in}{9.557222in}}{\pgfqpoint{4.717778in}{9.557222in}}%
\pgfpathlineto{\pgfqpoint{3.351906in}{9.557222in}}%
\pgfpathquadraticcurveto{\pgfqpoint{3.324128in}{9.557222in}}{\pgfqpoint{3.324128in}{9.529444in}}%
\pgfpathlineto{\pgfqpoint{3.324128in}{9.135619in}}%
\pgfpathquadraticcurveto{\pgfqpoint{3.324128in}{9.107841in}}{\pgfqpoint{3.351906in}{9.107841in}}%
\pgfpathclose%
\pgfusepath{stroke,fill}%
\end{pgfscope}%
\begin{pgfscope}%
\pgfsetrectcap%
\pgfsetroundjoin%
\pgfsetlinewidth{1.505625pt}%
\definecolor{currentstroke}{rgb}{0.000000,0.000000,0.000000}%
\pgfsetstrokecolor{currentstroke}%
\pgfsetdash{}{0pt}%
\pgfpathmoveto{\pgfqpoint{3.379684in}{9.444755in}}%
\pgfpathlineto{\pgfqpoint{3.657462in}{9.444755in}}%
\pgfusepath{stroke}%
\end{pgfscope}%
\begin{pgfscope}%
\pgftext[x=3.768573in,y=9.396144in,left,base]{\sffamily\fontsize{10.000000}{12.000000}\selectfont control}%
\end{pgfscope}%
\begin{pgfscope}%
\pgfsetbuttcap%
\pgfsetroundjoin%
\pgfsetlinewidth{1.505625pt}%
\definecolor{currentstroke}{rgb}{0.000000,0.000000,0.000000}%
\pgfsetstrokecolor{currentstroke}%
\pgfsetdash{{5.550000pt}{2.400000pt}}{0.000000pt}%
\pgfpathmoveto{\pgfqpoint{3.379684in}{9.240897in}}%
\pgfpathlineto{\pgfqpoint{3.657462in}{9.240897in}}%
\pgfusepath{stroke}%
\end{pgfscope}%
\begin{pgfscope}%
\pgftext[x=3.768573in,y=9.192286in,left,base]{\sffamily\fontsize{10.000000}{12.000000}\selectfont experimental}%
\end{pgfscope}%
\begin{pgfscope}%
\pgfsetbuttcap%
\pgfsetmiterjoin%
\definecolor{currentfill}{rgb}{1.000000,1.000000,1.000000}%
\pgfsetfillcolor{currentfill}%
\pgfsetlinewidth{0.000000pt}%
\definecolor{currentstroke}{rgb}{0.000000,0.000000,0.000000}%
\pgfsetstrokecolor{currentstroke}%
\pgfsetstrokeopacity{0.000000}%
\pgfsetdash{}{0pt}%
\pgfpathmoveto{\pgfqpoint{0.757955in}{3.866111in}}%
\pgfpathlineto{\pgfqpoint{4.815000in}{3.866111in}}%
\pgfpathlineto{\pgfqpoint{4.815000in}{6.343333in}}%
\pgfpathlineto{\pgfqpoint{0.757955in}{6.343333in}}%
\pgfpathclose%
\pgfusepath{fill}%
\end{pgfscope}%
\begin{pgfscope}%
\pgfsetbuttcap%
\pgfsetroundjoin%
\definecolor{currentfill}{rgb}{0.000000,0.000000,0.000000}%
\pgfsetfillcolor{currentfill}%
\pgfsetlinewidth{0.803000pt}%
\definecolor{currentstroke}{rgb}{0.000000,0.000000,0.000000}%
\pgfsetstrokecolor{currentstroke}%
\pgfsetdash{}{0pt}%
\pgfsys@defobject{currentmarker}{\pgfqpoint{0.000000in}{-0.048611in}}{\pgfqpoint{0.000000in}{0.000000in}}{%
\pgfpathmoveto{\pgfqpoint{0.000000in}{0.000000in}}%
\pgfpathlineto{\pgfqpoint{0.000000in}{-0.048611in}}%
\pgfusepath{stroke,fill}%
}%
\begin{pgfscope}%
\pgfsys@transformshift{1.172880in}{3.866111in}%
\pgfsys@useobject{currentmarker}{}%
\end{pgfscope}%
\end{pgfscope}%
\begin{pgfscope}%
\pgftext[x=1.172880in,y=3.768889in,,top]{\sffamily\fontsize{10.000000}{12.000000}\selectfont \(\displaystyle 2\)}%
\end{pgfscope}%
\begin{pgfscope}%
\pgfsetbuttcap%
\pgfsetroundjoin%
\definecolor{currentfill}{rgb}{0.000000,0.000000,0.000000}%
\pgfsetfillcolor{currentfill}%
\pgfsetlinewidth{0.803000pt}%
\definecolor{currentstroke}{rgb}{0.000000,0.000000,0.000000}%
\pgfsetstrokecolor{currentstroke}%
\pgfsetdash{}{0pt}%
\pgfsys@defobject{currentmarker}{\pgfqpoint{0.000000in}{-0.048611in}}{\pgfqpoint{0.000000in}{0.000000in}}{%
\pgfpathmoveto{\pgfqpoint{0.000000in}{0.000000in}}%
\pgfpathlineto{\pgfqpoint{0.000000in}{-0.048611in}}%
\pgfusepath{stroke,fill}%
}%
\begin{pgfscope}%
\pgfsys@transformshift{1.633908in}{3.866111in}%
\pgfsys@useobject{currentmarker}{}%
\end{pgfscope}%
\end{pgfscope}%
\begin{pgfscope}%
\pgftext[x=1.633908in,y=3.768889in,,top]{\sffamily\fontsize{10.000000}{12.000000}\selectfont \(\displaystyle 4\)}%
\end{pgfscope}%
\begin{pgfscope}%
\pgfsetbuttcap%
\pgfsetroundjoin%
\definecolor{currentfill}{rgb}{0.000000,0.000000,0.000000}%
\pgfsetfillcolor{currentfill}%
\pgfsetlinewidth{0.803000pt}%
\definecolor{currentstroke}{rgb}{0.000000,0.000000,0.000000}%
\pgfsetstrokecolor{currentstroke}%
\pgfsetdash{}{0pt}%
\pgfsys@defobject{currentmarker}{\pgfqpoint{0.000000in}{-0.048611in}}{\pgfqpoint{0.000000in}{0.000000in}}{%
\pgfpathmoveto{\pgfqpoint{0.000000in}{0.000000in}}%
\pgfpathlineto{\pgfqpoint{0.000000in}{-0.048611in}}%
\pgfusepath{stroke,fill}%
}%
\begin{pgfscope}%
\pgfsys@transformshift{2.094936in}{3.866111in}%
\pgfsys@useobject{currentmarker}{}%
\end{pgfscope}%
\end{pgfscope}%
\begin{pgfscope}%
\pgftext[x=2.094936in,y=3.768889in,,top]{\sffamily\fontsize{10.000000}{12.000000}\selectfont \(\displaystyle 6\)}%
\end{pgfscope}%
\begin{pgfscope}%
\pgfsetbuttcap%
\pgfsetroundjoin%
\definecolor{currentfill}{rgb}{0.000000,0.000000,0.000000}%
\pgfsetfillcolor{currentfill}%
\pgfsetlinewidth{0.803000pt}%
\definecolor{currentstroke}{rgb}{0.000000,0.000000,0.000000}%
\pgfsetstrokecolor{currentstroke}%
\pgfsetdash{}{0pt}%
\pgfsys@defobject{currentmarker}{\pgfqpoint{0.000000in}{-0.048611in}}{\pgfqpoint{0.000000in}{0.000000in}}{%
\pgfpathmoveto{\pgfqpoint{0.000000in}{0.000000in}}%
\pgfpathlineto{\pgfqpoint{0.000000in}{-0.048611in}}%
\pgfusepath{stroke,fill}%
}%
\begin{pgfscope}%
\pgfsys@transformshift{2.555963in}{3.866111in}%
\pgfsys@useobject{currentmarker}{}%
\end{pgfscope}%
\end{pgfscope}%
\begin{pgfscope}%
\pgftext[x=2.555963in,y=3.768889in,,top]{\sffamily\fontsize{10.000000}{12.000000}\selectfont \(\displaystyle 8\)}%
\end{pgfscope}%
\begin{pgfscope}%
\pgfsetbuttcap%
\pgfsetroundjoin%
\definecolor{currentfill}{rgb}{0.000000,0.000000,0.000000}%
\pgfsetfillcolor{currentfill}%
\pgfsetlinewidth{0.803000pt}%
\definecolor{currentstroke}{rgb}{0.000000,0.000000,0.000000}%
\pgfsetstrokecolor{currentstroke}%
\pgfsetdash{}{0pt}%
\pgfsys@defobject{currentmarker}{\pgfqpoint{0.000000in}{-0.048611in}}{\pgfqpoint{0.000000in}{0.000000in}}{%
\pgfpathmoveto{\pgfqpoint{0.000000in}{0.000000in}}%
\pgfpathlineto{\pgfqpoint{0.000000in}{-0.048611in}}%
\pgfusepath{stroke,fill}%
}%
\begin{pgfscope}%
\pgfsys@transformshift{3.016991in}{3.866111in}%
\pgfsys@useobject{currentmarker}{}%
\end{pgfscope}%
\end{pgfscope}%
\begin{pgfscope}%
\pgftext[x=3.016991in,y=3.768889in,,top]{\sffamily\fontsize{10.000000}{12.000000}\selectfont \(\displaystyle 10\)}%
\end{pgfscope}%
\begin{pgfscope}%
\pgfsetbuttcap%
\pgfsetroundjoin%
\definecolor{currentfill}{rgb}{0.000000,0.000000,0.000000}%
\pgfsetfillcolor{currentfill}%
\pgfsetlinewidth{0.803000pt}%
\definecolor{currentstroke}{rgb}{0.000000,0.000000,0.000000}%
\pgfsetstrokecolor{currentstroke}%
\pgfsetdash{}{0pt}%
\pgfsys@defobject{currentmarker}{\pgfqpoint{0.000000in}{-0.048611in}}{\pgfqpoint{0.000000in}{0.000000in}}{%
\pgfpathmoveto{\pgfqpoint{0.000000in}{0.000000in}}%
\pgfpathlineto{\pgfqpoint{0.000000in}{-0.048611in}}%
\pgfusepath{stroke,fill}%
}%
\begin{pgfscope}%
\pgfsys@transformshift{3.478019in}{3.866111in}%
\pgfsys@useobject{currentmarker}{}%
\end{pgfscope}%
\end{pgfscope}%
\begin{pgfscope}%
\pgftext[x=3.478019in,y=3.768889in,,top]{\sffamily\fontsize{10.000000}{12.000000}\selectfont \(\displaystyle 12\)}%
\end{pgfscope}%
\begin{pgfscope}%
\pgfsetbuttcap%
\pgfsetroundjoin%
\definecolor{currentfill}{rgb}{0.000000,0.000000,0.000000}%
\pgfsetfillcolor{currentfill}%
\pgfsetlinewidth{0.803000pt}%
\definecolor{currentstroke}{rgb}{0.000000,0.000000,0.000000}%
\pgfsetstrokecolor{currentstroke}%
\pgfsetdash{}{0pt}%
\pgfsys@defobject{currentmarker}{\pgfqpoint{0.000000in}{-0.048611in}}{\pgfqpoint{0.000000in}{0.000000in}}{%
\pgfpathmoveto{\pgfqpoint{0.000000in}{0.000000in}}%
\pgfpathlineto{\pgfqpoint{0.000000in}{-0.048611in}}%
\pgfusepath{stroke,fill}%
}%
\begin{pgfscope}%
\pgfsys@transformshift{3.939047in}{3.866111in}%
\pgfsys@useobject{currentmarker}{}%
\end{pgfscope}%
\end{pgfscope}%
\begin{pgfscope}%
\pgftext[x=3.939047in,y=3.768889in,,top]{\sffamily\fontsize{10.000000}{12.000000}\selectfont \(\displaystyle 14\)}%
\end{pgfscope}%
\begin{pgfscope}%
\pgfsetbuttcap%
\pgfsetroundjoin%
\definecolor{currentfill}{rgb}{0.000000,0.000000,0.000000}%
\pgfsetfillcolor{currentfill}%
\pgfsetlinewidth{0.803000pt}%
\definecolor{currentstroke}{rgb}{0.000000,0.000000,0.000000}%
\pgfsetstrokecolor{currentstroke}%
\pgfsetdash{}{0pt}%
\pgfsys@defobject{currentmarker}{\pgfqpoint{0.000000in}{-0.048611in}}{\pgfqpoint{0.000000in}{0.000000in}}{%
\pgfpathmoveto{\pgfqpoint{0.000000in}{0.000000in}}%
\pgfpathlineto{\pgfqpoint{0.000000in}{-0.048611in}}%
\pgfusepath{stroke,fill}%
}%
\begin{pgfscope}%
\pgfsys@transformshift{4.400075in}{3.866111in}%
\pgfsys@useobject{currentmarker}{}%
\end{pgfscope}%
\end{pgfscope}%
\begin{pgfscope}%
\pgftext[x=4.400075in,y=3.768889in,,top]{\sffamily\fontsize{10.000000}{12.000000}\selectfont \(\displaystyle 16\)}%
\end{pgfscope}%
\begin{pgfscope}%
\pgftext[x=2.786477in,y=3.578921in,,top]{\sffamily\fontsize{10.000000}{12.000000}\selectfont Number of Iterations}%
\end{pgfscope}%
\begin{pgfscope}%
\pgfsetbuttcap%
\pgfsetroundjoin%
\definecolor{currentfill}{rgb}{0.000000,0.000000,0.000000}%
\pgfsetfillcolor{currentfill}%
\pgfsetlinewidth{0.803000pt}%
\definecolor{currentstroke}{rgb}{0.000000,0.000000,0.000000}%
\pgfsetstrokecolor{currentstroke}%
\pgfsetdash{}{0pt}%
\pgfsys@defobject{currentmarker}{\pgfqpoint{-0.048611in}{0.000000in}}{\pgfqpoint{0.000000in}{0.000000in}}{%
\pgfpathmoveto{\pgfqpoint{0.000000in}{0.000000in}}%
\pgfpathlineto{\pgfqpoint{-0.048611in}{0.000000in}}%
\pgfusepath{stroke,fill}%
}%
\begin{pgfscope}%
\pgfsys@transformshift{0.757955in}{4.116225in}%
\pgfsys@useobject{currentmarker}{}%
\end{pgfscope}%
\end{pgfscope}%
\begin{pgfscope}%
\pgftext[x=0.344374in,y=4.063464in,left,base]{\sffamily\fontsize{10.000000}{12.000000}\selectfont \(\displaystyle 0.002\)}%
\end{pgfscope}%
\begin{pgfscope}%
\pgfsetbuttcap%
\pgfsetroundjoin%
\definecolor{currentfill}{rgb}{0.000000,0.000000,0.000000}%
\pgfsetfillcolor{currentfill}%
\pgfsetlinewidth{0.803000pt}%
\definecolor{currentstroke}{rgb}{0.000000,0.000000,0.000000}%
\pgfsetstrokecolor{currentstroke}%
\pgfsetdash{}{0pt}%
\pgfsys@defobject{currentmarker}{\pgfqpoint{-0.048611in}{0.000000in}}{\pgfqpoint{0.000000in}{0.000000in}}{%
\pgfpathmoveto{\pgfqpoint{0.000000in}{0.000000in}}%
\pgfpathlineto{\pgfqpoint{-0.048611in}{0.000000in}}%
\pgfusepath{stroke,fill}%
}%
\begin{pgfscope}%
\pgfsys@transformshift{0.757955in}{4.465673in}%
\pgfsys@useobject{currentmarker}{}%
\end{pgfscope}%
\end{pgfscope}%
\begin{pgfscope}%
\pgftext[x=0.344374in,y=4.412911in,left,base]{\sffamily\fontsize{10.000000}{12.000000}\selectfont \(\displaystyle 0.003\)}%
\end{pgfscope}%
\begin{pgfscope}%
\pgfsetbuttcap%
\pgfsetroundjoin%
\definecolor{currentfill}{rgb}{0.000000,0.000000,0.000000}%
\pgfsetfillcolor{currentfill}%
\pgfsetlinewidth{0.803000pt}%
\definecolor{currentstroke}{rgb}{0.000000,0.000000,0.000000}%
\pgfsetstrokecolor{currentstroke}%
\pgfsetdash{}{0pt}%
\pgfsys@defobject{currentmarker}{\pgfqpoint{-0.048611in}{0.000000in}}{\pgfqpoint{0.000000in}{0.000000in}}{%
\pgfpathmoveto{\pgfqpoint{0.000000in}{0.000000in}}%
\pgfpathlineto{\pgfqpoint{-0.048611in}{0.000000in}}%
\pgfusepath{stroke,fill}%
}%
\begin{pgfscope}%
\pgfsys@transformshift{0.757955in}{4.815121in}%
\pgfsys@useobject{currentmarker}{}%
\end{pgfscope}%
\end{pgfscope}%
\begin{pgfscope}%
\pgftext[x=0.344374in,y=4.762359in,left,base]{\sffamily\fontsize{10.000000}{12.000000}\selectfont \(\displaystyle 0.004\)}%
\end{pgfscope}%
\begin{pgfscope}%
\pgfsetbuttcap%
\pgfsetroundjoin%
\definecolor{currentfill}{rgb}{0.000000,0.000000,0.000000}%
\pgfsetfillcolor{currentfill}%
\pgfsetlinewidth{0.803000pt}%
\definecolor{currentstroke}{rgb}{0.000000,0.000000,0.000000}%
\pgfsetstrokecolor{currentstroke}%
\pgfsetdash{}{0pt}%
\pgfsys@defobject{currentmarker}{\pgfqpoint{-0.048611in}{0.000000in}}{\pgfqpoint{0.000000in}{0.000000in}}{%
\pgfpathmoveto{\pgfqpoint{0.000000in}{0.000000in}}%
\pgfpathlineto{\pgfqpoint{-0.048611in}{0.000000in}}%
\pgfusepath{stroke,fill}%
}%
\begin{pgfscope}%
\pgfsys@transformshift{0.757955in}{5.164568in}%
\pgfsys@useobject{currentmarker}{}%
\end{pgfscope}%
\end{pgfscope}%
\begin{pgfscope}%
\pgftext[x=0.344374in,y=5.111807in,left,base]{\sffamily\fontsize{10.000000}{12.000000}\selectfont \(\displaystyle 0.005\)}%
\end{pgfscope}%
\begin{pgfscope}%
\pgfsetbuttcap%
\pgfsetroundjoin%
\definecolor{currentfill}{rgb}{0.000000,0.000000,0.000000}%
\pgfsetfillcolor{currentfill}%
\pgfsetlinewidth{0.803000pt}%
\definecolor{currentstroke}{rgb}{0.000000,0.000000,0.000000}%
\pgfsetstrokecolor{currentstroke}%
\pgfsetdash{}{0pt}%
\pgfsys@defobject{currentmarker}{\pgfqpoint{-0.048611in}{0.000000in}}{\pgfqpoint{0.000000in}{0.000000in}}{%
\pgfpathmoveto{\pgfqpoint{0.000000in}{0.000000in}}%
\pgfpathlineto{\pgfqpoint{-0.048611in}{0.000000in}}%
\pgfusepath{stroke,fill}%
}%
\begin{pgfscope}%
\pgfsys@transformshift{0.757955in}{5.514016in}%
\pgfsys@useobject{currentmarker}{}%
\end{pgfscope}%
\end{pgfscope}%
\begin{pgfscope}%
\pgftext[x=0.344374in,y=5.461254in,left,base]{\sffamily\fontsize{10.000000}{12.000000}\selectfont \(\displaystyle 0.006\)}%
\end{pgfscope}%
\begin{pgfscope}%
\pgfsetbuttcap%
\pgfsetroundjoin%
\definecolor{currentfill}{rgb}{0.000000,0.000000,0.000000}%
\pgfsetfillcolor{currentfill}%
\pgfsetlinewidth{0.803000pt}%
\definecolor{currentstroke}{rgb}{0.000000,0.000000,0.000000}%
\pgfsetstrokecolor{currentstroke}%
\pgfsetdash{}{0pt}%
\pgfsys@defobject{currentmarker}{\pgfqpoint{-0.048611in}{0.000000in}}{\pgfqpoint{0.000000in}{0.000000in}}{%
\pgfpathmoveto{\pgfqpoint{0.000000in}{0.000000in}}%
\pgfpathlineto{\pgfqpoint{-0.048611in}{0.000000in}}%
\pgfusepath{stroke,fill}%
}%
\begin{pgfscope}%
\pgfsys@transformshift{0.757955in}{5.863464in}%
\pgfsys@useobject{currentmarker}{}%
\end{pgfscope}%
\end{pgfscope}%
\begin{pgfscope}%
\pgftext[x=0.344374in,y=5.810702in,left,base]{\sffamily\fontsize{10.000000}{12.000000}\selectfont \(\displaystyle 0.007\)}%
\end{pgfscope}%
\begin{pgfscope}%
\pgfsetbuttcap%
\pgfsetroundjoin%
\definecolor{currentfill}{rgb}{0.000000,0.000000,0.000000}%
\pgfsetfillcolor{currentfill}%
\pgfsetlinewidth{0.803000pt}%
\definecolor{currentstroke}{rgb}{0.000000,0.000000,0.000000}%
\pgfsetstrokecolor{currentstroke}%
\pgfsetdash{}{0pt}%
\pgfsys@defobject{currentmarker}{\pgfqpoint{-0.048611in}{0.000000in}}{\pgfqpoint{0.000000in}{0.000000in}}{%
\pgfpathmoveto{\pgfqpoint{0.000000in}{0.000000in}}%
\pgfpathlineto{\pgfqpoint{-0.048611in}{0.000000in}}%
\pgfusepath{stroke,fill}%
}%
\begin{pgfscope}%
\pgfsys@transformshift{0.757955in}{6.212911in}%
\pgfsys@useobject{currentmarker}{}%
\end{pgfscope}%
\end{pgfscope}%
\begin{pgfscope}%
\pgftext[x=0.344374in,y=6.160150in,left,base]{\sffamily\fontsize{10.000000}{12.000000}\selectfont \(\displaystyle 0.008\)}%
\end{pgfscope}%
\begin{pgfscope}%
\pgftext[x=0.288818in,y=5.104722in,,bottom,rotate=90.000000]{\sffamily\fontsize{10.000000}{12.000000}\selectfont Proportional Contribution}%
\end{pgfscope}%
\begin{pgfscope}%
\pgfpathrectangle{\pgfqpoint{0.757955in}{3.866111in}}{\pgfqpoint{4.057045in}{2.477222in}}%
\pgfusepath{clip}%
\pgfsetrectcap%
\pgfsetroundjoin%
\pgfsetlinewidth{1.505625pt}%
\definecolor{currentstroke}{rgb}{0.000000,0.000000,0.000000}%
\pgfsetstrokecolor{currentstroke}%
\pgfsetdash{}{0pt}%
\pgfpathmoveto{\pgfqpoint{0.942366in}{5.142939in}}%
\pgfpathlineto{\pgfqpoint{1.172880in}{5.296891in}}%
\pgfpathlineto{\pgfqpoint{1.403394in}{6.230732in}}%
\pgfpathlineto{\pgfqpoint{1.633908in}{5.755498in}}%
\pgfpathlineto{\pgfqpoint{1.864422in}{5.851740in}}%
\pgfpathlineto{\pgfqpoint{2.094936in}{5.165205in}}%
\pgfpathlineto{\pgfqpoint{2.325450in}{4.894877in}}%
\pgfpathlineto{\pgfqpoint{2.555963in}{4.699420in}}%
\pgfpathlineto{\pgfqpoint{2.786477in}{4.853393in}}%
\pgfpathlineto{\pgfqpoint{3.016991in}{4.586063in}}%
\pgfpathlineto{\pgfqpoint{3.247505in}{4.373881in}}%
\pgfpathlineto{\pgfqpoint{3.478019in}{4.335424in}}%
\pgfpathlineto{\pgfqpoint{3.708533in}{4.441936in}}%
\pgfpathlineto{\pgfqpoint{3.939047in}{4.211994in}}%
\pgfpathlineto{\pgfqpoint{4.169561in}{4.494500in}}%
\pgfpathlineto{\pgfqpoint{4.400075in}{4.540289in}}%
\pgfpathlineto{\pgfqpoint{4.630589in}{4.180299in}}%
\pgfusepath{stroke}%
\end{pgfscope}%
\begin{pgfscope}%
\pgfpathrectangle{\pgfqpoint{0.757955in}{3.866111in}}{\pgfqpoint{4.057045in}{2.477222in}}%
\pgfusepath{clip}%
\pgfsetrectcap%
\pgfsetroundjoin%
\pgfsetlinewidth{1.505625pt}%
\definecolor{currentstroke}{rgb}{0.500000,0.500000,0.500000}%
\pgfsetstrokecolor{currentstroke}%
\pgfsetdash{}{0pt}%
\pgfpathmoveto{\pgfqpoint{0.942366in}{4.698379in}}%
\pgfpathlineto{\pgfqpoint{1.172880in}{4.806132in}}%
\pgfpathlineto{\pgfqpoint{1.403394in}{5.515544in}}%
\pgfpathlineto{\pgfqpoint{1.633908in}{5.197115in}}%
\pgfpathlineto{\pgfqpoint{1.864422in}{5.278999in}}%
\pgfpathlineto{\pgfqpoint{2.094936in}{4.676828in}}%
\pgfpathlineto{\pgfqpoint{2.325450in}{4.472873in}}%
\pgfpathlineto{\pgfqpoint{2.555963in}{4.434840in}}%
\pgfpathlineto{\pgfqpoint{2.786477in}{4.607072in}}%
\pgfpathlineto{\pgfqpoint{3.016991in}{4.327648in}}%
\pgfpathlineto{\pgfqpoint{3.247505in}{4.184798in}}%
\pgfpathlineto{\pgfqpoint{3.478019in}{4.104906in}}%
\pgfpathlineto{\pgfqpoint{3.708533in}{4.227261in}}%
\pgfpathlineto{\pgfqpoint{3.939047in}{4.056070in}}%
\pgfpathlineto{\pgfqpoint{4.169561in}{4.252276in}}%
\pgfpathlineto{\pgfqpoint{4.400075in}{4.285880in}}%
\pgfpathlineto{\pgfqpoint{4.630589in}{4.033165in}}%
\pgfusepath{stroke}%
\end{pgfscope}%
\begin{pgfscope}%
\pgfpathrectangle{\pgfqpoint{0.757955in}{3.866111in}}{\pgfqpoint{4.057045in}{2.477222in}}%
\pgfusepath{clip}%
\pgfsetbuttcap%
\pgfsetroundjoin%
\pgfsetlinewidth{1.505625pt}%
\definecolor{currentstroke}{rgb}{0.000000,0.000000,0.000000}%
\pgfsetstrokecolor{currentstroke}%
\pgfsetdash{{5.550000pt}{2.400000pt}}{0.000000pt}%
\pgfpathmoveto{\pgfqpoint{0.942366in}{4.957147in}}%
\pgfpathlineto{\pgfqpoint{1.172880in}{5.198740in}}%
\pgfpathlineto{\pgfqpoint{1.403394in}{5.958550in}}%
\pgfpathlineto{\pgfqpoint{1.633908in}{5.436895in}}%
\pgfpathlineto{\pgfqpoint{1.864422in}{5.495479in}}%
\pgfpathlineto{\pgfqpoint{2.094936in}{5.319282in}}%
\pgfpathlineto{\pgfqpoint{2.325450in}{4.706425in}}%
\pgfpathlineto{\pgfqpoint{2.555963in}{4.454023in}}%
\pgfpathlineto{\pgfqpoint{2.786477in}{4.527833in}}%
\pgfpathlineto{\pgfqpoint{3.016991in}{4.340777in}}%
\pgfpathlineto{\pgfqpoint{3.247505in}{4.178626in}}%
\pgfpathlineto{\pgfqpoint{3.478019in}{4.118223in}}%
\pgfpathlineto{\pgfqpoint{3.708533in}{4.073052in}}%
\pgfpathlineto{\pgfqpoint{3.939047in}{3.978712in}}%
\pgfpathlineto{\pgfqpoint{4.169561in}{4.089035in}}%
\pgfpathlineto{\pgfqpoint{4.400075in}{4.209550in}}%
\pgfpathlineto{\pgfqpoint{4.630589in}{4.224394in}}%
\pgfusepath{stroke}%
\end{pgfscope}%
\begin{pgfscope}%
\pgfpathrectangle{\pgfqpoint{0.757955in}{3.866111in}}{\pgfqpoint{4.057045in}{2.477222in}}%
\pgfusepath{clip}%
\pgfsetbuttcap%
\pgfsetroundjoin%
\pgfsetlinewidth{1.505625pt}%
\definecolor{currentstroke}{rgb}{0.500000,0.500000,0.500000}%
\pgfsetstrokecolor{currentstroke}%
\pgfsetdash{{5.550000pt}{2.400000pt}}{0.000000pt}%
\pgfpathmoveto{\pgfqpoint{0.942366in}{4.908068in}}%
\pgfpathlineto{\pgfqpoint{1.172880in}{5.124166in}}%
\pgfpathlineto{\pgfqpoint{1.403394in}{5.809257in}}%
\pgfpathlineto{\pgfqpoint{1.633908in}{5.390982in}}%
\pgfpathlineto{\pgfqpoint{1.864422in}{5.373242in}}%
\pgfpathlineto{\pgfqpoint{2.094936in}{5.200857in}}%
\pgfpathlineto{\pgfqpoint{2.325450in}{4.582002in}}%
\pgfpathlineto{\pgfqpoint{2.555963in}{4.430230in}}%
\pgfpathlineto{\pgfqpoint{2.786477in}{4.556160in}}%
\pgfpathlineto{\pgfqpoint{3.016991in}{4.325850in}}%
\pgfpathlineto{\pgfqpoint{3.247505in}{4.204655in}}%
\pgfpathlineto{\pgfqpoint{3.478019in}{4.121694in}}%
\pgfpathlineto{\pgfqpoint{3.708533in}{4.095249in}}%
\pgfpathlineto{\pgfqpoint{3.939047in}{4.016280in}}%
\pgfpathlineto{\pgfqpoint{4.169561in}{4.059271in}}%
\pgfpathlineto{\pgfqpoint{4.400075in}{4.248756in}}%
\pgfpathlineto{\pgfqpoint{4.630589in}{4.246104in}}%
\pgfusepath{stroke}%
\end{pgfscope}%
\begin{pgfscope}%
\pgfsetrectcap%
\pgfsetmiterjoin%
\pgfsetlinewidth{0.803000pt}%
\definecolor{currentstroke}{rgb}{0.000000,0.000000,0.000000}%
\pgfsetstrokecolor{currentstroke}%
\pgfsetdash{}{0pt}%
\pgfpathmoveto{\pgfqpoint{0.757955in}{3.866111in}}%
\pgfpathlineto{\pgfqpoint{0.757955in}{6.343333in}}%
\pgfusepath{stroke}%
\end{pgfscope}%
\begin{pgfscope}%
\pgfsetrectcap%
\pgfsetmiterjoin%
\pgfsetlinewidth{0.803000pt}%
\definecolor{currentstroke}{rgb}{0.000000,0.000000,0.000000}%
\pgfsetstrokecolor{currentstroke}%
\pgfsetdash{}{0pt}%
\pgfpathmoveto{\pgfqpoint{4.815000in}{3.866111in}}%
\pgfpathlineto{\pgfqpoint{4.815000in}{6.343333in}}%
\pgfusepath{stroke}%
\end{pgfscope}%
\begin{pgfscope}%
\pgfsetrectcap%
\pgfsetmiterjoin%
\pgfsetlinewidth{0.803000pt}%
\definecolor{currentstroke}{rgb}{0.000000,0.000000,0.000000}%
\pgfsetstrokecolor{currentstroke}%
\pgfsetdash{}{0pt}%
\pgfpathmoveto{\pgfqpoint{0.757955in}{3.866111in}}%
\pgfpathlineto{\pgfqpoint{4.815000in}{3.866111in}}%
\pgfusepath{stroke}%
\end{pgfscope}%
\begin{pgfscope}%
\pgfsetrectcap%
\pgfsetmiterjoin%
\pgfsetlinewidth{0.803000pt}%
\definecolor{currentstroke}{rgb}{0.000000,0.000000,0.000000}%
\pgfsetstrokecolor{currentstroke}%
\pgfsetdash{}{0pt}%
\pgfpathmoveto{\pgfqpoint{0.757955in}{6.343333in}}%
\pgfpathlineto{\pgfqpoint{4.815000in}{6.343333in}}%
\pgfusepath{stroke}%
\end{pgfscope}%
\begin{pgfscope}%
\pgftext[x=2.786477in,y=6.426667in,,base]{\sffamily\fontsize{12.000000}{14.400000}\selectfont Expected Penalty Contributions}%
\end{pgfscope}%
\begin{pgfscope}%
\pgfsetbuttcap%
\pgfsetmiterjoin%
\definecolor{currentfill}{rgb}{1.000000,1.000000,1.000000}%
\pgfsetfillcolor{currentfill}%
\pgfsetfillopacity{0.800000}%
\pgfsetlinewidth{1.003750pt}%
\definecolor{currentstroke}{rgb}{0.800000,0.800000,0.800000}%
\pgfsetstrokecolor{currentstroke}%
\pgfsetstrokeopacity{0.800000}%
\pgfsetdash{}{0pt}%
\pgfpathmoveto{\pgfqpoint{2.387060in}{5.237164in}}%
\pgfpathlineto{\pgfqpoint{4.717778in}{5.237164in}}%
\pgfpathquadraticcurveto{\pgfqpoint{4.745556in}{5.237164in}}{\pgfqpoint{4.745556in}{5.264942in}}%
\pgfpathlineto{\pgfqpoint{4.745556in}{6.246111in}}%
\pgfpathquadraticcurveto{\pgfqpoint{4.745556in}{6.273889in}}{\pgfqpoint{4.717778in}{6.273889in}}%
\pgfpathlineto{\pgfqpoint{2.387060in}{6.273889in}}%
\pgfpathquadraticcurveto{\pgfqpoint{2.359282in}{6.273889in}}{\pgfqpoint{2.359282in}{6.246111in}}%
\pgfpathlineto{\pgfqpoint{2.359282in}{5.264942in}}%
\pgfpathquadraticcurveto{\pgfqpoint{2.359282in}{5.237164in}}{\pgfqpoint{2.387060in}{5.237164in}}%
\pgfpathclose%
\pgfusepath{stroke,fill}%
\end{pgfscope}%
\begin{pgfscope}%
\pgfsetrectcap%
\pgfsetroundjoin%
\pgfsetlinewidth{1.505625pt}%
\definecolor{currentstroke}{rgb}{0.000000,0.000000,0.000000}%
\pgfsetstrokecolor{currentstroke}%
\pgfsetdash{}{0pt}%
\pgfpathmoveto{\pgfqpoint{2.414838in}{6.147346in}}%
\pgfpathlineto{\pgfqpoint{2.692615in}{6.147346in}}%
\pgfusepath{stroke}%
\end{pgfscope}%
\begin{pgfscope}%
\pgftext[x=2.803726in,y=6.098735in,left,base]{\sffamily\fontsize{10.000000}{12.000000}\selectfont \(\displaystyle 1 - E[r_{t, CLIP}^+]\), control}%
\end{pgfscope}%
\begin{pgfscope}%
\pgfsetrectcap%
\pgfsetroundjoin%
\pgfsetlinewidth{1.505625pt}%
\definecolor{currentstroke}{rgb}{0.500000,0.500000,0.500000}%
\pgfsetstrokecolor{currentstroke}%
\pgfsetdash{}{0pt}%
\pgfpathmoveto{\pgfqpoint{2.414838in}{5.898581in}}%
\pgfpathlineto{\pgfqpoint{2.692615in}{5.898581in}}%
\pgfusepath{stroke}%
\end{pgfscope}%
\begin{pgfscope}%
\pgftext[x=2.803726in,y=5.849970in,left,base]{\sffamily\fontsize{10.000000}{12.000000}\selectfont \(\displaystyle E[r_{t, CLIP}^-] - 1\), control}%
\end{pgfscope}%
\begin{pgfscope}%
\pgfsetbuttcap%
\pgfsetroundjoin%
\pgfsetlinewidth{1.505625pt}%
\definecolor{currentstroke}{rgb}{0.000000,0.000000,0.000000}%
\pgfsetstrokecolor{currentstroke}%
\pgfsetdash{{5.550000pt}{2.400000pt}}{0.000000pt}%
\pgfpathmoveto{\pgfqpoint{2.414838in}{5.649817in}}%
\pgfpathlineto{\pgfqpoint{2.692615in}{5.649817in}}%
\pgfusepath{stroke}%
\end{pgfscope}%
\begin{pgfscope}%
\pgftext[x=2.803726in,y=5.601206in,left,base]{\sffamily\fontsize{10.000000}{12.000000}\selectfont \(\displaystyle 1 - E[r_{t, CLIP}^+]\), experimental}%
\end{pgfscope}%
\begin{pgfscope}%
\pgfsetbuttcap%
\pgfsetroundjoin%
\pgfsetlinewidth{1.505625pt}%
\definecolor{currentstroke}{rgb}{0.500000,0.500000,0.500000}%
\pgfsetstrokecolor{currentstroke}%
\pgfsetdash{{5.550000pt}{2.400000pt}}{0.000000pt}%
\pgfpathmoveto{\pgfqpoint{2.414838in}{5.401052in}}%
\pgfpathlineto{\pgfqpoint{2.692615in}{5.401052in}}%
\pgfusepath{stroke}%
\end{pgfscope}%
\begin{pgfscope}%
\pgftext[x=2.803726in,y=5.352441in,left,base]{\sffamily\fontsize{10.000000}{12.000000}\selectfont \(\displaystyle E[r_{t, CLIP}^-] - 1 \), experimental}%
\end{pgfscope}%
\begin{pgfscope}%
\pgfsetbuttcap%
\pgfsetmiterjoin%
\definecolor{currentfill}{rgb}{1.000000,1.000000,1.000000}%
\pgfsetfillcolor{currentfill}%
\pgfsetlinewidth{0.000000pt}%
\definecolor{currentstroke}{rgb}{0.000000,0.000000,0.000000}%
\pgfsetstrokecolor{currentstroke}%
\pgfsetstrokeopacity{0.000000}%
\pgfsetdash{}{0pt}%
\pgfpathmoveto{\pgfqpoint{0.757955in}{0.582778in}}%
\pgfpathlineto{\pgfqpoint{4.815000in}{0.582778in}}%
\pgfpathlineto{\pgfqpoint{4.815000in}{3.060000in}}%
\pgfpathlineto{\pgfqpoint{0.757955in}{3.060000in}}%
\pgfpathclose%
\pgfusepath{fill}%
\end{pgfscope}%
\begin{pgfscope}%
\pgfsetbuttcap%
\pgfsetroundjoin%
\definecolor{currentfill}{rgb}{0.000000,0.000000,0.000000}%
\pgfsetfillcolor{currentfill}%
\pgfsetlinewidth{0.803000pt}%
\definecolor{currentstroke}{rgb}{0.000000,0.000000,0.000000}%
\pgfsetstrokecolor{currentstroke}%
\pgfsetdash{}{0pt}%
\pgfsys@defobject{currentmarker}{\pgfqpoint{0.000000in}{-0.048611in}}{\pgfqpoint{0.000000in}{0.000000in}}{%
\pgfpathmoveto{\pgfqpoint{0.000000in}{0.000000in}}%
\pgfpathlineto{\pgfqpoint{0.000000in}{-0.048611in}}%
\pgfusepath{stroke,fill}%
}%
\begin{pgfscope}%
\pgfsys@transformshift{1.172880in}{0.582778in}%
\pgfsys@useobject{currentmarker}{}%
\end{pgfscope}%
\end{pgfscope}%
\begin{pgfscope}%
\pgftext[x=1.172880in,y=0.485556in,,top]{\sffamily\fontsize{10.000000}{12.000000}\selectfont \(\displaystyle 2\)}%
\end{pgfscope}%
\begin{pgfscope}%
\pgfsetbuttcap%
\pgfsetroundjoin%
\definecolor{currentfill}{rgb}{0.000000,0.000000,0.000000}%
\pgfsetfillcolor{currentfill}%
\pgfsetlinewidth{0.803000pt}%
\definecolor{currentstroke}{rgb}{0.000000,0.000000,0.000000}%
\pgfsetstrokecolor{currentstroke}%
\pgfsetdash{}{0pt}%
\pgfsys@defobject{currentmarker}{\pgfqpoint{0.000000in}{-0.048611in}}{\pgfqpoint{0.000000in}{0.000000in}}{%
\pgfpathmoveto{\pgfqpoint{0.000000in}{0.000000in}}%
\pgfpathlineto{\pgfqpoint{0.000000in}{-0.048611in}}%
\pgfusepath{stroke,fill}%
}%
\begin{pgfscope}%
\pgfsys@transformshift{1.633908in}{0.582778in}%
\pgfsys@useobject{currentmarker}{}%
\end{pgfscope}%
\end{pgfscope}%
\begin{pgfscope}%
\pgftext[x=1.633908in,y=0.485556in,,top]{\sffamily\fontsize{10.000000}{12.000000}\selectfont \(\displaystyle 4\)}%
\end{pgfscope}%
\begin{pgfscope}%
\pgfsetbuttcap%
\pgfsetroundjoin%
\definecolor{currentfill}{rgb}{0.000000,0.000000,0.000000}%
\pgfsetfillcolor{currentfill}%
\pgfsetlinewidth{0.803000pt}%
\definecolor{currentstroke}{rgb}{0.000000,0.000000,0.000000}%
\pgfsetstrokecolor{currentstroke}%
\pgfsetdash{}{0pt}%
\pgfsys@defobject{currentmarker}{\pgfqpoint{0.000000in}{-0.048611in}}{\pgfqpoint{0.000000in}{0.000000in}}{%
\pgfpathmoveto{\pgfqpoint{0.000000in}{0.000000in}}%
\pgfpathlineto{\pgfqpoint{0.000000in}{-0.048611in}}%
\pgfusepath{stroke,fill}%
}%
\begin{pgfscope}%
\pgfsys@transformshift{2.094936in}{0.582778in}%
\pgfsys@useobject{currentmarker}{}%
\end{pgfscope}%
\end{pgfscope}%
\begin{pgfscope}%
\pgftext[x=2.094936in,y=0.485556in,,top]{\sffamily\fontsize{10.000000}{12.000000}\selectfont \(\displaystyle 6\)}%
\end{pgfscope}%
\begin{pgfscope}%
\pgfsetbuttcap%
\pgfsetroundjoin%
\definecolor{currentfill}{rgb}{0.000000,0.000000,0.000000}%
\pgfsetfillcolor{currentfill}%
\pgfsetlinewidth{0.803000pt}%
\definecolor{currentstroke}{rgb}{0.000000,0.000000,0.000000}%
\pgfsetstrokecolor{currentstroke}%
\pgfsetdash{}{0pt}%
\pgfsys@defobject{currentmarker}{\pgfqpoint{0.000000in}{-0.048611in}}{\pgfqpoint{0.000000in}{0.000000in}}{%
\pgfpathmoveto{\pgfqpoint{0.000000in}{0.000000in}}%
\pgfpathlineto{\pgfqpoint{0.000000in}{-0.048611in}}%
\pgfusepath{stroke,fill}%
}%
\begin{pgfscope}%
\pgfsys@transformshift{2.555963in}{0.582778in}%
\pgfsys@useobject{currentmarker}{}%
\end{pgfscope}%
\end{pgfscope}%
\begin{pgfscope}%
\pgftext[x=2.555963in,y=0.485556in,,top]{\sffamily\fontsize{10.000000}{12.000000}\selectfont \(\displaystyle 8\)}%
\end{pgfscope}%
\begin{pgfscope}%
\pgfsetbuttcap%
\pgfsetroundjoin%
\definecolor{currentfill}{rgb}{0.000000,0.000000,0.000000}%
\pgfsetfillcolor{currentfill}%
\pgfsetlinewidth{0.803000pt}%
\definecolor{currentstroke}{rgb}{0.000000,0.000000,0.000000}%
\pgfsetstrokecolor{currentstroke}%
\pgfsetdash{}{0pt}%
\pgfsys@defobject{currentmarker}{\pgfqpoint{0.000000in}{-0.048611in}}{\pgfqpoint{0.000000in}{0.000000in}}{%
\pgfpathmoveto{\pgfqpoint{0.000000in}{0.000000in}}%
\pgfpathlineto{\pgfqpoint{0.000000in}{-0.048611in}}%
\pgfusepath{stroke,fill}%
}%
\begin{pgfscope}%
\pgfsys@transformshift{3.016991in}{0.582778in}%
\pgfsys@useobject{currentmarker}{}%
\end{pgfscope}%
\end{pgfscope}%
\begin{pgfscope}%
\pgftext[x=3.016991in,y=0.485556in,,top]{\sffamily\fontsize{10.000000}{12.000000}\selectfont \(\displaystyle 10\)}%
\end{pgfscope}%
\begin{pgfscope}%
\pgfsetbuttcap%
\pgfsetroundjoin%
\definecolor{currentfill}{rgb}{0.000000,0.000000,0.000000}%
\pgfsetfillcolor{currentfill}%
\pgfsetlinewidth{0.803000pt}%
\definecolor{currentstroke}{rgb}{0.000000,0.000000,0.000000}%
\pgfsetstrokecolor{currentstroke}%
\pgfsetdash{}{0pt}%
\pgfsys@defobject{currentmarker}{\pgfqpoint{0.000000in}{-0.048611in}}{\pgfqpoint{0.000000in}{0.000000in}}{%
\pgfpathmoveto{\pgfqpoint{0.000000in}{0.000000in}}%
\pgfpathlineto{\pgfqpoint{0.000000in}{-0.048611in}}%
\pgfusepath{stroke,fill}%
}%
\begin{pgfscope}%
\pgfsys@transformshift{3.478019in}{0.582778in}%
\pgfsys@useobject{currentmarker}{}%
\end{pgfscope}%
\end{pgfscope}%
\begin{pgfscope}%
\pgftext[x=3.478019in,y=0.485556in,,top]{\sffamily\fontsize{10.000000}{12.000000}\selectfont \(\displaystyle 12\)}%
\end{pgfscope}%
\begin{pgfscope}%
\pgfsetbuttcap%
\pgfsetroundjoin%
\definecolor{currentfill}{rgb}{0.000000,0.000000,0.000000}%
\pgfsetfillcolor{currentfill}%
\pgfsetlinewidth{0.803000pt}%
\definecolor{currentstroke}{rgb}{0.000000,0.000000,0.000000}%
\pgfsetstrokecolor{currentstroke}%
\pgfsetdash{}{0pt}%
\pgfsys@defobject{currentmarker}{\pgfqpoint{0.000000in}{-0.048611in}}{\pgfqpoint{0.000000in}{0.000000in}}{%
\pgfpathmoveto{\pgfqpoint{0.000000in}{0.000000in}}%
\pgfpathlineto{\pgfqpoint{0.000000in}{-0.048611in}}%
\pgfusepath{stroke,fill}%
}%
\begin{pgfscope}%
\pgfsys@transformshift{3.939047in}{0.582778in}%
\pgfsys@useobject{currentmarker}{}%
\end{pgfscope}%
\end{pgfscope}%
\begin{pgfscope}%
\pgftext[x=3.939047in,y=0.485556in,,top]{\sffamily\fontsize{10.000000}{12.000000}\selectfont \(\displaystyle 14\)}%
\end{pgfscope}%
\begin{pgfscope}%
\pgfsetbuttcap%
\pgfsetroundjoin%
\definecolor{currentfill}{rgb}{0.000000,0.000000,0.000000}%
\pgfsetfillcolor{currentfill}%
\pgfsetlinewidth{0.803000pt}%
\definecolor{currentstroke}{rgb}{0.000000,0.000000,0.000000}%
\pgfsetstrokecolor{currentstroke}%
\pgfsetdash{}{0pt}%
\pgfsys@defobject{currentmarker}{\pgfqpoint{0.000000in}{-0.048611in}}{\pgfqpoint{0.000000in}{0.000000in}}{%
\pgfpathmoveto{\pgfqpoint{0.000000in}{0.000000in}}%
\pgfpathlineto{\pgfqpoint{0.000000in}{-0.048611in}}%
\pgfusepath{stroke,fill}%
}%
\begin{pgfscope}%
\pgfsys@transformshift{4.400075in}{0.582778in}%
\pgfsys@useobject{currentmarker}{}%
\end{pgfscope}%
\end{pgfscope}%
\begin{pgfscope}%
\pgftext[x=4.400075in,y=0.485556in,,top]{\sffamily\fontsize{10.000000}{12.000000}\selectfont \(\displaystyle 16\)}%
\end{pgfscope}%
\begin{pgfscope}%
\pgftext[x=2.786477in,y=0.295587in,,top]{\sffamily\fontsize{10.000000}{12.000000}\selectfont Number of Iterations}%
\end{pgfscope}%
\begin{pgfscope}%
\pgfsetbuttcap%
\pgfsetroundjoin%
\definecolor{currentfill}{rgb}{0.000000,0.000000,0.000000}%
\pgfsetfillcolor{currentfill}%
\pgfsetlinewidth{0.803000pt}%
\definecolor{currentstroke}{rgb}{0.000000,0.000000,0.000000}%
\pgfsetstrokecolor{currentstroke}%
\pgfsetdash{}{0pt}%
\pgfsys@defobject{currentmarker}{\pgfqpoint{-0.048611in}{0.000000in}}{\pgfqpoint{0.000000in}{0.000000in}}{%
\pgfpathmoveto{\pgfqpoint{0.000000in}{0.000000in}}%
\pgfpathlineto{\pgfqpoint{-0.048611in}{0.000000in}}%
\pgfusepath{stroke,fill}%
}%
\begin{pgfscope}%
\pgfsys@transformshift{0.757955in}{1.020878in}%
\pgfsys@useobject{currentmarker}{}%
\end{pgfscope}%
\end{pgfscope}%
\begin{pgfscope}%
\pgftext[x=0.452399in,y=0.968117in,left,base]{\sffamily\fontsize{10.000000}{12.000000}\selectfont \(\displaystyle 200\)}%
\end{pgfscope}%
\begin{pgfscope}%
\pgfsetbuttcap%
\pgfsetroundjoin%
\definecolor{currentfill}{rgb}{0.000000,0.000000,0.000000}%
\pgfsetfillcolor{currentfill}%
\pgfsetlinewidth{0.803000pt}%
\definecolor{currentstroke}{rgb}{0.000000,0.000000,0.000000}%
\pgfsetstrokecolor{currentstroke}%
\pgfsetdash{}{0pt}%
\pgfsys@defobject{currentmarker}{\pgfqpoint{-0.048611in}{0.000000in}}{\pgfqpoint{0.000000in}{0.000000in}}{%
\pgfpathmoveto{\pgfqpoint{0.000000in}{0.000000in}}%
\pgfpathlineto{\pgfqpoint{-0.048611in}{0.000000in}}%
\pgfusepath{stroke,fill}%
}%
\begin{pgfscope}%
\pgfsys@transformshift{0.757955in}{1.558813in}%
\pgfsys@useobject{currentmarker}{}%
\end{pgfscope}%
\end{pgfscope}%
\begin{pgfscope}%
\pgftext[x=0.452399in,y=1.506051in,left,base]{\sffamily\fontsize{10.000000}{12.000000}\selectfont \(\displaystyle 400\)}%
\end{pgfscope}%
\begin{pgfscope}%
\pgfsetbuttcap%
\pgfsetroundjoin%
\definecolor{currentfill}{rgb}{0.000000,0.000000,0.000000}%
\pgfsetfillcolor{currentfill}%
\pgfsetlinewidth{0.803000pt}%
\definecolor{currentstroke}{rgb}{0.000000,0.000000,0.000000}%
\pgfsetstrokecolor{currentstroke}%
\pgfsetdash{}{0pt}%
\pgfsys@defobject{currentmarker}{\pgfqpoint{-0.048611in}{0.000000in}}{\pgfqpoint{0.000000in}{0.000000in}}{%
\pgfpathmoveto{\pgfqpoint{0.000000in}{0.000000in}}%
\pgfpathlineto{\pgfqpoint{-0.048611in}{0.000000in}}%
\pgfusepath{stroke,fill}%
}%
\begin{pgfscope}%
\pgfsys@transformshift{0.757955in}{2.096748in}%
\pgfsys@useobject{currentmarker}{}%
\end{pgfscope}%
\end{pgfscope}%
\begin{pgfscope}%
\pgftext[x=0.452399in,y=2.043986in,left,base]{\sffamily\fontsize{10.000000}{12.000000}\selectfont \(\displaystyle 600\)}%
\end{pgfscope}%
\begin{pgfscope}%
\pgfsetbuttcap%
\pgfsetroundjoin%
\definecolor{currentfill}{rgb}{0.000000,0.000000,0.000000}%
\pgfsetfillcolor{currentfill}%
\pgfsetlinewidth{0.803000pt}%
\definecolor{currentstroke}{rgb}{0.000000,0.000000,0.000000}%
\pgfsetstrokecolor{currentstroke}%
\pgfsetdash{}{0pt}%
\pgfsys@defobject{currentmarker}{\pgfqpoint{-0.048611in}{0.000000in}}{\pgfqpoint{0.000000in}{0.000000in}}{%
\pgfpathmoveto{\pgfqpoint{0.000000in}{0.000000in}}%
\pgfpathlineto{\pgfqpoint{-0.048611in}{0.000000in}}%
\pgfusepath{stroke,fill}%
}%
\begin{pgfscope}%
\pgfsys@transformshift{0.757955in}{2.634682in}%
\pgfsys@useobject{currentmarker}{}%
\end{pgfscope}%
\end{pgfscope}%
\begin{pgfscope}%
\pgftext[x=0.452399in,y=2.581921in,left,base]{\sffamily\fontsize{10.000000}{12.000000}\selectfont \(\displaystyle 800\)}%
\end{pgfscope}%
\begin{pgfscope}%
\pgftext[x=0.396843in,y=1.821389in,,bottom,rotate=90.000000]{\sffamily\fontsize{10.000000}{12.000000}\selectfont Loss Contribution}%
\end{pgfscope}%
\begin{pgfscope}%
\pgfpathrectangle{\pgfqpoint{0.757955in}{0.582778in}}{\pgfqpoint{4.057045in}{2.477222in}}%
\pgfusepath{clip}%
\pgfsetrectcap%
\pgfsetroundjoin%
\pgfsetlinewidth{1.505625pt}%
\definecolor{currentstroke}{rgb}{0.000000,0.000000,0.000000}%
\pgfsetstrokecolor{currentstroke}%
\pgfsetdash{}{0pt}%
\pgfpathmoveto{\pgfqpoint{0.942366in}{1.032893in}}%
\pgfpathlineto{\pgfqpoint{1.172880in}{1.162184in}}%
\pgfpathlineto{\pgfqpoint{1.403394in}{1.522457in}}%
\pgfpathlineto{\pgfqpoint{1.633908in}{1.563793in}}%
\pgfpathlineto{\pgfqpoint{1.864422in}{1.878299in}}%
\pgfpathlineto{\pgfqpoint{2.094936in}{1.707088in}}%
\pgfpathlineto{\pgfqpoint{2.325450in}{1.643349in}}%
\pgfpathlineto{\pgfqpoint{2.555963in}{1.585759in}}%
\pgfpathlineto{\pgfqpoint{2.786477in}{1.546794in}}%
\pgfpathlineto{\pgfqpoint{3.016991in}{1.338148in}}%
\pgfpathlineto{\pgfqpoint{3.247505in}{1.048192in}}%
\pgfpathlineto{\pgfqpoint{3.478019in}{0.997883in}}%
\pgfpathlineto{\pgfqpoint{3.708533in}{1.027000in}}%
\pgfpathlineto{\pgfqpoint{3.939047in}{0.818214in}}%
\pgfpathlineto{\pgfqpoint{4.169561in}{0.872782in}}%
\pgfpathlineto{\pgfqpoint{4.400075in}{0.868082in}}%
\pgfpathlineto{\pgfqpoint{4.630589in}{0.709825in}}%
\pgfusepath{stroke}%
\end{pgfscope}%
\begin{pgfscope}%
\pgfpathrectangle{\pgfqpoint{0.757955in}{0.582778in}}{\pgfqpoint{4.057045in}{2.477222in}}%
\pgfusepath{clip}%
\pgfsetrectcap%
\pgfsetroundjoin%
\pgfsetlinewidth{1.505625pt}%
\definecolor{currentstroke}{rgb}{0.500000,0.500000,0.500000}%
\pgfsetstrokecolor{currentstroke}%
\pgfsetdash{}{0pt}%
\pgfpathmoveto{\pgfqpoint{0.942366in}{0.766141in}}%
\pgfpathlineto{\pgfqpoint{1.172880in}{1.054984in}}%
\pgfpathlineto{\pgfqpoint{1.403394in}{1.554208in}}%
\pgfpathlineto{\pgfqpoint{1.633908in}{1.722508in}}%
\pgfpathlineto{\pgfqpoint{1.864422in}{2.141162in}}%
\pgfpathlineto{\pgfqpoint{2.094936in}{2.013130in}}%
\pgfpathlineto{\pgfqpoint{2.325450in}{2.144265in}}%
\pgfpathlineto{\pgfqpoint{2.555963in}{2.328375in}}%
\pgfpathlineto{\pgfqpoint{2.786477in}{2.947399in}}%
\pgfpathlineto{\pgfqpoint{3.016991in}{2.307796in}}%
\pgfpathlineto{\pgfqpoint{3.247505in}{1.755362in}}%
\pgfpathlineto{\pgfqpoint{3.478019in}{1.757751in}}%
\pgfpathlineto{\pgfqpoint{3.708533in}{1.760388in}}%
\pgfpathlineto{\pgfqpoint{3.939047in}{1.433593in}}%
\pgfpathlineto{\pgfqpoint{4.169561in}{1.671180in}}%
\pgfpathlineto{\pgfqpoint{4.400075in}{1.421528in}}%
\pgfpathlineto{\pgfqpoint{4.630589in}{1.310368in}}%
\pgfusepath{stroke}%
\end{pgfscope}%
\begin{pgfscope}%
\pgfpathrectangle{\pgfqpoint{0.757955in}{0.582778in}}{\pgfqpoint{4.057045in}{2.477222in}}%
\pgfusepath{clip}%
\pgfsetbuttcap%
\pgfsetroundjoin%
\pgfsetlinewidth{1.505625pt}%
\definecolor{currentstroke}{rgb}{0.000000,0.000000,0.000000}%
\pgfsetstrokecolor{currentstroke}%
\pgfsetdash{{5.550000pt}{2.400000pt}}{0.000000pt}%
\pgfpathmoveto{\pgfqpoint{0.942366in}{1.005541in}}%
\pgfpathlineto{\pgfqpoint{1.172880in}{1.121104in}}%
\pgfpathlineto{\pgfqpoint{1.403394in}{1.491470in}}%
\pgfpathlineto{\pgfqpoint{1.633908in}{1.487978in}}%
\pgfpathlineto{\pgfqpoint{1.864422in}{1.711708in}}%
\pgfpathlineto{\pgfqpoint{2.094936in}{1.786854in}}%
\pgfpathlineto{\pgfqpoint{2.325450in}{1.553588in}}%
\pgfpathlineto{\pgfqpoint{2.555963in}{1.378675in}}%
\pgfpathlineto{\pgfqpoint{2.786477in}{1.308106in}}%
\pgfpathlineto{\pgfqpoint{3.016991in}{1.173027in}}%
\pgfpathlineto{\pgfqpoint{3.247505in}{1.003711in}}%
\pgfpathlineto{\pgfqpoint{3.478019in}{0.888584in}}%
\pgfpathlineto{\pgfqpoint{3.708533in}{0.828122in}}%
\pgfpathlineto{\pgfqpoint{3.939047in}{0.748727in}}%
\pgfpathlineto{\pgfqpoint{4.169561in}{0.708284in}}%
\pgfpathlineto{\pgfqpoint{4.400075in}{0.695379in}}%
\pgfpathlineto{\pgfqpoint{4.630589in}{0.740214in}}%
\pgfusepath{stroke}%
\end{pgfscope}%
\begin{pgfscope}%
\pgfpathrectangle{\pgfqpoint{0.757955in}{0.582778in}}{\pgfqpoint{4.057045in}{2.477222in}}%
\pgfusepath{clip}%
\pgfsetbuttcap%
\pgfsetroundjoin%
\pgfsetlinewidth{1.505625pt}%
\definecolor{currentstroke}{rgb}{0.500000,0.500000,0.500000}%
\pgfsetstrokecolor{currentstroke}%
\pgfsetdash{{5.550000pt}{2.400000pt}}{0.000000pt}%
\pgfpathmoveto{\pgfqpoint{0.942366in}{0.794694in}}%
\pgfpathlineto{\pgfqpoint{1.172880in}{1.138051in}}%
\pgfpathlineto{\pgfqpoint{1.403394in}{1.695458in}}%
\pgfpathlineto{\pgfqpoint{1.633908in}{1.847137in}}%
\pgfpathlineto{\pgfqpoint{1.864422in}{2.226524in}}%
\pgfpathlineto{\pgfqpoint{2.094936in}{2.561004in}}%
\pgfpathlineto{\pgfqpoint{2.325450in}{2.190858in}}%
\pgfpathlineto{\pgfqpoint{2.555963in}{2.317231in}}%
\pgfpathlineto{\pgfqpoint{2.786477in}{2.710295in}}%
\pgfpathlineto{\pgfqpoint{3.016991in}{2.049482in}}%
\pgfpathlineto{\pgfqpoint{3.247505in}{1.855122in}}%
\pgfpathlineto{\pgfqpoint{3.478019in}{1.794037in}}%
\pgfpathlineto{\pgfqpoint{3.708533in}{1.490300in}}%
\pgfpathlineto{\pgfqpoint{3.939047in}{1.225812in}}%
\pgfpathlineto{\pgfqpoint{4.169561in}{1.240451in}}%
\pgfpathlineto{\pgfqpoint{4.400075in}{1.309451in}}%
\pgfpathlineto{\pgfqpoint{4.630589in}{1.248355in}}%
\pgfusepath{stroke}%
\end{pgfscope}%
\begin{pgfscope}%
\pgfsetrectcap%
\pgfsetmiterjoin%
\pgfsetlinewidth{0.803000pt}%
\definecolor{currentstroke}{rgb}{0.000000,0.000000,0.000000}%
\pgfsetstrokecolor{currentstroke}%
\pgfsetdash{}{0pt}%
\pgfpathmoveto{\pgfqpoint{0.757955in}{0.582778in}}%
\pgfpathlineto{\pgfqpoint{0.757955in}{3.060000in}}%
\pgfusepath{stroke}%
\end{pgfscope}%
\begin{pgfscope}%
\pgfsetrectcap%
\pgfsetmiterjoin%
\pgfsetlinewidth{0.803000pt}%
\definecolor{currentstroke}{rgb}{0.000000,0.000000,0.000000}%
\pgfsetstrokecolor{currentstroke}%
\pgfsetdash{}{0pt}%
\pgfpathmoveto{\pgfqpoint{4.815000in}{0.582778in}}%
\pgfpathlineto{\pgfqpoint{4.815000in}{3.060000in}}%
\pgfusepath{stroke}%
\end{pgfscope}%
\begin{pgfscope}%
\pgfsetrectcap%
\pgfsetmiterjoin%
\pgfsetlinewidth{0.803000pt}%
\definecolor{currentstroke}{rgb}{0.000000,0.000000,0.000000}%
\pgfsetstrokecolor{currentstroke}%
\pgfsetdash{}{0pt}%
\pgfpathmoveto{\pgfqpoint{0.757955in}{0.582778in}}%
\pgfpathlineto{\pgfqpoint{4.815000in}{0.582778in}}%
\pgfusepath{stroke}%
\end{pgfscope}%
\begin{pgfscope}%
\pgfsetrectcap%
\pgfsetmiterjoin%
\pgfsetlinewidth{0.803000pt}%
\definecolor{currentstroke}{rgb}{0.000000,0.000000,0.000000}%
\pgfsetstrokecolor{currentstroke}%
\pgfsetdash{}{0pt}%
\pgfpathmoveto{\pgfqpoint{0.757955in}{3.060000in}}%
\pgfpathlineto{\pgfqpoint{4.815000in}{3.060000in}}%
\pgfusepath{stroke}%
\end{pgfscope}%
\begin{pgfscope}%
\pgftext[x=2.786477in,y=3.143333in,,base]{\sffamily\fontsize{12.000000}{14.400000}\selectfont Actual Penalty Contributions}%
\end{pgfscope}%
\begin{pgfscope}%
\pgfsetbuttcap%
\pgfsetmiterjoin%
\definecolor{currentfill}{rgb}{1.000000,1.000000,1.000000}%
\pgfsetfillcolor{currentfill}%
\pgfsetfillopacity{0.800000}%
\pgfsetlinewidth{1.003750pt}%
\definecolor{currentstroke}{rgb}{0.800000,0.800000,0.800000}%
\pgfsetstrokecolor{currentstroke}%
\pgfsetstrokeopacity{0.800000}%
\pgfsetdash{}{0pt}%
\pgfpathmoveto{\pgfqpoint{1.177020in}{2.133460in}}%
\pgfpathlineto{\pgfqpoint{4.717778in}{2.133460in}}%
\pgfpathquadraticcurveto{\pgfqpoint{4.745556in}{2.133460in}}{\pgfqpoint{4.745556in}{2.161238in}}%
\pgfpathlineto{\pgfqpoint{4.745556in}{2.962778in}}%
\pgfpathquadraticcurveto{\pgfqpoint{4.745556in}{2.990556in}}{\pgfqpoint{4.717778in}{2.990556in}}%
\pgfpathlineto{\pgfqpoint{1.177020in}{2.990556in}}%
\pgfpathquadraticcurveto{\pgfqpoint{1.149242in}{2.990556in}}{\pgfqpoint{1.149242in}{2.962778in}}%
\pgfpathlineto{\pgfqpoint{1.149242in}{2.161238in}}%
\pgfpathquadraticcurveto{\pgfqpoint{1.149242in}{2.133460in}}{\pgfqpoint{1.177020in}{2.133460in}}%
\pgfpathclose%
\pgfusepath{stroke,fill}%
\end{pgfscope}%
\begin{pgfscope}%
\pgfsetrectcap%
\pgfsetroundjoin%
\pgfsetlinewidth{1.505625pt}%
\definecolor{currentstroke}{rgb}{0.000000,0.000000,0.000000}%
\pgfsetstrokecolor{currentstroke}%
\pgfsetdash{}{0pt}%
\pgfpathmoveto{\pgfqpoint{1.204798in}{2.878088in}}%
\pgfpathlineto{\pgfqpoint{1.482575in}{2.878088in}}%
\pgfusepath{stroke}%
\end{pgfscope}%
\begin{pgfscope}%
\pgftext[x=1.593687in,y=2.829477in,left,base]{\sffamily\fontsize{10.000000}{12.000000}\selectfont Positive Penalty Contribution, control}%
\end{pgfscope}%
\begin{pgfscope}%
\pgfsetrectcap%
\pgfsetroundjoin%
\pgfsetlinewidth{1.505625pt}%
\definecolor{currentstroke}{rgb}{0.500000,0.500000,0.500000}%
\pgfsetstrokecolor{currentstroke}%
\pgfsetdash{}{0pt}%
\pgfpathmoveto{\pgfqpoint{1.204798in}{2.674231in}}%
\pgfpathlineto{\pgfqpoint{1.482575in}{2.674231in}}%
\pgfusepath{stroke}%
\end{pgfscope}%
\begin{pgfscope}%
\pgftext[x=1.593687in,y=2.625620in,left,base]{\sffamily\fontsize{10.000000}{12.000000}\selectfont Negative Penalty Contribution, control}%
\end{pgfscope}%
\begin{pgfscope}%
\pgfsetbuttcap%
\pgfsetroundjoin%
\pgfsetlinewidth{1.505625pt}%
\definecolor{currentstroke}{rgb}{0.000000,0.000000,0.000000}%
\pgfsetstrokecolor{currentstroke}%
\pgfsetdash{{5.550000pt}{2.400000pt}}{0.000000pt}%
\pgfpathmoveto{\pgfqpoint{1.204798in}{2.470374in}}%
\pgfpathlineto{\pgfqpoint{1.482575in}{2.470374in}}%
\pgfusepath{stroke}%
\end{pgfscope}%
\begin{pgfscope}%
\pgftext[x=1.593687in,y=2.421762in,left,base]{\sffamily\fontsize{10.000000}{12.000000}\selectfont Positive Penalty Contribution, experimental}%
\end{pgfscope}%
\begin{pgfscope}%
\pgfsetbuttcap%
\pgfsetroundjoin%
\pgfsetlinewidth{1.505625pt}%
\definecolor{currentstroke}{rgb}{0.500000,0.500000,0.500000}%
\pgfsetstrokecolor{currentstroke}%
\pgfsetdash{{5.550000pt}{2.400000pt}}{0.000000pt}%
\pgfpathmoveto{\pgfqpoint{1.204798in}{2.266516in}}%
\pgfpathlineto{\pgfqpoint{1.482575in}{2.266516in}}%
\pgfusepath{stroke}%
\end{pgfscope}%
\begin{pgfscope}%
\pgftext[x=1.593687in,y=2.217905in,left,base]{\sffamily\fontsize{10.000000}{12.000000}\selectfont Negative Penalty Contribution, experimental}%
\end{pgfscope}%
\end{pgfpicture}%
\makeatother%
\endgroup%
}\\
    \caption{Results: Hopper-v2 environment}
    \label{fig:3}
\end{figure}
\begin{figure}
    \centering
    \scalebox{0.45}{%% Creator: Matplotlib, PGF backend
%%
%% To include the figure in your LaTeX document, write
%%   \input{<filename>.pgf}
%%
%% Make sure the required packages are loaded in your preamble
%%   \usepackage{pgf}
%%
%% Figures using additional raster images can only be included by \input if
%% they are in the same directory as the main LaTeX file. For loading figures
%% from other directories you can use the `import` package
%%   \usepackage{import}
%% and then include the figures with
%%   \import{<path to file>}{<filename>.pgf}
%%
%% Matplotlib used the following preamble
%%   \usepackage{fontspec}
%%   \setmainfont{DejaVu Serif}
%%   \setsansfont{DejaVu Sans}
%%   \setmonofont{DejaVu Sans Mono}
%%
\begingroup%
\makeatletter%
\begin{pgfpicture}%
\pgfpathrectangle{\pgfpointorigin}{\pgfqpoint{6.400000in}{4.800000in}}%
\pgfusepath{use as bounding box, clip}%
\begin{pgfscope}%
\pgfsetbuttcap%
\pgfsetmiterjoin%
\definecolor{currentfill}{rgb}{1.000000,1.000000,1.000000}%
\pgfsetfillcolor{currentfill}%
\pgfsetlinewidth{0.000000pt}%
\definecolor{currentstroke}{rgb}{1.000000,1.000000,1.000000}%
\pgfsetstrokecolor{currentstroke}%
\pgfsetdash{}{0pt}%
\pgfpathmoveto{\pgfqpoint{0.000000in}{0.000000in}}%
\pgfpathlineto{\pgfqpoint{6.400000in}{0.000000in}}%
\pgfpathlineto{\pgfqpoint{6.400000in}{4.800000in}}%
\pgfpathlineto{\pgfqpoint{0.000000in}{4.800000in}}%
\pgfpathclose%
\pgfusepath{fill}%
\end{pgfscope}%
\begin{pgfscope}%
\pgfsetbuttcap%
\pgfsetmiterjoin%
\definecolor{currentfill}{rgb}{1.000000,1.000000,1.000000}%
\pgfsetfillcolor{currentfill}%
\pgfsetlinewidth{0.000000pt}%
\definecolor{currentstroke}{rgb}{0.000000,0.000000,0.000000}%
\pgfsetstrokecolor{currentstroke}%
\pgfsetstrokeopacity{0.000000}%
\pgfsetdash{}{0pt}%
\pgfpathmoveto{\pgfqpoint{0.827140in}{2.907778in}}%
\pgfpathlineto{\pgfqpoint{6.215000in}{2.907778in}}%
\pgfpathlineto{\pgfqpoint{6.215000in}{4.426667in}}%
\pgfpathlineto{\pgfqpoint{0.827140in}{4.426667in}}%
\pgfpathclose%
\pgfusepath{fill}%
\end{pgfscope}%
\begin{pgfscope}%
\pgfsetbuttcap%
\pgfsetroundjoin%
\definecolor{currentfill}{rgb}{0.000000,0.000000,0.000000}%
\pgfsetfillcolor{currentfill}%
\pgfsetlinewidth{0.803000pt}%
\definecolor{currentstroke}{rgb}{0.000000,0.000000,0.000000}%
\pgfsetstrokecolor{currentstroke}%
\pgfsetdash{}{0pt}%
\pgfsys@defobject{currentmarker}{\pgfqpoint{0.000000in}{-0.048611in}}{\pgfqpoint{0.000000in}{0.000000in}}{%
\pgfpathmoveto{\pgfqpoint{0.000000in}{0.000000in}}%
\pgfpathlineto{\pgfqpoint{0.000000in}{-0.048611in}}%
\pgfusepath{stroke,fill}%
}%
\begin{pgfscope}%
\pgfsys@transformshift{1.022567in}{2.907778in}%
\pgfsys@useobject{currentmarker}{}%
\end{pgfscope}%
\end{pgfscope}%
\begin{pgfscope}%
\pgftext[x=1.022567in,y=2.810556in,,top]{\sffamily\fontsize{10.000000}{12.000000}\selectfont \(\displaystyle 0\)}%
\end{pgfscope}%
\begin{pgfscope}%
\pgfsetbuttcap%
\pgfsetroundjoin%
\definecolor{currentfill}{rgb}{0.000000,0.000000,0.000000}%
\pgfsetfillcolor{currentfill}%
\pgfsetlinewidth{0.803000pt}%
\definecolor{currentstroke}{rgb}{0.000000,0.000000,0.000000}%
\pgfsetstrokecolor{currentstroke}%
\pgfsetdash{}{0pt}%
\pgfsys@defobject{currentmarker}{\pgfqpoint{0.000000in}{-0.048611in}}{\pgfqpoint{0.000000in}{0.000000in}}{%
\pgfpathmoveto{\pgfqpoint{0.000000in}{0.000000in}}%
\pgfpathlineto{\pgfqpoint{0.000000in}{-0.048611in}}%
\pgfusepath{stroke,fill}%
}%
\begin{pgfscope}%
\pgfsys@transformshift{1.641009in}{2.907778in}%
\pgfsys@useobject{currentmarker}{}%
\end{pgfscope}%
\end{pgfscope}%
\begin{pgfscope}%
\pgftext[x=1.641009in,y=2.810556in,,top]{\sffamily\fontsize{10.000000}{12.000000}\selectfont \(\displaystyle 25000\)}%
\end{pgfscope}%
\begin{pgfscope}%
\pgfsetbuttcap%
\pgfsetroundjoin%
\definecolor{currentfill}{rgb}{0.000000,0.000000,0.000000}%
\pgfsetfillcolor{currentfill}%
\pgfsetlinewidth{0.803000pt}%
\definecolor{currentstroke}{rgb}{0.000000,0.000000,0.000000}%
\pgfsetstrokecolor{currentstroke}%
\pgfsetdash{}{0pt}%
\pgfsys@defobject{currentmarker}{\pgfqpoint{0.000000in}{-0.048611in}}{\pgfqpoint{0.000000in}{0.000000in}}{%
\pgfpathmoveto{\pgfqpoint{0.000000in}{0.000000in}}%
\pgfpathlineto{\pgfqpoint{0.000000in}{-0.048611in}}%
\pgfusepath{stroke,fill}%
}%
\begin{pgfscope}%
\pgfsys@transformshift{2.259450in}{2.907778in}%
\pgfsys@useobject{currentmarker}{}%
\end{pgfscope}%
\end{pgfscope}%
\begin{pgfscope}%
\pgftext[x=2.259450in,y=2.810556in,,top]{\sffamily\fontsize{10.000000}{12.000000}\selectfont \(\displaystyle 50000\)}%
\end{pgfscope}%
\begin{pgfscope}%
\pgfsetbuttcap%
\pgfsetroundjoin%
\definecolor{currentfill}{rgb}{0.000000,0.000000,0.000000}%
\pgfsetfillcolor{currentfill}%
\pgfsetlinewidth{0.803000pt}%
\definecolor{currentstroke}{rgb}{0.000000,0.000000,0.000000}%
\pgfsetstrokecolor{currentstroke}%
\pgfsetdash{}{0pt}%
\pgfsys@defobject{currentmarker}{\pgfqpoint{0.000000in}{-0.048611in}}{\pgfqpoint{0.000000in}{0.000000in}}{%
\pgfpathmoveto{\pgfqpoint{0.000000in}{0.000000in}}%
\pgfpathlineto{\pgfqpoint{0.000000in}{-0.048611in}}%
\pgfusepath{stroke,fill}%
}%
\begin{pgfscope}%
\pgfsys@transformshift{2.877891in}{2.907778in}%
\pgfsys@useobject{currentmarker}{}%
\end{pgfscope}%
\end{pgfscope}%
\begin{pgfscope}%
\pgftext[x=2.877891in,y=2.810556in,,top]{\sffamily\fontsize{10.000000}{12.000000}\selectfont \(\displaystyle 75000\)}%
\end{pgfscope}%
\begin{pgfscope}%
\pgfsetbuttcap%
\pgfsetroundjoin%
\definecolor{currentfill}{rgb}{0.000000,0.000000,0.000000}%
\pgfsetfillcolor{currentfill}%
\pgfsetlinewidth{0.803000pt}%
\definecolor{currentstroke}{rgb}{0.000000,0.000000,0.000000}%
\pgfsetstrokecolor{currentstroke}%
\pgfsetdash{}{0pt}%
\pgfsys@defobject{currentmarker}{\pgfqpoint{0.000000in}{-0.048611in}}{\pgfqpoint{0.000000in}{0.000000in}}{%
\pgfpathmoveto{\pgfqpoint{0.000000in}{0.000000in}}%
\pgfpathlineto{\pgfqpoint{0.000000in}{-0.048611in}}%
\pgfusepath{stroke,fill}%
}%
\begin{pgfscope}%
\pgfsys@transformshift{3.496332in}{2.907778in}%
\pgfsys@useobject{currentmarker}{}%
\end{pgfscope}%
\end{pgfscope}%
\begin{pgfscope}%
\pgftext[x=3.496332in,y=2.810556in,,top]{\sffamily\fontsize{10.000000}{12.000000}\selectfont \(\displaystyle 100000\)}%
\end{pgfscope}%
\begin{pgfscope}%
\pgfsetbuttcap%
\pgfsetroundjoin%
\definecolor{currentfill}{rgb}{0.000000,0.000000,0.000000}%
\pgfsetfillcolor{currentfill}%
\pgfsetlinewidth{0.803000pt}%
\definecolor{currentstroke}{rgb}{0.000000,0.000000,0.000000}%
\pgfsetstrokecolor{currentstroke}%
\pgfsetdash{}{0pt}%
\pgfsys@defobject{currentmarker}{\pgfqpoint{0.000000in}{-0.048611in}}{\pgfqpoint{0.000000in}{0.000000in}}{%
\pgfpathmoveto{\pgfqpoint{0.000000in}{0.000000in}}%
\pgfpathlineto{\pgfqpoint{0.000000in}{-0.048611in}}%
\pgfusepath{stroke,fill}%
}%
\begin{pgfscope}%
\pgfsys@transformshift{4.114774in}{2.907778in}%
\pgfsys@useobject{currentmarker}{}%
\end{pgfscope}%
\end{pgfscope}%
\begin{pgfscope}%
\pgftext[x=4.114774in,y=2.810556in,,top]{\sffamily\fontsize{10.000000}{12.000000}\selectfont \(\displaystyle 125000\)}%
\end{pgfscope}%
\begin{pgfscope}%
\pgfsetbuttcap%
\pgfsetroundjoin%
\definecolor{currentfill}{rgb}{0.000000,0.000000,0.000000}%
\pgfsetfillcolor{currentfill}%
\pgfsetlinewidth{0.803000pt}%
\definecolor{currentstroke}{rgb}{0.000000,0.000000,0.000000}%
\pgfsetstrokecolor{currentstroke}%
\pgfsetdash{}{0pt}%
\pgfsys@defobject{currentmarker}{\pgfqpoint{0.000000in}{-0.048611in}}{\pgfqpoint{0.000000in}{0.000000in}}{%
\pgfpathmoveto{\pgfqpoint{0.000000in}{0.000000in}}%
\pgfpathlineto{\pgfqpoint{0.000000in}{-0.048611in}}%
\pgfusepath{stroke,fill}%
}%
\begin{pgfscope}%
\pgfsys@transformshift{4.733215in}{2.907778in}%
\pgfsys@useobject{currentmarker}{}%
\end{pgfscope}%
\end{pgfscope}%
\begin{pgfscope}%
\pgftext[x=4.733215in,y=2.810556in,,top]{\sffamily\fontsize{10.000000}{12.000000}\selectfont \(\displaystyle 150000\)}%
\end{pgfscope}%
\begin{pgfscope}%
\pgfsetbuttcap%
\pgfsetroundjoin%
\definecolor{currentfill}{rgb}{0.000000,0.000000,0.000000}%
\pgfsetfillcolor{currentfill}%
\pgfsetlinewidth{0.803000pt}%
\definecolor{currentstroke}{rgb}{0.000000,0.000000,0.000000}%
\pgfsetstrokecolor{currentstroke}%
\pgfsetdash{}{0pt}%
\pgfsys@defobject{currentmarker}{\pgfqpoint{0.000000in}{-0.048611in}}{\pgfqpoint{0.000000in}{0.000000in}}{%
\pgfpathmoveto{\pgfqpoint{0.000000in}{0.000000in}}%
\pgfpathlineto{\pgfqpoint{0.000000in}{-0.048611in}}%
\pgfusepath{stroke,fill}%
}%
\begin{pgfscope}%
\pgfsys@transformshift{5.351656in}{2.907778in}%
\pgfsys@useobject{currentmarker}{}%
\end{pgfscope}%
\end{pgfscope}%
\begin{pgfscope}%
\pgftext[x=5.351656in,y=2.810556in,,top]{\sffamily\fontsize{10.000000}{12.000000}\selectfont \(\displaystyle 175000\)}%
\end{pgfscope}%
\begin{pgfscope}%
\pgfsetbuttcap%
\pgfsetroundjoin%
\definecolor{currentfill}{rgb}{0.000000,0.000000,0.000000}%
\pgfsetfillcolor{currentfill}%
\pgfsetlinewidth{0.803000pt}%
\definecolor{currentstroke}{rgb}{0.000000,0.000000,0.000000}%
\pgfsetstrokecolor{currentstroke}%
\pgfsetdash{}{0pt}%
\pgfsys@defobject{currentmarker}{\pgfqpoint{0.000000in}{-0.048611in}}{\pgfqpoint{0.000000in}{0.000000in}}{%
\pgfpathmoveto{\pgfqpoint{0.000000in}{0.000000in}}%
\pgfpathlineto{\pgfqpoint{0.000000in}{-0.048611in}}%
\pgfusepath{stroke,fill}%
}%
\begin{pgfscope}%
\pgfsys@transformshift{5.970097in}{2.907778in}%
\pgfsys@useobject{currentmarker}{}%
\end{pgfscope}%
\end{pgfscope}%
\begin{pgfscope}%
\pgftext[x=5.970097in,y=2.810556in,,top]{\sffamily\fontsize{10.000000}{12.000000}\selectfont \(\displaystyle 200000\)}%
\end{pgfscope}%
\begin{pgfscope}%
\pgftext[x=3.521070in,y=2.620587in,,top]{\sffamily\fontsize{10.000000}{12.000000}\selectfont Timestep}%
\end{pgfscope}%
\begin{pgfscope}%
\pgfsetbuttcap%
\pgfsetroundjoin%
\definecolor{currentfill}{rgb}{0.000000,0.000000,0.000000}%
\pgfsetfillcolor{currentfill}%
\pgfsetlinewidth{0.803000pt}%
\definecolor{currentstroke}{rgb}{0.000000,0.000000,0.000000}%
\pgfsetstrokecolor{currentstroke}%
\pgfsetdash{}{0pt}%
\pgfsys@defobject{currentmarker}{\pgfqpoint{-0.048611in}{0.000000in}}{\pgfqpoint{0.000000in}{0.000000in}}{%
\pgfpathmoveto{\pgfqpoint{0.000000in}{0.000000in}}%
\pgfpathlineto{\pgfqpoint{-0.048611in}{0.000000in}}%
\pgfusepath{stroke,fill}%
}%
\begin{pgfscope}%
\pgfsys@transformshift{0.827140in}{3.090080in}%
\pgfsys@useobject{currentmarker}{}%
\end{pgfscope}%
\end{pgfscope}%
\begin{pgfscope}%
\pgftext[x=0.591029in,y=3.037318in,left,base]{\sffamily\fontsize{10.000000}{12.000000}\selectfont \(\displaystyle 10\)}%
\end{pgfscope}%
\begin{pgfscope}%
\pgfsetbuttcap%
\pgfsetroundjoin%
\definecolor{currentfill}{rgb}{0.000000,0.000000,0.000000}%
\pgfsetfillcolor{currentfill}%
\pgfsetlinewidth{0.803000pt}%
\definecolor{currentstroke}{rgb}{0.000000,0.000000,0.000000}%
\pgfsetstrokecolor{currentstroke}%
\pgfsetdash{}{0pt}%
\pgfsys@defobject{currentmarker}{\pgfqpoint{-0.048611in}{0.000000in}}{\pgfqpoint{0.000000in}{0.000000in}}{%
\pgfpathmoveto{\pgfqpoint{0.000000in}{0.000000in}}%
\pgfpathlineto{\pgfqpoint{-0.048611in}{0.000000in}}%
\pgfusepath{stroke,fill}%
}%
\begin{pgfscope}%
\pgfsys@transformshift{0.827140in}{3.451179in}%
\pgfsys@useobject{currentmarker}{}%
\end{pgfscope}%
\end{pgfscope}%
\begin{pgfscope}%
\pgftext[x=0.591029in,y=3.398418in,left,base]{\sffamily\fontsize{10.000000}{12.000000}\selectfont \(\displaystyle 20\)}%
\end{pgfscope}%
\begin{pgfscope}%
\pgfsetbuttcap%
\pgfsetroundjoin%
\definecolor{currentfill}{rgb}{0.000000,0.000000,0.000000}%
\pgfsetfillcolor{currentfill}%
\pgfsetlinewidth{0.803000pt}%
\definecolor{currentstroke}{rgb}{0.000000,0.000000,0.000000}%
\pgfsetstrokecolor{currentstroke}%
\pgfsetdash{}{0pt}%
\pgfsys@defobject{currentmarker}{\pgfqpoint{-0.048611in}{0.000000in}}{\pgfqpoint{0.000000in}{0.000000in}}{%
\pgfpathmoveto{\pgfqpoint{0.000000in}{0.000000in}}%
\pgfpathlineto{\pgfqpoint{-0.048611in}{0.000000in}}%
\pgfusepath{stroke,fill}%
}%
\begin{pgfscope}%
\pgfsys@transformshift{0.827140in}{3.812279in}%
\pgfsys@useobject{currentmarker}{}%
\end{pgfscope}%
\end{pgfscope}%
\begin{pgfscope}%
\pgftext[x=0.591029in,y=3.759517in,left,base]{\sffamily\fontsize{10.000000}{12.000000}\selectfont \(\displaystyle 30\)}%
\end{pgfscope}%
\begin{pgfscope}%
\pgfsetbuttcap%
\pgfsetroundjoin%
\definecolor{currentfill}{rgb}{0.000000,0.000000,0.000000}%
\pgfsetfillcolor{currentfill}%
\pgfsetlinewidth{0.803000pt}%
\definecolor{currentstroke}{rgb}{0.000000,0.000000,0.000000}%
\pgfsetstrokecolor{currentstroke}%
\pgfsetdash{}{0pt}%
\pgfsys@defobject{currentmarker}{\pgfqpoint{-0.048611in}{0.000000in}}{\pgfqpoint{0.000000in}{0.000000in}}{%
\pgfpathmoveto{\pgfqpoint{0.000000in}{0.000000in}}%
\pgfpathlineto{\pgfqpoint{-0.048611in}{0.000000in}}%
\pgfusepath{stroke,fill}%
}%
\begin{pgfscope}%
\pgfsys@transformshift{0.827140in}{4.173378in}%
\pgfsys@useobject{currentmarker}{}%
\end{pgfscope}%
\end{pgfscope}%
\begin{pgfscope}%
\pgftext[x=0.591029in,y=4.120616in,left,base]{\sffamily\fontsize{10.000000}{12.000000}\selectfont \(\displaystyle 40\)}%
\end{pgfscope}%
\begin{pgfscope}%
\pgftext[x=0.535473in,y=3.667222in,,bottom,rotate=90.000000]{\sffamily\fontsize{10.000000}{12.000000}\selectfont Average Return}%
\end{pgfscope}%
\begin{pgfscope}%
\pgfpathrectangle{\pgfqpoint{0.827140in}{2.907778in}}{\pgfqpoint{5.387860in}{1.518889in}}%
\pgfusepath{clip}%
\pgfsetrectcap%
\pgfsetroundjoin%
\pgfsetlinewidth{1.505625pt}%
\definecolor{currentstroke}{rgb}{0.000000,0.000000,0.000000}%
\pgfsetstrokecolor{currentstroke}%
\pgfsetdash{}{0pt}%
\pgfpathmoveto{\pgfqpoint{1.072043in}{3.089967in}}%
\pgfpathlineto{\pgfqpoint{1.121518in}{3.326625in}}%
\pgfpathlineto{\pgfqpoint{1.170993in}{3.280462in}}%
\pgfpathlineto{\pgfqpoint{1.220469in}{3.051276in}}%
\pgfpathlineto{\pgfqpoint{1.269944in}{3.530676in}}%
\pgfpathlineto{\pgfqpoint{1.319419in}{3.399974in}}%
\pgfpathlineto{\pgfqpoint{1.368895in}{3.601444in}}%
\pgfpathlineto{\pgfqpoint{1.418370in}{3.559872in}}%
\pgfpathlineto{\pgfqpoint{1.467845in}{3.641191in}}%
\pgfpathlineto{\pgfqpoint{1.517320in}{3.523331in}}%
\pgfpathlineto{\pgfqpoint{1.566796in}{3.607597in}}%
\pgfpathlineto{\pgfqpoint{1.616271in}{3.546747in}}%
\pgfpathlineto{\pgfqpoint{1.665746in}{3.579112in}}%
\pgfpathlineto{\pgfqpoint{1.715222in}{3.654051in}}%
\pgfpathlineto{\pgfqpoint{1.764697in}{3.706667in}}%
\pgfpathlineto{\pgfqpoint{1.814172in}{3.695815in}}%
\pgfpathlineto{\pgfqpoint{1.863648in}{3.692529in}}%
\pgfpathlineto{\pgfqpoint{1.913123in}{3.875775in}}%
\pgfpathlineto{\pgfqpoint{1.962598in}{3.764031in}}%
\pgfpathlineto{\pgfqpoint{2.012073in}{3.888618in}}%
\pgfpathlineto{\pgfqpoint{2.086286in}{3.814992in}}%
\pgfpathlineto{\pgfqpoint{2.135762in}{3.980908in}}%
\pgfpathlineto{\pgfqpoint{2.185237in}{3.884774in}}%
\pgfpathlineto{\pgfqpoint{2.234712in}{3.899734in}}%
\pgfpathlineto{\pgfqpoint{2.284188in}{3.969384in}}%
\pgfpathlineto{\pgfqpoint{2.333663in}{3.973683in}}%
\pgfpathlineto{\pgfqpoint{2.383138in}{3.941120in}}%
\pgfpathlineto{\pgfqpoint{2.432613in}{3.937203in}}%
\pgfpathlineto{\pgfqpoint{2.482089in}{4.023017in}}%
\pgfpathlineto{\pgfqpoint{2.531564in}{4.008590in}}%
\pgfpathlineto{\pgfqpoint{2.581039in}{4.003596in}}%
\pgfpathlineto{\pgfqpoint{2.630515in}{4.027798in}}%
\pgfpathlineto{\pgfqpoint{2.679990in}{4.010362in}}%
\pgfpathlineto{\pgfqpoint{2.729465in}{3.994907in}}%
\pgfpathlineto{\pgfqpoint{2.778941in}{4.013084in}}%
\pgfpathlineto{\pgfqpoint{2.828416in}{4.018608in}}%
\pgfpathlineto{\pgfqpoint{2.877891in}{4.006856in}}%
\pgfpathlineto{\pgfqpoint{2.927366in}{4.039902in}}%
\pgfpathlineto{\pgfqpoint{2.976842in}{3.983786in}}%
\pgfpathlineto{\pgfqpoint{3.026317in}{4.075968in}}%
\pgfpathlineto{\pgfqpoint{3.075792in}{4.053147in}}%
\pgfpathlineto{\pgfqpoint{3.150005in}{4.036773in}}%
\pgfpathlineto{\pgfqpoint{3.199481in}{4.060205in}}%
\pgfpathlineto{\pgfqpoint{3.248956in}{4.052788in}}%
\pgfpathlineto{\pgfqpoint{3.298431in}{4.041756in}}%
\pgfpathlineto{\pgfqpoint{3.347906in}{4.130818in}}%
\pgfpathlineto{\pgfqpoint{3.397382in}{4.039006in}}%
\pgfpathlineto{\pgfqpoint{3.446857in}{4.063764in}}%
\pgfpathlineto{\pgfqpoint{3.496332in}{4.081241in}}%
\pgfpathlineto{\pgfqpoint{3.545808in}{4.126738in}}%
\pgfpathlineto{\pgfqpoint{3.595283in}{4.129463in}}%
\pgfpathlineto{\pgfqpoint{3.644758in}{4.121546in}}%
\pgfpathlineto{\pgfqpoint{3.694234in}{4.146750in}}%
\pgfpathlineto{\pgfqpoint{3.743709in}{4.067236in}}%
\pgfpathlineto{\pgfqpoint{3.793184in}{4.057497in}}%
\pgfpathlineto{\pgfqpoint{3.842659in}{4.077150in}}%
\pgfpathlineto{\pgfqpoint{3.892135in}{4.117834in}}%
\pgfpathlineto{\pgfqpoint{3.941610in}{4.096051in}}%
\pgfpathlineto{\pgfqpoint{3.991085in}{4.135020in}}%
\pgfpathlineto{\pgfqpoint{4.040561in}{4.120122in}}%
\pgfpathlineto{\pgfqpoint{4.090036in}{4.103301in}}%
\pgfpathlineto{\pgfqpoint{4.139511in}{4.121318in}}%
\pgfpathlineto{\pgfqpoint{4.213724in}{4.118540in}}%
\pgfpathlineto{\pgfqpoint{4.263200in}{4.151868in}}%
\pgfpathlineto{\pgfqpoint{4.312675in}{4.142352in}}%
\pgfpathlineto{\pgfqpoint{4.362150in}{4.202047in}}%
\pgfpathlineto{\pgfqpoint{4.411625in}{4.112187in}}%
\pgfpathlineto{\pgfqpoint{4.461101in}{4.135816in}}%
\pgfpathlineto{\pgfqpoint{4.510576in}{4.136383in}}%
\pgfpathlineto{\pgfqpoint{4.560051in}{4.141512in}}%
\pgfpathlineto{\pgfqpoint{4.609527in}{4.106873in}}%
\pgfpathlineto{\pgfqpoint{4.659002in}{4.075298in}}%
\pgfpathlineto{\pgfqpoint{4.708477in}{4.143858in}}%
\pgfpathlineto{\pgfqpoint{4.757952in}{4.101981in}}%
\pgfpathlineto{\pgfqpoint{4.807428in}{4.170822in}}%
\pgfpathlineto{\pgfqpoint{4.856903in}{4.151274in}}%
\pgfpathlineto{\pgfqpoint{4.906378in}{4.171882in}}%
\pgfpathlineto{\pgfqpoint{4.955854in}{4.252809in}}%
\pgfpathlineto{\pgfqpoint{5.005329in}{4.193055in}}%
\pgfpathlineto{\pgfqpoint{5.054804in}{4.231077in}}%
\pgfpathlineto{\pgfqpoint{5.104280in}{4.133099in}}%
\pgfpathlineto{\pgfqpoint{5.153755in}{4.150789in}}%
\pgfpathlineto{\pgfqpoint{5.203230in}{4.179286in}}%
\pgfpathlineto{\pgfqpoint{5.277443in}{4.224952in}}%
\pgfpathlineto{\pgfqpoint{5.326918in}{4.143213in}}%
\pgfpathlineto{\pgfqpoint{5.376394in}{4.201554in}}%
\pgfpathlineto{\pgfqpoint{5.425869in}{4.220991in}}%
\pgfpathlineto{\pgfqpoint{5.475344in}{4.184106in}}%
\pgfpathlineto{\pgfqpoint{5.524820in}{4.291053in}}%
\pgfpathlineto{\pgfqpoint{5.574295in}{4.286278in}}%
\pgfpathlineto{\pgfqpoint{5.623770in}{4.241376in}}%
\pgfpathlineto{\pgfqpoint{5.673245in}{4.237307in}}%
\pgfpathlineto{\pgfqpoint{5.722721in}{4.262062in}}%
\pgfpathlineto{\pgfqpoint{5.772196in}{4.320505in}}%
\pgfpathlineto{\pgfqpoint{5.821671in}{4.252341in}}%
\pgfpathlineto{\pgfqpoint{5.871147in}{4.244027in}}%
\pgfpathlineto{\pgfqpoint{5.920622in}{4.294836in}}%
\pgfpathlineto{\pgfqpoint{5.970097in}{4.243085in}}%
\pgfusepath{stroke}%
\end{pgfscope}%
\begin{pgfscope}%
\pgfpathrectangle{\pgfqpoint{0.827140in}{2.907778in}}{\pgfqpoint{5.387860in}{1.518889in}}%
\pgfusepath{clip}%
\pgfsetbuttcap%
\pgfsetroundjoin%
\pgfsetlinewidth{1.505625pt}%
\definecolor{currentstroke}{rgb}{0.000000,0.000000,0.000000}%
\pgfsetstrokecolor{currentstroke}%
\pgfsetdash{{5.550000pt}{2.400000pt}}{0.000000pt}%
\pgfpathmoveto{\pgfqpoint{1.072043in}{3.089967in}}%
\pgfpathlineto{\pgfqpoint{1.121518in}{3.152850in}}%
\pgfpathlineto{\pgfqpoint{1.170993in}{3.473419in}}%
\pgfpathlineto{\pgfqpoint{1.220469in}{2.976818in}}%
\pgfpathlineto{\pgfqpoint{1.269944in}{3.488501in}}%
\pgfpathlineto{\pgfqpoint{1.319419in}{3.393493in}}%
\pgfpathlineto{\pgfqpoint{1.368895in}{3.596414in}}%
\pgfpathlineto{\pgfqpoint{1.418370in}{3.564554in}}%
\pgfpathlineto{\pgfqpoint{1.467845in}{3.637939in}}%
\pgfpathlineto{\pgfqpoint{1.517320in}{3.520340in}}%
\pgfpathlineto{\pgfqpoint{1.566796in}{3.596686in}}%
\pgfpathlineto{\pgfqpoint{1.616271in}{3.545896in}}%
\pgfpathlineto{\pgfqpoint{1.665746in}{3.602395in}}%
\pgfpathlineto{\pgfqpoint{1.715222in}{3.662956in}}%
\pgfpathlineto{\pgfqpoint{1.764697in}{3.711456in}}%
\pgfpathlineto{\pgfqpoint{1.814172in}{3.711266in}}%
\pgfpathlineto{\pgfqpoint{1.863648in}{3.690939in}}%
\pgfpathlineto{\pgfqpoint{1.913123in}{3.844326in}}%
\pgfpathlineto{\pgfqpoint{1.962598in}{3.753779in}}%
\pgfpathlineto{\pgfqpoint{2.012073in}{3.871485in}}%
\pgfpathlineto{\pgfqpoint{2.086286in}{3.773251in}}%
\pgfpathlineto{\pgfqpoint{2.135762in}{3.846174in}}%
\pgfpathlineto{\pgfqpoint{2.185237in}{3.919052in}}%
\pgfpathlineto{\pgfqpoint{2.234712in}{3.865140in}}%
\pgfpathlineto{\pgfqpoint{2.284188in}{3.948426in}}%
\pgfpathlineto{\pgfqpoint{2.333663in}{3.938540in}}%
\pgfpathlineto{\pgfqpoint{2.383138in}{3.944314in}}%
\pgfpathlineto{\pgfqpoint{2.432613in}{3.922280in}}%
\pgfpathlineto{\pgfqpoint{2.482089in}{4.030739in}}%
\pgfpathlineto{\pgfqpoint{2.531564in}{4.007788in}}%
\pgfpathlineto{\pgfqpoint{2.581039in}{3.997903in}}%
\pgfpathlineto{\pgfqpoint{2.630515in}{4.018089in}}%
\pgfpathlineto{\pgfqpoint{2.679990in}{4.016531in}}%
\pgfpathlineto{\pgfqpoint{2.729465in}{3.998839in}}%
\pgfpathlineto{\pgfqpoint{2.778941in}{4.069474in}}%
\pgfpathlineto{\pgfqpoint{2.828416in}{4.009062in}}%
\pgfpathlineto{\pgfqpoint{2.877891in}{4.007461in}}%
\pgfpathlineto{\pgfqpoint{2.927366in}{4.058331in}}%
\pgfpathlineto{\pgfqpoint{2.976842in}{3.986426in}}%
\pgfpathlineto{\pgfqpoint{3.026317in}{4.094353in}}%
\pgfpathlineto{\pgfqpoint{3.075792in}{4.056319in}}%
\pgfpathlineto{\pgfqpoint{3.150005in}{4.023854in}}%
\pgfpathlineto{\pgfqpoint{3.199481in}{4.067352in}}%
\pgfpathlineto{\pgfqpoint{3.248956in}{4.062768in}}%
\pgfpathlineto{\pgfqpoint{3.298431in}{4.060648in}}%
\pgfpathlineto{\pgfqpoint{3.347906in}{4.137604in}}%
\pgfpathlineto{\pgfqpoint{3.397382in}{4.039769in}}%
\pgfpathlineto{\pgfqpoint{3.446857in}{4.055076in}}%
\pgfpathlineto{\pgfqpoint{3.496332in}{4.086129in}}%
\pgfpathlineto{\pgfqpoint{3.545808in}{4.130070in}}%
\pgfpathlineto{\pgfqpoint{3.595283in}{4.131462in}}%
\pgfpathlineto{\pgfqpoint{3.644758in}{4.157708in}}%
\pgfpathlineto{\pgfqpoint{3.694234in}{4.126187in}}%
\pgfpathlineto{\pgfqpoint{3.743709in}{4.059631in}}%
\pgfpathlineto{\pgfqpoint{3.793184in}{4.050118in}}%
\pgfpathlineto{\pgfqpoint{3.842659in}{3.984551in}}%
\pgfpathlineto{\pgfqpoint{3.892135in}{4.107318in}}%
\pgfpathlineto{\pgfqpoint{3.941610in}{4.101636in}}%
\pgfpathlineto{\pgfqpoint{3.991085in}{4.141099in}}%
\pgfpathlineto{\pgfqpoint{4.040561in}{4.132042in}}%
\pgfpathlineto{\pgfqpoint{4.090036in}{4.097782in}}%
\pgfpathlineto{\pgfqpoint{4.139511in}{4.134974in}}%
\pgfpathlineto{\pgfqpoint{4.213724in}{4.143068in}}%
\pgfpathlineto{\pgfqpoint{4.263200in}{4.176733in}}%
\pgfpathlineto{\pgfqpoint{4.312675in}{4.163517in}}%
\pgfpathlineto{\pgfqpoint{4.362150in}{4.209399in}}%
\pgfpathlineto{\pgfqpoint{4.411625in}{4.120101in}}%
\pgfpathlineto{\pgfqpoint{4.461101in}{4.153654in}}%
\pgfpathlineto{\pgfqpoint{4.510576in}{4.173275in}}%
\pgfpathlineto{\pgfqpoint{4.560051in}{4.171911in}}%
\pgfpathlineto{\pgfqpoint{4.609527in}{4.124501in}}%
\pgfpathlineto{\pgfqpoint{4.659002in}{4.108895in}}%
\pgfpathlineto{\pgfqpoint{4.708477in}{4.179985in}}%
\pgfpathlineto{\pgfqpoint{4.757952in}{4.141930in}}%
\pgfpathlineto{\pgfqpoint{4.807428in}{4.213040in}}%
\pgfpathlineto{\pgfqpoint{4.856903in}{4.190927in}}%
\pgfpathlineto{\pgfqpoint{4.906378in}{4.179646in}}%
\pgfpathlineto{\pgfqpoint{4.955854in}{4.261925in}}%
\pgfpathlineto{\pgfqpoint{5.005329in}{4.238752in}}%
\pgfpathlineto{\pgfqpoint{5.054804in}{4.232946in}}%
\pgfpathlineto{\pgfqpoint{5.104280in}{4.152639in}}%
\pgfpathlineto{\pgfqpoint{5.153755in}{4.199681in}}%
\pgfpathlineto{\pgfqpoint{5.203230in}{4.223420in}}%
\pgfpathlineto{\pgfqpoint{5.277443in}{4.238710in}}%
\pgfpathlineto{\pgfqpoint{5.326918in}{4.156258in}}%
\pgfpathlineto{\pgfqpoint{5.376394in}{4.184227in}}%
\pgfpathlineto{\pgfqpoint{5.425869in}{4.313446in}}%
\pgfpathlineto{\pgfqpoint{5.475344in}{4.272134in}}%
\pgfpathlineto{\pgfqpoint{5.524820in}{4.357626in}}%
\pgfpathlineto{\pgfqpoint{5.574295in}{4.322768in}}%
\pgfpathlineto{\pgfqpoint{5.623770in}{4.285235in}}%
\pgfpathlineto{\pgfqpoint{5.673245in}{4.356655in}}%
\pgfpathlineto{\pgfqpoint{5.722721in}{4.319134in}}%
\pgfpathlineto{\pgfqpoint{5.772196in}{4.355339in}}%
\pgfpathlineto{\pgfqpoint{5.821671in}{4.300642in}}%
\pgfpathlineto{\pgfqpoint{5.871147in}{4.308055in}}%
\pgfpathlineto{\pgfqpoint{5.920622in}{4.341176in}}%
\pgfpathlineto{\pgfqpoint{5.970097in}{4.320909in}}%
\pgfusepath{stroke}%
\end{pgfscope}%
\begin{pgfscope}%
\pgfsetrectcap%
\pgfsetmiterjoin%
\pgfsetlinewidth{0.803000pt}%
\definecolor{currentstroke}{rgb}{0.000000,0.000000,0.000000}%
\pgfsetstrokecolor{currentstroke}%
\pgfsetdash{}{0pt}%
\pgfpathmoveto{\pgfqpoint{0.827140in}{2.907778in}}%
\pgfpathlineto{\pgfqpoint{0.827140in}{4.426667in}}%
\pgfusepath{stroke}%
\end{pgfscope}%
\begin{pgfscope}%
\pgfsetrectcap%
\pgfsetmiterjoin%
\pgfsetlinewidth{0.803000pt}%
\definecolor{currentstroke}{rgb}{0.000000,0.000000,0.000000}%
\pgfsetstrokecolor{currentstroke}%
\pgfsetdash{}{0pt}%
\pgfpathmoveto{\pgfqpoint{6.215000in}{2.907778in}}%
\pgfpathlineto{\pgfqpoint{6.215000in}{4.426667in}}%
\pgfusepath{stroke}%
\end{pgfscope}%
\begin{pgfscope}%
\pgfsetrectcap%
\pgfsetmiterjoin%
\pgfsetlinewidth{0.803000pt}%
\definecolor{currentstroke}{rgb}{0.000000,0.000000,0.000000}%
\pgfsetstrokecolor{currentstroke}%
\pgfsetdash{}{0pt}%
\pgfpathmoveto{\pgfqpoint{0.827140in}{2.907778in}}%
\pgfpathlineto{\pgfqpoint{6.215000in}{2.907778in}}%
\pgfusepath{stroke}%
\end{pgfscope}%
\begin{pgfscope}%
\pgfsetrectcap%
\pgfsetmiterjoin%
\pgfsetlinewidth{0.803000pt}%
\definecolor{currentstroke}{rgb}{0.000000,0.000000,0.000000}%
\pgfsetstrokecolor{currentstroke}%
\pgfsetdash{}{0pt}%
\pgfpathmoveto{\pgfqpoint{0.827140in}{4.426667in}}%
\pgfpathlineto{\pgfqpoint{6.215000in}{4.426667in}}%
\pgfusepath{stroke}%
\end{pgfscope}%
\begin{pgfscope}%
\pgftext[x=3.521070in,y=4.510000in,,base]{\sffamily\fontsize{12.000000}{14.400000}\selectfont Performance}%
\end{pgfscope}%
\begin{pgfscope}%
\pgfsetbuttcap%
\pgfsetmiterjoin%
\definecolor{currentfill}{rgb}{1.000000,1.000000,1.000000}%
\pgfsetfillcolor{currentfill}%
\pgfsetfillopacity{0.800000}%
\pgfsetlinewidth{1.003750pt}%
\definecolor{currentstroke}{rgb}{0.800000,0.800000,0.800000}%
\pgfsetstrokecolor{currentstroke}%
\pgfsetstrokeopacity{0.800000}%
\pgfsetdash{}{0pt}%
\pgfpathmoveto{\pgfqpoint{4.751906in}{2.977222in}}%
\pgfpathlineto{\pgfqpoint{6.117778in}{2.977222in}}%
\pgfpathquadraticcurveto{\pgfqpoint{6.145556in}{2.977222in}}{\pgfqpoint{6.145556in}{3.005000in}}%
\pgfpathlineto{\pgfqpoint{6.145556in}{3.398826in}}%
\pgfpathquadraticcurveto{\pgfqpoint{6.145556in}{3.426603in}}{\pgfqpoint{6.117778in}{3.426603in}}%
\pgfpathlineto{\pgfqpoint{4.751906in}{3.426603in}}%
\pgfpathquadraticcurveto{\pgfqpoint{4.724128in}{3.426603in}}{\pgfqpoint{4.724128in}{3.398826in}}%
\pgfpathlineto{\pgfqpoint{4.724128in}{3.005000in}}%
\pgfpathquadraticcurveto{\pgfqpoint{4.724128in}{2.977222in}}{\pgfqpoint{4.751906in}{2.977222in}}%
\pgfpathclose%
\pgfusepath{stroke,fill}%
\end{pgfscope}%
\begin{pgfscope}%
\pgfsetrectcap%
\pgfsetroundjoin%
\pgfsetlinewidth{1.505625pt}%
\definecolor{currentstroke}{rgb}{0.000000,0.000000,0.000000}%
\pgfsetstrokecolor{currentstroke}%
\pgfsetdash{}{0pt}%
\pgfpathmoveto{\pgfqpoint{4.779684in}{3.314136in}}%
\pgfpathlineto{\pgfqpoint{5.057462in}{3.314136in}}%
\pgfusepath{stroke}%
\end{pgfscope}%
\begin{pgfscope}%
\pgftext[x=5.168573in,y=3.265525in,left,base]{\sffamily\fontsize{10.000000}{12.000000}\selectfont control}%
\end{pgfscope}%
\begin{pgfscope}%
\pgfsetbuttcap%
\pgfsetroundjoin%
\pgfsetlinewidth{1.505625pt}%
\definecolor{currentstroke}{rgb}{0.000000,0.000000,0.000000}%
\pgfsetstrokecolor{currentstroke}%
\pgfsetdash{{5.550000pt}{2.400000pt}}{0.000000pt}%
\pgfpathmoveto{\pgfqpoint{4.779684in}{3.110279in}}%
\pgfpathlineto{\pgfqpoint{5.057462in}{3.110279in}}%
\pgfusepath{stroke}%
\end{pgfscope}%
\begin{pgfscope}%
\pgftext[x=5.168573in,y=3.061667in,left,base]{\sffamily\fontsize{10.000000}{12.000000}\selectfont experimental}%
\end{pgfscope}%
\begin{pgfscope}%
\pgfsetbuttcap%
\pgfsetmiterjoin%
\definecolor{currentfill}{rgb}{1.000000,1.000000,1.000000}%
\pgfsetfillcolor{currentfill}%
\pgfsetlinewidth{0.000000pt}%
\definecolor{currentstroke}{rgb}{0.000000,0.000000,0.000000}%
\pgfsetstrokecolor{currentstroke}%
\pgfsetstrokeopacity{0.000000}%
\pgfsetdash{}{0pt}%
\pgfpathmoveto{\pgfqpoint{0.827140in}{0.582778in}}%
\pgfpathlineto{\pgfqpoint{6.215000in}{0.582778in}}%
\pgfpathlineto{\pgfqpoint{6.215000in}{2.101667in}}%
\pgfpathlineto{\pgfqpoint{0.827140in}{2.101667in}}%
\pgfpathclose%
\pgfusepath{fill}%
\end{pgfscope}%
\begin{pgfscope}%
\pgfsetbuttcap%
\pgfsetroundjoin%
\definecolor{currentfill}{rgb}{0.000000,0.000000,0.000000}%
\pgfsetfillcolor{currentfill}%
\pgfsetlinewidth{0.803000pt}%
\definecolor{currentstroke}{rgb}{0.000000,0.000000,0.000000}%
\pgfsetstrokecolor{currentstroke}%
\pgfsetdash{}{0pt}%
\pgfsys@defobject{currentmarker}{\pgfqpoint{0.000000in}{-0.048611in}}{\pgfqpoint{0.000000in}{0.000000in}}{%
\pgfpathmoveto{\pgfqpoint{0.000000in}{0.000000in}}%
\pgfpathlineto{\pgfqpoint{0.000000in}{-0.048611in}}%
\pgfusepath{stroke,fill}%
}%
\begin{pgfscope}%
\pgfsys@transformshift{1.021547in}{0.582778in}%
\pgfsys@useobject{currentmarker}{}%
\end{pgfscope}%
\end{pgfscope}%
\begin{pgfscope}%
\pgftext[x=1.021547in,y=0.485556in,,top]{\sffamily\fontsize{10.000000}{12.000000}\selectfont \(\displaystyle 0\)}%
\end{pgfscope}%
\begin{pgfscope}%
\pgfsetbuttcap%
\pgfsetroundjoin%
\definecolor{currentfill}{rgb}{0.000000,0.000000,0.000000}%
\pgfsetfillcolor{currentfill}%
\pgfsetlinewidth{0.803000pt}%
\definecolor{currentstroke}{rgb}{0.000000,0.000000,0.000000}%
\pgfsetstrokecolor{currentstroke}%
\pgfsetdash{}{0pt}%
\pgfsys@defobject{currentmarker}{\pgfqpoint{0.000000in}{-0.048611in}}{\pgfqpoint{0.000000in}{0.000000in}}{%
\pgfpathmoveto{\pgfqpoint{0.000000in}{0.000000in}}%
\pgfpathlineto{\pgfqpoint{0.000000in}{-0.048611in}}%
\pgfusepath{stroke,fill}%
}%
\begin{pgfscope}%
\pgfsys@transformshift{2.031456in}{0.582778in}%
\pgfsys@useobject{currentmarker}{}%
\end{pgfscope}%
\end{pgfscope}%
\begin{pgfscope}%
\pgftext[x=2.031456in,y=0.485556in,,top]{\sffamily\fontsize{10.000000}{12.000000}\selectfont \(\displaystyle 20\)}%
\end{pgfscope}%
\begin{pgfscope}%
\pgfsetbuttcap%
\pgfsetroundjoin%
\definecolor{currentfill}{rgb}{0.000000,0.000000,0.000000}%
\pgfsetfillcolor{currentfill}%
\pgfsetlinewidth{0.803000pt}%
\definecolor{currentstroke}{rgb}{0.000000,0.000000,0.000000}%
\pgfsetstrokecolor{currentstroke}%
\pgfsetdash{}{0pt}%
\pgfsys@defobject{currentmarker}{\pgfqpoint{0.000000in}{-0.048611in}}{\pgfqpoint{0.000000in}{0.000000in}}{%
\pgfpathmoveto{\pgfqpoint{0.000000in}{0.000000in}}%
\pgfpathlineto{\pgfqpoint{0.000000in}{-0.048611in}}%
\pgfusepath{stroke,fill}%
}%
\begin{pgfscope}%
\pgfsys@transformshift{3.041364in}{0.582778in}%
\pgfsys@useobject{currentmarker}{}%
\end{pgfscope}%
\end{pgfscope}%
\begin{pgfscope}%
\pgftext[x=3.041364in,y=0.485556in,,top]{\sffamily\fontsize{10.000000}{12.000000}\selectfont \(\displaystyle 40\)}%
\end{pgfscope}%
\begin{pgfscope}%
\pgfsetbuttcap%
\pgfsetroundjoin%
\definecolor{currentfill}{rgb}{0.000000,0.000000,0.000000}%
\pgfsetfillcolor{currentfill}%
\pgfsetlinewidth{0.803000pt}%
\definecolor{currentstroke}{rgb}{0.000000,0.000000,0.000000}%
\pgfsetstrokecolor{currentstroke}%
\pgfsetdash{}{0pt}%
\pgfsys@defobject{currentmarker}{\pgfqpoint{0.000000in}{-0.048611in}}{\pgfqpoint{0.000000in}{0.000000in}}{%
\pgfpathmoveto{\pgfqpoint{0.000000in}{0.000000in}}%
\pgfpathlineto{\pgfqpoint{0.000000in}{-0.048611in}}%
\pgfusepath{stroke,fill}%
}%
\begin{pgfscope}%
\pgfsys@transformshift{4.051272in}{0.582778in}%
\pgfsys@useobject{currentmarker}{}%
\end{pgfscope}%
\end{pgfscope}%
\begin{pgfscope}%
\pgftext[x=4.051272in,y=0.485556in,,top]{\sffamily\fontsize{10.000000}{12.000000}\selectfont \(\displaystyle 60\)}%
\end{pgfscope}%
\begin{pgfscope}%
\pgfsetbuttcap%
\pgfsetroundjoin%
\definecolor{currentfill}{rgb}{0.000000,0.000000,0.000000}%
\pgfsetfillcolor{currentfill}%
\pgfsetlinewidth{0.803000pt}%
\definecolor{currentstroke}{rgb}{0.000000,0.000000,0.000000}%
\pgfsetstrokecolor{currentstroke}%
\pgfsetdash{}{0pt}%
\pgfsys@defobject{currentmarker}{\pgfqpoint{0.000000in}{-0.048611in}}{\pgfqpoint{0.000000in}{0.000000in}}{%
\pgfpathmoveto{\pgfqpoint{0.000000in}{0.000000in}}%
\pgfpathlineto{\pgfqpoint{0.000000in}{-0.048611in}}%
\pgfusepath{stroke,fill}%
}%
\begin{pgfscope}%
\pgfsys@transformshift{5.061180in}{0.582778in}%
\pgfsys@useobject{currentmarker}{}%
\end{pgfscope}%
\end{pgfscope}%
\begin{pgfscope}%
\pgftext[x=5.061180in,y=0.485556in,,top]{\sffamily\fontsize{10.000000}{12.000000}\selectfont \(\displaystyle 80\)}%
\end{pgfscope}%
\begin{pgfscope}%
\pgfsetbuttcap%
\pgfsetroundjoin%
\definecolor{currentfill}{rgb}{0.000000,0.000000,0.000000}%
\pgfsetfillcolor{currentfill}%
\pgfsetlinewidth{0.803000pt}%
\definecolor{currentstroke}{rgb}{0.000000,0.000000,0.000000}%
\pgfsetstrokecolor{currentstroke}%
\pgfsetdash{}{0pt}%
\pgfsys@defobject{currentmarker}{\pgfqpoint{0.000000in}{-0.048611in}}{\pgfqpoint{0.000000in}{0.000000in}}{%
\pgfpathmoveto{\pgfqpoint{0.000000in}{0.000000in}}%
\pgfpathlineto{\pgfqpoint{0.000000in}{-0.048611in}}%
\pgfusepath{stroke,fill}%
}%
\begin{pgfscope}%
\pgfsys@transformshift{6.071088in}{0.582778in}%
\pgfsys@useobject{currentmarker}{}%
\end{pgfscope}%
\end{pgfscope}%
\begin{pgfscope}%
\pgftext[x=6.071088in,y=0.485556in,,top]{\sffamily\fontsize{10.000000}{12.000000}\selectfont \(\displaystyle 100\)}%
\end{pgfscope}%
\begin{pgfscope}%
\pgftext[x=3.521070in,y=0.295587in,,top]{\sffamily\fontsize{10.000000}{12.000000}\selectfont Number of Iterations}%
\end{pgfscope}%
\begin{pgfscope}%
\pgfsetbuttcap%
\pgfsetroundjoin%
\definecolor{currentfill}{rgb}{0.000000,0.000000,0.000000}%
\pgfsetfillcolor{currentfill}%
\pgfsetlinewidth{0.803000pt}%
\definecolor{currentstroke}{rgb}{0.000000,0.000000,0.000000}%
\pgfsetstrokecolor{currentstroke}%
\pgfsetdash{}{0pt}%
\pgfsys@defobject{currentmarker}{\pgfqpoint{-0.048611in}{0.000000in}}{\pgfqpoint{0.000000in}{0.000000in}}{%
\pgfpathmoveto{\pgfqpoint{0.000000in}{0.000000in}}%
\pgfpathlineto{\pgfqpoint{-0.048611in}{0.000000in}}%
\pgfusepath{stroke,fill}%
}%
\begin{pgfscope}%
\pgfsys@transformshift{0.827140in}{1.124540in}%
\pgfsys@useobject{currentmarker}{}%
\end{pgfscope}%
\end{pgfscope}%
\begin{pgfscope}%
\pgftext[x=0.413559in,y=1.071779in,left,base]{\sffamily\fontsize{10.000000}{12.000000}\selectfont \(\displaystyle 0.005\)}%
\end{pgfscope}%
\begin{pgfscope}%
\pgfsetbuttcap%
\pgfsetroundjoin%
\definecolor{currentfill}{rgb}{0.000000,0.000000,0.000000}%
\pgfsetfillcolor{currentfill}%
\pgfsetlinewidth{0.803000pt}%
\definecolor{currentstroke}{rgb}{0.000000,0.000000,0.000000}%
\pgfsetstrokecolor{currentstroke}%
\pgfsetdash{}{0pt}%
\pgfsys@defobject{currentmarker}{\pgfqpoint{-0.048611in}{0.000000in}}{\pgfqpoint{0.000000in}{0.000000in}}{%
\pgfpathmoveto{\pgfqpoint{0.000000in}{0.000000in}}%
\pgfpathlineto{\pgfqpoint{-0.048611in}{0.000000in}}%
\pgfusepath{stroke,fill}%
}%
\begin{pgfscope}%
\pgfsys@transformshift{0.827140in}{1.694692in}%
\pgfsys@useobject{currentmarker}{}%
\end{pgfscope}%
\end{pgfscope}%
\begin{pgfscope}%
\pgftext[x=0.413559in,y=1.641930in,left,base]{\sffamily\fontsize{10.000000}{12.000000}\selectfont \(\displaystyle 0.010\)}%
\end{pgfscope}%
\begin{pgfscope}%
\pgftext[x=0.358003in,y=1.342222in,,bottom,rotate=90.000000]{\sffamily\fontsize{10.000000}{12.000000}\selectfont Proportional Contribution}%
\end{pgfscope}%
\begin{pgfscope}%
\pgfpathrectangle{\pgfqpoint{0.827140in}{0.582778in}}{\pgfqpoint{5.387860in}{1.518889in}}%
\pgfusepath{clip}%
\pgfsetrectcap%
\pgfsetroundjoin%
\pgfsetlinewidth{1.505625pt}%
\definecolor{currentstroke}{rgb}{0.000000,0.000000,0.000000}%
\pgfsetstrokecolor{currentstroke}%
\pgfsetdash{}{0pt}%
\pgfpathmoveto{\pgfqpoint{1.072043in}{1.372025in}}%
\pgfpathlineto{\pgfqpoint{1.122538in}{1.302922in}}%
\pgfpathlineto{\pgfqpoint{1.173034in}{1.264960in}}%
\pgfpathlineto{\pgfqpoint{1.223529in}{0.954773in}}%
\pgfpathlineto{\pgfqpoint{1.274024in}{0.972798in}}%
\pgfpathlineto{\pgfqpoint{1.324520in}{0.808786in}}%
\pgfpathlineto{\pgfqpoint{1.375015in}{0.762251in}}%
\pgfpathlineto{\pgfqpoint{1.425511in}{0.764161in}}%
\pgfpathlineto{\pgfqpoint{1.476006in}{0.722031in}}%
\pgfpathlineto{\pgfqpoint{1.526501in}{1.049198in}}%
\pgfpathlineto{\pgfqpoint{1.576997in}{0.893326in}}%
\pgfpathlineto{\pgfqpoint{1.627492in}{1.025931in}}%
\pgfpathlineto{\pgfqpoint{1.677988in}{0.900178in}}%
\pgfpathlineto{\pgfqpoint{1.728483in}{1.293020in}}%
\pgfpathlineto{\pgfqpoint{1.778978in}{0.876992in}}%
\pgfpathlineto{\pgfqpoint{1.829474in}{0.987327in}}%
\pgfpathlineto{\pgfqpoint{1.879969in}{0.850491in}}%
\pgfpathlineto{\pgfqpoint{1.930465in}{1.140981in}}%
\pgfpathlineto{\pgfqpoint{1.980960in}{0.933753in}}%
\pgfpathlineto{\pgfqpoint{2.031456in}{1.154901in}}%
\pgfpathlineto{\pgfqpoint{2.081951in}{0.917704in}}%
\pgfpathlineto{\pgfqpoint{2.132446in}{0.943144in}}%
\pgfpathlineto{\pgfqpoint{2.182942in}{1.100398in}}%
\pgfpathlineto{\pgfqpoint{2.233437in}{0.837605in}}%
\pgfpathlineto{\pgfqpoint{2.283933in}{0.746039in}}%
\pgfpathlineto{\pgfqpoint{2.334428in}{0.848790in}}%
\pgfpathlineto{\pgfqpoint{2.384923in}{1.163925in}}%
\pgfpathlineto{\pgfqpoint{2.435419in}{1.119952in}}%
\pgfpathlineto{\pgfqpoint{2.485914in}{0.824002in}}%
\pgfpathlineto{\pgfqpoint{2.536410in}{0.796389in}}%
\pgfpathlineto{\pgfqpoint{2.586905in}{1.147617in}}%
\pgfpathlineto{\pgfqpoint{2.637400in}{0.975078in}}%
\pgfpathlineto{\pgfqpoint{2.687896in}{1.116683in}}%
\pgfpathlineto{\pgfqpoint{2.738391in}{0.790691in}}%
\pgfpathlineto{\pgfqpoint{2.788887in}{1.260708in}}%
\pgfpathlineto{\pgfqpoint{2.839382in}{1.128960in}}%
\pgfpathlineto{\pgfqpoint{2.889877in}{0.741990in}}%
\pgfpathlineto{\pgfqpoint{2.940373in}{0.897645in}}%
\pgfpathlineto{\pgfqpoint{2.990868in}{0.975556in}}%
\pgfpathlineto{\pgfqpoint{3.041364in}{0.954862in}}%
\pgfpathlineto{\pgfqpoint{3.091859in}{0.856921in}}%
\pgfpathlineto{\pgfqpoint{3.142354in}{0.792089in}}%
\pgfpathlineto{\pgfqpoint{3.192850in}{1.466567in}}%
\pgfpathlineto{\pgfqpoint{3.243345in}{0.762594in}}%
\pgfpathlineto{\pgfqpoint{3.293841in}{1.234989in}}%
\pgfpathlineto{\pgfqpoint{3.344336in}{1.102546in}}%
\pgfpathlineto{\pgfqpoint{3.394832in}{0.858029in}}%
\pgfpathlineto{\pgfqpoint{3.445327in}{1.030473in}}%
\pgfpathlineto{\pgfqpoint{3.495822in}{0.926327in}}%
\pgfpathlineto{\pgfqpoint{3.546318in}{0.789848in}}%
\pgfpathlineto{\pgfqpoint{3.596813in}{0.677915in}}%
\pgfpathlineto{\pgfqpoint{3.647309in}{1.112589in}}%
\pgfpathlineto{\pgfqpoint{3.697804in}{1.044128in}}%
\pgfpathlineto{\pgfqpoint{3.748299in}{1.189005in}}%
\pgfpathlineto{\pgfqpoint{3.798795in}{1.169401in}}%
\pgfpathlineto{\pgfqpoint{3.849290in}{0.898030in}}%
\pgfpathlineto{\pgfqpoint{3.899786in}{0.837865in}}%
\pgfpathlineto{\pgfqpoint{3.950281in}{1.098617in}}%
\pgfpathlineto{\pgfqpoint{4.000776in}{0.909539in}}%
\pgfpathlineto{\pgfqpoint{4.051272in}{1.112247in}}%
\pgfpathlineto{\pgfqpoint{4.101767in}{1.386200in}}%
\pgfpathlineto{\pgfqpoint{4.152263in}{1.791077in}}%
\pgfpathlineto{\pgfqpoint{4.202758in}{0.982814in}}%
\pgfpathlineto{\pgfqpoint{4.253253in}{0.962823in}}%
\pgfpathlineto{\pgfqpoint{4.303749in}{0.759592in}}%
\pgfpathlineto{\pgfqpoint{4.354244in}{0.923200in}}%
\pgfpathlineto{\pgfqpoint{4.404740in}{1.207535in}}%
\pgfpathlineto{\pgfqpoint{4.455235in}{1.154989in}}%
\pgfpathlineto{\pgfqpoint{4.505730in}{0.687696in}}%
\pgfpathlineto{\pgfqpoint{4.556226in}{1.326355in}}%
\pgfpathlineto{\pgfqpoint{4.606721in}{1.354559in}}%
\pgfpathlineto{\pgfqpoint{4.657217in}{1.227746in}}%
\pgfpathlineto{\pgfqpoint{4.707712in}{1.041187in}}%
\pgfpathlineto{\pgfqpoint{4.758208in}{1.265670in}}%
\pgfpathlineto{\pgfqpoint{4.808703in}{1.241924in}}%
\pgfpathlineto{\pgfqpoint{4.859198in}{1.216928in}}%
\pgfpathlineto{\pgfqpoint{4.909694in}{1.315446in}}%
\pgfpathlineto{\pgfqpoint{4.960189in}{1.292290in}}%
\pgfpathlineto{\pgfqpoint{5.010685in}{1.342667in}}%
\pgfpathlineto{\pgfqpoint{5.061180in}{0.984824in}}%
\pgfpathlineto{\pgfqpoint{5.111675in}{0.939445in}}%
\pgfpathlineto{\pgfqpoint{5.162171in}{1.768039in}}%
\pgfpathlineto{\pgfqpoint{5.212666in}{1.807432in}}%
\pgfpathlineto{\pgfqpoint{5.263162in}{1.263284in}}%
\pgfpathlineto{\pgfqpoint{5.313657in}{1.185803in}}%
\pgfpathlineto{\pgfqpoint{5.364152in}{1.243571in}}%
\pgfpathlineto{\pgfqpoint{5.414648in}{1.060216in}}%
\pgfpathlineto{\pgfqpoint{5.465143in}{1.434648in}}%
\pgfpathlineto{\pgfqpoint{5.515639in}{1.413032in}}%
\pgfpathlineto{\pgfqpoint{5.566134in}{0.852172in}}%
\pgfpathlineto{\pgfqpoint{5.616629in}{1.438920in}}%
\pgfpathlineto{\pgfqpoint{5.667125in}{2.032626in}}%
\pgfpathlineto{\pgfqpoint{5.717620in}{1.126803in}}%
\pgfpathlineto{\pgfqpoint{5.768116in}{1.298133in}}%
\pgfpathlineto{\pgfqpoint{5.818611in}{0.811881in}}%
\pgfpathlineto{\pgfqpoint{5.869106in}{1.082679in}}%
\pgfpathlineto{\pgfqpoint{5.919602in}{0.984147in}}%
\pgfpathlineto{\pgfqpoint{5.970097in}{1.322558in}}%
\pgfusepath{stroke}%
\end{pgfscope}%
\begin{pgfscope}%
\pgfpathrectangle{\pgfqpoint{0.827140in}{0.582778in}}{\pgfqpoint{5.387860in}{1.518889in}}%
\pgfusepath{clip}%
\pgfsetrectcap%
\pgfsetroundjoin%
\pgfsetlinewidth{1.505625pt}%
\definecolor{currentstroke}{rgb}{0.500000,0.500000,0.500000}%
\pgfsetstrokecolor{currentstroke}%
\pgfsetdash{}{0pt}%
\pgfpathmoveto{\pgfqpoint{1.072043in}{1.090484in}}%
\pgfpathlineto{\pgfqpoint{1.122538in}{1.140997in}}%
\pgfpathlineto{\pgfqpoint{1.173034in}{1.082144in}}%
\pgfpathlineto{\pgfqpoint{1.223529in}{0.802006in}}%
\pgfpathlineto{\pgfqpoint{1.274024in}{0.876060in}}%
\pgfpathlineto{\pgfqpoint{1.324520in}{0.760280in}}%
\pgfpathlineto{\pgfqpoint{1.375015in}{0.721940in}}%
\pgfpathlineto{\pgfqpoint{1.425511in}{0.669250in}}%
\pgfpathlineto{\pgfqpoint{1.476006in}{0.687195in}}%
\pgfpathlineto{\pgfqpoint{1.526501in}{0.934490in}}%
\pgfpathlineto{\pgfqpoint{1.576997in}{0.771722in}}%
\pgfpathlineto{\pgfqpoint{1.627492in}{0.874882in}}%
\pgfpathlineto{\pgfqpoint{1.677988in}{0.808576in}}%
\pgfpathlineto{\pgfqpoint{1.728483in}{1.046468in}}%
\pgfpathlineto{\pgfqpoint{1.778978in}{0.783584in}}%
\pgfpathlineto{\pgfqpoint{1.829474in}{0.924725in}}%
\pgfpathlineto{\pgfqpoint{1.879969in}{0.777435in}}%
\pgfpathlineto{\pgfqpoint{1.930465in}{0.921485in}}%
\pgfpathlineto{\pgfqpoint{1.980960in}{0.834027in}}%
\pgfpathlineto{\pgfqpoint{2.031456in}{0.998316in}}%
\pgfpathlineto{\pgfqpoint{2.081951in}{0.879051in}}%
\pgfpathlineto{\pgfqpoint{2.132446in}{0.773011in}}%
\pgfpathlineto{\pgfqpoint{2.182942in}{0.920534in}}%
\pgfpathlineto{\pgfqpoint{2.233437in}{0.831177in}}%
\pgfpathlineto{\pgfqpoint{2.283933in}{0.693192in}}%
\pgfpathlineto{\pgfqpoint{2.334428in}{0.776375in}}%
\pgfpathlineto{\pgfqpoint{2.384923in}{1.001089in}}%
\pgfpathlineto{\pgfqpoint{2.435419in}{1.075207in}}%
\pgfpathlineto{\pgfqpoint{2.485914in}{0.767383in}}%
\pgfpathlineto{\pgfqpoint{2.536410in}{0.732144in}}%
\pgfpathlineto{\pgfqpoint{2.586905in}{0.959647in}}%
\pgfpathlineto{\pgfqpoint{2.637400in}{0.901714in}}%
\pgfpathlineto{\pgfqpoint{2.687896in}{1.044479in}}%
\pgfpathlineto{\pgfqpoint{2.738391in}{0.729824in}}%
\pgfpathlineto{\pgfqpoint{2.788887in}{1.046885in}}%
\pgfpathlineto{\pgfqpoint{2.839382in}{1.032313in}}%
\pgfpathlineto{\pgfqpoint{2.889877in}{0.745973in}}%
\pgfpathlineto{\pgfqpoint{2.940373in}{0.804459in}}%
\pgfpathlineto{\pgfqpoint{2.990868in}{0.880487in}}%
\pgfpathlineto{\pgfqpoint{3.041364in}{0.857061in}}%
\pgfpathlineto{\pgfqpoint{3.091859in}{0.864479in}}%
\pgfpathlineto{\pgfqpoint{3.142354in}{0.753135in}}%
\pgfpathlineto{\pgfqpoint{3.192850in}{1.300664in}}%
\pgfpathlineto{\pgfqpoint{3.243345in}{0.715635in}}%
\pgfpathlineto{\pgfqpoint{3.293841in}{1.081872in}}%
\pgfpathlineto{\pgfqpoint{3.344336in}{0.951867in}}%
\pgfpathlineto{\pgfqpoint{3.394832in}{0.837451in}}%
\pgfpathlineto{\pgfqpoint{3.445327in}{0.990511in}}%
\pgfpathlineto{\pgfqpoint{3.495822in}{0.860847in}}%
\pgfpathlineto{\pgfqpoint{3.546318in}{0.751975in}}%
\pgfpathlineto{\pgfqpoint{3.596813in}{0.651818in}}%
\pgfpathlineto{\pgfqpoint{3.647309in}{0.918771in}}%
\pgfpathlineto{\pgfqpoint{3.697804in}{0.899677in}}%
\pgfpathlineto{\pgfqpoint{3.748299in}{1.007933in}}%
\pgfpathlineto{\pgfqpoint{3.798795in}{1.014823in}}%
\pgfpathlineto{\pgfqpoint{3.849290in}{0.812926in}}%
\pgfpathlineto{\pgfqpoint{3.899786in}{0.829068in}}%
\pgfpathlineto{\pgfqpoint{3.950281in}{0.987219in}}%
\pgfpathlineto{\pgfqpoint{4.000776in}{0.764193in}}%
\pgfpathlineto{\pgfqpoint{4.051272in}{0.917011in}}%
\pgfpathlineto{\pgfqpoint{4.101767in}{1.183388in}}%
\pgfpathlineto{\pgfqpoint{4.152263in}{1.476955in}}%
\pgfpathlineto{\pgfqpoint{4.202758in}{0.831096in}}%
\pgfpathlineto{\pgfqpoint{4.253253in}{0.843950in}}%
\pgfpathlineto{\pgfqpoint{4.303749in}{0.699916in}}%
\pgfpathlineto{\pgfqpoint{4.354244in}{0.917255in}}%
\pgfpathlineto{\pgfqpoint{4.404740in}{1.039549in}}%
\pgfpathlineto{\pgfqpoint{4.455235in}{1.013096in}}%
\pgfpathlineto{\pgfqpoint{4.505730in}{0.747978in}}%
\pgfpathlineto{\pgfqpoint{4.556226in}{1.248050in}}%
\pgfpathlineto{\pgfqpoint{4.606721in}{1.303521in}}%
\pgfpathlineto{\pgfqpoint{4.657217in}{1.039420in}}%
\pgfpathlineto{\pgfqpoint{4.707712in}{0.988275in}}%
\pgfpathlineto{\pgfqpoint{4.758208in}{1.038731in}}%
\pgfpathlineto{\pgfqpoint{4.808703in}{1.144235in}}%
\pgfpathlineto{\pgfqpoint{4.859198in}{1.003386in}}%
\pgfpathlineto{\pgfqpoint{4.909694in}{1.159160in}}%
\pgfpathlineto{\pgfqpoint{4.960189in}{1.282201in}}%
\pgfpathlineto{\pgfqpoint{5.010685in}{1.082085in}}%
\pgfpathlineto{\pgfqpoint{5.061180in}{0.917421in}}%
\pgfpathlineto{\pgfqpoint{5.111675in}{0.908268in}}%
\pgfpathlineto{\pgfqpoint{5.162171in}{1.508270in}}%
\pgfpathlineto{\pgfqpoint{5.212666in}{1.650025in}}%
\pgfpathlineto{\pgfqpoint{5.263162in}{1.067719in}}%
\pgfpathlineto{\pgfqpoint{5.313657in}{1.023618in}}%
\pgfpathlineto{\pgfqpoint{5.364152in}{1.014591in}}%
\pgfpathlineto{\pgfqpoint{5.414648in}{0.944451in}}%
\pgfpathlineto{\pgfqpoint{5.465143in}{1.166489in}}%
\pgfpathlineto{\pgfqpoint{5.515639in}{1.279877in}}%
\pgfpathlineto{\pgfqpoint{5.566134in}{0.788183in}}%
\pgfpathlineto{\pgfqpoint{5.616629in}{1.348460in}}%
\pgfpathlineto{\pgfqpoint{5.667125in}{1.811155in}}%
\pgfpathlineto{\pgfqpoint{5.717620in}{1.011664in}}%
\pgfpathlineto{\pgfqpoint{5.768116in}{1.228369in}}%
\pgfpathlineto{\pgfqpoint{5.818611in}{0.883381in}}%
\pgfpathlineto{\pgfqpoint{5.869106in}{0.960820in}}%
\pgfpathlineto{\pgfqpoint{5.919602in}{0.882119in}}%
\pgfpathlineto{\pgfqpoint{5.970097in}{1.068784in}}%
\pgfusepath{stroke}%
\end{pgfscope}%
\begin{pgfscope}%
\pgfpathrectangle{\pgfqpoint{0.827140in}{0.582778in}}{\pgfqpoint{5.387860in}{1.518889in}}%
\pgfusepath{clip}%
\pgfsetbuttcap%
\pgfsetroundjoin%
\pgfsetlinewidth{1.505625pt}%
\definecolor{currentstroke}{rgb}{0.000000,0.000000,0.000000}%
\pgfsetstrokecolor{currentstroke}%
\pgfsetdash{{5.550000pt}{2.400000pt}}{0.000000pt}%
\pgfpathmoveto{\pgfqpoint{1.072043in}{1.283012in}}%
\pgfpathlineto{\pgfqpoint{1.122538in}{1.234264in}}%
\pgfpathlineto{\pgfqpoint{1.173034in}{1.230297in}}%
\pgfpathlineto{\pgfqpoint{1.223529in}{0.923830in}}%
\pgfpathlineto{\pgfqpoint{1.274024in}{0.972703in}}%
\pgfpathlineto{\pgfqpoint{1.324520in}{0.867730in}}%
\pgfpathlineto{\pgfqpoint{1.375015in}{0.710111in}}%
\pgfpathlineto{\pgfqpoint{1.425511in}{0.686937in}}%
\pgfpathlineto{\pgfqpoint{1.476006in}{0.787721in}}%
\pgfpathlineto{\pgfqpoint{1.526501in}{0.871783in}}%
\pgfpathlineto{\pgfqpoint{1.576997in}{0.992686in}}%
\pgfpathlineto{\pgfqpoint{1.627492in}{0.836349in}}%
\pgfpathlineto{\pgfqpoint{1.677988in}{0.779225in}}%
\pgfpathlineto{\pgfqpoint{1.728483in}{1.030509in}}%
\pgfpathlineto{\pgfqpoint{1.778978in}{0.922117in}}%
\pgfpathlineto{\pgfqpoint{1.829474in}{0.982604in}}%
\pgfpathlineto{\pgfqpoint{1.879969in}{0.791081in}}%
\pgfpathlineto{\pgfqpoint{1.930465in}{1.142531in}}%
\pgfpathlineto{\pgfqpoint{1.980960in}{1.001141in}}%
\pgfpathlineto{\pgfqpoint{2.031456in}{1.217764in}}%
\pgfpathlineto{\pgfqpoint{2.081951in}{1.047261in}}%
\pgfpathlineto{\pgfqpoint{2.132446in}{0.945346in}}%
\pgfpathlineto{\pgfqpoint{2.182942in}{0.971047in}}%
\pgfpathlineto{\pgfqpoint{2.233437in}{0.924290in}}%
\pgfpathlineto{\pgfqpoint{2.283933in}{0.789747in}}%
\pgfpathlineto{\pgfqpoint{2.334428in}{0.786020in}}%
\pgfpathlineto{\pgfqpoint{2.384923in}{1.016125in}}%
\pgfpathlineto{\pgfqpoint{2.435419in}{1.023883in}}%
\pgfpathlineto{\pgfqpoint{2.485914in}{0.794520in}}%
\pgfpathlineto{\pgfqpoint{2.536410in}{0.775013in}}%
\pgfpathlineto{\pgfqpoint{2.586905in}{1.260873in}}%
\pgfpathlineto{\pgfqpoint{2.637400in}{0.899328in}}%
\pgfpathlineto{\pgfqpoint{2.687896in}{0.910255in}}%
\pgfpathlineto{\pgfqpoint{2.738391in}{0.774735in}}%
\pgfpathlineto{\pgfqpoint{2.788887in}{1.154751in}}%
\pgfpathlineto{\pgfqpoint{2.839382in}{1.003343in}}%
\pgfpathlineto{\pgfqpoint{2.889877in}{0.864309in}}%
\pgfpathlineto{\pgfqpoint{2.940373in}{0.704418in}}%
\pgfpathlineto{\pgfqpoint{2.990868in}{0.894715in}}%
\pgfpathlineto{\pgfqpoint{3.041364in}{0.945988in}}%
\pgfpathlineto{\pgfqpoint{3.091859in}{0.850108in}}%
\pgfpathlineto{\pgfqpoint{3.142354in}{0.806007in}}%
\pgfpathlineto{\pgfqpoint{3.192850in}{1.315741in}}%
\pgfpathlineto{\pgfqpoint{3.243345in}{0.766939in}}%
\pgfpathlineto{\pgfqpoint{3.293841in}{1.106692in}}%
\pgfpathlineto{\pgfqpoint{3.344336in}{0.839249in}}%
\pgfpathlineto{\pgfqpoint{3.394832in}{0.773493in}}%
\pgfpathlineto{\pgfqpoint{3.445327in}{1.174170in}}%
\pgfpathlineto{\pgfqpoint{3.495822in}{0.987382in}}%
\pgfpathlineto{\pgfqpoint{3.546318in}{0.941214in}}%
\pgfpathlineto{\pgfqpoint{3.596813in}{0.839030in}}%
\pgfpathlineto{\pgfqpoint{3.647309in}{1.020894in}}%
\pgfpathlineto{\pgfqpoint{3.697804in}{0.920969in}}%
\pgfpathlineto{\pgfqpoint{3.748299in}{1.042589in}}%
\pgfpathlineto{\pgfqpoint{3.798795in}{1.179650in}}%
\pgfpathlineto{\pgfqpoint{3.849290in}{1.814304in}}%
\pgfpathlineto{\pgfqpoint{3.899786in}{0.940827in}}%
\pgfpathlineto{\pgfqpoint{3.950281in}{0.896537in}}%
\pgfpathlineto{\pgfqpoint{4.000776in}{0.743411in}}%
\pgfpathlineto{\pgfqpoint{4.051272in}{0.954499in}}%
\pgfpathlineto{\pgfqpoint{4.101767in}{1.432618in}}%
\pgfpathlineto{\pgfqpoint{4.152263in}{0.917629in}}%
\pgfpathlineto{\pgfqpoint{4.202758in}{0.907117in}}%
\pgfpathlineto{\pgfqpoint{4.253253in}{0.837097in}}%
\pgfpathlineto{\pgfqpoint{4.303749in}{0.843119in}}%
\pgfpathlineto{\pgfqpoint{4.354244in}{1.154620in}}%
\pgfpathlineto{\pgfqpoint{4.404740in}{1.063938in}}%
\pgfpathlineto{\pgfqpoint{4.455235in}{1.011087in}}%
\pgfpathlineto{\pgfqpoint{4.505730in}{0.984987in}}%
\pgfpathlineto{\pgfqpoint{4.556226in}{0.966394in}}%
\pgfpathlineto{\pgfqpoint{4.606721in}{1.139173in}}%
\pgfpathlineto{\pgfqpoint{4.657217in}{0.790227in}}%
\pgfpathlineto{\pgfqpoint{4.707712in}{1.024811in}}%
\pgfpathlineto{\pgfqpoint{4.758208in}{1.565471in}}%
\pgfpathlineto{\pgfqpoint{4.808703in}{0.751322in}}%
\pgfpathlineto{\pgfqpoint{4.859198in}{1.060034in}}%
\pgfpathlineto{\pgfqpoint{4.909694in}{1.335060in}}%
\pgfpathlineto{\pgfqpoint{4.960189in}{0.733898in}}%
\pgfpathlineto{\pgfqpoint{5.010685in}{1.130988in}}%
\pgfpathlineto{\pgfqpoint{5.061180in}{1.134669in}}%
\pgfpathlineto{\pgfqpoint{5.111675in}{1.042150in}}%
\pgfpathlineto{\pgfqpoint{5.162171in}{1.693007in}}%
\pgfpathlineto{\pgfqpoint{5.212666in}{1.254666in}}%
\pgfpathlineto{\pgfqpoint{5.263162in}{0.971881in}}%
\pgfpathlineto{\pgfqpoint{5.313657in}{0.911619in}}%
\pgfpathlineto{\pgfqpoint{5.364152in}{0.877286in}}%
\pgfpathlineto{\pgfqpoint{5.414648in}{0.965612in}}%
\pgfpathlineto{\pgfqpoint{5.465143in}{1.093667in}}%
\pgfpathlineto{\pgfqpoint{5.515639in}{0.870829in}}%
\pgfpathlineto{\pgfqpoint{5.566134in}{1.110917in}}%
\pgfpathlineto{\pgfqpoint{5.616629in}{1.987972in}}%
\pgfpathlineto{\pgfqpoint{5.667125in}{1.830340in}}%
\pgfpathlineto{\pgfqpoint{5.717620in}{0.959289in}}%
\pgfpathlineto{\pgfqpoint{5.768116in}{1.274940in}}%
\pgfpathlineto{\pgfqpoint{5.818611in}{1.545964in}}%
\pgfpathlineto{\pgfqpoint{5.869106in}{1.373613in}}%
\pgfpathlineto{\pgfqpoint{5.919602in}{1.529360in}}%
\pgfpathlineto{\pgfqpoint{5.970097in}{1.445647in}}%
\pgfusepath{stroke}%
\end{pgfscope}%
\begin{pgfscope}%
\pgfpathrectangle{\pgfqpoint{0.827140in}{0.582778in}}{\pgfqpoint{5.387860in}{1.518889in}}%
\pgfusepath{clip}%
\pgfsetbuttcap%
\pgfsetroundjoin%
\pgfsetlinewidth{1.505625pt}%
\definecolor{currentstroke}{rgb}{0.500000,0.500000,0.500000}%
\pgfsetstrokecolor{currentstroke}%
\pgfsetdash{{5.550000pt}{2.400000pt}}{0.000000pt}%
\pgfpathmoveto{\pgfqpoint{1.072043in}{1.153014in}}%
\pgfpathlineto{\pgfqpoint{1.122538in}{1.236806in}}%
\pgfpathlineto{\pgfqpoint{1.173034in}{1.186535in}}%
\pgfpathlineto{\pgfqpoint{1.223529in}{0.879452in}}%
\pgfpathlineto{\pgfqpoint{1.274024in}{0.983712in}}%
\pgfpathlineto{\pgfqpoint{1.324520in}{0.885898in}}%
\pgfpathlineto{\pgfqpoint{1.375015in}{0.733100in}}%
\pgfpathlineto{\pgfqpoint{1.425511in}{0.658706in}}%
\pgfpathlineto{\pgfqpoint{1.476006in}{0.769184in}}%
\pgfpathlineto{\pgfqpoint{1.526501in}{0.901070in}}%
\pgfpathlineto{\pgfqpoint{1.576997in}{0.919519in}}%
\pgfpathlineto{\pgfqpoint{1.627492in}{0.792646in}}%
\pgfpathlineto{\pgfqpoint{1.677988in}{0.786212in}}%
\pgfpathlineto{\pgfqpoint{1.728483in}{0.985116in}}%
\pgfpathlineto{\pgfqpoint{1.778978in}{0.927892in}}%
\pgfpathlineto{\pgfqpoint{1.829474in}{1.034943in}}%
\pgfpathlineto{\pgfqpoint{1.879969in}{0.788634in}}%
\pgfpathlineto{\pgfqpoint{1.930465in}{1.031293in}}%
\pgfpathlineto{\pgfqpoint{1.980960in}{0.996938in}}%
\pgfpathlineto{\pgfqpoint{2.031456in}{1.181739in}}%
\pgfpathlineto{\pgfqpoint{2.081951in}{1.126880in}}%
\pgfpathlineto{\pgfqpoint{2.132446in}{0.847179in}}%
\pgfpathlineto{\pgfqpoint{2.182942in}{0.888875in}}%
\pgfpathlineto{\pgfqpoint{2.233437in}{0.964006in}}%
\pgfpathlineto{\pgfqpoint{2.283933in}{0.761751in}}%
\pgfpathlineto{\pgfqpoint{2.334428in}{0.761046in}}%
\pgfpathlineto{\pgfqpoint{2.384923in}{0.943593in}}%
\pgfpathlineto{\pgfqpoint{2.435419in}{1.091071in}}%
\pgfpathlineto{\pgfqpoint{2.485914in}{0.773817in}}%
\pgfpathlineto{\pgfqpoint{2.536410in}{0.764766in}}%
\pgfpathlineto{\pgfqpoint{2.586905in}{1.187933in}}%
\pgfpathlineto{\pgfqpoint{2.637400in}{0.901435in}}%
\pgfpathlineto{\pgfqpoint{2.687896in}{0.933749in}}%
\pgfpathlineto{\pgfqpoint{2.738391in}{0.749236in}}%
\pgfpathlineto{\pgfqpoint{2.788887in}{1.036046in}}%
\pgfpathlineto{\pgfqpoint{2.839382in}{1.007269in}}%
\pgfpathlineto{\pgfqpoint{2.889877in}{0.878303in}}%
\pgfpathlineto{\pgfqpoint{2.940373in}{0.721186in}}%
\pgfpathlineto{\pgfqpoint{2.990868in}{0.908490in}}%
\pgfpathlineto{\pgfqpoint{3.041364in}{0.944812in}}%
\pgfpathlineto{\pgfqpoint{3.091859in}{0.916546in}}%
\pgfpathlineto{\pgfqpoint{3.142354in}{0.802219in}}%
\pgfpathlineto{\pgfqpoint{3.192850in}{1.285423in}}%
\pgfpathlineto{\pgfqpoint{3.243345in}{0.741075in}}%
\pgfpathlineto{\pgfqpoint{3.293841in}{1.088220in}}%
\pgfpathlineto{\pgfqpoint{3.344336in}{0.829872in}}%
\pgfpathlineto{\pgfqpoint{3.394832in}{0.854087in}}%
\pgfpathlineto{\pgfqpoint{3.445327in}{1.220899in}}%
\pgfpathlineto{\pgfqpoint{3.495822in}{0.978863in}}%
\pgfpathlineto{\pgfqpoint{3.546318in}{0.877438in}}%
\pgfpathlineto{\pgfqpoint{3.596813in}{0.822715in}}%
\pgfpathlineto{\pgfqpoint{3.647309in}{0.916814in}}%
\pgfpathlineto{\pgfqpoint{3.697804in}{0.864805in}}%
\pgfpathlineto{\pgfqpoint{3.748299in}{1.022793in}}%
\pgfpathlineto{\pgfqpoint{3.798795in}{1.066663in}}%
\pgfpathlineto{\pgfqpoint{3.849290in}{1.605934in}}%
\pgfpathlineto{\pgfqpoint{3.899786in}{0.895964in}}%
\pgfpathlineto{\pgfqpoint{3.950281in}{0.939750in}}%
\pgfpathlineto{\pgfqpoint{4.000776in}{0.701119in}}%
\pgfpathlineto{\pgfqpoint{4.051272in}{0.913447in}}%
\pgfpathlineto{\pgfqpoint{4.101767in}{1.339144in}}%
\pgfpathlineto{\pgfqpoint{4.152263in}{0.905855in}}%
\pgfpathlineto{\pgfqpoint{4.202758in}{0.817321in}}%
\pgfpathlineto{\pgfqpoint{4.253253in}{0.837031in}}%
\pgfpathlineto{\pgfqpoint{4.303749in}{0.782467in}}%
\pgfpathlineto{\pgfqpoint{4.354244in}{1.111810in}}%
\pgfpathlineto{\pgfqpoint{4.404740in}{1.053129in}}%
\pgfpathlineto{\pgfqpoint{4.455235in}{0.992491in}}%
\pgfpathlineto{\pgfqpoint{4.505730in}{0.878376in}}%
\pgfpathlineto{\pgfqpoint{4.556226in}{0.937417in}}%
\pgfpathlineto{\pgfqpoint{4.606721in}{1.165588in}}%
\pgfpathlineto{\pgfqpoint{4.657217in}{0.805266in}}%
\pgfpathlineto{\pgfqpoint{4.707712in}{1.075733in}}%
\pgfpathlineto{\pgfqpoint{4.758208in}{1.467367in}}%
\pgfpathlineto{\pgfqpoint{4.808703in}{0.777569in}}%
\pgfpathlineto{\pgfqpoint{4.859198in}{0.988442in}}%
\pgfpathlineto{\pgfqpoint{4.909694in}{1.459814in}}%
\pgfpathlineto{\pgfqpoint{4.960189in}{0.760752in}}%
\pgfpathlineto{\pgfqpoint{5.010685in}{1.012874in}}%
\pgfpathlineto{\pgfqpoint{5.061180in}{1.129599in}}%
\pgfpathlineto{\pgfqpoint{5.111675in}{1.110217in}}%
\pgfpathlineto{\pgfqpoint{5.162171in}{1.750732in}}%
\pgfpathlineto{\pgfqpoint{5.212666in}{1.317501in}}%
\pgfpathlineto{\pgfqpoint{5.263162in}{0.912113in}}%
\pgfpathlineto{\pgfqpoint{5.313657in}{0.953990in}}%
\pgfpathlineto{\pgfqpoint{5.364152in}{0.991723in}}%
\pgfpathlineto{\pgfqpoint{5.414648in}{0.900719in}}%
\pgfpathlineto{\pgfqpoint{5.465143in}{0.975841in}}%
\pgfpathlineto{\pgfqpoint{5.515639in}{0.890229in}}%
\pgfpathlineto{\pgfqpoint{5.566134in}{1.066299in}}%
\pgfpathlineto{\pgfqpoint{5.616629in}{1.880527in}}%
\pgfpathlineto{\pgfqpoint{5.667125in}{1.675241in}}%
\pgfpathlineto{\pgfqpoint{5.717620in}{0.926934in}}%
\pgfpathlineto{\pgfqpoint{5.768116in}{1.312236in}}%
\pgfpathlineto{\pgfqpoint{5.818611in}{1.622351in}}%
\pgfpathlineto{\pgfqpoint{5.869106in}{1.361648in}}%
\pgfpathlineto{\pgfqpoint{5.919602in}{1.511935in}}%
\pgfpathlineto{\pgfqpoint{5.970097in}{1.255508in}}%
\pgfusepath{stroke}%
\end{pgfscope}%
\begin{pgfscope}%
\pgfsetrectcap%
\pgfsetmiterjoin%
\pgfsetlinewidth{0.803000pt}%
\definecolor{currentstroke}{rgb}{0.000000,0.000000,0.000000}%
\pgfsetstrokecolor{currentstroke}%
\pgfsetdash{}{0pt}%
\pgfpathmoveto{\pgfqpoint{0.827140in}{0.582778in}}%
\pgfpathlineto{\pgfqpoint{0.827140in}{2.101667in}}%
\pgfusepath{stroke}%
\end{pgfscope}%
\begin{pgfscope}%
\pgfsetrectcap%
\pgfsetmiterjoin%
\pgfsetlinewidth{0.803000pt}%
\definecolor{currentstroke}{rgb}{0.000000,0.000000,0.000000}%
\pgfsetstrokecolor{currentstroke}%
\pgfsetdash{}{0pt}%
\pgfpathmoveto{\pgfqpoint{6.215000in}{0.582778in}}%
\pgfpathlineto{\pgfqpoint{6.215000in}{2.101667in}}%
\pgfusepath{stroke}%
\end{pgfscope}%
\begin{pgfscope}%
\pgfsetrectcap%
\pgfsetmiterjoin%
\pgfsetlinewidth{0.803000pt}%
\definecolor{currentstroke}{rgb}{0.000000,0.000000,0.000000}%
\pgfsetstrokecolor{currentstroke}%
\pgfsetdash{}{0pt}%
\pgfpathmoveto{\pgfqpoint{0.827140in}{0.582778in}}%
\pgfpathlineto{\pgfqpoint{6.215000in}{0.582778in}}%
\pgfusepath{stroke}%
\end{pgfscope}%
\begin{pgfscope}%
\pgfsetrectcap%
\pgfsetmiterjoin%
\pgfsetlinewidth{0.803000pt}%
\definecolor{currentstroke}{rgb}{0.000000,0.000000,0.000000}%
\pgfsetstrokecolor{currentstroke}%
\pgfsetdash{}{0pt}%
\pgfpathmoveto{\pgfqpoint{0.827140in}{2.101667in}}%
\pgfpathlineto{\pgfqpoint{6.215000in}{2.101667in}}%
\pgfusepath{stroke}%
\end{pgfscope}%
\begin{pgfscope}%
\pgftext[x=3.521070in,y=2.185000in,,base]{\sffamily\fontsize{12.000000}{14.400000}\selectfont Expected Penalty Contributions}%
\end{pgfscope}%
\begin{pgfscope}%
\pgfsetbuttcap%
\pgfsetmiterjoin%
\definecolor{currentfill}{rgb}{1.000000,1.000000,1.000000}%
\pgfsetfillcolor{currentfill}%
\pgfsetfillopacity{0.800000}%
\pgfsetlinewidth{1.003750pt}%
\definecolor{currentstroke}{rgb}{0.800000,0.800000,0.800000}%
\pgfsetstrokecolor{currentstroke}%
\pgfsetstrokeopacity{0.800000}%
\pgfsetdash{}{0pt}%
\pgfpathmoveto{\pgfqpoint{0.924362in}{0.995498in}}%
\pgfpathlineto{\pgfqpoint{3.255080in}{0.995498in}}%
\pgfpathquadraticcurveto{\pgfqpoint{3.282858in}{0.995498in}}{\pgfqpoint{3.282858in}{1.023276in}}%
\pgfpathlineto{\pgfqpoint{3.282858in}{2.004444in}}%
\pgfpathquadraticcurveto{\pgfqpoint{3.282858in}{2.032222in}}{\pgfqpoint{3.255080in}{2.032222in}}%
\pgfpathlineto{\pgfqpoint{0.924362in}{2.032222in}}%
\pgfpathquadraticcurveto{\pgfqpoint{0.896585in}{2.032222in}}{\pgfqpoint{0.896585in}{2.004444in}}%
\pgfpathlineto{\pgfqpoint{0.896585in}{1.023276in}}%
\pgfpathquadraticcurveto{\pgfqpoint{0.896585in}{0.995498in}}{\pgfqpoint{0.924362in}{0.995498in}}%
\pgfpathclose%
\pgfusepath{stroke,fill}%
\end{pgfscope}%
\begin{pgfscope}%
\pgfsetrectcap%
\pgfsetroundjoin%
\pgfsetlinewidth{1.505625pt}%
\definecolor{currentstroke}{rgb}{0.000000,0.000000,0.000000}%
\pgfsetstrokecolor{currentstroke}%
\pgfsetdash{}{0pt}%
\pgfpathmoveto{\pgfqpoint{0.952140in}{1.905679in}}%
\pgfpathlineto{\pgfqpoint{1.229918in}{1.905679in}}%
\pgfusepath{stroke}%
\end{pgfscope}%
\begin{pgfscope}%
\pgftext[x=1.341029in,y=1.857068in,left,base]{\sffamily\fontsize{10.000000}{12.000000}\selectfont \(\displaystyle 1 - E[r_{t, CLIP}^+]\), control}%
\end{pgfscope}%
\begin{pgfscope}%
\pgfsetrectcap%
\pgfsetroundjoin%
\pgfsetlinewidth{1.505625pt}%
\definecolor{currentstroke}{rgb}{0.500000,0.500000,0.500000}%
\pgfsetstrokecolor{currentstroke}%
\pgfsetdash{}{0pt}%
\pgfpathmoveto{\pgfqpoint{0.952140in}{1.656915in}}%
\pgfpathlineto{\pgfqpoint{1.229918in}{1.656915in}}%
\pgfusepath{stroke}%
\end{pgfscope}%
\begin{pgfscope}%
\pgftext[x=1.341029in,y=1.608303in,left,base]{\sffamily\fontsize{10.000000}{12.000000}\selectfont \(\displaystyle E[r_{t, CLIP}^-] - 1\), control}%
\end{pgfscope}%
\begin{pgfscope}%
\pgfsetbuttcap%
\pgfsetroundjoin%
\pgfsetlinewidth{1.505625pt}%
\definecolor{currentstroke}{rgb}{0.000000,0.000000,0.000000}%
\pgfsetstrokecolor{currentstroke}%
\pgfsetdash{{5.550000pt}{2.400000pt}}{0.000000pt}%
\pgfpathmoveto{\pgfqpoint{0.952140in}{1.408150in}}%
\pgfpathlineto{\pgfqpoint{1.229918in}{1.408150in}}%
\pgfusepath{stroke}%
\end{pgfscope}%
\begin{pgfscope}%
\pgftext[x=1.341029in,y=1.359539in,left,base]{\sffamily\fontsize{10.000000}{12.000000}\selectfont \(\displaystyle 1 - E[r_{t, CLIP}^+]\), experimental}%
\end{pgfscope}%
\begin{pgfscope}%
\pgfsetbuttcap%
\pgfsetroundjoin%
\pgfsetlinewidth{1.505625pt}%
\definecolor{currentstroke}{rgb}{0.500000,0.500000,0.500000}%
\pgfsetstrokecolor{currentstroke}%
\pgfsetdash{{5.550000pt}{2.400000pt}}{0.000000pt}%
\pgfpathmoveto{\pgfqpoint{0.952140in}{1.159386in}}%
\pgfpathlineto{\pgfqpoint{1.229918in}{1.159386in}}%
\pgfusepath{stroke}%
\end{pgfscope}%
\begin{pgfscope}%
\pgftext[x=1.341029in,y=1.110775in,left,base]{\sffamily\fontsize{10.000000}{12.000000}\selectfont \(\displaystyle E[r_{t, CLIP}^-] - 1 \), experimental}%
\end{pgfscope}%
\end{pgfpicture}%
\makeatother%
\endgroup%
}\\
    \caption{Results: Swimmer-v2 environment}
    \label{fig:4}
\end{figure}

\section{Conclusions}
The results of this research so far suggest that, within the framework of a PPO
algorithm, using two dynamic clipping parameters that are optimized to address
problem-specific imbalances is a promising technique to improve the performance
of PPO. 

There are many different directions in which we can continue to develop this
research. In terms of runtime and computational complexity, it would likely be
useful to use a problem-independent heuristic method to minimize the
discrepancy. Additionally, the discrepancy can likely be more effectively
reduced by applying $\epsilon$-optimization on a per-state, rather than
per-batch, basis. Another important consideration is how the expected
discrepancy relates to the actual penalty contribution discrepancy.

\section{Acknowledgements}
I would like to thank my mentor, Dr. Sicun Gao, for being very attentive and
supportive of my work, and for directing me to relevant research at the
cutting-edge of reinforcement learning. I would also like to thank the Academic
Enrichment Program at UCSD, especially Dr. Kirsten Kung, for hosting me during
my work this summer as a member of the UC Scholars Program.

%\theendnotes
\bibliographystyle{plain}
\bibliography{bib}

\end{document}
